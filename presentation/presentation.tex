%\documentclass[first,firstsupp,handout,compress,notes,navigation]{ETHclass}
%\documentclass[first,firstsupp,handout,lastsupp]{ETHclass}
\documentclass[first,firstsupp,handout,last]{ETHclass}
%\documentclass[first,firstsupp]{ETHclass}

% use sensible encoding
\usepackage[T1]{fontenc}

%% Input encoding 'utf8'. In some cases you might need 'utf8x' for
%% extra symbols. Not all editors, especially on Windows, are UTF-8
%% capable, so you may want to use 'latin1' instead.
\usepackage[utf8x]{inputenc}

%% The AMS-LaTeX extensions for mathematical typesetting.  Do not
%% remove.
\usepackage{amsmath,amssymb,amsfonts,mathrsfs}

\usepackage{amsthm}
%% additional package extending mathematical typesetting.
\usepackage{thmtools}
\usepackage{thm-restate}

%% Some more packages that you may want to use.  Have a look at the
%% file, and consult the package docs for each.
\input{extrapackages}

%% Our layout configuration.  DO NOT CHANGE.
%\input{layoutsetup}

%% Theorem environments.  You will have to adapt this for a German
%% thesis.
%% Theorem-like environments

%% This can be changed according to language. You can comment out the ones you
%% don't need.

\numberwithin{equation}{chapter}

%% German theorems
%\newtheorem{satz}{Satz}[chapter]
%\newtheorem{beispiel}[satz]{Beispiel}
%\newtheorem{bemerkung}[satz]{Bemerkung}
%\newtheorem{korrolar}[satz]{Korrolar}
%\newtheorem{definition}[satz]{Definition}
%\newtheorem{lemma}[satz]{Lemma}
%\newtheorem{proposition}[satz]{Proposition}

%% English variants
\newtheorem{theorem}{Theorem}[chapter]
\newtheorem{example}[theorem]{Example}
\newtheorem{remark}[theorem]{Remark}
\newtheorem{corollary}[theorem]{Corollary}
\newtheorem{lemma}[theorem]{Lemma}
\newtheorem{proposition}[theorem]{Proposition}
\newtheorem{observation}[theorem]{Observation}

\theoremstyle{definition}
\theorembodyfont{\normalfont}
%% end def with blacksquare symbol
\theoremsymbol{\ensuremath{\lozenge}}
\newtheorem{definition}[theorem]{Definition}

%% Proof environment with a small square as a "qed" symbol
\theoremstyle{nonumberplain}
\theorembodyfont{\normalfont}
\theoremsymbol{\ensuremath{\square}}
\theoremseparator{.}
\newtheorem{proof}{Proof}

%\newtheorem{beweis}{Beweis}

\declaretheorem[name=Theorem, numberwithin=chapter]{thm}


%% Helpful macros.
%%% Custom commands
%% ===============

%% Special characters for number sets, e.g. real or complex numbers.
\newcommand{\C}{\mathbb{C}}
\newcommand{\K}{\mathbb{K}}
\newcommand{\N}{\mathbb{N}}
\newcommand{\Q}{\mathbb{Q}}
\newcommand{\R}{\mathbb{R}}
\newcommand{\Z}{\mathbb{Z}}
\newcommand{\X}{\mathbb{X}}

\newcommand{\cX}{\mathcal{X}}
\newcommand{\cH}{\mathcal{H}}
\newcommand{\cW}{\mathcal{W}}
\newcommand{\cG}{\mathcal{G}}
\newcommand{\cB}{\mathcal{B}}
\newcommand{\cP}{\mathcal{P}}
\newcommand{\cR}{\mathcal{R}}
\newcommand{\cD}{\mathcal{D}}

%define our own code commands
%use capital latters as most of these commands is already defined
\renewcommand{\For}{\textbf{for }}
\renewcommand{\If}{\textbf{if }}
\renewcommand{\Else}{\textbf{else }}
\renewcommand{\Return}{\textbf{return }}
\newcommand{\Then}{\textbf{then }}
\newcommand{\Do}{\textbf{do: }}
\renewcommand{\And}{\textbf{and }}
\newcommand{\Or}{\textbf{or }}
\newcommand{\Run}{\textbf{run }}
\newcommand{\To}{\textbf{to }}

%% Fixed/scaling delimiter examples (see mathtools documentation)
\DeclarePairedDelimiter\abs{\lvert}{\rvert}
\DeclarePairedDelimiter\norm{\lVert}{\rVert}

%% Use the alternative epsilon per default and define the old one as \oldepsilon
\let\oldepsilon\epsilon
\renewcommand{\epsilon}{\ensuremath\varepsilon}

%% Also set the alternate phi as default.
\let\oldphi\phi
\renewcommand{\phi}{\ensuremath{\varphi}}

% New command that introduces a tab
\newcommand{\itab}[1]{\hspace{0em}\rlap{#1}}
\newcommand{\tab}[1]{\hspace{.2\textwidth}\rlap{#1}}

\DeclareMathOperator{\la0}{\leftarrow}
\DeclareMathOperator{\ra0}{\rightarrow}

\DeclareMathOperator{\hash}{\mathit{hash}}
\DeclareMathOperator{\CanonicalSuccess}{\mathit{CanonicalSuccess}}
\DeclareMathOperator{\Success}{\mathit{Success}}

%the DWVP for the permutation
\DeclareMathOperator{\PiDWVP}{\Pi_{DWVP}}
%the DWPV for the k-wise product of permutations
\DeclareMathOperator{\kPiDWVP}{\Pi_{DWVP}^{(k)}}

% Options for beamer:
%
% 9,10,11,12,13,14,17pt  Fontsizes
%
% compress: navigation bar becomes smaller
% t       : place contents of frames on top (alternative: b,c)
% handout : handoutversion
% notes   : show notes
% notes=onlyslideswithnotes
%
%hyperref={bookmarksopen,bookmarksnumbered} : Needed for menues in
%                                             acrobat. Also need
%                                             pdftex as option or
%                                             compile with
% pdflatex '\PassOptionsToPackage{pdftex,bookmarksopen,bookmarksnumbered}{hyperref} \input{file}'

%\usepackage{beamerseminar}
%\usepackage[accumulated]{beamerseminar}
                                % remove ``accumulated'' option
                                % for original behaviour
%\usepackage{beamerbasenotes}
%\setbeamertemplate{note page}[plain]
%\setbeameroption{notes on second screen}

%\setbeamertemplate{note page}[plain]
\setbeamertemplate{note page}{\ \\[.3cm]
\textbf{\color{blue}Notes:}\\%[0.1cm]
{\footnotesize %\tiny
\insertnote}}
%\setbeameroption{notes on second screen}


\setbeamertemplate{navigation symbols}{} % suppresses all navigation symbols:
% \setbeamertemplate{navigation symbols}[horizontal] % Organizes the navigation symbols horizontally.
% \setbeamertemplate{navigation symbols}[vertical] % Organizes the navigation symbols vertically.
% \setbeamertemplate{navigation symbols}[only frame symbol] % Shows only the navigational symbol for navigating frames.

\setlayoutscale{0.5}
\setparametertextfont{\scriptsize}
\setlabelfont{\scriptsize}

\begin{document}

\title{Hardness amplification for weakly verifiable cryptographic primitives}
\author{Grzegorz M\k{a}kosa}
\advisors{Advisors: Prof. Dr. Thomas Holenstein, Dr. Robin Künzler}
\department{Department of Computer Science}

%\institute{D-MTEC - ETH Zurich}

\begin{frame}
\maketitle
\end{frame}
%\note{}

\begin{frame}[t]
\frametitle{Agenda}
\begin{itemize}
  \item<+-> Motivation and problem statement
  \item<+-> Background and related work
  \item<+-> My contribution
  \item<+-> Results
  \item<+-> Discussion
% \item<+-> Item order 1
% \item<+-> Item order 1
%   \begin{itemize}
%   \item Item order 2
%   \item Item order 2
%         \begin{itemize}
%         \item Item order 3
%         \end{itemize}
%   \item Item
%   \end{itemize}
% \item<+-> Item order 1
% \item<+-> Item order 1
% \item<+-> Item order 1
\end{itemize}
\end{frame}

\begin{frame}
  \frametitle{Hardness Amplification}
\end{frame}
\note{Describe hardness amplification theorem, maybe the implication that we use (it makes sense to say that when we state the theorem), why is it important -- building a weakly securite primitives is easier}

\begin{frame}
  \frametitle{Weakly verifiable cryptographic primitives}
\end{frame}
\note{Give examples of weakly verifiable cryptographic primitives, motivate why it is not possible to use the traditional approach of Yao / why it is necessary to extend it.}

\begin{frame}
  \frametitle{Previous works HS}
\end{frame}
\note{Set up ground for introducing my results}

\begin{frame}
  \frametitle{Previous works DIJK}
\end{frame}
\note{Set up ground for introducing my results}

\begin{frame}
  \frametitle{My contribution I}
\end{frame}
\note{State theorem}

\begin{frame}
  \frametitle{My contribution II}
\end{frame}
\note{Give intuition behind the result}

\begin{frame}
  \frametitle{Discussion}
\end{frame}
\note{Give intuition behind the result}

\begin{frame}
\frametitle{Questions}
\end{frame}

% \frame{
%   \frametitle{Frame with Box}
%   \vspace{-0.2cm}
%   \begin{block}{\textbf{Box Title}}
%    Text within the box...
%   \end{block}
% }
%\note{}


% \frame{
% \frametitle{Include Figures}

% \includegraphics[width=0.4\textwidth]{penguin}
% \begin{itemize}
% \item This is a pdf figure called `penguin.pdf'
% \end{itemize}
% }
% %\note{}

% \begin{frame}[allowframebreaks=0.8]
% \frametitle{Overcrowded Slide: allowframebreaks=0.8}
%   Word1 Word2 Word3 Word4 Word5 Word6 Word7 Word8 Word9 Word10 Word11
%   Word12 Word13 Word14 Word15 Word16 Word17 Word18 Word19 Word20
%   Word21 Word22 Word23 Word24 Word25 Word26 Word27 Word28 Word29
%   Word30 Word31 Word32 Word33 Word34 Word35 Word36 Word37 Word38
%   Word39 Word40 Word41 Word42 Word43 Word44 Word45 Word46 Word47
%   Word48 Word49 Word50 Word51 Word52 Word53 Word54 Word55 Word56
%   Word57 Word58 Word59 Word60 Word61 Word62 Word63 Word64 Word65
%   Word66 Word67 Word68 Word69 Word70 Word71 Word72 Word73 Word74
%   Word75 Word76 Word77 Word78 Word79 Word80 Word81 Word82 Word83
%   Word84 Word85 Word86 Word87
% \end{frame}

% \begin{frame}[shrink=10]
% \frametitle{Overcrowded Slide: shrink=10}
%   Word1 Word2 Word3 Word4 Word5 Word6 Word7 Word8 Word9 Word10 Word11
%   Word12 Word13 Word14 Word15 Word16 Word17 Word18 Word19 Word20
%   Word21 Word22 Word23 Word24 Word25 Word26 Word27 Word28 Word29
%   Word30 Word31 Word32 Word33 Word34 Word35 Word36 Word37 Word38
%   Word39 Word40 Word41 Word42 Word43 Word44 Word45 Word46 Word47
%   Word48 Word49 Word50 Word51 Word52 Word53 Word54 Word55 Word56
%   Word57 Word58 Word59 Word60 Word61 Word62 Word63 Word64 Word65
%   Word66 Word67 Word68 Word69 Word70 Word71 Word72 Word73 Word74
%   Word75 Word76 Word77 Word78 Word79 Word80 Word81 Word82 Word83
%   Word84 Word85 Word86 Word87
% \end{frame}

\end{document}