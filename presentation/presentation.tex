%\documentclass[first,firstsupp,handout,compress,notes,navigation]{ETHclass}
%\documentclass[first,firstsupp,handout,lastsupp]{ETHclass}
%\documentclass[handout,notes]{ETHclass}
\documentclass[first,firstsupp,handout,last]{ETHclass}
%\documentclass[first,firstsupp]{ETHclass}

% use sensible encoding
\usepackage[T1]{fontenc}

%% Input encoding 'utf8'. In some cases you might need 'utf8x' for
%% extra symbols. Not all editors, especially on Windows, are UTF-8
%% capable, so you may want to use 'latin1' instead.
\usepackage[utf8x]{inputenc}

%% The AMS-LaTeX extensions for mathematical typesetting.  Do not
%% remove.
\usepackage{amsmath,amssymb,amsfonts,mathrsfs}

\usepackage{amsthm}
%% additional package extending mathematical typesetting.
\usepackage{thmtools}
\usepackage{thm-restate}

\usepackage{textpos}
% for graphics
\usepackage{tikz}
\usetikzlibrary{arrows, decorations.markings}
% for double arrows a la chef
% adapt line thickness and line width, if needed
\tikzstyle{vecArrow} = [thick, decoration={markings,mark=at position
   1 with {\arrow[semithick]{open triangle 60}}},
   double distance=1.4pt, shorten >= 5.5pt,
   preaction = {decorate},
   postaction = {draw,line width=1.4pt, white,shorten >= 4.5pt}]
\tikzstyle{innerWhite} = [semithick, white,line width=1.4pt, shorten >= 4.5pt]
\usetikzlibrary{decorations.shapes}

%% Some more packages that you may want to use.  Have a look at the
%% file, and consult the package docs for each.
%% See the TeXed file for more explanations

%% [OPT] Multi-rowed cells in tabulars
%\usepackage{multirow}

%% [REC] Intelligent cross reference package. This allows for nice
%% combined references that include the reference and a hint to where
%% to look for it.
\usepackage{varioref}

%% [OPT] Easily changeable quotes with \enquote{Text}
%\usepackage[german=swiss]{csquotes}

%% [REC] Format dates and time depending on locale
\usepackage{datetime}

%% [OPT] Provides a \cancel{} command to stroke through mathematics.
%\usepackage{cancel}

%% [NEED] This allows for additional typesetting tools in mathmode.
%% See its excellent documentation.
\usepackage{mathtools}

%% [ADV] Conditional commands
%\usepackage{ifthen}

%% [OPT] Manual large braces or other delimiters.
%\usepackage{bigdelim, bigstrut}

%% [REC] Alternate vector arrows. Use the command \vv{} to get scaled
%% vector arrows.
%\usepackage[h]{esvect}

%% [NEED] Some extensions to tabulars and array environments.
\usepackage{array}

%% [OPT] Postscript support via pstricks graphics package. Very
%% diverse applications.
%\usepackage{pstricks,pst-all}

%% [?] This seems to allow us to define some additional counters.
%\usepackage{etex}

%% [ADV] XY-Pic to typeset some matrix-style graphics
%\usepackage[all]{xy}

%% [OPT] This is needed to generate an index at the end of the
%% document.
%\usepackage{makeidx}

%% [OPT] Fancy package for source code listings.  The template text
%% needs it for some LaTeX snippets; remove/adapt the \lstset when you
%% remove the template content.
\usepackage{listings}
\lstset{language=TeX,basicstyle={\normalfont\ttfamily}}

%% [REC] Fancy character protrusion.  Must be loaded after all fonts.
\usepackage[activate]{pdfcprot}

%% [REC] Nicer tables.  Read the excellent documentation.
\usepackage{booktabs}

%%pseudocode and algorithms
\usepackage{algpseudocode}
\usepackage{algorithm}

%% define comments in single line
\algnewcommand{\LineComment}[1]{\State \(\triangleright\) #1}

%%placing in the right place
\usepackage{float}

\usepackage{caption}

\DeclareCaptionFormat{algor}{
  \hrulefill\par\offinterlineskip\vskip1pt
    \textbf{#1#2}#3\offinterlineskip\hrulefill}
\DeclareCaptionStyle{algori}{singlelinecheck=off,format=algor,labelsep=space}
\captionsetup[algorithm]{style=algori}

\algnewcommand\algorithmicinput{\textbf{Input:}}
\algnewcommand\Input{\item[\algorithmicinput]}

\algnewcommand\algorithmicauxinput{\textbf{Auxiliary input:}}
\algnewcommand\AuxInput{\item[\algorithmicauxinput]}

%indention in algorithms
\newcommand{\pushcode}[1][1]{\hskip\dimexpr#1\algorithmicindent\relax}

% hyphen in math mode
\def\hyph{\text{-}}



%% Our layout configuration.  DO NOT CHANGE.
%%% Memoir layout setup

%% NOTE: You are strongly advised not to change any of them unless you
%% know what you are doing.  These settings strongly interact in the
%% final look of the document.

% Dependencies
\usepackage{ETHlogo}

% Turn extra space before chapter headings off.
\setlength{\beforechapskip}{0pt}

\nonzeroparskip
\parindent=0pt
\defaultlists

% Chapter style redefinition
\makeatletter

\if@twoside
  \pagestyle{Ruled}
  \copypagestyle{chapter}{Ruled}
\else
  \pagestyle{ruled}
  \copypagestyle{chapter}{ruled}
\fi
\makeoddhead{chapter}{}{}{}
\makeevenhead{chapter}{}{}{}
\makeheadrule{chapter}{\textwidth}{0pt}
\copypagestyle{abstract}{empty}

\makechapterstyle{bianchimod}{%
  \chapterstyle{default}
  \renewcommand*{\chapnamefont}{\normalfont\Large\sffamily}
  \renewcommand*{\chapnumfont}{\normalfont\Large\sffamily}
  \renewcommand*{\printchaptername}{%
    \chapnamefont\centering\@chapapp}
  \renewcommand*{\printchapternum}{\chapnumfont {\thechapter}}
  \renewcommand*{\chaptitlefont}{\normalfont\huge\sffamily}
  \renewcommand*{\printchaptertitle}[1]{%
    \hrule\vskip\onelineskip \centering \chaptitlefont\textbf{\vphantom{gyM}##1}\par}
  \renewcommand*{\afterchaptertitle}{\vskip\onelineskip \hrule\vskip
    \afterchapskip}
  \renewcommand*{\printchapternonum}{%
    \vphantom{\chapnumfont {9}}\afterchapternum}}

% Use the newly defined style
\chapterstyle{bianchimod}

\setsecheadstyle{\Large\bfseries\sffamily}
\setsubsecheadstyle{\large\bfseries\sffamily}
\setsubsubsecheadstyle{\bfseries\sffamily}
\setparaheadstyle{\normalsize\bfseries\sffamily}
\setsubparaheadstyle{\normalsize\itshape\sffamily}
\setsubparaindent{0pt}

% Set captions to a more separated style for clearness
\captionnamefont{\sffamily\bfseries\footnotesize}
\captiontitlefont{\sffamily\footnotesize}
\setlength{\intextsep}{16pt}
\setlength{\belowcaptionskip}{1pt}

% Set section and TOC numbering depth to subsection
\setsecnumdepth{subsection}
\settocdepth{subsection}

%% Titlepage adjustments
\pretitle{\vspace{0pt plus 0.7fill}\begin{center}\HUGE\sffamily\bfseries}
\posttitle{\end{center}\par}
\preauthor{\par\begin{center}\let\and\\\Large\sffamily}
\postauthor{\end{center}}
\predate{\par\begin{center}\Large\sffamily}
\postdate{\end{center}}

\def\@advisors{}
\newcommand{\advisors}[1]{\def\@advisors{#1}}
\def\@department{}
\newcommand{\department}[1]{\def\@department{#1}}
\def\@thesistype{}
\newcommand{\thesistype}[1]{\def\@thesistype{#1}}

\renewcommand{\maketitlehooka}{\noindent\ETHlogo[2in]}

\renewcommand{\maketitlehookb}{\vspace{1in}%
  \par\begin{center}\Large\sffamily\@thesistype\end{center}}

\renewcommand{\maketitlehookd}{%
  \vfill\par
  \begin{flushright}
    \sffamily
    \@advisors\par
    \@department, ETH Z\"urich
  \end{flushright}
}

\checkandfixthelayout

\setlength{\droptitle}{-48pt}

\makeatother

% This defines how theorems should look. Best leave as is.
\theoremstyle{plain}
\setlength\theorempostskipamount{0pt}

\usepackage[framemethod=tikz]{mdframed}
\newdimen\linenumbersep

\newcommand{\linenumber}[1]{%
  \linenumbersep 4pt%
  \advance\linenumbersep\mdflength{innerleftmargin}%
  \advance\linenumbersep\mdflength{innerlinewidth}%
  \advance\linenumbersep\mdflength{middlelinewidth}%
  \advance\linenumbersep\mdflength{outerlinewidth}%
  \advance\linenumbersep\mdflength{linewidth}%
  \makebox[0pt][r]{{\rmfamily\tiny#1}\hspace*{\linenumbersep}}}

\newenvironment{codeblock}%
   {\medskip\begin{mdframed}\setlength{\parindent}{0cm}}%
   {\end{mdframed}\medskip}
\newcommand{\Ind}{\mbox{}}
\newcommand{\IndI}{\mbox\qquad}
\newcommand{\IndII}{\mbox\qquad\qquad}
\newcommand{\IndIII}{\mbox\qquad\qquad\qquad}
\newcommand{\IndIIII}{\mbox\qquad\qquad\qquad\qquad}

\newenvironment{todo}%
   {\medskip\begin{mdframed}\setlength{\parindent}{0cm}}%
   {\end{mdframed}\medskip}


%%% Local Variables:
%%% mode: latex
%%% TeX-master: "thesis"
%%% End:


%% Theorem environments.  You will have to adapt this for a German
%% thesis.
%% Theorem-like environments

%% This can be changed according to language. You can comment out the ones you
%% don't need.

\numberwithin{equation}{chapter}

% declare theorem style
\declaretheoremstyle[
spaceabove=6pt, spacebelow=6pt,
headfont=\normalfont\bfseries,
notefont=\mdseries, notebraces={(}{)},
bodyfont=\normalfont\itshape,
postheadspace=1em,
%qed=\qedsymbol
]{thm_sty}

% declare definition style
\declaretheoremstyle[
spaceabove=6pt, spacebelow=6pt,
headfont=\normalfont\bfseries,
notefont=\mdseries, notebraces={(}{)},
bodyfont=\normalfont,
postheadspace=1em,
qed=\ensuremath{\lozenge}
]{def_sty}


%% English variants
\declaretheorem[style=thm_sty, name=Theorem, numberwithin=chapter]{theorem}
\declaretheorem[style=thm_sty, name=Observation, sibling=theorem]{observation}
\declaretheorem[style=thm_sty, name=Lemma, sibling=theorem]{lemma}
\declaretheorem[style=def_sty, name=Definition, sibling=theorem]{definition}
%\declaretheorem[style=def_sty, name=Proof]{proof}

%\newtheorem{theorem}{Theorem}[chapter]
% \newtheorem{example}[theorem]{Example}
% \newtheorem{remark}[theorem]{Remark}
% \newtheorem{corollary}[theorem]{Corollary}
% \newtheorem{lemma}[theorem]{Lemma}
% \newtheorem{proposition}[theorem]{Proposition}
% \newtheorem{observation}[theorem]{Observation}

%\theoremstyle{definition}
%\theorembodyfont{\normalfont}
%% end def with blacksquare symbol
%\theoremsymbol{\ensuremath{\lozenge}}
%\newtheorem{definition}[theorem]{Definition}

%%Proof environment with a small square as a "qed" symbol
%\theoremstyle{nonumberplain}
%\theorembodyfont{\normalfont}
%\theoremsymbol{\ensuremath{\square}}
%\theoremseparator{.}
%\newtheorem{proof}{Proof}



%% Helpful macros.
%% Custom commands
%% ===============

%% Special characters for number sets, e.g. real or complex numbers.
\newcommand{\C}{\mathbb{C}}
\newcommand{\D}{\mathbb{D}}
\newcommand{\K}{\mathbb{K}}
\newcommand{\N}{\mathbb{N}}
\newcommand{\M}{\mathbb{M}}
\newcommand{\Q}{\mathbb{Q}}
\newcommand{\R}{\mathbb{R}}
\newcommand{\T}{\mathbb{T}}
\newcommand{\X}{\mathbb{X}}
\newcommand{\Z}{\mathbb{Z}}


\newcommand{\cA}{\mathcal{A}}
\newcommand{\cB}{\mathcal{B}}
\newcommand{\cX}{\mathcal{X}}
\newcommand{\cH}{\mathcal{H}}
\newcommand{\cW}{\mathcal{W}}
\newcommand{\cG}{\mathcal{G}}
\newcommand{\cP}{\mathcal{P}}
\newcommand{\cR}{\mathcal{R}}
\newcommand{\cD}{\mathcal{D}}
\newcommand{\cF}{\mathcal{F}}
\newcommand{\cM}{\mathcal{M}}
\newcommand{\cK}{\mathcal{K}}
\newcommand{\cT}{\mathcal{T}}
\newcommand{\cS}{\mathcal{S}}
\newcommand{\cQ}{\mathcal{Q}}
\newcommand{\cV}{\mathcal{V}}
\newcommand{\cI}{\mathcal{I}}

%define our own code commands
%use capital latters as most of these commands is already defined
\renewcommand{\For}{\textbf{for }}
\renewcommand{\If}{\textbf{if }}
\renewcommand{\Else}{\textbf{else }}
\renewcommand{\Return}{\textbf{return }}
\newcommand{\Then}{\textbf{then }}
\newcommand{\Do}{\textbf{do: }}
\renewcommand{\And}{\textbf{and }}
\newcommand{\Or}{\textbf{or }}
\newcommand{\Run}{\textbf{run }}
\newcommand{\To}{\textbf{to }}
\renewcommand{\Repeat}{\textbf{Repeat }}

%% Fixed/scaling delimiter examples (see mathtools documentation)
\DeclarePairedDelimiter\abs{\lvert}{\rvert}
\DeclarePairedDelimiter\norm{\lVert}{\rVert}

%% Use the alternative epsilon per default and define the old one as \oldepsilon
\let\oldepsilon\epsilon
\renewcommand{\epsilon}{\ensuremath\varepsilon}

%% Also set the alternate phi as default.
\let\oldphi\phi
\renewcommand{\phi}{\ensuremath{\varphi}}

% New command that introduces a tab
\newcommand{\itab}[1]{\hspace{0em}\rlap{#1}}
\newcommand{\tab}[1]{\hspace{.2\textwidth}\rlap{#1}}

\DeclareMathOperator{\la0}{\leftarrow}
\DeclareMathOperator{\ra0}{\rightarrow}

\DeclareMathOperator{\hash}{\mathit{hash}}
\DeclareMathOperator{\CanonicalSuccess}{\mathit{CanonicalSuccess}}
\DeclareMathOperator{\Success}{\mathit{Success}}
\DeclareMathOperator{\poly}{\mathit{poly}}
\DeclareMathOperator{\p}{\mathit{p}}
\DeclareMathOperator{\Time}{\mathit{Time}}
\DeclareMathOperator{\Gen}{\text{Gen}}
\DeclareMathOperator{\FindHash}{\text{FindHash}}
\DeclareMathOperator{\Bad}{\mathit{Bad}}
\DeclareMathOperator{\Good}{\mathit{Good}}
\DeclareMathOperator{\trans}{\mathit{trans}}
\DeclareMathOperator{\Canonical}{\mathit{Canonical}}


%the DWVP for the permutation
\DeclareMathOperator{\PiDWVP}{\Pi_{DWVP}}
%the DWPV for the k-wise product of permutations
\DeclareMathOperator{\kPiDWVP}{\Pi_{DWVP}^{(k)}}

\setbeamertemplate{note page}{\ \\[.3cm]
\textbf{\color{blue}Notes:}\\%[0.1cm]
{\footnotesize %\tiny
\insertnote}}

% suppresses all navigation symbols:
\setbeamertemplate{navigation symbols}{}

\setlayoutscale{0.5}
\setparametertextfont{\scriptsize}
\setlabelfont{\scriptsize}

\begin{document}

\title{Hardness amplification for weakly verifiable cryptographic primitives}
\author{Grzegorz M\k{a}kosa}
\advisors{Advisors: Prof. Dr. Thomas Holenstein, Dr. Robin Künzler}
\department{Department of Computer Science}

\begin{frame}
\maketitle
\end{frame}

\begin{frame}[t]
\frametitle{Agenda}
\begin{itemize}
  \item<+-> Motivation and problem statement
  \item<+-> Background and related work
  \item<+-> My contribution
  \item<+-> Results
  \item<+-> Discussion
\end{itemize}
\end{frame}

\begin{frame}[fragile,t]
  \frametitle{Hardness Amplification}
% \begin{columns}
% \begin{column}{.48\textwidth}
% \color{red}\rule{\linewidth}{4pt}

% Left Part
% \end{column}%
% \hfill%
% \begin{column}{.48\textwidth}
% \color{blue}\rule{\linewidth}{4pt}

% Right Part
% \end{column}%
% \end{columns}
\tikzset{decorate sep/.style 2 args={decorate,decoration={shape backgrounds,shape=circle,shape size=#1,shape sep=#2}}}

Is solving parallel repetition of problems substantially harder than a single instance of a problem?
\vspace{50pt}
  \[\begin{tikzpicture}[remember picture,overlay]
% left node
\node (rect) at (-4,0) [draw,thin, minimum width=2cm,minimum height=1cm] {$S$};
%\node at (-5.5,0.25) {$f(x)$};
\draw (-5.5,0) -- (-5,0);
\draw (-2.5,0) -- (-3.0,0);
% arrow
 \draw[vecArrow] (-1.5,0) to (0.75,0);
% right nodes
\node (rect) at (3,2) [draw,thin, minimum width=2cm,minimum height=1cm] {$S_1$};
\draw (1.5,2) -- (2,  2);
\draw (4.0,2) -- (4.5,2);

\node (rect) at (3,0.5) [draw,thin, minimum width=2cm,minimum height=1cm] {$S_2$};
\draw (1.5,0.5) -- (2,  0.5);
\draw (4.0,0.5) -- (4.5,0.5);

\draw[decorate sep={1mm}{5mm},fill] (3,-.3) -- (3,-1.5);

\node (rect) at (3,-2.) [draw,thin, minimum width=2cm,minimum height=1cm] {$S_k$};
\draw (1.5,-2) -- (2,  -2);
\draw (4.0,-2) -- (4.5,-2);

\end{tikzpicture}\]
%
\note{
  \begin{enumerate}
    \item Describe hardness amplification theorem in general,
    \item The implication that we use to prove the theorem (it makes sense to say that when we state the theorem),
    \item Why is it important - building a weakly secure primitives is easier
  \end{enumerate}
}
\end{frame}

\begin{frame}[t]
  \frametitle{Weakly Verifiable Puzzles}
  \begin{enumerate}
    \item An algorithm $G$ generates a puzzle $p$ together with some secrecy information $s$.
    \item A solver given $p$ has to find a correct solution.
    \item It is hard for the solver to verify the correctness of a solution given only $p$.
    \item A verification algorithm has access to $s$ which makes the task of checking the correctness of a solution easy.
  \end{enumerate}
  \note{
    \begin{enumerate}
    \end{enumerate}
  }
\end{frame}

\begin{frame}[c]
  \frametitle{Weakly Verifiable Primitives - Example}
%  \includegraphics[width=0.30\textwidth,natwidth=100,natheight=100]{captcha.png}
%\pgfdeclareimage{captcha}{captcha.png}
\[\begin{tikzpicture}[remember picture,overlay]
% generation
\node (rect) at (-4,0) [draw,thin, minimum width=2cm,minimum height=1cm] {$\Gen$};
\draw (-5.5,0) -- (-5,0);
\draw (-1, 1.5) -- (-2.5, 1.5) -- (-2.5, 0.25) -- (-3, 0.25);
\draw (-3, -0.25) -- (3, -0.25);

% solver's input i.e. CAPTCHA
\node at (-1.9, 1.6) {\pgfbox[center,bottom]{\includegraphics[width=0.16\textwidth,natwidth=100,natheight=100]{captcha.png}}};

%one of the verifies input's the unique solution
\node[draw] at (0,0) {Solution YF4JY};

% solver
\node (rect) at (0,1.5) [draw,thin, minimum width=2cm,minimum height=1cm] {Solver};
\draw (1, 1.5) -- (2.5, 1.5) -- (2.5,0);

% verification algorithm
\node (rect) at (4,0) [draw,thin, minimum width=2cm,minimum height=1cm] {Verif.};
\draw (5, 0) -- (5.5,0);

\end{tikzpicture}\]


  \note{
    \begin{enumerate}
      \item Give examples of weakly verifiable cryptographic primitives (that are really weakly verifiable),
        motivate why it is not possible to use the traditional approach of Yao / why it is necessary to extend it.
      \item Use CAPTCHAs as an example (maybe but a nice picture)
    \end{enumerate}
  }
\end{frame}

\begin{frame}
  \frametitle{Dynamic Cryptographic Primitives}
  \note{
    \begin{enumerate}
    \item Give examples of dynamic weakly verifiable cryptographic primitives MAC seems to be easily understandable
    \end{enumerate}
  }
\end{frame}

\begin{frame}
  \frametitle{Interactive Cryptographic Primitives}
  \note{
    \begin{enumerate}
    \item Bit commitment protocols
    \end{enumerate}
  }
\end{frame}

\begin{frame}
  \frametitle{Previous work of Cannetti, Halevi, and Steiner}
  \note{
    \begin{enumerate}
    \item It is the only understandable result
    \item Explaining how to deal with weakly verifiable property in this example seems to be possible, it is not possible for others
    \item Very general overview of the work
    \end{enumerate}
  }
\end{frame}

\begin{frame}
  \frametitle{Previous work DIJK}
\note{
  \begin{enumerate}
    \item Threshold functions
    \item Dynamic puzzles recall what it is.
    \item Give a general overview of the method i.e. what they do
  \end{enumerate}
}
\end{frame}

\begin{frame}
  \frametitle{Previous work HS}
  \note{
    \begin{enumerate}
     \item Use general monotone binary function -- has to introduce
     \item Works also for interactive puzzles
     \item Extends proof of CHS
    \end{enumerate}
}
\end{frame}

\begin{frame}
  \frametitle{My contribution I}
  \note{
    \begin{enumerate}
    \item State formally the theorem being proven? Does it make sense without formal introduction of weakly verifiable puzzles
    \item Maybe use pictures to visualize the main theorem?
    \end{enumerate}
}
\end{frame}

\begin{frame}
  \frametitle{My contribution II}
  \note {
    \begin{enumerate}
      \item Give a very short overview of how you prove your theorem
      \item Have to state the bounds that I get and say that it may not be optimal
    \end{enumerate}
}
\end{frame}
\note{Give intuition behind the result}

\begin{frame}
  \frametitle{Discussion}
  \note{
    \begin{enumerate}
      \item Summarize the work
      \item Say that it may not be optimal
    \end{enumerate}
}
\end{frame}

\begin{frame}
\frametitle{Questions}
\end{frame}

\note{
  \begin{enumerate}
    \item What is weak one-way function:
      there exists a polynomial that lower bounds the failure probability of every polynomial time algorithm. Goldreich p.35.
    \item What is strong one-way function?

    \item Why it may not be optimal?

    \item What did you try to show that it is possible to improve your results?

    \item Is it computational or information theoretic result?

    \item What kind of security-games for MAC are modeled by DWVP?

    \item Why sequential repetition implies security amplification?

    \item What if the number of puzzles in the sequential repetition is very big i.e. our result holds only with high probability?

    \item Does repeating the whole algorithm may help and increase the success probability?

    \item What is the cryptographic primitive?

    \item What is the cryptographic scheme?

    \item Is there a simple proof for sequential repetition?

    \item Why is it not possible to fix all coordinates? Give examples in $n$-dim space.

    \item Try to justify why we have this form of the theorem, what does it mean? and what w
  \end{enumerate}
}

\note{
  \begin{enumerate}

  \item Where does the proof break when you want to apply it to the large surplus

  \item When does the proof breaks? -- exactly and why we do not have to care about it too much

  \item What it the intuition behind the surplus?

  \item Be manage to explain the function on the right hand-side.

  \item Why do we need to consider two surpluses

  \item How does the optimization problem for gap amplification looks like?

  \item Why we cannot perfect hardness amplification?

  \item Why $\Pr[c \in \cG_1 \setminus \cG_0]$ is small but we still have a large surplus

  \item Why can we assume that the verification algorithm can be deterministic

  \item Followup question: what is proven in DIJK09

  \item Followup question: why is the hint circuit H probabilistic

  \item Is defining the puzzle only by poser is not artificial as an instance is defined by a pair poser-solver

  \item There are no examples of puzzles that are both interactive and dynamic what about this?

  \item Number of rounds for which parallel repetition works as i.e. the intuition behind the > 3-round protocols

  \end{enumerate}
}


\note {
  \begin{enumerate}

  \item Why does obsevation 5.1 is true?

  \item What is function what is relation

  \item Why WVP does not imply one-way function

  \item Is it possible that sampling does not work?

  \item Is number of hint and verification queries limitation for the solver or the verify

  \item Why do we iterate $\frac{1}{\epsilon}$ times in $\Gen$, is it efficient, when it might not be efficient what happens with $h+v$ then?

  \item Think about different computational contexts for this theorem. Under what conditions for $h,v,\epsilon$ does it make sense

  \item Be able to explain the definition in DIJK09.

  \item What is the Coron's proof about? Why it does not work?

  \item $k$-wise independent functions; how do they look like?

  \item Random questions about hash functions

  \item (if not covered in the presentation) is your result optimal?

  \item Relation with soundness error of four-round protocols

  \end{enumerate}
}
% \frame{
%   \frametitle{Frame with Box}
%   \vspace{-0.2cm}
%   \begin{block}{\textbf{Box Title}}
%    Text within the box...
%   \end{block}
% }
%\note{}


% \frame{
% \frametitle{Include Figures}

% \includegraphics[width=0.4\textwidth]{penguin}
% \begin{itemize}
% \item This is a pdf figure called `penguin.pdf'
% \end{itemize}
% }
% %\note{}

% \begin{frame}[allowframebreaks=0.8]
% \frametitle{Overcrowded Slide: allowframebreaks=0.8}
%   Word1 Word2 Word3 Word4 Word5 Word6 Word7 Word8 Word9 Word10 Word11
%   Word12 Word13 Word14 Word15 Word16 Word17 Word18 Word19 Word20
%   Word21 Word22 Word23 Word24 Word25 Word26 Word27 Word28 Word29
%   Word30 Word31 Word32 Word33 Word34 Word35 Word36 Word37 Word38
%   Word39 Word40 Word41 Word42 Word43 Word44 Word45 Word46 Word47
%   Word48 Word49 Word50 Word51 Word52 Word53 Word54 Word55 Word56
%   Word57 Word58 Word59 Word60 Word61 Word62 Word63 Word64 Word65
%   Word66 Word67 Word68 Word69 Word70 Word71 Word72 Word73 Word74
%   Word75 Word76 Word77 Word78 Word79 Word80 Word81 Word82 Word83
%   Word84 Word85 Word86 Word87
% \end{frame}

% \begin{frame}[shrink=10]
% \frametitle{Overcrowded Slide: shrink=10}
%   Word1 Word2 Word3 Word4 Word5 Word6 Word7 Word8 Word9 Word10 Word11
%   Word12 Word13 Word14 Word15 Word16 Word17 Word18 Word19 Word20
%   Word21 Word22 Word23 Word24 Word25 Word26 Word27 Word28 Word29
%   Word30 Word31 Word32 Word33 Word34 Word35 Word36 Word37 Word38
%   Word39 Word40 Word41 Word42 Word43 Word44 Word45 Word46 Word47
%   Word48 Word49 Word50 Word51 Word52 Word53 Word54 Word55 Word56
%   Word57 Word58 Word59 Word60 Word61 Word62 Word63 Word64 Word65
%   Word66 Word67 Word68 Word69 Word70 Word71 Word72 Word73 Word74
%   Word75 Word76 Word77 Word78 Word79 Word80 Word81 Word82 Word83
%   Word84 Word85 Word86 Word87
% \end{frame}

% \item<+-> Item order 1
% \item<+-> Item order 1
%   \begin{itemize}
%   \item Item order 2
%   \item Item order 2
%         \begin{itemize}
%         \item Item order 3
%         \end{itemize}
%   \item Item
%   \end{itemize}
% \item<+-> Item order 1
% \item<+-> Item order 1
% \item<+-> Item order 1

\end{document}