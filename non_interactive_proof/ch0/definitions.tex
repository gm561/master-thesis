%FIXME is x a puzzle?
\begin{definition}[Dynamic weakly verifiable puzzle.]
  \label{def:dwvp}
  A dynamic weakly verifiable puzzle (DWVP) is defined by a probabilistic algorithm $P$
  called a problem poser. We write $P(\pi)$ to denote the algorithm $P$ with the randomness $\pi$.
  The algorithm $P$ outputs circuits $\Gamma_{V}$, $\Gamma_{H}$ and a puzzle $x \in \{0,1\}^{*}$.
  The circuit $\Gamma_{V}$ takes as input $q \in Q$, an answer $y \in \{0,1\}^*$,
  and outputs $1$ if $y$ is a correct solution of $x$ for $q$ and $0$ otherwise.
  The circuit $\Gamma_H$ on input $q \in Q$ outputs a hint such that $\Gamma_V(q,\Gamma_H(q)) = 1$.

  A problem solver $S$ is a probabilistic algorithm that takes as input a puzzle $x$,
  and has oracle access to $\Gamma_V$ and $\Gamma_H$.
  The randomness used by $S$ is denoted by $\rho$. The execution of $S$ with the input $x$ and the randomness $\rho$
  is denoted by $S(x, \rho)$. The queries of $S$ to $\Gamma_V$ are called verification queries, and to $\Gamma_H$ hint queries.
  The solver $S$ can ask at most $h$ hint queries, $v$ verification queries, and successfully solves the puzzle if and only if
  it makes a verification query $(q,y)$ such that $\Gamma_V(q,y) = 1$, when it has not previously asked for a hint query on $q$.
\end{definition}
%
%
\begin{definition}[$k$-wise direct-product of DWVPs.]
Let $g: \{0,1\}^{k} \rightarrow \{0,1\}$ be a monotone function and $P^{(1)}$ a problem poser as in Definition \ref{def:dwvp}.
The $k$-wise direct product of $P^{(1)}$ is a DWVP defined by a probabilistic algorithm $P^{(g)}$.
Let $\pi^{(k)} := (\pi_1, \dots, \pi_k)$ be the randomness used by $P^{(g)}$, and $P^{(g)}(\pi^{(k)})$ denote the execution of $P^{(g)}$ with the randomness $\pi^{(k)}$.
The poser $P^{(g)}$ outputs:
a verification circuit
\begin{align*}
  \Gamma_V^{(g)} (q, y_1, \dots, y_k) := g(\Gamma_V^{1}(q, y_1), \dots, \Gamma_V^{k}(q, y_k)),
\end{align*}
a hint circuit
\begin{align*}
  \Gamma_H^{(k)} (q) := (\Gamma_H^{1}(q), \dots, \Gamma_H^{k}(q)),
\end{align*}
and a puzzle $x^{(k)} := (x_1, \dots, x_k)$, where the $i$-th instance $(x_i, \Gamma_V^{i}, \Gamma_H^{i} ) := P^{(1)}(\pi_i)$.
\end{definition}
%
% A dynamic weakly verifiable puzzle is a special case of a $k$-wise direct product, when $k$ equals one and $g$ is the identity function.
% We consider the following random experiment in which a $k$-wise direct product of DWVP defined by $P^{(k)}$ is solved by a circuit $C$.
% The circuit $C$, takes as input $x^{(k)}$, random bitstring $\rho$, and has oracle access to $\Gamma^{(g)}_V, \Gamma^{(k)}_H$
%
We consider the following random experiment in which a DWVP, defined by $P$, is solved by a random circuit $C$.
\begin{codeblock}
  \textbf{Experiment $Success^{P, C^{(\cdot, \cdot)}}(\pi, \rho) $}
  \medskip
  \hrule
  \medskip
  \textbf{Oracle:} A problem poser $P$, a solver circuit $C^{(\cdot,\cdot)}$.\\
  \textbf{Input:}  Bitstrings $\pi$, $\rho$.\\
  \textbf{Output:} A bit $b \in \{0,1\}$.
  \medskip\hrule\medskip
  $(x, \Gamma_V, \Gamma_H) := P(\pi)$ \\
  Run $C^{\Gamma_V,\Gamma_H}(x, \rho)$ \\
  \IndI Let $Q_{Solved} := \{q: \text{ $C^{\Gamma_V, \Gamma_H}$ asked a verification query $(q,y)$ and $\Gamma_V(q, y) = 1$} \}$\\
  \IndI Let $Q_{Hint} := \{q: \text{$C^{\Gamma_V, \Gamma_H}$ asked a hint query on q} \}$\\
  \textbf{If} $\exists q \in Q_{solved} : q \notin Q_{Hint}$ \then \\
  \IndI \textbf{return} $1$\\
  \textbf{else} \\
  \IndI \textbf{return} $0$\\
\end{codeblock}
%
The success probability of $C$ in solving DWVP posed by $P$ is
\begin{align}
 \underset{\pi, \rho}{\Pr}[Success^{P,C^{(\cdot, \cdot)}}(\pi, \rho) = 1].
\end{align}
%
\begin{todo}
  \textbf{TODO:} Do we have to fix $P$ here?
  \textbf{TODO:} Do the circuit bound is well defined?
  \textbf{TODO:} What happens when $8(h+v) \left(\underset{\mu \leftarrow \mu_\delta^k}{\Pr}[g(\mu) = 1] + \varepsilon\right) \geq 1$ then the formula does not work
  \textbf{TODO:} Define $\mu_{\delta}^{(k)}$
\end{todo}
%
\begin{theorem}{\textbf{Security amplification for a dynamic weakly verifiable puzzle.}}
\label{th:sec_amp_for_dwvp}\\
Let $P^{(1)}$ be a fixed problem poser as in Definition \ref{def:dwvp}. There exists a probabilistic
algorithm $Gen(C, g, \varepsilon, \delta, n, v, h)$ which takes as input a solver circuit $C$ for the $k$-wise
direct product of $P^{(1)}$, a monotone function $g:\{0,1\}^k \rightarrow \{0,1\}$, parameters $\varepsilon, \delta,n$,
the number of verification queries $v$, and hint queries $h$ asked by $C$, and outputs a circuit $D$
such that the following holds: \\
If $C$ is such that \\
  \begin{align*}
    \underset{\pi^{(k)}, \rho}{\Pr}[Success^{P^{(g)}, C}(\pi^{(k)}, \rho) = 1] \geq 8(h+v) \left(\underset{\mu \leftarrow \mu_\delta^k}{\Pr}[g(\mu) = 1] + \varepsilon\right)
  \end{align*}
then $D$ satisfies almost surely
  \begin{align*}
    \underset{\pi, \rho}{\Pr}[Success^{P^{(1)},D}(\pi, \rho) = 1] \geq (\delta + \frac{\varepsilon}{6k}).
  \end{align*}
Additionally, $D$ and $Gen$ require only oracle access to $g$ and $C$. Furthermore, $D$ asks at most $h$ hint queries, $v$ verification queries and
$Size(D) \leq Size(C) \cdot \Theta(\frac{6k}{\varepsilon})$ and $Time(Gen) = poly(k, \frac{1}{\varepsilon}, n, v, h)$.
\end{theorem}
%
%
The Theorem \ref{th:sec_amp_for_dwvp} implies that if there is no good solver for $P^{(1)}$, then a good solver for $P^{(k)}$ does not exist.

The idea of the algorithm $Gen$ is to find $k-1$ puzzles and a position for an input puzzle $x$, such that
when $C$ runs with these $k-1$ puzzles, and $x$ placed on the right position, then $x$ is often successfully solved.
\begin{todo}
  \textbf{TODO:} Write it more clearly
\end{todo}
To find such a position for $x$ and $k-1$ puzzles $Gen$ runs $C$ repeatedly on different $k-1$ tuples of puzzles.
Even if $Gen$ finds a set of puzzles and a position for $x$, such that $x$ is often solved it may still not
constitute a valid solution, as an additional requirement is needed that this happens often for $q$
on which a hint query was not asked before. To satisfy this requirement we split $Q$.

%%% Local Variables:
%%% mode: latex
%%% TeX-master: "../master"
%%% End:
