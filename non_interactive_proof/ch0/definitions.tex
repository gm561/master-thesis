\begin{definition} {\textbf{Dynamic weakly verifiable puzzle (non interactive version)}}\\
  A dynamic weakly verifiable puzzle (DWVP) is defined by a probabilistic algorithm $P(\pi)$,
  called a problem poser, that takes as input chosen uniformly at random bitstring $\pi \in \{0,1\}^l$,
  and produces circuits $\Gamma_{V}$, $\Gamma_{H}$ and a puzzle $x \in \{0,1\}^{*}$.
  The circuit $\Gamma_{V}$ takes as input $q \in Q$ and an answer $y \in \{0,1\}^*$.
  If $\Gamma_V(q,y) = 1$ then $y$ is a correct solution of a puzzle $x$ for $q$.
  The circuit $\Gamma_H$ on input $q$ provides a hint such that $\Gamma_V(q,\Gamma_H(q)) = 1$.
  The probabilistic algorithm $S$, called a solver, has oracle access to $\Gamma_V$ and $\Gamma_H$.
  The calls of $S$ to $\Gamma_V$ are verification queries and to $\Gamma_H$ are hint queries.
  The solver $S$ can ask at most $h$ hint queries, $v$ verification queries, and successfully solves DWVP if and only if
  it makes a verification query $(q,y)$ such that $\Gamma_V(q,y) = 1$, when it has not previously asked for a hint query on this $q$.
\end{definition}
%
%
\begin{definition}{\textbf{$k$-wise direct product of dynamic weakly verifiable puzzles}}\\
Let $g: \{0,1\}^{k} \rightarrow \{0,1\}$ be a monotone function, and $P^{(1)}$ a problem poser used to generate an instance of DWVP.
A $k$-wise direct product of dynamic weakly verifiable puzzles ($DWVP^k$) is defined by a probabilistic algorithm $P^{(g)}\left(\pi_1, \dots, \pi_k \right)$,
where $(\pi_1, \dots, \pi_k) \in \{0,1\}^{k \cdot l}$ is chosen uniformly at random.
The algorithm $P^{(g)}\left(\pi_1, \dots, \pi_k \right)$ generates $k$ independent instances of dynamic weakly verifiable puzzles,
where the $i$-th instance $(x_i, \Gamma_V^{(i)}, \Gamma_H^{(i)} )$ is produced by executing $P^{(1)}(\pi_i)$.
Finally, $P^{(g)}$ outputs a verification circuit
\begin{align*}
  \Gamma_V^{(g)} (q, y_1, \dots, y_k) := g(\Gamma_V^{1}(q, y_1), \dots, \Gamma_V^{k}(q, y_k)),
\end{align*}
a hint circuit
\begin{align*}
  \Gamma_H^{(k)} (q) := (\Gamma_H^{1}(q), \dots, \Gamma_H^{k}(q)),
\end{align*}
and a puzzle $x^{(k)} := (x_1, \dots, x_k)$.

The probabilistic algorithm $S$, called a solver, has oracle access to $\Gamma_V^{(g)}, \Gamma_H^{(k)}$.
The solver $S$ can ask at most $v$ verification queries to $\Gamma_V^{(g)}$, $h$ hint queries to $\Gamma_H^{(k)}$ and successfully solves the puzzle $x^{(k)}$
if and only if it asks a verification query $(q, y^{(k)}) := (q, y_1, \dots, y_k)$ such that $\Gamma_V^{(g)}(q, y_1, \dots, y_k) = 1$, and it has not previously asked for a hint query on this $q$.
\end{definition}
%
A dynamic weakly verifiable puzzle is special case of $k$-wise direct product, when
$k$ equals one and $g$ is identity function $g$.
Therefore, we can consider following random experiment in which a $k$-wise direct product of DWVP (or for $k$ equal one a single DWVP) 
defined by $P^{(k)}$ is solved by a circuit $C$ that takes as input puzzles and possibly a random bitstring.
%
\begin{codeblock}
  \textbf{Experiment $A^{P^{(\cdot)}, C^{(\cdot, \cdot)}}(\pi^{(\cdot)})$}
  \medskip
  \hrule
  \medskip
  \textbf{Oracle:} A problem poser $P^{(\cdot)}$ and a solver circuit $C^{(\cdot,\cdot)}$.\\
  \textbf{Input:}  Bitstrings $\pi^{(\cdot)}$ and $r$.
  \medskip\hrule\medskip
  $(x^{(\cdot)}, \Gamma_V^{(\cdot)}, \Gamma_H^{(\cdot)}) := P^{(\cdot)}(\pi^{(\cdot)})$ \\
  Run $C^{\Gamma_V^{(\cdot)},\Gamma_H^{(\cdot)}}(x^{(\cdot)}, r)$ \\
  \IndI Let $Q_{Solved} := \{q: \text{$C^{\Gamma_V^{(\cdot)}, \Gamma_V^{(\cdot)}}$ asked a verification query $(q,y^{(\cdot)})$ and $\Gamma_V^{(\cdot)}(q, y^{(\cdot)}) = 1$} \}$\\
  \IndI Let $Q_{Hint} := \{q: \text{$C^{\Gamma_V^{(\cdot)}, \Gamma_H^{(\cdot)}}$ asked a hint query on q} \}$\\
  \textbf{If} $\exists q \in Q_{solved} : q \notin Q_{Hint}$ \then \\
  \IndI \textbf{return} $1$\\
  \textbf{else} \\
  \IndI \textbf{return} $0$\\
\end{codeblock}
%
\begin{theorem}{\textbf{Security amplification for a dynamic weakly verifiable puzzle.}}\\
  \label{th:sec_amp_for_dwvp}
  For a fixed problem poser $P^{(1)}$ there exists an algorithm $Gen(C, g, \varepsilon, \delta, n, v, h)$ which takes as input a solver circuit $C$ for $k$-wise
  direct product of DWVP, a monotone function $g$, parameters $\varepsilon, \delta,n$, the number of verification $v$, and hint $h$ queries asked by $C$, and outputs a circuit $D$
  such that following holds: \\
  If $C$ is such that \\
  \begin{align*}
    \underset{(\pi_1, \dots, \pi_k) \in \{0,1\}^{kl}}{\Pr}[A^{P^{(g)}, C}(\pi_1, \dots, \pi_k, r) = 1]
    \geq \frac{(h+v)}{8} \left(\underset{\mu \leftarrow \mu_\delta^k}{\Pr}[g(\mu) = 1] + \varepsilon\right)
  \end{align*}
  then $D$ satisfies almost surely
  \begin{align*}
    \underset{\pi \in \{0,1\}^{l}}{\Pr}[A^{P^{(1)},D}(\pi, r) = 1] \geq (\delta + \frac{\varepsilon}{6k})
  \end{align*}
  Additionally, $D$ and $Gen$ require only oracle access to $g$ and $C$. Furthermore, $D$ asks at most $h$ hint queries, $v$ verification queries and
  $Size(D) \leq Size(C) \cdot \Theta(\frac{6k}{\varepsilon})$ and $Time(Gen) = poly(k, \frac{1}{\varepsilon}, n, v, h)$.
\end{theorem}
%%% Local Variables:
%%% mode: latex
%%% TeX-master: "../master"
%%% End:
