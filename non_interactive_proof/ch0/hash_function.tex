\begin{todo}
  \textbf{TODO:} Assumptions on circuit $C$ it does not ask verification queries on $q$ for which it asked a hint query.
\end{todo}

Let $hash:Q\rightarrow\{0,1,\dots, 2(h+v)-1\}$, then a set $P_{hash} \subseteq Q$,
defined with respect to $hash$, is the set of preimages of $0$ for function $hash$.
The idea is that the set $P_{hash}$ contains $q$ on which $C$ is not allowed to ask hint queries.
Therefore, if $C$ makes a verification query on $q \in P_{hash}$ we know that no hint query is ever asked on this $q$.
In the experiment $CanonicalSuccess$ a circuit $C$ succeeds if and only if it ask a verification query on $q \in P_{hash}$
and no hint query is asked on $q \in P_{hash}$.
%
\begin{todo}
  \textbf{TODO:} Define the enumeration of queries.
\end{todo}
%
\begin{codeblock}
  \textbf{Experiment $CanonicalSuccess^{P, C^{(\cdot, \cdot)}, hash}(\pi, \rho)$}
  \medskip

  \hrule

  \medskip
  \textbf{Oracle:} A problem poser $P$. A solver circuit $C^{(\cdot, \cdot)}$.\\
  \IndII A function $hash: Q \leftarrow \{0, \dots, 2(h+v) - 1\}$.\\
  \textbf{Input:}  Bitstrings: $\pi$, $\rho$. \\
  \textbf{Output:} A bit $b \in \{0,1\}$.

  \medskip\hrule\medskip
  $(x, \Gamma_V, \Gamma_H) := P(\pi)$\\
  Run $C^{\Gamma_V, \Gamma_H} (x, \rho)$ \\
  \IndI Let $(q_j,y_j)$ be the first verification query such that $C^{\Gamma_V, \Gamma_H}(q_j, y_j) = 1$, or \\
  \IndI an arbitrary verification query if it fails.\\
  \\
  \textbf{If} $(\forall i < j :  q_i \notin P_{hash} )$ and $q_j \in P_{hash}$ and $\Gamma_V(q_j, y_j) = 1$ \\
  \IndI \textbf{return} 1\\
  \textbf{else}\\
  \IndI \textbf{return} 0
\end{codeblock}
%
For fixed $hash$ and $P^{(1)}$ a canonical success of $C$ for $(\pi^{(k )}, \rho)$ is a situation when \\ $CanonicalSuccess^{P^{(g)}, C^{(\cdot, \cdot)}, hash}(\pi^{(k)}, \rho) = 1$.
We show that if for a fixed $P^{(1)}$ a solver circuit $C$ often succeeds in the experiment $Success$ for $P^{(g)}$,
then it also often successful in the experiment $CanonicalSuccess$ for $P^{(g)}$.
%
\begin{lemma}
\label{lemma:hash_function_probability}
\textbf{Success probability in solving a $k$-wise direct product of DWVP with respect to a function \textit{hash}.} \\
For fixed $P^{(1)}$ let $C$ succeed in the experiment $Success$ for $P^{(g)}$ with probability $\gamma$,
asking at most $h$ hint queries and $v$ verification queries.
There exists a probabilistic algorithm, with oracle access to $C$, that runs in time $O((h+v)^4/\gamma^4)$,
and with high probability outputs a function $hash: Q \rightarrow \{0, \dots, 2(h+v)-1\}$ such that the canonical success probability of
$C$ with respect to $P_{hash}$ is at least $\frac{\gamma}{8(h+v)}$.
\end{lemma}
%
\begin{proof}
Let $\cH$ be a family of pairwise independent hash functions $Q \rightarrow \{0,1, \dots,2(h+v)-1\}$.
For all $i \neq j \in \{1, \dots, (h+v)\}$ and $k,l \in \{0,1,\dots,2(h+v)-1\}$ by pairwise independence property of $\cH$,
we have
\begin{align}
  \label{eq:hash_pr}
 \forall q_i,q_j \in Q : \underset{\textit{hash} \leftarrow \cH}{\Pr}[hash(q_i) = k \mid hash(q_j) = l] = \underset{\textit{hash} \leftarrow \cH}{\Pr}[hash(q_i) = k] = \frac{1}{2(h+v)}.
\end{align}
Let $P^{(g)}, C, (\pi_1, \dots, \pi_k)$ be fixed. We consider the experiment $E$ and define a binary random variable $X$ for the event that $hash(q_j) = 0$, and for
every query $q_i$ asked before $q_j : hash(q_i) \neq 0$.
Conditioned on the event $hash(q_i) = 0$, we get
\begin{align*}
  \underset{\textit{hash} \leftarrow \cH}{\Pr}[X=1] &= \underset{\textit{hash} \leftarrow \cH}{\Pr}[hash(q_j) = 0 \land \forall i < j : hash(q_i) \neq 0] \\
  &=\underset{\textit{hash} \leftarrow \cH}{\Pr}[\forall i < j : hash(q_i) \neq 0 \mid hash(q_j) = 0] \underset{\textit{hash} \leftarrow \cH}{\Pr}[hash(q_j) = 0].
\end{align*}
Now we use (\ref{eq:hash_pr}) twice and obtain
\begin{align*}
\underset{\textit{hash} \leftarrow \cH}{\Pr}[X=1] &=
\frac{1}{2(h+v)}\left(1 -\underset{\textit{hash} \leftarrow \cH}{\Pr}[\exists i < j : hash(q_i) = 0 \mid hash(q_j) = 0] \right) \\
 &= \frac{1}{2(h+v)} \left( 1 -\underset{\textit{hash} \leftarrow \cH}{\Pr}[\exists i < j : hash(q_i) = 0] \right).
\end{align*}
Finally, we use union bound and $j \leq (h+v)$ to get
\begin{align*}
\underset{\textit{hash} \leftarrow \cH}{\Pr}[X=1] \geq
\frac{1}{2(h+v)} \left( 1 - \sum_{i < j} \underset{\textit{hash} \leftarrow \cH}{\Pr}[hash(q_i) = 0] \right) \geq \frac{1}{4(h+v)}.
\end{align*}
Let $G_A$ (correspondingly $G_E$) denote the set of all $(\pi_1, \dots, \pi_k)$ for which $C$ succeeds in the random experiment $A$ ($E$).
For fixed $(\pi_1, \dots, \pi_k)$, if $C$ succeeds canonically, then it also succeeds in the random experiment $A$.
Hence, $G_E \subseteq G_A$ and we get
\begin{align}
  \label{ineq:hash_high_prob}
\underset{\substack{\textit{hash} \leftarrow \cH \\ (\pi_1, \dots, \pi_k)}}{\Pr}[E^{P^{(g)}, C^{(\cdot, \cdot)}, hash}(\pi_1, \dots, \pi_k) = 1] =
\mathbb{E}_{(\pi_1, \dots, \pi_k) \in G_A}\left[\underset{\substack{\textit{hash} \leftarrow \cH}}{\Pr}[X = 1]\right]
\geq \frac{\gamma}{4(h+v)}.
\end{align}
%
\begin{codeblock}
  \textbf{Algorithm: FindHash}
  \medskip
  \hrule
  \medskip

  \textbf{Oracle:} A solver circuit $C^{(\cdot, \cdot)}$ for a $k$-wise direct product of DWVP. \\
  \textbf{Input:} A set $\cH$.\\
  \textbf{Output:} A function $hash \in \cH$.
  \medskip\hrule\medskip
  For $i = 1$ to $32(h+v)^2/\gamma^2$ \\
  \IndI $hash \xleftarrow{\$} \cH$ \\
  \IndI $count := 0$ \\
  \IndI \For $j := 1$ to $32(h+v)^2/\gamma^2$ \\
  \IndII $\pi^{(k)} \xleftarrow{\$} \{0,1\}^{kl} $\\
  \IndII \If $CanonicalSuccess^{P^{(g)}, C^{(\cdot, \cdot)}, hash}(\pi^{(k)}) = 1$ \then \\
  \IndIII $count := count + 1$\\
  \IndI \If $\frac{\gamma^2}{32(h+v)^2} count \geq \frac{\gamma}{6(h+v)}$ \\
  \IndII \return $hash$\\
  \return $\bot$
\end{codeblock}
We show that \textbf{FindHash} chooses $hash$ such
that the canonical success probability of $C$
with respect to $P_{hash}$ is at least $\frac{\gamma}{4(h+v)}$ almost surely.
Let $\cH_{Good}$ denote a family of functions $hash \in \cH$ for which
\begin{align*}
\underset{\pi^{(k)}}{\Pr}[CanonicalSuccess^{P^{(g)}, C^{(\cdot, \cdot)}, hash}(\pi^{(k)}) = 1] \geq \frac{\gamma}{4(h+v)},
\end{align*}
and $\cH_{Bad}$ be the family of functions $hash \in \cH$ such that
\begin{align*}
\underset{\pi^{(k)}}{\Pr}[CanonicalSuccess^{P^{(g)}, C^{(\cdot, \cdot)}, hash}(\pi^{(k)}) = 1] \leq \frac{\gamma}{8(h+v)}.
\end{align*}
Additionally, for a fixed $hash$, we define binary random variables $X_1, \dots, X_N$ such that
\begin{align*}
  X_i =
  \begin{cases}
    1 & \text{if in $i$th iteration variable $count$ is increased}\\
    0 & \text{otherwise .}
  \end{cases}
\end{align*}
We first show that it is unlikely that \textbf{FindHash} returns $hash \in \cH_{Bad}$.
For $hash \in \cH_{Bad}$ we have $\mathbb{E}_{\pi^{(k)}}[X_i] < \frac{\gamma}{8(h+v)}$.
Therefore, for any fixed $hash \in \cH_{Bad}$ using the Chernoff bound we get
\begin{align*}
  \underset{(\pi_1, \dots, \pi_k)}{\Pr} \left[\frac{1}{N} \sum_{i=1}^{N} X_i \geq \frac{\gamma}{6(h+v)} \right] \leq
  \underset{(\pi_1, \dots, \pi_k)}{\Pr}\left[\frac{1}{N} \sum_{i=1}^{N} X_i \geq (1 + \frac{1}{3}) \mathbb{E}[X_i]\right] \leq
  e^{-{\frac{\gamma}{4(h+v)}} N /27}.
\end{align*}
%
The probability that $hash \in \cH_{Good}$, when picked, is not returned amounts
\begin{align*}
  \underset{(\pi_1, \dots, \pi_k)}{\Pr}\left[\frac{1}{N} \sum_{i=1}^{N} X_i \leq \frac{\gamma}{6(h+v)}\right] \leq
  \underset{(\pi_1, \dots, \pi_k)}{\Pr}\left[\frac{1}{N} \sum_{i=1}^{N} X_i \leq (1 - \frac{1}{3})\mathbb{E}[X_i]\right] \leq e^{-{\frac{\gamma}{4(h+v)}} N /27}.
\end{align*}
%
Finally, we show that \textbf{FindHash} picks in one of its iteration $hash \in \cH_{Good}$ almost surely.
% The random variable $X$ is binary distributed. Thus, we have
% \begin{align*}
%   \underset{\substack{\textit{hash} \leftarrow \cH \\ (\pi_1, \dots, \pi_k)}}{\mathbb{E}}[X] \geq \frac{\gamma}{4(h+v)}
% \end{align*}
Let $Y_i$ be a binary random variable such that
\begin{align*}
  Y_i =
  \begin{cases}
    1 & \text{if in $i$th iteration $hash \in \cH_{Good}$ is picked} \\
    0 & \text{otherwise .}
  \end{cases}
\end{align*}
From equation (\ref{ineq:hash_high_prob}) we know that $\underset{hash \leftarrow \cH}{\Pr}[Y_i = 1] = \mathbb{E}[Y_i] \geq \frac{\gamma}{4(h+v)}$, almost surely.
Thus, we get
\begin{align*}
  \underset{hash \leftarrow \cH}{\Pr}\left[\sum_{i=1}^{K} Y_i = 0\right] &\leq \left(1 - \frac{\gamma}{4(h+v)}\right)^{K} \leq e^{-\frac{\gamma}{4(h+v)} K}.
\end{align*}
The bound stated in the Lemma \ref{lemma:hash_function_probability} is achieved for $\delta = \frac{1}{3}$ and $K = N = 32(h+v)^2/\gamma^2$.
\end{proof}
%%% Local Variables:
%%% mode: latex
%%% TeX-master: "../master"
%%% End:
