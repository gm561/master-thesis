From theorem (\ref{th:sec_amp_for_dwvp}) we conclude that if there is no good algorithm for a single DWVP
then it is not possible to find a good algorithm for $k$-wise direct product of DWVP.

The algorithm $Gen$ tries to find $k-1$ puzzles and a position for an input puzzle $x$, such that
when $C$ runs with $k-1$ puzzles and $x$ placed on a right position, then $x$ is solved correctly often.
To find a good position for $x$ and good remaining $k-1$ puzzles we need to run $C$ several times.
It may happen that in one of this runs $C$ ask for a hint query on some index $q$, and
in one of the later runs we find a set of puzzles and a position for $x$ such that $x$ is solved often.
However, we need an additional requirement that this happens often for $q$ on which a hint query was not asked before.
To satisfy this new requirement we split the set $Q$.

Let $hash:Q\rightarrow\{0,1,\dots, 2(h+v)-1\}$, then a set $P_{hash} \subseteq Q$,
defined with respect to $hash$, is a preimage of $0$ for function $hash$.
The set $P_{hash}$ contains $q$ on which $C$ is not allowed to ask hint queries.
Therefore, if $C$ makes a verification query on $q \in P_{hash}$ we know that no hint query is ever asked on this $q$.
In the experiment $E$ a circuit $C$ succeeds if and only if it ask a verification query on $q \in P_{hash}$.
%
\begin{codeblock}
  \textbf{Experiment $E^{P^{(g)}, C^{(.)(.)}, hash}(\pi_1, \dots, \pi_k, r)$} \\
  Solving $k$-wise direct product of DWVP with respect to the set $P_{hash}$
  \medskip

  \hrule

  \medskip
  \textbf{Oracle:} Problem poser for k-wise direct product $P^{(g)}$ \\
  \IndI A solver circuit for $k$-wise direct product $C^{(\cdot, \cdot)}$ \\
  \IndI A function $hash: Q \leftarrow \{0, \dots, 2(h+v) - 1\}$\\
  \textbf{Input:} Random bitstrings: $(\pi_1, \dots, \pi_k) \in \{0,1\}^{kl}$ and $r$.\\

  \medskip\hrule\medskip

  $\pi^{(k)} := \left(\pi_1, \dots, \pi_k \right)$\\
  $(x^{k}, \Gamma_V^{(g)}, \Gamma_H^{(k)}) := P^{(g)}(\pi^{k})$\\
  Run $C^{\Gamma_V^{(g)}, \Gamma_H^{(k)}} (x^{(k)}, r)$ \\
  \IndI Let $(q_j,y_j^{(k)})$ be the first successful verification query if $C^{\Gamma_V^{(g)}, \Gamma_H^{(k)}}$ succeeds or \\
  \IndI an arbitrary verification query when it fails.\\
  \textbf{If} $(\forall i < j :  q_i \notin P_{hash} )$ and $q_j \in P_{hash}$ and $\Gamma_V^{(g)}(q_j, y_j^{(k)}) = 1$ \\
  \IndI \textbf{return} 1\\
  \textbf{else}\\
  \IndI \textbf{return} 0\\
\end{codeblock}
%
For fixed $hash$ and $P^{(1)}$ a canonical success of $C$ is a situation when $E^{P^{(g)}, C^{(.)(.)}, hash}(\pi_1, \dots, \pi_k, r) = 1$.
We show that if $C$ often solves successfully the $k$-wise direct product of DWVP, then it also often succeeds canonically.
%
\begin{lemma}
\label{lemma:hash_function_probability}
\textbf{Success probability in solving a $k$-wise direct product of DWVP with respect to a function \textit{hash}.} \\
For a fixed $P^{(g)}$ let $C$ succeed in solving a $k$-wise direct product of DWVP produced by $P^{(g)}$
with probability $\gamma$, asking at most $h$ hint and $v$ verification queries.
There exists a probabilistic algorithm, with oracle access to $C$, that runs in time $O((h+v)^4/\gamma^4)$
and with high probability outputs a function $hash: Q \rightarrow \{0, \dots, 2(h+v)-1\}$ such that canonical success probability of
$C$ with respect to $P_{hash}$ is at least $\frac{\gamma}{8(h+v)}$.
\end{lemma}
%
\begin{proof}
Let $\cH$ be a family of pairwise independent hash functions $Q \rightarrow \{0,1, \dots,2(h+v)-1\}$.
For all $i \neq j \in \{1, \dots, (h+v)\}$ and $k,l \in \{0,1,\dots,2(h+v)-1\}$ by pairwise independence property of $\cH$
we have the following
\begin{align}
  \label{eq:hash_pr}
 \forall q_i,q_j \in Q : \underset{\textit{hash} \leftarrow \cH}{\Pr}[hash(q_i) = k \mid hash(q_j) = l] = \underset{\textit{hash} \leftarrow \cH}{\Pr}[hash(q_i) = k] = \frac{1}{2(h+v)}.
\end{align}
For a fixed $P^{(g)}, C$ and $(\pi_1, \dots, \pi_k)$ in the random experiment $E$ we define a binary random variable $X$ for the event that $hash(q_j) = 0$, and for
every query $q_i$ asked before $q_j$ $hash(q_i) \neq 0$.
Conditioned on the event $hash(q_i) = 0$, we get
\begin{align*}
  \underset{\textit{hash} \leftarrow \cH}{\Pr}[X=1] &= \underset{\textit{hash} \leftarrow \cH}{\Pr}[hash(q_j) = 0 \land \forall i < j : hash(q_i) \neq 0] \\
  &=\underset{\textit{hash} \leftarrow \cH}{\Pr}[\forall i < j : hash(q_i) \neq 0 \mid hash(q_j) = 0] \underset{\textit{hash} \leftarrow \cH}{\Pr}[hash(q_j) = 0].
\end{align*}
Now we use (\ref{eq:hash_pr}) twice and obtain
\begin{align*}
\underset{\textit{hash} \leftarrow \cH}{\Pr}[X=1] &=
\frac{1}{2(h+v)}\left(1 -\underset{\textit{hash} \leftarrow \cH}{\Pr}[\exists i < j : hash(q_i) = 0 \mid hash(q_j) = 0] \right) \\
 &= \frac{1}{2(h+v)} \left( 1 -\underset{\textit{hash} \leftarrow \cH}{\Pr}[\exists i < j : hash(q_i) = 0] \right).
\end{align*}
Finally, we use union bound and $j \leq (h+v)$ to get
\begin{align*}
\underset{\textit{hash} \leftarrow \cH}{\Pr}[X=1] \geq
\frac{1}{2(h+v)} \left( 1 - \sum_{i < j} \underset{\textit{hash} \leftarrow \cH}{\Pr}[hash(q_i) = 0] \right) \geq \frac{1}{4(h+v)}
\end{align*}
Let $G_A$ ($G_E$) denote the set of all $(\pi_1, \dots, \pi_k)$ for which $C$ succeeds in the random experiment $A$ ($E$).
If for fixed $(\pi_1, \dots, \pi_k)$ $C$ succeeds in the random experiment $E$, then it also succeeds in the random experiment $A$.
Hence, $G_E \subseteq G_A$ and we get
\begin{align}
  \label{ineq:hash_high_prob}
\underset{\substack{\textit{hash} \leftarrow \cH \\ (\pi_1, \dots, \pi_k)}}{\Pr}[E^{P^{(g)}, C^{(\cdot, \cdot)}, hash}(\pi_1, \dots, \pi_k) = 1] =
\mathbb{E}_{(\pi_1, \dots, \pi_k) \in G_A}\left[\underset{\substack{\textit{hash} \leftarrow \cH}}{\Pr}[X = 1]\right]
\geq \frac{\gamma}{4(h+v)}.
\end{align}
%
\begin{codeblock}
  \textbf{Algorithm: FindHash}
  \medskip
  \hrule
  \medskip

  \textbf{Oracle:} A solver circuit for a $k$-wise direct product of DWVP $C^{(\cdot, \cdot)}$. \\
  \textbf{Input:} A set $\cH$.
  \medskip\hrule\medskip
  For $i = 1$ to $32(h+v)^2/\gamma^2$ \\
  \IndI $hash \xleftarrow{\$} \cH$ \\
  \IndI $count := 0$ \\
  \IndI \For $j := 1$ to $32(h+v)^2/\gamma^2$ \\
  \IndII $(\pi_1, \dots, \pi_k) \xleftarrow{\$} \{0,1\}^{kl} $\\
  \IndII \If $E^{P^{(g)}, C^{(\cdot, \cdot)}, hash}(\pi_1, \dots, \pi_k) = 1$ \then \\
  \IndIII $count := count + 1$\\
  \IndI \If $\frac{\gamma^2}{32(h+v)^2} count \geq \frac{\gamma}{6(h+v)}$ \\
  \IndII \return $hash$\\
  \return $\bot$
\end{codeblock}
We show that \textbf{FindHash} chooses a function $hash$ such
that the canonical success probability of $C$
with respect to set $P_{hash}$ is at least $\frac{\gamma}{4(h+v)}$ almost surely.
Let $\cH_{Good}$ denote $hash \in \cH$ for which
\begin{align*}
\underset{(\pi_1, \dots, \pi_k)}{\Pr}[E^{P^{(g)}, C^{(\cdot, \cdot)}, hash}(\pi_1, \dots, \pi_k) = 1] \geq \frac{\gamma}{4(h+v)},
\end{align*}
and $\cH_{Bad}$ be the family $hash \in \cH$ such that
\begin{align*}
\underset{(\pi_1, \dots, \pi_k)}{\Pr}[E^{P^{(g)}, C^{(\cdot, \cdot)}, hash}(\pi_1, \dots, \pi_k) = 1] \leq \frac{\gamma}{8(h+v)}.
\end{align*}
Additionally, for a fixed $hash$, we define the binary random variables $X_1, \dots, X_i, \dots, X_N$ such that
\begin{align*}
  X_i =
  \begin{cases}
    1 & \text{if in $i$th iteration variable $count$ is increased}\\
    0 & \text{otherwise .}
  \end{cases}
\end{align*}
We first show that it is unlikely that \textbf{FindHash} returns $hash \in \cH_{Bad}$.
For $hash \in \cH_{Bad}$ we have $\mathbb{E}_{(\pi_1, \dots, \pi_k)}[X_i] < \frac{\gamma}{8(h+v)}$.
Therefore, for any fixed $hash \in \cH_{Bad}$ using the Chernoff bound we get
\begin{align*}
  \underset{(\pi_1, \dots, \pi_k)}{\Pr} \left[\frac{1}{N} \sum_{i=1}^{N} X_i \geq \frac{\gamma}{6(h+v)} \right] \leq
  \underset{(\pi_1, \dots, \pi_k)}{\Pr}\left[\frac{1}{N} \sum_{i=1}^{N} X_i \geq (1 + \frac{1}{3}) \mathbb{E}[X_i]\right] \leq
  e^{-{\frac{\gamma}{4(h+v)}} N /27}.
\end{align*}
%
The probability that $hash \in \cH_{Good}$, when picked, is not returned is
\begin{align*}
  \underset{(\pi_1, \dots, \pi_k)}{\Pr}\left[\frac{1}{N} \sum_{i=1}^{N} X_i \leq \frac{\gamma}{6(h+v)}\right] \leq
  \underset{(\pi_1, \dots, \pi_k)}{\Pr}\left[\frac{1}{N} \sum_{i=1}^{N} X_i \leq (1 - \frac{1}{3})\mathbb{E}[X_i]\right] \leq e^{-{\frac{\gamma}{4(h+v)}} N /27}.
\end{align*}
%
Finally, we show that \textbf{FindHash} picks in one of its iteration a hash function that is in $\cH_{Good}$ almost surely.
% The random variable $X$ is binary distributed. Thus, we have
% \begin{align*}
%   \underset{\substack{\textit{hash} \leftarrow \cH \\ (\pi_1, \dots, \pi_k)}}{\mathbb{E}}[X] \geq \frac{\gamma}{4(h+v)}
% \end{align*}
Let $Y_i$ be a binary random variable such that
\begin{align*}
  Y_i =
  \begin{cases}
    1 & \text{if in $i$th iteration $hash \in \cH_{Good}$ is picked} \\
    0 & \text{otherwise .}
  \end{cases}
\end{align*}
From equation (\ref{ineq:hash_high_prob}) we know that $\underset{hash \leftarrow \cH}{\Pr}[Y_i = 1] = \mathbb{E}[Y_i] \geq \frac{\gamma}{4(h+v)}$, almost surely.
Thus, we get
\begin{align*}
  \underset{hash \leftarrow \cH}{\Pr}\left[\sum_{i=1}^{K} Y_i = 0\right] &\leq \left(1 - \frac{\gamma}{4(h+v)}\right)^{K} \leq e^{-\frac{\gamma}{4(h+v)} K}.
\end{align*}
%and we use Chernoff bound to obtain
% \begin{align*}
%   \underset{hash \leftarrow \cH}{\Pr}\left[\frac{1}{K} \sum_{i=1}^{K} Y_i = 0\right] &\leq
%   \underset{hash \leftarrow \cH}{\Pr}\left[\frac{1}{K} \sum_{i=1}^{K} Y_i \leq (1-\delta) \frac{\gamma}{4(h+v)}\right] \\
%   &\leq \underset{hash \leftarrow \cH}{\Pr}\left[\frac{1}{K} \sum_{i=1}^{K} Y_i \leq (1-\delta) \mathbb{E}[Y_i] \right] \leq e^{-\delta^2K \mathbb{E}[Y_i]/2 }
% \end{align*}
The bound stated in the Lemma \ref{lemma:hash_function_probability} is achieved for $\delta = \frac{1}{2}$ and $K = N = 32(h+v)^2/\gamma^2$.
\end{proof}
%
\begin{codeblock}
  \textbf{Random experiment $F^{P^{(1)}, D, hash}(\pi)$} \\
  Solving a single DWVP with respect to the set $P_{hash}$
  \medskip

  \hrule

  \medskip

  \textbf{Oracle:}
  A problem poser $P^{(1)}$ for DWVP. \\
  \IndI A solver circuit $D$ for a single DWVP. \\
  \IndI A function $hash: Q \rightarrow \{0,1,\dots, 2(h+v)-1\}$. \\
  \textbf{Input:} A random bitstrings $\pi \in \{0,1\}^l$, $r \in \{0,1\}^{*}$.
  \medskip\hrule\medskip

  $(x, \Gamma_v, \Gamma_H) := P^{(1)}(\pi)$ \\
  Run $D^{\Gamma_V, \Gamma_H}(x,r)$ \\
  \IndI Let $(\widetilde{q_j},\widetilde{r_j})$ be the first successful verification query if $D^{\Gamma_V, \Gamma_H}(x)$ succeeds or \\
  \IndI an arbitrary verification query when it fails. \\
  \If $(\forall i < j :  q_i \notin  P_{hash} ) \land q_j \in P_{hash} \land \Gamma_V(q_j) = 1$ \then \\
  \IndI \return 1 \\
  \textbf{else}\\
  \IndI \return 0
\end{codeblock}
%%% Local Variables:
%%% mode: latex
%%% TeX-master: "../master"
%%% End:
