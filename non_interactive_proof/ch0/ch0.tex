\begin{definition} {\textbf{Dynamic weakly verifiable puzzle (non interactive version)}}\\
  A dynamic weakly verifiable puzzle (DWVP) is defined by a probabilistic algorithm $P(\pi)$,
  called a problem poser, that takes as input chosen uniformly at random bitstring $\pi \in \{0,1\}^l$,
  and produces circuits $\Gamma_{V}$, $\Gamma_{H}$ and a puzzle $x \in \{0,1\}^{*}$.
  The circuit $\Gamma_{V}$ takes as its input $q \in Q$ and an answer $y$.
  If $\Gamma_V(q,y) = 1$ then $y$ is a correct solution of puzzle $x$ for $q$.
  The circuit $\Gamma_H$ on input $q$ provides a hint such that $\Gamma_V(q,\Gamma_H(q)) = 1$.
  The algorithm $S$, called a solver, has oracle access to $\Gamma_V$ and $\Gamma_H$.
  The calls of $S$ to $\Gamma_V$ are called verification queries and the calls to $\Gamma_H$ are hint queries.
  The solver $S$ can ask at most $h$ hint queries, $v$ verification queries, and successfully solves a DWVP if and only if
  it makes a verification query $(q,y)$ such that $\Gamma_V(q,y) = 1$, when it has not previously asked for a hint query on this $q$.
\end{definition}
%
%
\begin{definition}{\textbf{$k$-wise direct product of dynamic weakly verifiable puzzles}}\\
Let $g: \{0,1\}^{k} \rightarrow \{0,1\}$ be a monotone function, and $P^{(1)}$ a probabilistic algorithm used to generate an instance of DWVP.
A $k$-wise direct product of dynamic weakly verifiable puzzles is defined by a probabilistic algorithm $P^{(g)}\left(\pi_1, \dots, \pi_k \right)$,
where $(\pi_1, \dots, \pi_k) \in \{0,1\}^{k \cdot l}$ are chosen uniformly at random.
$P^{(g)}\left(\pi_1, \dots, \pi_k \right)$ sequentially generates $k$ independent instances of dynamic weakly verifiable puzzles,
where in the $i$-th round $P^{(g)}$ runs $P^{(1)}(\pi_i)$ and obtains $(x_i, \Gamma_V^{(i)}, \Gamma_H^{(i)} )$.
Finally, $P^{(g)}$ outputs a verification circuit
\begin{align*}
  \Gamma_V^{(g)} (q, y_1, \dots, y_k) := g(\Gamma_V^{(1)}(q, y_1), \dots, \Gamma_V^{(k)}(q, y_k)),
\end{align*}
a hint circuit
\begin{align*}
  \Gamma_H^{(k)} (q) := (\Gamma_H^{(1)}(q), \dots, \Gamma_H^{(k)}(q)),
\end{align*}
and a puzzle $x^{(k)} := (x_1, \dots, x_k)$.
\\
The probabilistic algorithm $S$, called a solver, has oracle access to $\Gamma_V^{(g)}, \Gamma_H^{(k)}$.
The solver $S$ can ask at most $v$ verification queries to $\Gamma_V^{(g)}$, $h$ hint queries to $\Gamma_H^{(k)}$ and successfully solves the puzzle $x^{(k)}$
if and only if it asks a verification query $(q, y_1, \dots, y_k)$ such that $\Gamma_V^{(g)}(q, y_1, \dots, y_k) = 1$, and it has not previously asked for a hint query on this $q$.
\end{definition}
%
%
\begin{codeblock}
  \textbf{Experiment $A^{P^{(\cdot)}, C^{(\cdot, \cdot)}}(\pi^{(\cdot)})$}
  \medskip

  \hrule

  \medskip

  \textbf{Oracle:} A problem poser $P^{(\cdot)}$ and a solver circuit $D^{(\cdot,\cdot)}$.\\
  \textbf{Input:}  A bitstring $\pi^{(\cdot)}$.

  \medskip\hrule\medskip

  $(x^{(\cdot)}, \Gamma_V^{(\cdot)}, \Gamma_H^{(\cdot)}) := P^{(\cdot)}(\pi^{(\cdot)})$ \\
  Run $D^{(\Gamma_V^{(\cdot)},\Gamma_H^{(\cdot)})}(x^{(\cdot)})$ \\
  \IndI $Q_{Solved} := \{q: \text{$D^{\Gamma_V^{(\cdot)}, \Gamma_V^{(\cdot)}}(x^{(\cdot)})$ asked a verification query $(q,y^{(\cdot)})$ and $\Gamma_V^{(\cdot)}(q, y^{(\cdot)}) = 1$} \}$\\
  \IndI $Q_{Hint} := \{q: \text{$D^{\Gamma_V^{(\cdot)}, \Gamma_H^{(\cdot)}}(x^{(\cdot)})$ asked a hint query on q} \}$\\
  \textbf{If} $\exists q \in Q_{solved} : q \notin Q_{Hint}$\\
  \IndI \textbf{return} $1$\\
  \textbf{else}\\
  \IndI \textbf{return} $0$\\
\end{codeblock}
%
%
% TODO oracle access to P^{g} of Gen
\begin{theorem}{\textbf{Security amplification of a dynamic weakly verifiable puzzle.}}\\
  For a fixed problem poser $P^{(1)}$ there exists an algorithm $Gen(C, g, \varepsilon, \delta, n, v, h)$ which takes as input a solver circuit $C$ for $k$-wise
  direct product of DWVP, a monotone function $g$, parameters $\varepsilon, \delta,n$, the number of verification $v$, and hint $h$ queries asked by $C$, and outputs a circuit $D$
  such that following holds: \\
  If $C$ is such that \\
  \begin{align*}
    \underset{(\pi_1, \dots, \pi_k) \in \{0,1\}^{kl}}{\Pr}[A^{P^{(g)}, C}(\pi_1, \dots, \pi_k) = 1] \geq \underset{\mu \leftarrow \mu_\delta^k}{\Pr}[g(\mu) = 1] + \varepsilon
  \end{align*}
  then $D$ satisfies almost surely
  \begin{align*}
    \underset{\pi \in \{0,1\}^{l}}{\Pr}[A^{P^{(1)},D}(\pi) = 1] \geq (\delta + \frac{\varepsilon}{6k})
  \end{align*}
  Additionally, $D$ and $Gen$ require only oracle access to $g$ and $C$. Furthermore, $D$ asks at most $h$ hint queries, $v$ verification queries and
  $Size(D) \leq Size(C)\frac{6k}{\varepsilon}$ and $Time(Gen) = poly(k, \frac{1}{\varepsilon}, n, v, h)$.
\end{theorem}
% Then canonical success probability of $C$ in a random experiment $A^{P^{(g)}, C, Hash}(\pi_1, \dots, \pi_k)$ is \\
% $\underset{(\pi_1, \dots, \pi_k)}{\Pr}[A^{P^{(g)}, C, Hash}(\pi_1, \dots, \pi_k)]$.
%
% TODO: circuits or algorithms
% TODO: size of circuits?
% TODO: need def. of canonical success and P_hash
% TODO: number of calls to oracle circuit C
%
Let $hash:Q\rightarrow\{0,1,\dots, 2(h+v)-1\}$ and $P_{hash}$, defined with respect to $hash$, is a preimage of $0$ for function $hash$.
% Proof of good probability success with respect to the hash function

% Let $hash:Q\rightarrow\{0,1,\dots, 2(h+v)-1\}$, then a set $P_{hash} \subseteq Q$,
% defined with respect to $hash$, is the set of preimages of $0$ for $hash$.
Let $hash:Q\rightarrow\{0,1,\dots, 2(h+v)-1\}$,
the idea is to partition $Q$ such that the set of preimages of $0$ for $hash$ contains $q \in Q$ on which $C$ is not allowed to ask hint queries,
and the first successful verification query $(q,y)$ of $C$ is such that $hash(q) = 0$.
Therefore, if $C$ makes a verification query $(q,y)$ such that $hash(q) = 0$, then we know that no hint query is ever asked on this $q$.

In the experiment $CanonicalSuccess$ we denote the $i$-th query of $C$ by $q_i$ if it is a hint query, and by $(q_i, y_i)$ if it is a verification query.
A solver circuit $C$ succeeds in the experiment $CanonicalSuccess$ if it asks a successful verification query $(q_j, y_j)$ such that $hash(q_j) = 0$,
and no hint query $q_i$ is asked before $(q_j, y_j)$ such that $hash(q_i) = 0$.
%
\begin{codeblock}
  \textbf{Experiment $CanonicalSuccess^{P, C, hash}(\pi, \rho)$}
  \medskip

  \hrule

  \medskip
  \textbf{Oracle:} A problem poser $P$, a solver circuit $C = (C_1, C_2)$.\\
  \IndII A function $hash: Q \rightarrow \{0, \dots, 2(h+v) - 1\}$.\\
  \textbf{Input:}  Bitstrings $\pi \in \{0,1\}^n$ and $\rho \in \{0,1\}^*$. \\
  \textbf{Output:} A bit $b \in \{0,1\}$.

  \medskip\hrule\medskip
  Run $\langle P(\pi), C_1(\rho) \rangle$ \\
  \IndI $(\Gamma_V, \Gamma_H) := \langle P(\pi), C_1(\rho) \rangle_{P}$ \\
  \IndI $x := \langle P(\pi), C_1(\rho) \rangle_{\text{trans}}$ \\ \\
  Run $C_2^{\Gamma_V, \Gamma_H} (x, \rho)$ \\
  \IndI Let $(q_j,y_j)$ be the first verification query of $C_2$ such that $\Gamma_v(q_j, y_j) = 1$.\\
  \IndI If $C_2$ does not succeed let $(q_j, y_j)$ be an arbitrary verification query.\\
  \\
  \textbf{If} $(\forall i < j :  hash(q_i) = 0)$ \And $(hash(q_j) = 1)$ \And $(\Gamma_V(q_j, y_j) = 1)$ \then \\
  \IndI \textbf{return} 1\\
  \textbf{else}\\
  \IndI \textbf{return} 0
\end{codeblock}
%
We define the \textit{canonical success probability} of a solver $C$ for $P$ with respect to a function $hash$ as
\begin{align}
 \underset{\pi, \rho}{\Pr}[CanonicalSuccess^{P, C, hash}(\pi, \rho) = 1].
\end{align}
%
For fixed $hash$ and a problem poser $P$ a \textit{canonical success} of $C$ for $\pi, \rho$ is a situation where $CanonicalSuccess^{P, C, hash}(\pi, \rho) = 1$.

We show that if a solver circuit $C$ for $P^{(g)}$ often succeeds in the experiment $Success$, then it is
also often successful in the experiment $CanonicalSuccess$.

\begin{lemma}\textbf{(\boldmath{Success probability in solving a $k$-wise direct product of $P^{(1)}$ with respect to a function $hash$.)}}
\label{lemma:hash_function_probability}
For fixed $P^{(g)}$ let $C$ be a solver for $P^{(g)}$ with the success probability at least $\gamma$,
asking at most $h$ hint queries and $v$ verification queries.
There exists a probabilistic algorithm \textbf{FindHash} that takes as input:
parameters $\gamma$, $n$, $k$, the number of verification queries $v$ and hint queries $h$, and has
oracle access to $C$ and $P^{(g)}$. Furthermore, \textbf{FindHash} runs in time $O((h+v)^4/\gamma^4)$,
and with high probability outputs a function $hash \in \cH$
such that the canonical success probability of $C$ with respect to $hash$ is at least $\frac{\gamma}{8(h+v)}$.
\end{lemma}
%
\begin{proof}
We fix $P^{(g)}$ and a solver $C$ for $P^{(g)}$ in the whole proof of Lemma \ref{lemma:hash_function_probability}.
Let $\cH$ be a family of pairwise independent hash functions $Q \rightarrow \{0,1, \dots,2(h+v)-1\}$.
For all $m,n \in \{1, \dots, (h+v)\}$ and $k,l \in \{0,1,\dots,2(h+v)-1\}$ by the pairwise independence property of $\cH$, we have
\begin{align}
  \label{eq:hash_pr}
 \forall q_m,q_n \in Q, q_m \neq q_n : \underset{\textit{hash} \leftarrow \cH}{\Pr}[hash(q_m) = k \mid hash(q_n) = l] = \underset{\textit{hash} \leftarrow \cH}{\Pr}[hash(q_m) = k] = \frac{1}{2(h+v)}.
\end{align}
%
Let $\cP_{Success}$ be a set containing all $(\pi^{(k)},\rho)$ for which $Success^{P^{(g)}, C}(\pi^{(k)}, \rho) = 1$.
We choose uniformly at random $hash \leftarrow \cH$, and consider the experiment $CanonicalSuccess^{P^{(g)}, C, hash}(\pi^{(k)}, \rho)$.
We are interested in the probability of the event that for a fixed $(\pi, rho) \in \cP_{Success}$ the solver $C$ succeeds canonically.
Let $(q_j, y_j)$ denote the first query such that $\Gamma_V(q_j, y_j) = 1$.
We have
\begin{align*}
  &\underset{\textit{hash} \leftarrow \cH}{\Pr}[hash(q_j) = 0 \land (\forall i < j : hash(q_i) \neq 0)]\\
  &\IndII = \underset{\textit{hash} \leftarrow \cH}{\Pr}[\forall i < j : hash(q_i) \neq 0 \mid hash(q_j) = 0] \underset{\textit{hash} \leftarrow \cH}{\Pr}[hash(q_j) = 0] \\
  &\IndII \stackrel{(\ref{eq:hash_pr})}{=} \frac{1}{2(h+v)}\left(1 -\underset{\textit{hash} \leftarrow \cH}{\Pr}[\exists i < j : hash(q_i) = 0 \mid hash(q_j) = 0] \right) \\
  &\IndII \stackrel{(\ref{eq:hash_pr})}{=} \frac{1}{2(h+v)} \left( 1 -\underset{\textit{hash} \leftarrow \cH}{\Pr}[\exists i < j : hash(q_i) = 0] \right) \\
  &\IndII \stackrel{(\text{u.b})}{\geq} \frac{1}{2(h+v)} \left( 1 - \sum_{i < j} \underset{\textit{hash} \leftarrow \cH}{\Pr}[hash(q_i) = 0] \right) \\
  &\IndII \stackrel{(\ref{eq:hash_pr})}{\geq} \frac{1}{4(h+v)}.
\end{align*}

We denote the set of those $(\pi^{(k)},\rho)$ for which $CanonicalSuccess^{P^{(g)}, C, hash}(\pi^{(k)}, \rho) = 1$ by $\cP_{Canonical}$.
For $(\pi^{(k)}, \rho)$ for which $C$ succeeds canonically, we have $Success^{P^{(g)}, C}(\pi^{(k)}, \rho) = 1$.
Hence, $\cP_{Canonical} \subseteq \cP_{Success}$, and we conclude
\begin{align}
  \label{ineq:hash_high_prob}
\underset{\substack{\textit{hash} \leftarrow \cH \\ \pi^{(k)}, \rho}}{\Pr}\left[CanonicalSuccess^{P^{(g)}, C, hash}(\pi^{(k)}, \rho) = 1\right] &=
\underset{{(\pi^{(k)},\rho) \in \cP_{Success}}}{\mathbb{E}}\left[\underset{\substack{\textit{hash} \leftarrow \cH}}{\Pr}[X = 1]\right] \notag\\
&\geq \frac{\gamma}{4(h+v)}.
\end{align}
%
\begin{codeblock}
  \textbf{Algorithm: FindHash}$(\gamma, n, k, h, v)$
  \medskip
  \hrule
  \medskip
  \textbf{Oracle:} A problem poser $P^{(g)}$, a solver circuit $C$ for $P^{(g)}$.\\
  \textbf{Input:} Parameters $\gamma, n, k, h,v $\\
  \textbf{Output:} A function $hash:Q \rightarrow \{0,1, \dots, 2(h+v)-1 \}$.
  \medskip\hrule\medskip
  Let $\cH$ be a family of pairwise independent hash functions $Q \rightarrow \{0,1,\dots, 2(h+v)-1\}$\\
  \For $i = 1$ \To $32(h+v)^2/\gamma^2$ \Do \\
  \IndI $hash \xleftarrow{\$} \cH$ \\
  \IndI $count := 0$ \\
  \IndI \For $j := 1$ to $32(h+v)^2/\gamma^2$ \Do \\
  \IndII $\pi^{(k)} \xleftarrow{\$} \{0,1\}^{kn} $\\
  \IndII $\rho \xleftarrow{\$} \{0,1\}^*$ \\
  \IndII \If $CanonicalSuccess^{P^{(g)}, C, hash}(\pi^{(k)}, \rho) = 1$ \then \\
  \IndIII $count := count + 1$\\
  \IndI \If $\frac{\gamma^2}{32(h+v)^2} count \geq \frac{\gamma}{6(h+v)}$ \then \\
  \IndII \return $hash$\\
  \return $\bot$
\end{codeblock}
We show that \textbf{FindHash} chooses $hash$ such that the canonical success probability of $C$
with respect to $hash$ is at least $\frac{\gamma}{4(h+v)}$ almost surely.
Let $\cH_{Good}$ denote a family of functions $hash \in \cH$ for which
\begin{align}
  \label{eq:hash_good}
\underset{\pi^{(k)}, \rho}{\Pr}\left[CanonicalSuccess^{P^{(g)}, C, hash}(\pi^{(k)}, \rho) = 1\right] \geq \frac{\gamma}{8(h+v)},
\end{align}
and $\cH_{Bad}$ be the family of functions $hash \in \cH$ such that
\begin{align}
  \label{eq:hash_bad}
\underset{\pi^{(k)}, \rho}{\Pr}\left[CanonicalSuccess^{P^{(g)}, C^{(\cdot, \cdot)}, hash}(\pi^{(k)}, \rho) = 1\right] \leq \frac{\gamma}{16(h+v)}.
\end{align}
%
Let $N$ denote the number of iterations of the inner loop of \textbf{FindHash}.
For a fixed $hash$, we define binary random variables $X_1, \dots, X_{N}$ such that
\begin{align*}
  X_i =
  \begin{cases}
    1 & \text{if in the $i$-th iteration of the inner loop $count$ is increased}\\
    0 & \text{otherwise.}
  \end{cases}
\end{align*}
We show now that \textbf{FindHash} is unlikely to return $hash \in \cH_{Bad}$.
For $hash \in \cH_{Bad}$ by (\ref{eq:hash_bad}) we have $\mathbb{E}_{\pi^{(k)},\rho}[X_i] \leq \frac{\gamma}{16(h+v)}$.
Therefore, for any fixed $hash \in \cH_{Bad}$ using the Chernoff bound we get
\footnote{For $X = \sum_{i=1}^N X_i$ and $0 < \delta \leq 1$ we use the Chernoff bounds in the form
$\Pr[X \geq (1+\delta) \mathbb{E}[X]] \leq e^{- \mathbb{E}[X] \delta^2/3}$ and
$\Pr[X \leq (1-\delta) \mathbb{E}[X]] \leq e^{- \mathbb{E}[X] \delta^2/2}$.}
\begin{align*}
  \underset{\pi^{(k)},\rho}{\Pr} \left[\frac{1}{N} \sum_{i=1}^{N} X_i \geq \frac{\gamma}{12(h+v)} \right] \leq
  \underset{\pi^{(k)}, \rho}{\Pr}\left[\frac{1}{N} \sum_{i=1}^{N} X_i \geq (1 + \frac{1}{4}) \mathbb{E}[X_i]\right] \leq
  e^{-{\frac{\gamma}{16(h+v)}} N /48} \leq e^{-\frac{1}{24}\frac{(h+v)}{\gamma}}.
\end{align*}
%
The probability that $hash \in \cH_{Good}$, when picked, is not returned amounts
\begin{align*}
  \underset{\pi^{(k)}, \rho}{\Pr}\left[\frac{1}{N} \sum_{i=1}^{N} X_i \leq \frac{\gamma}{12(h+v)}\right] \leq
  \underset{\pi^{(k)}, \rho}{\Pr}\left[\frac{1}{N} \sum_{i=1}^{N} X_i \leq (1 - \frac{1}{3})\mathbb{E}[X_i]\right]
  \leq e^{-{\frac{\gamma}{8(h+v)}} N / 18} \leq e^{-\frac{2}{9} \frac{(h+v)}{\gamma}},
\end{align*}
where we once more used the Chernoff bound.
Now we show that the probability of picking a $hash \in \cH_{Good}$ is at least $\frac{\gamma}{8(h+v)}$.
We proof this statement by contradiction. We assume otherwise, namely that
$\underset{hash \leftarrow \cH}{\Pr}[hash \in \cH_{Good}] < \frac{\gamma}{8(g+v)}$.
We have
\begin{align*}
  &\underset{\substack{hash \leftarrow \cH \\ \pi, \rho}}{\Pr}[CanonicalSuccess^{P,C,hash}(\pi, \rho) = 1] \\
  &\IndI = \underset{\substack{hash \leftarrow \cH \\ \pi, \rho}}{\Pr}[CanonicalSuccess^{P,C,hash}(\pi, \rho) = 1 \mid hash \in \cH_{Good}]
  \underset{hash \leftarrow \cH}{\Pr}[hash \in \cH_{Good}] \\
  & \IndII + \underset{\substack{hash \leftarrow \cH \\ \pi, \rho}}{\Pr}[CanonicalSuccess^{P,C,hash}(\pi, \rho) = 1 \mid hash \notin \cH_{Good}]
  \underset{hash \leftarrow \cH}{\Pr}[hash \notin \cH_{Good}] \\
  & \IndI \leq \underset{hash \leftarrow \cH}{\Pr}[hash \in \cH_{Good}] +
  \underset{\substack{hash \leftarrow \cH \\ \pi, \rho}}{\Pr}[CanonicalSuccess^{P,C,hash}(\pi, \rho) = 1 \mid hash \notin \cH_{Good}] \\
  & \IndI < \frac{\gamma}{8(h+v)} + \frac{\gamma}{8(h+v)} = \frac{\gamma}{4(h+v)}.
\end{align*}
But this contradicts (\ref{ineq:hash_high_prob}).
Finally, we show that \textbf{FindHash} picks in one of its iteration $hash \in \cH_{Good}$ almost surely.
Let $K$ be the number of iterations of the outer loop of \textbf{FindHash}.
Let $Y_i$ be a random variable for the event
that in the $i$-th iteration of the outer loop $hash \in \cH_{Good}$ is picked.
Using $\underset{hash \leftarrow \cH}{\Pr}[hash \in \cH_{Good}] < \frac{\gamma}{8(g+v)}$ and  $K \leq \frac{32(h+v)^2}{\gamma^2}$ we conclude
\begin{align*}
  \underset{hash \leftarrow \cH}{\Pr}[ \bigcap_{1 \leq i \leq K} Y_i ] \leq \left(1 - \frac{\gamma}{8(h+v)}\right)^{K}
    \leq e^{-\frac{\gamma}{8(h+v)} K}
    \leq e^{-\frac{4(h+v)}{\gamma}}.
\end{align*}
\end{proof}
%%% Local Variables:
%%% mode: latex
%%% TeX-master: "../master"
%%% End:

%
%
% TODO define puzzle and probabilities using x^{(k)}
%
\begin{codeblock}
  \textbf{Experiment $E^{P^{(g)}, C^{(.)(.)}, Hash}(\pi_1, \dots, \pi_k)$} \\
  Solving $k$-wise direct product of DWVP with respect to the set $P_{hash}$
  \medskip

  \hrule

  \medskip
  \textbf{Oracle:} Problem poser for k-wise direct product $P^{(g)}$ \\
  \IndI Solver circuit $C^{(.)(.)}$ with oracle access to hint and verification circuits \\
  \IndI Function $Hash: Q \leftarrow \{0, \dots, 2(h+v) - 1\}$\\
  \textbf{Input:} Random bitstring $(\pi_1, \dots, \pi_k) \in \{0,1\}^{lk}$\\

  \medskip\hrule\medskip

  $\pi^{(k)} := \left(\pi_1, \dots, \pi_k \right)$\\
  $(x^{k}, \Gamma_V^{(g)}, \Gamma_H^{(g)}) := P^{(g)}(\pi^{k})$\\
  Run $C^{\Gamma_V^{(g)}, \Gamma_H^{(g)}} (x^{(k)})$ \\
  \IndI Let $(q_j,y_j^{(k)})$ be the first successful verification query if $C^{\Gamma_V^{(g)}, \Gamma_H^{(g)}}$ succeeds or \\
  \IndI an arbitrary verification query when it fails.\\

  \textbf{If} $(\forall i < j :  Hash(q_i) \neq 0 )$ and $( Hash(q_j) = 1 \land \Gamma_V^{(g)}(q_j, y_j^{(k)}) = 1)$ \\
  \IndI \textbf{return} 1\\
  \textbf{else}\\
  \IndI \textbf{return} 0\\
\end{codeblock}
%
%
\begin{lemma}\textbf{Security amplification of a dynamic weakly verifiable puzzle with respect to set $P_{hash}$.} \\
  For a fixed dynamic weakly verifiable puzzle $P^{(1)}$ there exists an algorithm\\
  $Gen(C, g, \varepsilon, \delta, n, v, h, Hash)$, which takes as input a circuit $C$, a monotone function $g$,
  a function $Hash : Q \rightarrow \{0, \dots, 2(h+v)-1\}$, parameters $\varepsilon, \delta, n$,
  number of verification $v$, and hint $h$ queries asked by $C$, and outputs a circuit $D$
  such that following holds: \\
  If $C$ is such that \\
  \begin{align*}
    \underset{(\pi_1, \dots, \pi_k)}{\Pr}[E^{P^{(g)}, C, Hash}(\pi_1, \dots, \pi_k)] \geq \underset{\mu \leftarrow \mu_\delta^k}{\Pr}[g(\mu) = 1] + \varepsilon
  \end{align*}
  then $D$ satisfies almost surely
  \begin{align*}
    \underset{\pi}{\Pr}[F^{P^{(1)},D, Hash}(\pi) = 1] \geq (\delta + \frac{\varepsilon}{6k})
  \end{align*}
  and $Size(D) \leq Size(C)\frac{6k}{\varepsilon}$ and $Time(Gen) = poly(k, \frac{1}{\varepsilon}, n, v, h)$.
\end{lemma}
%
%
\begin{codeblock}
  \textbf{Random experiment $F^{P^{(1)}, D, Hash}(\pi)$} \\
  Solving a single DWVP with respect to the set $P_{hash}$
  \medskip

  \hrule

  \medskip

  \textbf{Oracle:} A circuit D, a function $Hash$, a dynamic weakly verifiable puzzle $P^{(1)}$\\
  \textbf{Input:} Random bitstring $\pi$
  %TODO length of the bitstring maybe it is fixed as it is input to P^{(1)} ? like l see the end of the paper by Imaginazzo.
  \medskip\hrule\medskip

  $(x, \Gamma_v, \Gamma_H) := P^{(1)}(\pi)$ \\
  Run $D^{\Gamma_V, \Gamma_H}(x)$ \\
  \IndI Let $(\widetilde{q_j},\widetilde{r_j})$ be the first successful verification query if $D^{\Gamma_V, \Gamma_H}(x)$ succeeds or \\
  \IndI an arbitrary verification query when it fails.\\
  \textbf{If} $(\forall i < j :  Hash(q_i) \neq 0 )$ and $Hash(q_j) = 1$ \\
  \IndI \textbf{return} 1\\
  \textbf{else}\\
  \IndI \textbf{return} 0\\

\end{codeblock}
%
%
% \begin{codeblock}
%   \textbf{Random experiment $G^{Hash}(\pi_1, \dots, \pi_k)$}
%   Random experiment F on the photos
%   \medskip

%   \hrule

%   \medskip

%   \textbf{Oracle:} A function $Hash$ \\
%   \textbf{Input:} Random bitstring $\pi$
%   %TODO length of the bitstring maybe it is fixed as it is input to P^{(1)} ? like l see the end of the paper by Imaginazzo.
%   \medskip\hrule\medskip
%   \textbf{For} $i=1$ to $k$ \\
%   \IndI $(x, \Gamma_V^{(i)}, \Gamma_H^{(i)}) = P^{(1)}(\pi_i)$ \\
%   Let $\Gamma_V^{(g)} $ be a circuit computing $g(\Gamma_v^{(1)}(q, r_1), \dots, \Gamma_V^{(k)}(q, r_k))$ \\
%   Let $\Gamma_H^{(g)} $ be a circuit computing $(\Gamma_v^{(1)}(q, r_1), \dots, \Gamma_V^{(k)}(q, r_k))$ \\
%   $(q,\widetilde{r}) = \widetilde{C}^{\Gamma_V^{(g)}, \Gamma_H^{(g)}, Hash} (x_1, \dots, x_k)$ \\
%   \textbf{If} $(q, \widetilde{r}) = \bot$ then \textbf{return} $\bot$ \\
%   \textbf{For} $i = 1$ to $k$ \\
%   \IndI $c_i = \Gamma_V^{(i)}(q,r_i)$ \\
%   \textbf{Return} $(c_1, \dots, c_k)$
% \end{codeblock}

\begin{codeblock}
  \textbf{Circuit $\widetilde{C}^{\Gamma_V^{(g)}, \Gamma_H^{(g)}, Hash} (x_1, \dots, x_k)$} \\
  Circuit $\widetilde{C}$ has good canonical success probability.
  \medskip

  \hrule

  \medskip

  \textbf{Oracle:} $\Gamma_V^{(g)}, \Gamma_H^{(g)}, Hash$ \\
  \textbf{Input:} k-wise direct product of puzzles $(x_1, \dots, x_k)$ \\

  \medskip\hrule\medskip
  Run $C^{(.), (.)}(x_1, \dots, x_k)$ \\
  \IndI \textbf{If} $C$ asks hint query $q$ \textbf{then}\\
  \IndII \textbf{If} $Hash(q) = 0$ \textbf{then}\\
  \IndIII \textbf{return} $\bot$\\
  \IndII \textbf{else}\\
  \IndIII answer with $\Gamma_H^{(g)}(q)$\\
  \\
  \IndI \textbf{If} $C$ asks verification query $(q, y_1, \dots, y_k)$ \textbf{then} \\
  \IndII \textbf{If} $hash(q) = 0$ \textbf{then} \\
  \IndIII \textbf{return} $(q, y_1, \dots, y_k)$ \\
  \IndII \textbf{else} \\
  \IndIII answer verification query with 0
  \textbf{return} $\bot$
\end{codeblock}


\begin{lemma}
  \begin{align*}
  \underset{(\pi_1, \dots, \pi_k)}{\Pr}[E^{P^{(g)}, C, Hash}(\pi_1, \dots, \pi_k) = 1] \leq \underset{(\pi_1, \dots, \pi_k)}{\Pr}[\Gamma_V^{(g)} (\widetilde{C}^{\Gamma_V^{(g)}, \Gamma_H^{(g)}, Hash}(\pi_1, \dots, \pi_k)) = 1]
  \end{align*}
\end{lemma}

\begin{proof}
If $E^{P^{(g)}, C, Hash}(\pi_1, \dots, \pi_k) = 1$ then circuit $\Gamma_V^{(g)} (\widetilde{C}^{\Gamma_V^{(g)}, \Gamma_H^{(g)}, Hash}(\pi_1, \dots, \pi_k)) = 1$.
\end{proof}

\begin{codeblock}
  \textbf{Algorithm $Gen(\widetilde{C},g,\epsilon,\delta,n)$}
  \medskip

  \hrule

  \medskip

  \textbf{Oracle:} $\widetilde{C}, g$ \\
  \textbf{Input:}  $\epsilon, \delta, n$\\
  \textbf{Output:} A circuit $D$

  \medskip\hrule\medskip

  \textbf{For} $i:=1$ to $\frac{6k}{\epsilon}\log(n)$ \\
  \IndI $\pi* \leftarrow \{0,1\}^{l}$\\
  \IndI $\widetilde{S}_{\pi^*,0} := EvaluateSurplus(\pi^*, 0)$\\
  \IndI $\widetilde{S}_{\pi^*,1} := EvaluateSurplus(\pi^*, 1)$\\
  \IndI \textbf{If} $\widetilde{S}_{\pi^*,0} \geq (1 - \frac{3}{4k}) \epsilon$ or $\widetilde{S}_{\pi^*,1} \geq (1 - \frac{3}{4k}) \epsilon$ \\
  \IndII $\widetilde{C}' := \widetilde{C}$ with the first input fixed on $\pi^*$\\
  \IndII\textbf{return} $Gen(\widetilde{C}', g, \epsilon, \delta, n)$ \\
  // all estimates are lower than $(1-\frac{3}{4k})\varepsilon$\\
  $SolvePuzzle(\pi, \widetilde{C})$ \\
  \\
  \textbf{EvaluateSurplus}($\pi^*, b$) \\
  \IndI \textbf{For} $i:=1$ to $N_k$ \\
  \IndII $\pi^{(k)} \leftarrow \{0,1\}^{lk}$\\
  \IndII $(c_1, \dots, c_k) := EvalutePuzzles(\pi^*, \pi^{(k)})$\\
  \IndII $\widetilde{S}_{\pi^*,b}^i := g(b, c_2, \dots, c_k) - \underset{(u_2, \dots, u_k)}{\Pr}[b, u_2, \dots, u_k] $\\
  \IndI \textbf{return} $\frac{1}{N_k} \sum_{i=1}^{N_k} \widetilde{S}_{\pi^*,b}^i$\\
  \\
  \textbf{EvalutePuzzles}($\pi^*, \pi^{(k)}$)\\
  \IndI $(x^{k}, \Gamma_V^{(g)}, \Gamma_H^{(g)}) := P^{(g)}(\pi^*, \pi_2, \dots, \pi_k)$ \\
  \IndI \textbf{For} $i=2$ to $k$\\
  \IndII $(x_1, \Gamma_v^{(i)}, \Gamma_H^{(i)}) := P^{(1)}(\pi_i)$\\
  \IndI $(q,y^{k}) := \widetilde{C}^{\Gamma_V^{(g)}, \Gamma_H^{(g)}}(x^*, x_2, \dots, x_k)$\\
  \IndI \textbf{For} $i=1$ to $k$\\
  \IndII $c_i := \Gamma_v^{i}(q, y_i)$\\
  \IndI \textbf{return} $(c_1, \dots, c_k)$\\
\end{codeblock}

\begin{codeblock}
  \textbf{Circuit $D^{\widetilde{C}}$}
  \medskip

  \hrule

  \medskip

  \textbf{Oracle:}  $\widetilde{C}, P^{(1)}$\\

  \medskip\hrule\medskip
  \textbf{For} $i:=1$ to $\frac{6k}{\epsilon} \log(\frac{6k}{\epsilon})$\\
  \IndI $\pi^{k} \leftarrow \{0,1\}^{k}$\\
  \IndI $(c_1, \dots, c_k) := EvaluatePuzzles(\pi, \pi^{(k)})$\\
  \IndI \textbf{If} $g(1,c_2, \dots, c_k) = 1$ and $g(0,c_2, \dots, c_k) = 0$\\
  \IndII $(q, y_1, \dots, y_k) := \widetilde{C}(\pi^*, \pi_2, \dots, \pi_k)$\\
  \IndII \textbf{return} $y_1$\\
  \textbf{return} $\bot$ \\

\end{codeblock}

% \begin{codeblock}
%   \textbf{$\text{Estimates}_{\pi^*}$}
%   \medskip

%   \hrule

%   \medskip

%   \textbf{Oracle:}  \\
%   \textbf{Input:}  \\

%   \medskip\hrule\medskip
%   \textbf{For} $i = 1$ to  $\frac{6k}{\varepsilon} \log(n)$\\
%   \IndI $\pi_i^* := \{0,1\}^{l}$ \tab{ // $\pi_i^*$ is picked uniformly at random}\\
%   \IndI \textbf{For} $j = 1$ to $K$ \\
%   \IndII $\pi^{k} := \{0,1\}^{kl}$ \tab{// $\pi^{k}$ is picked uniformly at random}\\
%   \IndII $(c_2, \dots, c_k) := H^{\widetilde{C}^{\Gamma_V^{(g)}, \Gamma_H^{(g)}}, P^{(1)}}(\pi_i^*, \pi_2, \dots, \pi_k)$\\
%   \IndI $\widetilde{S}_{\pi_i^*,0}^j := g(0, c_2, \dots, c_k)$ \\
%   \IndI $\widetilde{S}_{\pi_i^*,1}^j := g(1, c_2, \dots, c_k)$ \\

% \end{codeblock}

%%% Local Variables:
%%% mode: latex
%%% TeX-master: "master.tex"
%%% End:

%%% Local Variables: 
%%% mode: latex
%%% TeX-master: "../master"
%%% End: 
