\begin{codeblock}
  \textbf{Experiment $E^{P^{(g)}, C^{(.)(.)}, hash}(\pi_1, \dots, \pi_k)$} \\
  Solving $k$-wise direct product of DWVP with respect to the set $P_{hash}$
  \medskip

  \hrule

  \medskip
  \textbf{Oracle:} Problem poser for k-wise direct product $P^{(g)}$ \\
  \IndI A solver circuit for $k$-wise direct product $C^{(\cdot, \cdot)}$ \\
  \IndI A function $hash: Q \leftarrow \{0, \dots, 2(h+v) - 1\}$\\
  \textbf{Input:} Random bitstring $(\pi_1, \dots, \pi_k) \in \{0,1\}^{kl}$\\

  \medskip\hrule\medskip

  $\pi^{(k)} := \left(\pi_1, \dots, \pi_k \right)$\\
  $(x^{k}, \Gamma_V^{(g)}, \Gamma_H^{(k)}) := P^{(g)}(\pi^{k})$\\
  Run $C^{\Gamma_V^{(g)}, \Gamma_H^{(k)}} (x^{(k)})$ \\
  \IndI Let $(q_j,y_j^{(k)})$ be the first successful verification query if $C^{\Gamma_V^{(g)}, \Gamma_H^{(k)}}$ succeeds or \\
  \IndI an arbitrary verification query when it fails.\\
  \textbf{If} $(\forall i < j :  q_i \notin P_{hash} )$ and $q_j \in P_{hash}$ and $\Gamma_V^{(g)}(q_j, y_j^{(k)}) = 1$ \\
  \IndI \textbf{return} 1\\
  \textbf{else}\\
  \IndI \textbf{return} 0\\
\end{codeblock}
%
A canonical success is a situation when a solver $C$ for fixed $hash$ and $P^{(1)}$ succeeds in a random experiment $E$.
%
\begin{codeblock}
  \textbf{Random experiment $F^{P^{(1)}, D, hash}(\pi)$} \\
  Solving a single DWVP with respect to the set $P_{hash}$
  \medskip

  \hrule

  \medskip

  \textbf{Oracle:}
  A dynamic weakly verifiable puzzle $P^{(1)}$ \\
  \IndI A solver circuit for a single DWVP $D$ \\
  \IndI A function $hash: Q \rightarrow \{0,1,\dots, 2(h+v)-1\}$ \\
  \textbf{Input:} Random bitstring $\pi \in \{0,1\}^l$
  %TODO length of the bitstring maybe it is fixed as it is input to P^{(1)} ? like l see the end of the paper by Imaginazzo.
  \medskip\hrule\medskip

  $(x, \Gamma_v, \Gamma_H) := P^{(1)}(\pi)$ \\
  Run $D^{\Gamma_V, \Gamma_H}(x)$ \\
  \IndI Let $(\widetilde{q_j},\widetilde{r_j})$ be the first successful verification query if $D^{\Gamma_V, \Gamma_H}(x)$ succeeds or \\
  \IndI an arbitrary verification query when it fails. \\
  \If $(\forall i < j :  q_i \notin  P_{hash} )$ and $q_j \in P_{hash}$ and $\Gamma_V(q_j) = 1$ \then \\
  \IndI \return 1 \\
  \textbf{else}\\
  \IndI \return 0\\

\end{codeblock}
%
%
\begin{lemma}
  \label{lemma:sec_amp_for_p_hash}
  \textbf{Security amplification of a dynamic weakly verifiable puzzle with respect to set $P_{hash}$.} \\
  For a fixed dynamic weakly verifiable puzzle $P^{(1)}$ there exists an algorithm\\
  $Gen(C, g, \varepsilon, \delta, n, v, h, hash)$, which takes as input a circuit $C$, a monotone function $g$,
  a function $hash : Q \rightarrow \{0, \dots, 2(h+v)-1\}$, parameters $\varepsilon, \delta, n$,
  number of verification $v$, and hint $h$ queries asked by $C$, and outputs a circuit $D$
  such that following holds: \\
  If $C$ is such that \\
  \begin{align*}
    \underset{(\pi_1, \dots, \pi_k)}{\Pr}[E^{P^{(g)}, C, Hash}(\pi_1, \dots, \pi_k)=1] \geq \underset{\mu \leftarrow \mu_\delta^k}{\Pr}[g(\mu) = 1] + \varepsilon
  \end{align*}
  then $D$ satisfies almost surely
  \begin{align*}
    \underset{\pi}{\Pr}[F^{P^{(1)},D, Hash}(\pi) = 1] \geq (\delta + \frac{\varepsilon}{6k})
  \end{align*}
  and $Size(D) \leq Size(C)\frac{6k}{\varepsilon}$ and $Time(Gen) = poly(k, \frac{1}{\varepsilon}, n, v, h)$.
\end{lemma}

\begin{codeblock}
  \textbf{Circuit $\widetilde{C}^{\Gamma_V^{(g)}, \Gamma_H^{(g)}, hash, C} (x_1, \dots, x_k)$} \\
  Circuit $\widetilde{C}$ has good canonical success probability.
  \medskip

  \hrule

  \medskip

  \textbf{Oracle:} $\Gamma_V^{(g)}, \Gamma_H^{(g)}, hash, C$ \\
  \textbf{Input:} k-wise direct product of puzzles $(x_1, \dots, x_k)$

  \medskip\hrule\medskip
  Run $C^{(\cdot,\cdot)}(x_1, \dots, x_k)$ \\
  \IndI \If $C$ asks a hint query $q$ \then\\
  \IndII \If $q \in P_{hash}$ \then\\
  \IndIII \return $\bot$\\
  \IndII \textbf{else}\\
  \IndIII answer the hint query with $\Gamma_H^{(k)}(q)$\\
  \\
  \IndI \textbf{If} $C$ asks a verification query $(q, y_1, \dots, y_k)$ \textbf{then} \\
  \IndII \textbf{If} $q \in P_{hash}$ \textbf{then} \\
  \IndIII \text{ask the verification query} $(q, y_1, \dots, y_k)$ \\
  \IndIII \textbf{stop the execution} \\
  \IndII \textbf{else} \\
  \IndIII answer verification query with 0 \\
  \textbf{return} $\bot$
\end{codeblock}
The key difference between circuits $C$ and $\widetilde{C}$
is that if $\widetilde{C}$ asks a verification query $(q, y_1, \dots, y_k)$ then $q \in P_{hash}$.
This means that if $\widetilde{C}$ succeeds then it also succeeds canonically.

\begin{lemma}
  For fixed $P^{(g)}$ it is true that
  \begin{align*}
  \underset{(\pi_1, \dots, \pi_k)}{\Pr}[E^{P^{(g)}, C, Hash}(\pi_1, \dots, \pi_k) = 1] \leq \underset{(\pi_1, \dots, \pi_k)}{\Pr}[\Gamma_V^{(g)} (\widetilde{C}^{\Gamma_V^{(g)}, \Gamma_H^{(g)}, Hash}(\pi_1, \dots, \pi_k)) = 1].
  \end{align*}
\end{lemma}

\begin{proof}
We fix the randomness $(\pi_1, \dots, \pi_k)$ used in the random experiment $E$.
Let $x^{(k)} = (x_1, \dots, x_k)$ be a set of puzzles generated in the random experiment $E$ for the randomness $(\pi_1, \dots, \pi_k)$.
If $C$ succeeds canonically for the set of puzzles $x^{(k)}$, then also circuit $\widetilde{C}$ that runs $C$ on the same set of puzzles succeeds.
Using the definition of conditional expectation, we conclude that
\begin{align*}
  \underset{}{\Pr}[E^{P^{(g)}, C, hash}(\pi^{(k)}) = 1] &=
  \sum_{\pi^{(k)} \in \{0,1\}^{kl}}\underset{}{\Pr}[E^{P^{(g)}, C, hash}(\widetilde{\pi}^{(k)}) = 1 | \pi^{(k)} = \widetilde{\pi}^{(k)}] \underset{}{\Pr}[\pi^{(k)} = \widetilde{\pi}^{(k)}] \\
  &\leq
  \sum_{\pi^{(k)} \in \{0,1\}^{kl}}\underset{}{\Pr}[E^{P^{(g)}, \widetilde{C}, hash}(\widetilde{\pi}^{(k)}) = 1 | \pi^{(k)} = \widetilde{\pi}^{(k)}] \underset{}{\Pr}[\pi^{(k)} = \widetilde{\pi}^{(k)}] \\
  &= \underset{}{\Pr}[E^{P^{(g)}, \widetilde{C}, hash}(\pi^{(k)}) = 1]
\end{align*}

% If the first successful verification query is in $P_{hash}$, all previous hint queries were not in $P_{hash}$,
% and all previous verification queries were unsuccessful and are not in $P_{hash}$, then
%$\Gamma_V^{(g)} (\widetilde{C}^{\Gamma_V^{(g)}, \Gamma_H^{(g)}, Hash}(\pi_1, \dots, \pi_k)) = 1$.
\end{proof}

\begin{codeblock}
  \textbf{Algorithm $Gen(\widetilde{C},g,\epsilon,\delta,n)$}
  \medskip

  \hrule

  \medskip

  \textbf{Oracle:} $\widetilde{C}, g$ \\
  \textbf{Input:}  $\epsilon, \delta, n$\\
  \textbf{Output:} A circuit $D$

  \medskip\hrule\medskip

  \textbf{For} $i:=1$ to $\frac{6k}{\epsilon}\log(n)$ \\
  \IndI $\pi^* \leftarrow \{0,1\}^{l}$\\
  \IndI $\widetilde{S}_{\pi^*,0} := EvaluateSurplus(\pi^*, 0)$\\
  \IndI $\widetilde{S}_{\pi^*,1} := EvaluateSurplus(\pi^*, 1)$\\
  \IndI \textbf{If} $\widetilde{S}_{\pi^*,0} \geq (1 - \frac{3}{4k}) \epsilon$ or $\widetilde{S}_{\pi^*,1} \geq (1 - \frac{3}{4k}) \epsilon$ \\
  \IndII $\widetilde{C}' := \widetilde{C}$ with the first input fixed on $\pi^*$\\
  \IndII\textbf{return} $Gen(\widetilde{C}', g, \epsilon, \delta, n)$ \\
  // all estimates are lower than $(1-\frac{3}{4k})\varepsilon$\\
  \return $D^{\widetilde{C}}$ \\
  \\
  \textbf{EvaluateSurplus}($\pi^*, b$) \\
  \IndI \textbf{For} $i:=1$ to $N_k$ \\
  \IndII $\pi^{(k)} \leftarrow \{0,1\}^{lk}$\\
  \IndII $(c_1, \dots, c_k) := EvalutePuzzles(\pi^*, \pi^{(k)})$\\
  \IndII $\widetilde{S}_{\pi^*,b}^i := g(b, c_2, \dots, c_k) - \underset{(u_2, \dots, u_k)}{\Pr}[g(b, u_2, \dots, u_k) = 1] $\\
  \IndI \textbf{return} $\frac{1}{N_k} \sum_{i=1}^{N_k} \widetilde{S}_{\pi^*,b}^i$\\
  \\
  \textbf{EvalutePuzzles}($\pi^*, \pi^{(k)}$)\\
  \IndI $(x^{k}, \Gamma_V^{(g)}, \Gamma_H^{(g)}) := P^{(g)}(\pi^*, \pi_2, \dots, \pi_k)$ \\
  \IndI \textbf{For} $i=2$ to $k$\\
  \IndII $(x_1, \Gamma_v^{(i)}, \Gamma_H^{(i)}) := P^{(1)}(\pi_i)$\\
  \IndI $(q,y^{k}) := \widetilde{C}^{\Gamma_V^{(g)}, \Gamma_H^{(g)}}(x^*, x_2, \dots, x_k)$\\
  \IndI \textbf{For} $i=1$ to $k$\\
  \IndII $c_i := \Gamma_v^{i}(q, y_i)$\\
  \IndI \textbf{return} $(c_1, \dots, c_k)$\\
\end{codeblock}
%
\begin{codeblock}
  \textbf{Circuit $D^{\widetilde{C}}$}
  \medskip

  \hrule

  \medskip

  \textbf{Oracle:}  $\widetilde{C}, P^{(1)}$\\
  \textbf{Input:}  puzzle $x$

  \medskip\hrule\medskip
  \textbf{For} $i:=1$ to $\frac{6k}{\epsilon} \log(\frac{6k}{\epsilon})$\\
  \IndI $\pi^{k} \leftarrow \{0,1\}^{k}$\\
  \IndI $(c_1, \dots, c_k) := EvaluatePuzzles(\pi, \pi^{(k)})$\\
  \IndI \textbf{If} $g(1,c_2, \dots, c_k) = 1$ and $g(0,c_2, \dots, c_k) = 0$\\
  \IndII $(q, y_1, \dots, y_k) := \widetilde{C}(\pi^*, \pi_2, \dots, \pi_k)$\\
  \IndII \textbf{return} $y_1$\\
  \textbf{return} $\bot$ \\

\end{codeblock}
%
% The algorithm $Gen$ recursively builds the circuit that have high success probability in solving a dynamic weakly verifiable puzzle.
% When algorithm recurses it fixes a puzzle on the first position on the input of circuit $\widetilde{C}$ which yields a new circuit $\widetilde{C}'$.
% It happens only in the situation when for some fixed $\pi^*$ circuit $\widetilde{C}$ performs good on the remaining $k-1$ puzzles.
%
Let $(q, y_1, \dots, y_k)$ denote the output of $\widetilde{C}$.
Additionally, let us denote by $c_i = \Gamma_V(q, y_i)$ whether $(q,y_i)$ is a correct solution for a single puzzle.
We define surplus as the following quantity
\begin{align*}
S_{\pi^*, b} = \underset{\pi^{(k)}}{\Pr}[g(b, c_2, \dots, c_k) = 1] - \underset{\mu^{(k)}}{\Pr}[g(b, u_2, \dots, u_k) = 1]
\end{align*}
The surplus $S_{\pi^*, b}$ tells how good the algorithm $\widetilde{C}$ performs when the first puzzle is fixed, and the fact whether it is correctly solved is neglected.
The procedure \textbf{EvaluateSurplus} returns the estimate $\widetilde{S}_{\pi^*, b}$ that differs from $\widetilde{S}_{\pi*, b}$
by at most $\frac{\epsilon}{4k}$ almost surely. Therefore, we conclude $S_{\pi*,b} \geq (1-\frac{1}{k})\epsilon$.
We use a new monotone binary function $g'(b_2, \dots, b_k) := g(b, b_2, \dots, b_k)$, and fix the first input puzzle of $\widetilde{C}$ for the one generated
by using the randomness $\pi^*$. The new circuit satisfies the conditions of Lemma \ref{lemma:sec_amp_for_p_hash} which means that we can
use algorithm $Gen$ for the new circuit $\widetilde{C}$ and monotone function $g'$.
For $k=1$ function $g(b)$ is either identity or a constant function.
If $g$ is identity then the success probability of $\widetilde{C}$ is as least $(\delta + \epsilon}$
and we can simply use $\widetilde{C}$ to solve the single puzzle. If the function $g$ is constant the vacuously true.


%%% Local Variables:
%%% mode: latex
%%% TeX-master: "../master"
%%% End:
