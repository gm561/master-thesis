%General TODO:
% Define Gen as a program
%
%
%
%TODO runtime bound
%TODO class complexity bound
%Def : (Dynamic weakly verifiable puzzle (non interactive version))
\begin{definition} {\textbf{(Dynamic weakly verifiable puzzle (non interactive version))}}\\
  A dynamic weakly verifiable puzzle (DWVP) is defined by a probabilistic algorithm $P(\pi)$,
  called a problem poser, that takes as input chosen uniformly at random bitstring $\pi \in \{0,1\}^l$.
  The algorithm $P(\pi)$ produces circuits $\Gamma_{V}$, $\Gamma_{H}$ and a puzzle $x \in \{0,1\}^{*}$.
  The circuit $\Gamma_{V}$ takes as its input $q \in Q$ and an answer $y$.
  An answer $y$ is a correct solution for $q$ if and only if $\Gamma_V(q,y) = 1$.
  The circuit $\Gamma_H$ on input $q$ provides a hint such that $\Gamma_V(q,\Gamma_H(q)) = 1$.
  The algorithm $S$, called a solver, has oracle access to $\Gamma_V$ and $\Gamma_H$.
  The calls to $\Gamma_V$ are verification queries, the calls to $\Gamma_H$ are hint queries.
  The solver $S$ can ask at most $h$ hint queries, $v$ verification queries, and successfully solves a DWVP if and only if
  it makes a verification query $(q,r)$ such that $\Gamma_V(q,r) = 1$, when it has not previously asked for a hint query on this $q$.
\end{definition}
%
%
%
\begin{codeblock}
  \textbf{Experiment $B^{P^{(1)}, D}(\pi)$} \\
  Solving a dynamic weakly verifiable puzzle
  \medskip

  \hrule

  \medskip

  \textbf{Oracle:} Problem poser for a single instance of DWVP $P^{(g)}$, a solver circuit $D$. \\
  \textbf{Input:} A random bitstring $\pi \in \{0,1\}^{l}$.\\

  \medskip\hrule\medskip

  $(x, \Gamma_V, \Gamma_H) := P^{(1)}(\pi)$ \\
  Run $D^{(.)(.)}(x)$ with oracle access to $\Gamma_V$ and $\Gamma_H$ \\
  \IndI Let $(\widetilde{q},y)$ be the first verification query of $D^{\Gamma_H, \Gamma_V}(x)$ such that $\Gamma_V(\widetilde{q},y) = 1$ \\
  \IndI Define $Q_{Hint} := \{q: \text{$D^{\Gamma_H, \Gamma_V}(x)$ asked a hint query on q} \}$\\
  \textbf{If} $q \notin Q_{Hint}$\\
  \IndI \textbf{return} $1$\\
  \textbf{else}\\
  \IndI \textbf{return} $0$\\
\end{codeblock}
%
%
%
%Def: (k-wise direct product of dynamic weakly verifiable puzzles)
\begin{definition}{\textbf{(k-wise direct product of dynamic weakly verifiable puzzles)}}\\
Let $g: \{0,1\}^{k} \rightarrow \{0,1\}$ denote a monotone function, and $P^{(1)}$ an algorithm used to generate an instance of DWVP.
A $k$-wise direct product of dynamic weakly verifiable puzzles is defined by an algorithm $P^{(g)}\left(\pi_1, \dots, \pi_k \right)$,
where $(\pi_1, \dots, \pi_k) \in \{0,1\}^{kl}$ are chosen uniformly at random.
The algorithm $P^{(g)}\left(\pi_1, \dots, \pi_k \right)$ sequentially generates $k$ independent instances of dynamic weakly verifiable puzzles,
where in the $i$-th round $P^{(g)}$ runs $P^{(1)}(\pi_i)$ and obtains $(x_i, \Gamma_V^{(i)}, \Gamma_H^{(i)} )$.
Finally, $P^{(g)}$ outputs a verification circuit
\begin{align*}
  \Gamma_V^{(g)} (q, r_1, \dots, r_k) := g(\Gamma_V^{(1)}(q, r_1), \dots, \Gamma_V^{(k)}(q, r_k)),
\end{align*}
a hint circuit
\begin{align*}
  \Gamma_H^{(g)} (q) := (\Gamma_H^{(1)}(q), \dots, \Gamma_H^{(k)}(q)),
\end{align*}
and a puzzle $x^{(k)} = (x_1, \dots, x_k)$.
\end{definition}
%TODO define a circuit C for problem poser + assumptions about C
%
\begin{codeblock}
  \textbf{Experiment $A^{P^{(g)}, C^{(.) (.)}}(\pi^{(k)})$} \\
  Solving k-wise direct product of dynamic weakly verifiable puzzles.
  \medskip

  \hrule

  \medskip

  \textbf{Oracle:} Problem poser for k-wise direct product $P^{(g)}$, a solver circuit $C^{(.)(.)}$ with oracle access to hint and verification circuits. \\
  \textbf{Input:} Random bitstring $\pi^{(k)} \in \{0,1\}^{lk}$.\\

  \medskip\hrule\medskip

  $(x^{(k)}, \Gamma_V^{(g)}, \Gamma_H^{(g)}) := P^{(g)}(\pi^{(k)})$ \\
  Run $C^{(.)(.)}(x)$ with oracle access to $\Gamma_V$ and $\Gamma_H$ \\
  \IndI Let $(\widetilde{q},y)$ be the first verification query of $C^{\Gamma_V^{(g)}, \Gamma_H^{(g)}}(x)$ such that $\Gamma_V^{(g)}(\widetilde{q},y_1, \dots, y_k) = 1$ \\
  \IndI Define $Q_{Hint} := \{q: \text{$D^{\Gamma_V^{(g)}, \Gamma_H^{(g)}}(x^{(k)})$ asked a hint query on q} \}$\\
  \textbf{If} $q \notin Q_{Hint}$\\
  \IndI \textbf{return} $1$\\
  \textbf{else}\\
  \IndI \textbf{return} $0$\\

\end{codeblock}
%
%
%
%
\begin{theorem}{\textbf{Security amplification of a dynamic weakly verifiable puzzle.}}\\
  Fix a problem poser $P^{(1)}$.
  There exists an algorithm $Gen(C, g, \varepsilon, \delta, n, v, h)$ which takes as input a circuit $C$, a monotone function $g$, parameters
  $\varepsilon, \delta$, a security parameter $n$, number of verification $v$, and hint $h$ queries asked by $C$, and outputs a circuit $D$
  such that following holds: \\
  If $C$ is such that \\
  \begin{align*}
    \underset{(\pi_1, \dots, \pi_k) \in \{0,1\}^{lk}}{\Pr}[A^{P^{(g)}, C}(\pi_1, \dots, \pi_k) = 1] \geq \underset{\mu \leftarrow \mu_\delta^k}{\Pr}[g(\mu) = 1] + \varepsilon
  \end{align*}
  then $D$ satisfies almost surely
  \begin{align*}
    \underset{\pi \in \{0,1\}^{l}}{\Pr}[B^{P^{(1)},D}(\pi) = 1] \geq (\delta + \frac{\varepsilon}{6k})
  \end{align*}
  and $Size(D) \leq Size(C)\frac{6k}{\varepsilon}$ and $Time(Gen) = poly(k, \frac{1}{\varepsilon}, n, v, h)$.
\end{theorem}
%
%
%
% TODO define puzzle and probabilities using x^{(k)}
%
\begin{codeblock}
  \textbf{Experiment $E^{P^{(g)}, C^{(.)(.)}, Hash}(\pi_1, \dots, \pi_k)$} \\
  Solving $k$-wise direct product with respect to the set $P_{hash}$
  \medskip

  \hrule

  \medskip
  \textbf{Oracle:} Problem poser for k-wise direct product $P^{(g)}$ \\
  \IndI Solver circuit $C^{(.)(.)}$ with oracle access to hint and verification circuits \\
  \IndI Function $Hash: Q \leftarrow \{0, \dots, 2(h+v) - 1\}$\\
  \textbf{Input:} Random bitstring $(\pi_1, \dots, \pi_k) \in \{0,1\}^{lk}$\\

  \medskip\hrule\medskip

  $\pi^{(k)} := \left(\pi_1, \dots, \pi_k \right)$\\
  $(x^{k}, \Gamma_V^{(g)}, \Gamma_H^{(g)}) := P^{(g)}(\pi^{k})$\\
  Run $C^{\Gamma_V^{(g)}, \Gamma_H^{(g)}} (x^{(k)})$ \\
  \IndI Let $(q_j,y_j^{(k)})$ be the first successful verification query if $C^{\Gamma_V^{(g)}, \Gamma_H^{(g)}}$ succeeds or \\
  \IndI an arbitrary verification query when it fails.\\

  \textbf{If} $(\forall i < j :  Hash(q_i) \neq 0 )$ and $( Hash(q_j) = 1 \land \Gamma_V^{(g)}(q_j, y_j^{(k)}) = 1)$ \\
  \IndI \textbf{return} 1\\
  \textbf{else}\\
  \IndI \textbf{return} 0\\
\end{codeblock}
% Then canonical success probability of $C$ in a random experiment $A^{P^{(g)}, C, Hash}(\pi_1, \dots, \pi_k)$ is \\
% $\underset{(\pi_1, \dots, \pi_k)}{\Pr}[A^{P^{(g)}, C, Hash}(\pi_1, \dots, \pi_k)]$.
%
% TODO: circuits or algorithms
% TODO: size of circuits?
% TODO: need def. of canonical success and P_hash
% TODO: number of calls to oracle circuit C
%
\begin{lemma}
Fix $P^{(1)}$ and let $C$ be a circuit that succeeds in solving the $k$-wise direct product of DWVP produced by $P^{(1)}$
with probability $\varepsilon$ making $h$ hint and $v$ verification queries.
Then there exists a probabilistic algorithm, with oracle access to $C$, that runs in time $O((h+v)^4/\varepsilon^4)$
and with high probability outputs a function $Hash: Q \leftarrow \{0, 2(h+v)-1\}$ such that success probability of
$C$ in random experiment $E$ with respect to set $P_{Hash}$ is at least $\frac{\varepsilon}{8(h+v)}$.
\end{lemma}
%
%
%
\begin{lemma}
  For a fixed dynamic weakly verifiable puzzle $P^{(1)}$ there exists an algorithm\\
  $Gen(C, g, \varepsilon, \delta, n, v, h, Hash)$, which takes as input a circuit $C$, a monotone function $g$,
  a function $Hash : Q \leftarrow \{0, 2(h+v)-1\}$, parameters $\varepsilon, \delta, n$,
  number of verification $v$, and hint $h$ queries asked by $C$, and outputs a circuit $D$
  such that following holds: \\
  If $C$ is such that \\
  \begin{align*}
    \underset{(\pi_1, \dots, \pi_k)}{\Pr}[A^{P^{(g)}, C, Hash}(\pi_1, \dots, \pi_k)] \geq \underset{\mu \leftarrow \mu_\delta^k}{\Pr}[g(\mu) = 1] + \varepsilon
  \end{align*}
  then $D$ satisfies almost surely
  \begin{align*}
    \underset{}{\Pr}[B^{P^{(1)},D, Hash}(\pi) = 1] \geq (\delta + \frac{\varepsilon}{6k})
  \end{align*}
  and $Size(D) \leq Size(C)\frac{6k}{\varepsilon}$ and $Time(Gen) = poly(k, \frac{1}{\varepsilon}, n, v, h)$.
\end{lemma}
%
\begin{codeblock}
  \textbf{Random experiment $B^{P^{(1)}, C, Hash}(\pi)$}
  \medskip

  \hrule

  \medskip

  \textbf{Oracle:} A circuit C, a function $Hash$ and a dynamic weakly verifiable puzzle $P^{(1)}$\\
  \textbf{Input:} A random bitstring $\pi$ of length at most $Time ? $
  %TODO length of the bitstring maybe it is fixed as it is input to P^{(1)} ? like l see the end of the paper by Imaginazzo.
  \medskip\hrule\medskip

  $(x, \Gamma_v, \Gamma_H) := P^{(1)}(\pi)$ \\
  Run $D^{\Gamma_V, \Gamma_H}(x)$ \\
  \IndI Let $(\widetilde{q_j},\widetilde{r_j})$ be the first successful verification query if $D^{\Gamma_V, \Gamma_H}(x)$ succeeds or \\
  \IndI an arbitrary verification query when it fails.\\
  \textbf{If} $(\forall i < j :  Hash(q_i) \neq 0 )$ and $Hash(q_j) = 1$ \\
  \IndI \textbf{return} 1\\
  \textbf{else}\\
  \IndI \textbf{return} 0\\

\end{codeblock}

\begin{codeblock}
  \textbf{Random experiment $F^{Hash}(\pi_1, \dots, \pi_k)$}
  \medskip

  \hrule

  \medskip

  \textbf{Oracle:} A function $Hash$ \\
  \textbf{Input:} A random bitstring $\pi$ of length at most $Time ? $
  %TODO length of the bitstring maybe it is fixed as it is input to P^{(1)} ? like l see the end of the paper by Imaginazzo.
  \medskip\hrule\medskip
  \textbf{For} $i=1$ to $k$ \\
  \IndI $(x, \Gamma_V^{(i)}, \Gamma_H^{(i)}) = P^{(1)}(\pi_i)$
  \textbf{End}
  Let $\Gamma_V^{(g)} $ be a circuit computing $g(\Gamma_v^{(1)}(q, r_1), \dots, \Gamma_V^{(k)}(q, r_k))$ \\
  Let $\Gamma_H^{(g)} $ be a circuit computing $(\Gamma_v^{(1)}(q, r_1), \dots, \Gamma_V^{(k)}(q, r_k))$ \\
  $(q,\widetilde{r}) = \widetilde{C}^{\Gamma_V^{(g)}, \Gamma_H^{(g)}, Hash} (x_1, \dots, x_k)$ \\
  \textbf{If} $(q, \widetilde{r}) = \bot$ then return $\bot$ \\
  \textbf{For} $i = 1$ to $k$ \\
  \IndI $c_i = \Gamma_V^{(i)}(q,r_i)$ \\
  \textbf{End}\\
  \textbf{Return} $(c_1, \dots, c_k)$

\end{codeblock}



%%% Local Variables:
%%% mode: latex
%%% TeX-master: "master.tex"
%%% End:
