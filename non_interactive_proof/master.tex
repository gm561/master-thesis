
% \documentclass[11commpt]{article}

% %% We use the memoir class because it offers a many easy to use features.
\documentclass[11pt,a4paper,article,oneside]{memoir}

\counterwithout{section}{chapter}
\usepackage[margin=1in]{geometry}

%% Packages
%% ========

%% LaTeX Font encoding -- DO NOT CHANGE
%\usepackage[OT1]{fontenc}

%% Babel provides support for languages.  'english' uses British
%% English hyphenation and text snippets like "Figure" and
%% "Theorem". Use the option 'ngerman' if your document is in German.
%% Use 'american' for American English.  Note that if you change this,
%% the next LaTeX run may show spurious errors.  Simply run it again.
%% If they persist, remove the .aux file and try again.
\usepackage[english]{babel}

%% Input encoding 'utf8'. In some cases you might need 'utf8x' for
%% extra symbols. Not all editors, especially on Windows, are UTF-8
%% capable, so you may want to use 'latin1' instead.
\usepackage[utf8]{inputenc}

%% This changes default fonts for both text and math mode to use Herman Zapfs
%% excellent Palatino font.  Do not change this.
%\usepackage[sc]{mathpazo}

%% The AMS-LaTeX extensions for mathematical typesetting.  Do not
%% remove.
\usepackage{amsmath,amssymb,amsfonts}

%% NTheorem is a reimplementation of the AMS Theorem package. This
%% will allow us to typeset theorems like examples, proofs and
%% similar.  Do not remove.
%% NOTE: Must be loaded AFTER amsmath, or the \qed placement will
%% break
\usepackage[amsmath,thmmarks]{ntheorem}

%% LaTeX' own graphics handling
\usepackage{graphicx}

%% We unfortunately need this for the Rules chapter.  Remove it
%% afterwards; or at least NEVER use its underlining features.
\usepackage{soul}

%% For code snippets
\usepackage[framemethod=tikz]{mdframed}
\newdimen\linenumbersep

\newcommand{\linenumber}[1]{%
  \linenumbersep 4pt%
  \advance\linenumbersep\mdflength{innerleftmargin}%
  \advance\linenumbersep\mdflength{innerlinewidth}%
  \advance\linenumbersep\mdflength{middlelinewidth}%
  \advance\linenumbersep\mdflength{outerlinewidth}%
  \advance\linenumbersep\mdflength{linewidth}%
  \makebox[0pt][r]{{\rmfamily\tiny#1}\hspace*{\linenumbersep}}}

%\usepackage{mdframed}
\newenvironment{codeblock}%
   {\medskip\begin{mdframed}\setlength{\parindent}{0cm}}%
   {\end{mdframed}\medskip}
\newcommand{\Ind}{\mbox{}}
\newcommand{\IndI}{\mbox\qquad}
\newcommand{\IndII}{\mbox\qquad\qquad}
\newcommand{\IndIII}{\mbox\qquad\qquad\qquad}
\newcommand{\IndIIII}{\mbox\qquad\qquad\qquad\qquad}

%% Some more packages that you may want to use.  Have a look at the
%% file, and consult the package docs for each.
\input{extrapackages}

%% Our layout configuration.  DO NOT CHANGE.
%\input{layoutsetup}

%% Theorem environments.
%% thesis.
%% Theorem-like environments

%% This can be changed according to language. You can comment out the ones you
%% don't need.

\numberwithin{equation}{chapter}

%% German theorems
%\newtheorem{satz}{Satz}[chapter]
%\newtheorem{beispiel}[satz]{Beispiel}
%\newtheorem{bemerkung}[satz]{Bemerkung}
%\newtheorem{korrolar}[satz]{Korrolar}
%\newtheorem{definition}[satz]{Definition}
%\newtheorem{lemma}[satz]{Lemma}
%\newtheorem{proposition}[satz]{Proposition}

%% English variants
\newtheorem{theorem}{Theorem}[chapter]
\newtheorem{example}[theorem]{Example}
\newtheorem{remark}[theorem]{Remark}
\newtheorem{corollary}[theorem]{Corollary}
\newtheorem{lemma}[theorem]{Lemma}
\newtheorem{proposition}[theorem]{Proposition}
\newtheorem{observation}[theorem]{Observation}

\theoremstyle{definition}
\theorembodyfont{\normalfont}
%% end def with blacksquare symbol
\theoremsymbol{\ensuremath{\lozenge}}
\newtheorem{definition}[theorem]{Definition}

%% Proof environment with a small square as a "qed" symbol
\theoremstyle{nonumberplain}
\theorembodyfont{\normalfont}
\theoremsymbol{\ensuremath{\square}}
\theoremseparator{.}
\newtheorem{proof}{Proof}

%\newtheorem{beweis}{Beweis}

\declaretheorem[name=Theorem, numberwithin=chapter]{thm}


%% Helpful macros.
%% Custom commands
%% ===============

%% Special characters for number sets, e.g. real or complex numbers.
\newcommand{\C}{\mathbb{C}}
\newcommand{\K}{\mathbb{K}}
\newcommand{\N}{\mathbb{N}}
\newcommand{\Q}{\mathbb{Q}}
\newcommand{\R}{\mathbb{R}}
\newcommand{\Z}{\mathbb{Z}}
\newcommand{\X}{\mathbb{X}}

\newcommand{\cX}{\mathcal{X}}
\newcommand{\cH}{\mathcal{H}}
\newcommand{\cW}{\mathcal{W}}
\newcommand{\cG}{\mathcal{G}}
\newcommand{\cB}{\mathcal{B}}
\newcommand{\cP}{\mathcal{P}}
\newcommand{\cR}{\mathcal{R}}
\newcommand{\cD}{\mathcal{D}}

%define our own code commands
%use capital latters as most of these commands is already defined
\renewcommand{\For}{\textbf{for }}
\renewcommand{\If}{\textbf{if }}
\renewcommand{\Else}{\textbf{else }}
\renewcommand{\Return}{\textbf{return }}
\newcommand{\Then}{\textbf{then }}
\newcommand{\Do}{\textbf{do: }}
\renewcommand{\And}{\textbf{and }}
\newcommand{\Or}{\textbf{or }}
\newcommand{\Run}{\textbf{run }}
\newcommand{\To}{\textbf{to }}

%% Fixed/scaling delimiter examples (see mathtools documentation)
\DeclarePairedDelimiter\abs{\lvert}{\rvert}
\DeclarePairedDelimiter\norm{\lVert}{\rVert}

%% Use the alternative epsilon per default and define the old one as \oldepsilon
\let\oldepsilon\epsilon
\renewcommand{\epsilon}{\ensuremath\varepsilon}

%% Also set the alternate phi as default.
\let\oldphi\phi
\renewcommand{\phi}{\ensuremath{\varphi}}

% New command that introduces a tab
\newcommand{\itab}[1]{\hspace{0em}\rlap{#1}}
\newcommand{\tab}[1]{\hspace{.2\textwidth}\rlap{#1}}

\DeclareMathOperator{\la0}{\leftarrow}
\DeclareMathOperator{\ra0}{\rightarrow}

\DeclareMathOperator{\hash}{\mathit{hash}}
\DeclareMathOperator{\CanonicalSuccess}{\mathit{CanonicalSuccess}}
\DeclareMathOperator{\Success}{\mathit{Success}}

%the DWVP for the permutation
\DeclareMathOperator{\PiDWVP}{\Pi_{DWVP}}
%the DWPV for the k-wise product of permutations
\DeclareMathOperator{\kPiDWVP}{\Pi_{DWVP}^{(k)}}

\usepackage{enumerate}

\begin{document}

%\title{Latex template}
%\author{Grzegorz Makosa}
%\maketitle

\setcounter {myc}{1}
%TODO pairwise independent hash function
%TODO define Size and Time
\noindent
We write $\mu_{\delta}$ to denote the Bernoulli distribution, where outcome $1$ occurs with
probability $\delta$ and $0$ with probability $1-\delta$ where $0 \leq \delta \leq 1$.
Moreover, we use $\mu_{\delta}^k$ to denote a probability distribution over $k$-tuples,
where each bit of a $k$-tuple is drawn independently according to $\mu_{\delta}$.
Finally, let $u \leftarrow \mu_{\delta}^k$ denote that a $k$-tuple $u$ is chosen according to $\mu_{\delta}^k$.

The protocol execution between two probabilistic algorithms $A$ and $B$ is denoted by $\langle A, B \rangle$.
The output of $A$ in such a protocol execution is denoted by $\langle A, B \rangle_A$ and of $B$ by $\langle A, B \rangle_B$.
Finally, let $\langle A, B \rangle_{\mathit{trans}}$ denote the transcript of communication between $A$ and $B$.

We define a \textit{two phase circuit} $C := (C_1, C_2)$ as a circuit where in the first phase a circuit $C_1$
is executed and in the second phase a circuit $C_2$.

We say that an event happens \textit{almost surely} or with \textit{high probability} if
it occurs with probability at least $1 - 2^{-n} \mathit{poly}(n)$.

\begin{definition}[Dynamic weakly verifiable puzzle.]
  \label{def:dwvp}
  A dynamic weakly verifiable puzzle (DWVP) is defined by a family of probabilistic circuits $\{P_n\}$.
  A circuit belonging to $\{P_n\}$ is called a problem poser.
  A problem solver $C := (C_1, C_2)$ for $P_n$ is a probabilistic two phase circuit.
  We write $P_n(\pi)$ to denote the execution of $P_n$ with the randomness fixed to $\pi \in \{0,1\}^n$, and $(C_1,C_2)(\rho)$
  to denote the execution of both $C_1$ and $C_2$ with the randomness fixed to $\rho \in \{0,1\}^{*}$.

  In the first phase, the poser $P_n(\pi)$ and the solver $C_1(\rho)$ interact.
  As the result of the interaction $P_n(\pi)$ outputs a verification circuit $\Gamma_{V}$ and a hint circuit $\Gamma_{H}$.
  The circuit $C_1(\rho)$ produces no output.
  The circuit $\Gamma_{V}$ takes as input $q \in Q$, an answer $y \in \{0,1\}^*$,
  and outputs a bit. We say that an answer $(q,y)$ is a correct solution if and only if $\Gamma_V(q,y) = 1$.
  The circuit $\Gamma_H$ on input $q \in Q$ outputs a hint such that $\Gamma_V(q,\Gamma_H(q)) = 1$.

  In the second phase, $C_2$ takes as input $x := \langle P_n(\pi), C_1(\rho) \rangle_{\mathit{trans}}$,
  and has oracle access to $\Gamma_V$ and $\Gamma_H$.
  The execution of $C_2$ with the input $x$ and the randomness fixed to $\rho$
  is denoted by $C_2(x, \rho)$. The queries of $C_2$ to $\Gamma_V$ and $\Gamma_H$ are called verification queries and hint queries respectively.
  The circuit $C_2$ succeeds if and only if it makes a verification query $(q,y)$ such that $\Gamma_V(q,y) = 1$,
  and it has not previously asked for a hint query on $q$.
\end{definition}
%
\begin{definition}[$k$-wise direct-product of DWVPs.]
  Let $g: \{0,1\}^{k} \rightarrow \{0,1\}$ be a monotone function and $P_n^{(1)}$ a problem poser as in Definition \ref{def:dwvp}.
  The $k$-wise direct product of $P_n^{(1)}$ is a DWVP defined by a circuit $P_{kn}^{(g)}$.
  We write $P_{kn}^{(g)}(\pi^{(k)})$ to denote the execution of $P_{kn}^{(g)}$ with the randomness fixed to $\pi^{(k)} := (\pi_1, \dots, \pi_k)$
  where for each $1 \leq i \leq n : \pi_i \in \{0,1\}^n.$
  Let $(C_1, C_2)(\rho)$ be a solver for $P_{nk}^{(g)}$ as in Definition \ref{def:dwvp}.
  In the first phase, the algorithm $C_1(\rho)$ sequentially interacts in $k$ rounds with $P_{kn}^{(g)}(\pi^{(k)})$.
  In the $i$-th round $C_1(\rho)$ interacts with $P_n^{(1)}(\pi_i)$,
  and as the result $P_{n}^{(1)}(\pi_i)$ generates circuits $\Gamma_V^i, \Gamma_H^i$.
  Finally, after $k$ rounds $P_{kn}^{(g)}(\pi^{(k)})$ outputs a verification circuit
\begin{align*}
  \Gamma_V^{(g)} (q, y_1, \dots, y_k) := g(\Gamma_V^{1}(q, y_1), \dots, \Gamma_V^{k}(q, y_k))
\end{align*}
and a hint circuit
\begin{align*}
  \Gamma_H^{(k)} (q) := (\Gamma_H^{1}(q), \dots, \Gamma_H^{k}(q)).
\end{align*}
\end{definition}
%
If it is clear from a context, we omit the subscript $n$, and write $P(\pi)$ instead of $P_n(\pi)$ where $\pi \in \{0,1\}^{n}$.

A verification query $(q,y)$ of a solver $C$ for which a hint query on this $q$ has been asked before can not be a verification query that succeeds.
Therefore, without loss of generality, we make the assumption that $C$ does not ask verification queries on $q$
for which a hint query has been asked before. Furthermore, we assume that once $C$ asked a verification query that succeeds,
it does not ask any further hint or verification queries.

%
\begin{codeblock}
  \textbf{Experiment} $\Success^{P, C}(\pi, \rho)$
  \medskip \hrule \medskip
  \textbf{Oracle:} A problem poser $P$, a solver $C = (C_1, C_2)$ for $P$.\\
  \textbf{Input:}  Bitstrings $\pi \in \{0,1\}^n$, $\rho \in \{0,1\}^*$.\\
  \textbf{Output:} A bit $b \in \{0,1\}$.
  \medskip\hrule\medskip
  \Run $\langle P(\pi), C_1(\rho) \rangle$ \\
  \IndI $(\Gamma_V, \Gamma_H) := \langle P(\pi), C_1(\rho) \rangle_{P}$ \\
  \IndI $x := \langle P(\pi), C_1(\rho) \rangle_{\mathit{trans}}$ \\ \\
  \Run $C_2^{\Gamma_V,\Gamma_H}(x, \rho)$ \\
  \IndI \If $C_2^{\Gamma_V, \Gamma_H}(x, \rho)$ asks a verification query $(q, y)$ such that $\Gamma_V(q, y) = 1$ \Then \\
  \IndII \Return $1$ \\
  \Return $0$ \\
\end{codeblock}
%
We define the \textit{success probability} of $C$ in solving a puzzle defined by $P$ as
\begin{align}
 \underset{\pi, \rho}{\Pr}[\Success^{P,C}(\pi, \rho) = 1].
\end{align}
Furthermore, we say that $C$ succeeds for $\pi$, $\rho$ if $\Success^{P,C}(\pi, \rho) = 1$.
%
\begin{theorem}[Security amplification for dynamic weakly verifiable puzzles.]
\label{th:sec_amp_for_dwvp}
Let $P_{n}^{(1)}$ be a fixed problem poser as in Definition \ref{def:dwvp}
and $P_{kn}^{(g)}$ a problem poser for the $k$-wise direct product of $P_{n}^{(1)}$.
Furthermore, let $C$ be a problem solver for $P_{kn}^{(g)}$ asking at most $h$ hint queries and $v$ verification queries.
There exists a probabilistic algorithm Gen with oracle access to a solver circuit $C$,
a monotone function $g:\{0,1\}^k \rightarrow \{0,1\}$ and problem posers $P_{n}^{(1)}$, $P_{kn}^{(g)}$.
Additionally, $\mathit{Gen}$ takes as input parameters $\varepsilon$, $\delta$, $n$, $k$, $h$, $v$ and outputs a solver circuit $D$ for $P_{n}^{(1)}$
such that the following holds: \\
If $C$ is such that
  \begin{align*}
    \underset{\substack{\pi^{(k)} \in \{0,1\}^{kn} \\ \rho \in \{0,1\}^{*}}}{\Pr}\left[\mathit{Success}^{P_{kn}^{(g)}, C}(\pi^{(k)}, \rho) = 1\right]
    \geq 16(h+v)\left(\underset{u \leftarrow \mu_\delta^k}{\Pr}\left[g(u) = 1\right] + \varepsilon\right)
  \end{align*}
then $D$ satisfies almost surely
  \begin{align*}
    \underset{\substack{\pi \in \{0,1\}^{n} \\ \rho \in \{0,1\}^{*}}}
    {\Pr}\left[\mathit{Success}^{P_{n}^{(1)},D}(\pi, \rho) = 1\right] \geq (\delta + \frac{\varepsilon}{6k}).
  \end{align*}
Additionally, $D$ requires oracle access to $g$, $P_{n}^{(1)}$, $C$,
and asks at most $\frac{6k}{\epsilon}\log\left(\frac{6k}{\epsilon}\right) h$ hint queries and one verification query.
Finally, $\mathit{Size}(D) \leq \mathit{Size}(C) \cdot \frac{6k}{\varepsilon}$ and $\mathit{Time}(\mathit{Gen}) = \mathit{poly}(k, \frac{1}{\varepsilon}, n, v, h)$.
\end{theorem}
%
% The Theorem \ref{th:sec_amp_for_dwvp} implies that if there is no good solver for a puzzle defined by $P^{(1)}$, then a good solver for
% a $k$-wise direct product of $P^{(1)}$ does not exist.

% The idea of the algorithm $Gen$ is to output a circuit $D$ that solves the input puzzle often.
% We know that $C$ has good success probability for a $k$-wise product of $P^{(1)}$.
% The algorithm $Gen$ tries to find a puzzle such that when $C$ runs with this puzzle fixed
% on the first position, and disregards whether this puzzle is correctly solved
% then the assumptions of Theorem \ref{th:sec_amp_for_dwvp} are true for a $k-1$-wise direct product.
% If it is possible to find such a puzzle then $Gen$ could recurse and solve a smaller problem.
% In the optimistic case we can reach $k=1$, which means that we found a good circuit for solving a single
% puzzle by just fixing the initial puzzles of $C$.

% Otherwise, when the first position is disregarded then the success probability of $C$ is not substantially better.
% This is remarkable, as we know that $C$ performs good for $k$-wise product, it means that the first position is important,
% in the sense that $C$ solves the puzzle on that position unusually often.
% Therefore, it is reasonable to construct the circuit $D$ using $C$ by placing the input puzzle of $D$ on that position, and then
% finding remaining $k-1$ puzzles. These $k-1$ remaining puzzles are generated by the circuit $D$, hence it is possible to check
% whether they are correctly solved by the circuit $C$. We know that circuit $C$ has good success probability, and the puzzle on the first
% position is important. Therefore, if we are able to find a $k-1$ puzzles such that the fact whether the $k$-wise direct product is correctly
% solved depends on whether the puzzle on the first position is correctly solved then we can assume that $C$ is often correct on this first position.

% There are some problems with this approach, first we have to ensure that we can make a decision when the algorithm $Gen$ should recurse and when not
% correctly with high probability. Then, we have to show that it is possible to find a puzzles such that $C$ is often correct on the first position.
% Finally, we also have to be sure that we do not ask a hint query, on the final verification query to the oracle.
% To satisfy the last requirement we split $Q$.

%%% Local Variables:
%%% mode: latex
%%% TeX-master: "../master"
%%% End:


\begin{lemma}
\label{lemma:hash_function_probability}
\textbf{Success probability with respect to hash function.} \\
For a fixed $P^{(g)}$ let $C$ succeed in solving the $k$-wise direct product of DWVP produced by $P^{(g)}$
with probability $\gamma$ making $h$ hint and $v$ verification queries.
There exists a probabilistic algorithm, with oracle access to $C$, that runs in time $O((h+v)^4/\gamma^4)$
and with high probability outputs a function $hash: Q \rightarrow \{0, \dots, 2(h+v)-1\}$ such that success probability of
$C$ in random experiment $E$ with respect to the set $P_{hash}$ is at least $\frac{\gamma}{8(h+v)}$.
\end{lemma}
%
% Proof of existence of hash function with required properties
%
\begin{proof}
Let $\cH$ be a family of pairwise independent hash functions $Q \rightarrow \{0,1, \dots,2(h+v)-1\}$.
By a pairwise independence property of $\cH$ we know that for all $i \neq j \in \{1, \dots, (h+v)\}$ and $k,l \in \{0,1,\dots,2(h+v)-1\}$
we have the following
\begin{align}
  \label{eq:hash_pr}
 \forall q_i,q_j \in Q : \underset{\textit{hash} \leftarrow \cH}{\Pr}[hash(q_i) = k \mid hash(q_j) = l] = \underset{\textit{hash} \leftarrow \cH}{\Pr}[hash(q_i) = k] = \frac{1}{2(h+v)}.
\end{align}
For a fixed $P^{(g)}$ and $(\pi_1, \dots, \pi_k)$ in the random experiment $A$ we define a binary random variable $X$ for the event that $hash(q_j) = 0$, and for
every query $q_i$ asked before $q_j$ $hash(q_i) \neq 0$.
By definition of conditional probability
\begin{align*}
  \underset{\textit{hash} \leftarrow \cH}{\Pr}[X=1] &= \underset{\textit{hash} \leftarrow \cH}{\Pr}[hash(q_j) = 0 \land \forall i < j : hash(q_i) \neq 0] \\
  &=\underset{\textit{hash} \leftarrow \cH}{\Pr}[\forall i < j : hash(q_i) \neq 0 \mid hash(q_j) = 0] \underset{\textit{hash} \leftarrow \cH}{\Pr}[hash(q_j) = 0].
\end{align*}
Now we use (\ref{eq:hash_pr}) and obtain
\begin{align*}
\underset{\textit{hash} \leftarrow \cH}{\Pr}[X=1] =
\frac{1}{2(h+v)}\left(1 -\underset{\textit{hash} \leftarrow \cH}{\Pr}[\exists i < j : hash(q_i) = 0 \mid hash(q_j) = 0] \right)
\end{align*}
Using pairwise independence property we conclude
\begin{align*}
\underset{\textit{hash} \leftarrow \cH}{\Pr}[X=1] = \frac{1}{2(h+v)} \left( 1 -\underset{\textit{hash} \leftarrow \cH}{\Pr}[\exists i < j : hash(q_i) = 0] \right).
\end{align*}
Finally, we use union bound and the fact $j \leq (h+v)$ to get
\begin{align*}
\underset{\textit{hash} \leftarrow \cH}{\Pr}[X=1] \geq
\frac{1}{2(h+v)} \left( 1 - \sum_{i < j} \underset{\textit{hash} \leftarrow \cH}{\Pr}[hash(q_i) = 0] \right) \geq \frac{1}{4(h+v)}
\end{align*}
Let $G$ denote the set of all $(\pi_1, \dots, \pi_k)$ for which $C$ succeeds in the random experiment $A$.
Then
\begin{align*}
\underset{\substack{\textit{hash} \leftarrow \cH \\ (\pi_1, \dots, \pi_k)}}{\Pr}[X=1] &=
\sum_{(\pi_1, \dots, \pi_k) \in G} \underset{\textit{hash} \leftarrow \cH}{\Pr}[X=1 \mid (\pi_1, \dots, \pi_k)] \cdot \underset{(\widetilde{\pi}_1, \dots, \widetilde{\pi}_k)}{\Pr}[(\widetilde{\pi}_1, \dots, \widetilde{\pi}_k) = (\pi_1, \dots, \pi_k)]\\
&\geq \frac{1}{4(h+v)} \sum_{(\pi_1, \dots, \pi_k) \in G} \underset{(\widetilde{\pi}_1, \dots, \widetilde{\pi}_k)}{\Pr}[(\widetilde{\pi}_1, \dots, \widetilde{\pi}_k) = (\pi_1, \dots, \pi_k)] = \frac{\gamma}{4(h+v)}
\end{align*}

\begin{codeblock}
  \textbf{Algorithm: FindHash}

  \medskip

  \hrule

  \medskip

  %TODO define the circuit $C$ globally do not forget about limit on number of hint and verification queries
  \textbf{Oracle:} A solver circuit for $k$-wise direct product of DWVP $C^{(\cdot, \cdot)}$ with oracle access to hint and verification oracle.\\
  %TODO better describe this hash functions
  \textbf{Input:} $\cH$ a family of pairwise independent hash functions $Q \rightarrow \{0,1,\dots, 2(h+v)-1\}$
  \medskip\hrule\medskip
  For $i = 1$ to $16(h+v)^2/\gamma^2$ \\
  \IndI $hash \xleftarrow{\$} \cH$ \\
  \IndI $count := 0$ \\
  \IndI \For $j := 1$ to $16(h+v)^2/\gamma^2$ \\
  \IndII $(\pi_1, \dots, \pi_k) \xleftarrow{\$} \{0,1\}^{kl} $\\
  \IndII Run $A^{P^{(g)},C^{(\cdot,\cdot)}}(\pi_1, \dots, \pi_k)$\\
  \IndIII Let $(\widetilde{q},y^{(k)})$ be the first successful verification query. \\
  \IndIII Let $G$ be a set of all $q$ used in hint or verification queries asked before $(\widetilde{q},y^{(k)})$.\\
  \IndII \If $\Gamma_V^{(g)}(\widetilde{q},y^{(k)}) = 1 \land G \subseteq P_{hash}$\\
  \IndIII $count := count + 1$\\
  \IndI \If $count \geq 4(h+v)/\gamma$ \\
  \IndII \return $hash$\\
  \return $\bot$
\end{codeblock}
We show that the algorithm \textbf{FindHash} chooses a hash function such
that almost surly the success probability of $C$ in random experiment $E$
with respect to set $P_{hash}$ is at least $\frac{\gamma}{4(h+v)}$.
Let $\cH_{Good}$ denote the family of hash functions for which $\underset{(\pi_1, \dots, \pi_k)}{\Pr}[X] \geq \frac{\gamma}{4(h+v)}$
and $X_1, \dots, X_i$ be binary random variables such that for a fixed hash function
\begin{align*}
  X_i =
  \begin{cases}
    1 & \text{if in $i$th iteration variable $count$ is increased}\\
    0 & \text{otherwise .}
  \end{cases}
\end{align*}
We first show that it is unlikely that the algorithm \textbf{FindHash} returns $hash \notin \cH_{Good}$.
For $hash \notin \cH_{Good}$ we have $\mathbb{E}_{(\pi_1, \dots, \pi_k)}[X_i] < \frac{\gamma}{4(h+v)}$.
We use Chernoff inequality and obtain
%FIXME write down the definition of $X_i$ correctly
%FIXME write the number over which you sum correctly
\begin{align*}
  \underset{(\pi_1, \dots, \pi_k)}{\Pr} \left[\frac{1}{N} \sum_{i=1}^{N} X_i \geq (1 + \delta) \frac{\gamma}{4(h+v)} \right] \leq
  \underset{(\pi_1, \dots, \pi_k)}{\Pr}\left[\frac{1}{N} \sum_{i=1}^{N} X_i \geq (1 + \delta) \mathbb{E}[X_i]\right] \leq
  e^{-{\frac{\gamma}{4(h+v)}} N \delta^2 /3}
\end{align*}
%
The probability that $hash \in \cH_{Good}$ is not returned by the algorithm is
\begin{align*}
  \underset{(\pi_1, \dots, \pi_k)}{\Pr}[\frac{1}{N} \sum_{i=1}^{N} X_i \leq (1 - \delta) \frac{\gamma}{4(h+v)}] \leq
  \underset{(\pi_1, \dots, \pi_k)}{\Pr}[\frac{1}{N} \sum_{i=1}^{N} X_i \leq (1 - \delta) \mathbb{E}[X_i]] \leq e^{-{\frac{\gamma}{4(h+v)}} N \delta^2 /3}
\end{align*}
%
Finally, we show that almost surely \textbf{FindHash} picks in one of its iteration a hash function that is in $\cH_{Good}$.
From the fact that the random variable $X$ is binary distributed we have
\begin{align*}
  \underset{\substack{\textit{hash} \leftarrow \cH \\ (\pi_1, \dots, \pi_k)}}{\mathbb{E}}[X] \geq \frac{\gamma}{4(h+v)}
\end{align*}
Let $Y_i$ be a binary random variable
\begin{align*}
  Y_i =
  \begin{cases}
    1 & \text{in $i$th iteration $hash \in \cH_{Good}$ is picked} \\
    0 & \text{otherwise .}
  \end{cases}
\end{align*}
We make use of the fact that if a function from $\cH_{Good}$ is picked, then it is returned almost surely. Therefore,
$\mathbb{E}[Y_i] \geq \frac{\gamma}{4(h+v)}$ and we can use Chernoff bound to obtain
\begin{align*}
  \underset{hash \leftarrow \cH}{\Pr}\left[\frac{1}{K} \sum_{i=1}^{K} Y_i = 0\right] &\leq
  \underset{hash \leftarrow \cH}{\Pr}\left[\frac{1}{K} \sum_{i=1}^{K} Y_i \leq (1-\delta) \frac{\gamma}{4(h+v)}\right] \\
  &\leq \underset{hash \leftarrow \cH}{\Pr}\left[\frac{1}{K} \sum_{i=1}^{K} Y_i \leq (1-\delta) \mathbb{E}[Y_i] \right] \leq e^{-\delta^2K \mathbb{E}[Y_i]/2 }
\end{align*}
We see that the bound stated in the lemma \ref{lemma:hash_function_probability} is achieved for valid for $\delta = \frac{1}{2}$ and $K = N = 16(h+v)^2/\gamma^2$
\end{proof}
%%% Local Variables: 
%%% mode: latex
%%% TeX-master: "../master"
%%% End: 

\begin{codeblock}
  \textbf{Experiment $E^{P^{(g)}, C^{(.)(.)}, hash}(\pi_1, \dots, \pi_k)$} \\
  Solving $k$-wise direct product of DWVP with respect to the set $P_{hash}$
  \medskip

  \hrule

  \medskip
  \textbf{Oracle:} Problem poser for k-wise direct product $P^{(g)}$ \\
  \IndI A solver circuit for $k$-wise direct product $C^{(\cdot, \cdot)}$ \\
  \IndI A function $hash: Q \leftarrow \{0, \dots, 2(h+v) - 1\}$\\
  \textbf{Input:} Random bitstring $(\pi_1, \dots, \pi_k) \in \{0,1\}^{kl}$\\

  \medskip\hrule\medskip

  $\pi^{(k)} := \left(\pi_1, \dots, \pi_k \right)$\\
  $(x^{k}, \Gamma_V^{(g)}, \Gamma_H^{(k)}) := P^{(g)}(\pi^{k})$\\
  Run $C^{\Gamma_V^{(g)}, \Gamma_H^{(k)}} (x^{(k)})$ \\
  \IndI Let $(q_j,y_j^{(k)})$ be the first successful verification query if $C^{\Gamma_V^{(g)}, \Gamma_H^{(k)}}$ succeeds or \\
  \IndI an arbitrary verification query when it fails.\\
  \textbf{If} $(\forall i < j :  q_i \notin P_{hash} )$ and $q_j \in P_{hash}$ and $\Gamma_V^{(g)}(q_j, y_j^{(k)}) = 1$ \\
  \IndI \textbf{return} 1\\
  \textbf{else}\\
  \IndI \textbf{return} 0\\
\end{codeblock}
%
A canonical success is a situation when a solver $C$ for fixed $hash$ and $P^{(1)}$ succeeds in a random experiment $E$.
%
\begin{codeblock}
  \textbf{Random experiment $F^{P^{(1)}, D, hash}(\pi)$} \\
  Solving a single DWVP with respect to the set $P_{hash}$
  \medskip

  \hrule

  \medskip

  \textbf{Oracle:}
  A dynamic weakly verifiable puzzle $P^{(1)}$ \\
  \IndI A solver circuit for a single DWVP $D$ \\
  \IndI A function $hash: Q \rightarrow \{0,1,\dots, 2(h+v)-1\}$ \\
  \textbf{Input:} Random bitstring $\pi \in \{0,1\}^l$
  %TODO length of the bitstring maybe it is fixed as it is input to P^{(1)} ? like l see the end of the paper by Imaginazzo.
  \medskip\hrule\medskip

  $(x, \Gamma_v, \Gamma_H) := P^{(1)}(\pi)$ \\
  Run $D^{\Gamma_V, \Gamma_H}(x)$ \\
  \IndI Let $(\widetilde{q_j},\widetilde{r_j})$ be the first successful verification query if $D^{\Gamma_V, \Gamma_H}(x)$ succeeds or \\
  \IndI an arbitrary verification query when it fails. \\
  \If $(\forall i < j :  q_i \notin  P_{hash} )$ and $q_j \in P_{hash}$ and $\Gamma_V(q_j) = 1$ \then \\
  \IndI \return 1 \\
  \textbf{else}\\
  \IndI \return 0\\

\end{codeblock}
%
%
\begin{lemma}
  \label{lemma:sec_amp_for_p_hash}
  \textbf{Security amplification of a dynamic weakly verifiable puzzle with respect to set $P_{hash}$.} \\
  For a fixed dynamic weakly verifiable puzzle $P^{(1)}$ there exists an algorithm\\
  $Gen(C, g, \varepsilon, \delta, n, v, h, hash)$, which takes as input a circuit $C$, a monotone function $g$,
  a function $hash : Q \rightarrow \{0, \dots, 2(h+v)-1\}$, parameters $\varepsilon, \delta, n$,
  number of verification $v$, and hint $h$ queries asked by $C$, and outputs a circuit $D$
  such that following holds: \\
  If $C$ is such that \\
  \begin{align*}
    \underset{(\pi_1, \dots, \pi_k)}{\Pr}[E^{P^{(g)}, C, Hash}(\pi_1, \dots, \pi_k)=1] \geq \underset{\mu \leftarrow \mu_\delta^k}{\Pr}[g(\mu) = 1] + \varepsilon
  \end{align*}
  then $D$ satisfies almost surely
  \begin{align*}
    \underset{\pi}{\Pr}[F^{P^{(1)},D, Hash}(\pi) = 1] \geq (\delta + \frac{\varepsilon}{6k})
  \end{align*}
  and $Size(D) \leq Size(C)\frac{6k}{\varepsilon}$ and $Time(Gen) = poly(k, \frac{1}{\varepsilon}, n, v, h)$.
\end{lemma}

\begin{todo}
  \textbf{TODO:} The circuit should return the solutions to puzzles. Then we just need to call circuit $ \Gamma_v $ to eval. it.
  But there should be an assumption that the circuit always returns a tuple in $P_{hash}$ and does not ask hint or verification queries
  on this tuple.
\end{todo}

\begin{codeblock}
  \textbf{Circuit $\widetilde{C}^{\Gamma_V^{(g)}, \Gamma_H^{(g)}, hash, C} (x_1, \dots, x_k)$} \\
  Circuit $\widetilde{C}$ has good canonical success probability.
  \medskip

  \hrule

  \medskip

  \textbf{Oracle:} $\Gamma_V^{(g)}, \Gamma_H^{(g)}, hash, C$ \\
  \textbf{Input:} k-wise direct product of puzzles $(x_1, \dots, x_k)$

  \medskip\hrule\medskip
  Run $C^{(\cdot,\cdot)}(x_1, \dots, x_k)$ \\
  \IndI \If $C$ asks a hint query $q$ \then\\
  \IndII \If $q \in P_{hash}$ \then\\
  \IndIII \return $\bot$\\
  \IndII \textbf{else}\\
  \IndIII answer the hint query with $\Gamma_H^{(k)}(q)$\\
  \\
  \IndI \textbf{If} $C$ asks a verification query $(q, y_1, \dots, y_k)$ \textbf{then} \\
  \IndII \textbf{If} $q \in P_{hash}$ \textbf{then} \\
  \IndIII \text{ask the verification query} $(q, y_1, \dots, y_k)$ \\
  \IndIII \textbf{stop the execution} \\
  \IndII \textbf{else} \\
  \IndIII answer verification query with 0 \\
  \textbf{return} $\bot$
\end{codeblock}
The key difference between circuits $C$ and $\widetilde{C}$
is that if $\widetilde{C}$ asks a verification query $(q, y_1, \dots, y_k)$ then $q \in P_{hash}$.
This means that if $\widetilde{C}$ succeeds then it also succeeds canonically.

\begin{lemma}
  For fixed $P^{(g)}$ it is true that
  \begin{align*}
  \underset{(\pi_1, \dots, \pi_k)}{\Pr}[E^{P^{(g)}, C, Hash}(\pi_1, \dots, \pi_k) = 1] \leq \underset{(\pi_1, \dots, \pi_k)}{\Pr}[\Gamma_V^{(g)} (\widetilde{C}^{\Gamma_V^{(g)}, \Gamma_H^{(g)}, Hash}(\pi_1, \dots, \pi_k)) = 1].
  \end{align*}
\end{lemma}

\begin{proof}
We fix the randomness $(\pi_1, \dots, \pi_k)$ used in the random experiment $E$.
Let $x^{(k)} = (x_1, \dots, x_k)$ be a set of puzzles generated in the random experiment $E$ for the randomness $(\pi_1, \dots, \pi_k)$.
If $C$ succeeds canonically for the set of puzzles $x^{(k)}$, then also circuit $\widetilde{C}$ that runs $C$ on the same set of puzzles succeeds.
Using the definition of conditional expectation, we conclude that
\begin{align*}
  \underset{}{\Pr}[E^{P^{(g)}, C, hash}(\pi^{(k)}) = 1] &=
  \sum_{\pi^{(k)} \in \{0,1\}^{kl}}\underset{}{\Pr}[E^{P^{(g)}, C, hash}(\widetilde{\pi}^{(k)}) = 1 | \pi^{(k)} = \widetilde{\pi}^{(k)}] \underset{}{\Pr}[\pi^{(k)} = \widetilde{\pi}^{(k)}] \\
  &\leq
  \sum_{\pi^{(k)} \in \{0,1\}^{kl}}\underset{}{\Pr}[E^{P^{(g)}, \widetilde{C}, hash}(\widetilde{\pi}^{(k)}) = 1 | \pi^{(k)} = \widetilde{\pi}^{(k)}] \underset{}{\Pr}[\pi^{(k)} = \widetilde{\pi}^{(k)}] \\
  &= \underset{}{\Pr}[E^{P^{(g)}, \widetilde{C}, hash}(\pi^{(k)}) = 1]
\end{align*}

\end{proof}

\begin{codeblock}
  \textbf{Algorithm $Gen(\widetilde{C},g,\epsilon,\delta,n)$}
  \medskip

  \hrule

  \medskip

  \textbf{Oracle:} $\widetilde{C}, g$ \\
  \textbf{Input:}  $\epsilon, \delta, n$\\
  \textbf{Output:} A circuit $D$

  \medskip\hrule\medskip
  \If \text{the number of puzzles to solve equals one} \then \\
  \IndI \return $\widetilde{C}$ \\ \\
  \textbf{For} $i:=1$ to $\frac{6k}{\epsilon}\log(n)$ \\
  \IndI $\pi^* \leftarrow \{0,1\}^{l}$\\
  \IndI $\widetilde{S}_{\pi^*,0} := EvaluateSurplus(\pi^*, 0)$\\
  \IndI $\widetilde{S}_{\pi^*,1} := EvaluateSurplus(\pi^*, 1)$\\
  \IndI \textbf{If} $\widetilde{S}_{\pi^*,0} \geq (1 - \frac{3}{4k}) \epsilon$ or $\widetilde{S}_{\pi^*,1} \geq (1 - \frac{3}{4k}) \epsilon$ \\
  \IndII $\widetilde{C}' := \widetilde{C}$ with the first input fixed on $\pi^*$\\
  \IndII\textbf{return} $Gen(\widetilde{C}', g, \epsilon, \delta, n)$ \\
  // all estimates are lower than $(1-\frac{3}{4k})\varepsilon$\\
  \return $D^{\widetilde{C}}$ \\
  \\
  \textbf{EvaluateSurplus}($\pi^*, b$) \\
  \IndI \textbf{For} $i:=1$ to $N_k$ \\
  \IndII $(\pi_2, \dots, \pi_k) \xleftarrow{\$} \{0,1\}^{(k-1)l}$\\
  \IndII $(c_1, \dots, c_k) := EvalutePuzzles(\pi^*, \pi_2, \dots, \pi_k)$\\
  \IndII $\widetilde{S}_{\pi^*,b}^i := g(b, c_2, \dots, c_k) - \underset{(u_2, \dots, u_k)}{\Pr}[g(b, u_2, \dots, u_k) = 1] $\\
  \IndI \textbf{return} $\frac{1}{N_k} \sum_{i=1}^{N_k} \widetilde{S}_{\pi^*,b}^i$\\
  \\
  \textbf{EvalutePuzzles}($\pi^{(k)}$)\\
  \IndI $(x^{(k)}, \Gamma_V^{(g)}, \Gamma_H^{(k)}) := P^{(g)}(\pi^{(k)})$ \\
  \IndI \textbf{For} $i:=1$ to $k$\\
  \IndII $(x_i, \Gamma_V^{i}, \Gamma_H^{i}) := P^{(1)}(\pi_i)$\\
  \IndI $(q,y^{k}) := \widetilde{C}^{\Gamma_V^{(g)}, \Gamma_H^{(k)}}(x_1, x_2, \dots, x_k)$\\
  \IndI \textbf{For} $i:=1$ to $k$\\
  \IndII $c_i := \Gamma_v^{i}(q, y_i)$\\
  \IndI \textbf{return} $(c_1, \dots, c_k)$\\
\end{codeblock}
%
%
%
\begin{todo}
  \textbf{TODO:} Circuit $\widetilde{C}$ gets as input puzzle find a nice way to genereate the puzzles as it is used in many places in the code.
   Also make EvalutePuzzles more general maybe it should take $\widetilde{C}$ as input?
\end{todo}
\begin{codeblock}
  \textbf{Circuit $D^{\widetilde{C}}$}
  \medskip

  \hrule

  \medskip

  \textbf{Oracle:}  $\widetilde{C}, P^{(1)}$\\
  \textbf{Input:}  puzzle $x^*$, a random bitstring $r \in \{0,1\}^{*}$

  \medskip\hrule\medskip
  \textbf{For} $i:=1$ to $\frac{6k}{\epsilon} \log(\frac{6k}{\epsilon})$\\
  \IndI $\pi^{(k)} \leftarrow \{0,1\}^{kl}$ //read $k \cdot l$ bits from $r$  \\
  \IndI $(c_1, \dots, c_k) := EvaluatePuzzles(\pi^{(k)})$\\
  \IndI \textbf{If} $g(1,c_2, \dots, c_k) = 1$ and $g(0,c_2, \dots, c_k) = 0$\\
  \IndII $(q, y_1, \dots, y_k) := \widetilde{C}(x^*, x_2, \dots, x_k)$\\
  \IndII \textbf{return} $y_1$\\
  \textbf{return} $\bot$ \\

\end{codeblock}
%
% The algorithm $Gen$ recursively builds the circuit that have high success probability in solving a dynamic weakly verifiable puzzle.
% When algorithm recurses it fixes a puzzle on the first position on the input of circuit $\widetilde{C}$ which yields a new circuit $\widetilde{C}'$.
% It happens only in the situation when for some fixed $\pi^*$ circuit $\widetilde{C}$ performs good on the remaining $k-1$ puzzles.
%
For $k=1$ function $g(b)$ is either identity or a constant function.
If $g$ is identity then the success probability of $\widetilde{C}$ is as least $\delta + \epsilon$
and $\widetilde{C}$ can be directly used to solve a puzzle. If the function $g$ is constant the statement is vacuously true.

Let $(q, y_1, \dots, y_k)$ denote the output of $\widetilde{C}$.
Additionally, let us denote by $c_i = \Gamma_V(q, y_i)$ whether $(q,y_i)$ is a correct solution for a single puzzle.
We define surplus as the following quantity:
\begin{align}
  \label{eq:s_pi_b}
S_{\pi^*, b} = \underset{\pi^{(k)}}{\Pr}[g(b, c_2, \dots, c_k) = 1] - \underset{\mu^{(k)}}{\Pr}[g(b, u_2, \dots, u_k) = 1]
\end{align}
The surplus $S_{\pi^*, b}$ tells us how good the algorithm $\widetilde{C}$ performs when the first puzzle is fixed, and value of $c_1$ is neglected.
The procedure \textbf{EvaluateSurplus} returns the estimate for $\widetilde{S}_{\pi^*, b}$.
All puzzles used during obtaining the estimate are generated by $\textbf{EvaluatePuzzles}$.
Therefore, it is possible to provide answers for all hint and verification queries.
The returned estimate $\widetilde{S}_{\pi^*,b}$ that differs from $S_{\pi*, b}$
by at most $\frac{\epsilon}{4k}$ almost surely.
Therefore, if $\widetilde{S}_{\pi^*,b} \geq (1-\frac{3}{4k})\epsilon$ then
with high probability $S_{\pi*,b} \geq (1-\frac{1}{k})\epsilon$.
In this case we use a new monotone binary function $g'(b_2, \dots, b_k) := g(b, b_2, \dots, b_k)$, and fix the first puzzle of $\widetilde{C}$ for the one generated
by using the randomness $\pi^*$. The new circuit satisfies the conditions of Lemma \ref{lemma:sec_amp_for_p_hash} which means that we can
use algorithm $Gen$ for the new circuit $\widetilde{C}$ and monotone function $g'$.

If all estimates are less than $(1-\frac{1}{4k})\epsilon$, then intuitively $\widetilde{C}$
does not perform much better on the remaining $k-1$ puzzles than an algorithm that solves each puzzle independent with probability $\delta$.
However, from the assumption we know that on all $k$ puzzles $\widetilde{C}$ has high success probability.
It means that in this case the first puzzle has to be correctly solved with substantial probability.

\begin{todo}
  \textbf{TODO:} Explain the intuition why it may happen that we still can fail in the case of circuit $\widetilde{D}$.
\end{todo}

We have to show that the success probability when $Gen$ does not recurse is substantial.
We fix a randomness $\pi^*$ and thus also a puzzle $x^*$. For this fixed puzzle using (\ref{eq:s_pi_b}) we get
\begin{align}
\label{eq:diff_s01}
  &\underset{\mu_{\delta}^k}{\Pr}[g(1, \mu_2, \dots, \mu_k)=1] - \underset{\mu_{\delta}^k}{\Pr}[g(0, \mu_2, \dots, \mu_k)=1] = \notag\\
&\IndI  \underset{\pi^{(k)}}{\Pr}[g(1, c_2, \dots, c_k) =1 \mid \pi_1 = \pi^*] - \underset{\pi^k}{\Pr}[g(0, c_2, \dots, c_k) = 1 \mid \pi_1 = \pi^*] - (S_{\pi^*,1} - S_{\pi^*,0})
\end{align}
\begin{todo}
  \textbf{TODO:} Better explain why we can write $\Pr(g() = 1 \land g() = 0)$ as the equivalence for the difference.
\end{todo}
From the monotonicity of $g$ we know that for any set of tuples $(b_1, \dots, b_k)$
and sets $G_0 = \{ (b_1, b_2, \dots, b_k): g(0, b_2, \dots, b_k) = 1\}$, $ G_1 = \{(b_1, b_2, \dots, b_k) : g(1, b_2, \dots, b_k) = 1 \}$
we have $G_0 \subseteq G_1$. Hence, we can write (\ref{eq:diff_s01}):
\begin{align}
  \label{eq:diff_s01_next}
  &\underset{\mu_{\delta}^k}{\Pr}[g(1, \mu_2, \dots, \mu_k) = 1 \land g(0, \mu_2, \dots, \mu_k) = 0] = \notag\\
&\IndI  \underset{\pi^{(k)}}{\Pr}[g(1, c_2, \dots, c_k) = 1 \land g(0, c_2, \dots, c_k) = 0 \mid \pi_1 = \pi^*] - (S_{\pi^*,1} - S_{\pi^*,0}).
\end{align}
Let $G_{\mu^{(k)}}$ denote the event $g(1, \mu_2, \dots, \mu_k) = 1 \land g(0, \mu_2, \dots, \mu_k) = 0$, and correspondingly
$G_{\pi^{(k)}} := g(1, \pi_2, \dots, \pi_k) = 1 \land g(0, \pi_2, \dots, \pi_k) = 0$.
Then multiplying and dividing $\underset{}{\Pr}[\Gamma_v^{(g)}(D(x^*, \pi^{(k)})) = 1 \mid \pi_1 = \pi^*]$ by (\ref{eq:diff_s01_next}) we get
\begin{align}
\label{eq:pr_d_succ_0}
  \underset{r}{\Pr}[\Gamma_V^{(g)}(D(x^*, r)) = 1 \mid \pi_1 = \pi^*] &=
  \frac{\underset{r}{\Pr}[\Gamma_V^{(g)}(D(x^*, r)) = 1 \mid \pi_1 = \pi^*] \underset{\pi^{(k)}}{\Pr}[G_{\pi} \mid \pi_1 = \pi^*]} {\underset{\mu_{\delta}^{k}}{\Pr}[G_{\mu}]} \notag\\
  & \IndI - \frac{\underset{r}{\Pr}[\Gamma_V^{(g)}(D(x^*, r)) = 1 \mid \pi_1 = \pi^*](S_{\pi^*,1} - S_{\pi^*,0})}{\underset{\mu_{\delta}^{k}}{\Pr}[G_{\mu}]}
\end{align}
%
% \begin{todo}
%   \textbf{TODO:} Define $c_1, \dots, c_k$ correctly from the paper it is not known whether it is in the iteration or the final one
% \end{todo}
If output of circuit $D(x^*,r) \neq \bot$ then we denote $c_i := \Gamma_V^{i}(q, y_i)$.
We can write the first summand of (\ref{eq:pr_d_succ_0}) as
\begin{align}
  &\underset{r}{\Pr}[\Gamma_V^{(g)}(D(x^*,r)) = 1 \mid \pi_1 = \pi^*] \underset{\pi^{(k)}}{\Pr}[G_{\pi} \mid \pi_1 = \pi^*] = \notag\\
  &\IndI \underset{r}{\Pr}[D(x^*,r) \neq \bot \mid \pi_1 = \pi^*]
  \underset{\pi^{(k)}}{\Pr}[c_1 = 1 \mid G_{\pi}, \pi_1 = \pi^*]
  \underset{\pi^{(k)}}{\Pr}[G_{\pi} \mid \pi_1 = \pi^*]
\end{align}
where we make use of the fact that the event $G_{\pi}$ implies $D(x^*, r) \neq \bot$.
We consider two cases.
If $\underset{\pi^{k}}{\Pr}[g(1, c_2, \dots, c_k) = 1 \land g(0, c_2, \dots,c_k ) = 0 \mid \pi_1 = \pi^*] \leq \frac{\epsilon}{6k}$ then also
\begin{align}
  \underset{\pi^{(k)}}{\Pr}[c_1 = 1 \mid G_{\pi}, \pi_1 = \pi^*] \underset{\pi^{(k)}}{\Pr}[G_{\pi} \mid \pi_1 = \pi^*] \leq \frac{\epsilon}{6k}
\end{align}
and in the case when $\underset{\pi^{k}}{\Pr}[g(1, c_2, \dots, c_k) = 1 \land g(0, c_2, \dots,c_k ) = 0] > \frac{\epsilon}{6k}$ then circuit $D$ outputs $\bot$
only if it fails in all $\frac{6k}{\epsilon} \log(\frac{6k}{\epsilon})$ iterations to find $\pi^{(k)}$ such that $g(1, c_2, \dots, c_k) = 1 \land g(0, c_2, \dots, c_k) = 0$
which happens with probability
\begin{align}
\underset{r}{\Pr}[D(x^*,r) = \bot \mid \pi_1 = \pi^*] \leq (1 - \frac{\epsilon}{6k})^{\frac{6k}{\epsilon}\log(\frac{\epsilon}{6k})} \leq \frac{\epsilon}{6k}.
\end{align}
We conclude that in both cases we have
\begin{align}
  &\underset{r}{\Pr}[D(x^*,r) \neq \bot \mid \pi_1 = \pi^*]
  \underset{\pi^{(k)}}{\Pr}[c_1 = 1 \mid G_{\pi}, \pi_1 = \pi^*]
  \underset{\pi^{(k)}}{\Pr}[G_{\pi} \mid \pi_1 = \pi^*] \notag\\
  &\IndII \geq \underset{\pi^{(k)}}{\Pr}[c_1 = 1 \mid G_{\pi}, \pi_1 = \pi^*]\underset{\pi^{(k)}}{\Pr}[G_{\pi} \mid \pi_1 = \pi^*] - \frac{\epsilon}{6k}
\end{align}
Using definition of conditional probability we get
\begin{align*}
  &\underset{r}{\Pr}[D(x^*,r) \neq \bot \mid \pi_1 = \pi^*]
  \underset{\pi^{(k)}}{\Pr}[c_1 = 1 \mid G_{\pi}, \pi_1 = \pi^*]
  \underset{\pi^{(k)}}{\Pr}[G_{\pi} \mid \pi_1 = \pi^*] \notag\\
  &\IndII = \underset{\pi^{(k)}}{\Pr}[c_1 = 1 \land g(1, c_2,\dots, c_k) = 1 \land g(0, c_2, \dots, c_k) = 0 \mid \pi_1 = \pi^*] - \frac{\epsilon}{6k} \\
  &\IndII = \underset{\pi^{(k)}}{\Pr}[g(c_1, c_2,\dots, c_k) = 1 \land g(0, c_2, \dots, c_k) = 0 \mid \pi_1 = \pi^*] - \frac{\epsilon}{6k} \\
  &\IndII = \underset{\pi^{(k)}}{\Pr}[g(c_1, c_2,\dots, c_k) = 1 \mid \pi_1 = \pi^*] -  \underset{\pi^{(k)}}{\Pr}[g(0, c_2, \dots, c_k) = 0 \mid \pi_1 = \pi^*] - \frac{\epsilon}{6k} \\
\end{align*}
and finally by (\ref{eq:s_pi_b})
\begin{align}
  &\underset{r}{\Pr}[D(x^*,r) \neq \bot \mid \pi_1 = \pi^*]
  \underset{\pi^{(k)}}{\Pr}[c_1 = 1 \mid G_{\pi}, \pi_1 = \pi^*]
  \underset{\pi^{(k)}}{\Pr}[G_{\pi} \mid \pi_1 = \pi^*] \notag\\
  &\IndII = \underset{\pi^{(k)}}{\Pr}[g(c_1, c_2,\dots, c_k) = 1 \mid \pi_1 = \pi^*] -  \underset{\mu_{\delta}^{(k)}}{\Pr}[g(0, \mu_2, \dots, \mu_k) = 0 \mid \pi_1 = \pi^*]  - S_{\pi^*,0} - \frac{\epsilon}{6k}.
\end{align}
We insert this result into equation (\ref{eq:pr_d_succ_0}) to get
\begin{align}
\label{eq:pr_d_succ_1}
  &\underset{r,\pi}{\Pr}[D(x,r) = 1] = \mathbb{E_{\pi}}[\underset{r}{\Pr}[D(x,r) = 1 \mid \pi_1 = \pi^*]] \notag\\
&\IndI = \mathbb{E}_{\pi}\left[\frac{{\Pr}_{\pi^{(k)}}[g(c) = 1 \mid \pi_1 = \pi^*] -
{\Pr}_{\mu_{\delta}^{(k)}}[g(0, \mu_2, \dots, \mu_k) = 0 \mid \pi_1 = \pi^*] - \frac{\epsilon}{6k}} {{\Pr}_{\mu_{\delta}^{k}}[G_{\mu}]}\right] \notag\\
&\IndII - \mathbb{E}_{\pi}\left[\frac{
  S_{\pi^*,0} + \Pr_r [\Gamma_V^{(g)}(D(x^*,r)) = 1 \mid \pi_1 = \pi^*](S_{\pi^*,1} - S_{\pi^*,0})}
{{\Pr}_{\mu_{\delta}^{k}}[G_{\mu}]}\right]
\end{align}
For the second summand we want to show first that almost all estimates all low if we do not recurse.
Let assume that
\begin{align}
\underset{\pi}{\Pr}\left[\left(S_{\pi,0} \leq (1 - \frac{1}{2k})\epsilon\right) \land \left( S_{\pi,1} \leq (1-\frac{1}{2k})\epsilon\right)\right] < 1 - \frac{\epsilon}{6k},
\end{align}
then the algorithm would recurse almost surely.
Therefore, under the assumption that we do not recurse, we have almost surely
\begin{align}
\underset{\pi}{\Pr}\left[\left(S_{\pi,0} \leq (1 - \frac{1}{2k})\epsilon\right) \land \left( S_{\pi,1} \leq (1-\frac{1}{2k})\epsilon\right)\right] \geq 1 - \frac{\epsilon}{6k}.
\end{align}
Let us define a set
\begin{align}
  \X = \left\{ \pi :  \left(S_{\pi,0} \leq (1 - \frac{1}{2k})\epsilon\right) \land \left( S_{\pi,1} \leq (1-\frac{1}{2k})\epsilon \right) \right\}
\end{align}
and the complement of this set $\X^c$.
We bound the second summand in (\ref{eq:pr_d_succ_1})
\begin{align}
&\mathbb{E}_{\pi}\left[ S_{\pi^*,0} + \Pr_r [\Gamma_V^{(g)}(D(x^*,r)) = 1 \mid \pi_1 = \pi^*](S_{\pi^*,1} - S_{\pi^*,0}) \right] \notag\\
&\IndII = \mathbb{E}_{\pi \in \X^c}\left[ S_{\pi^*,0} + \Pr_r [\Gamma_V^{(g)}(D(x^*,r)) = 1 \mid \pi = \pi^*](S_{\pi^*,1} - S_{\pi^*,0}) \right] \notag\\
&\IndIII +  \mathbb{E}_{\pi \in \X}\left[ S_{\pi^*,0} + \Pr_r [\Gamma_V^{(g)}(D(x^*,r)) = 1 \mid \pi = \pi^*](S_{\pi^*,1} - S_{\pi^*,0}) \right] \\
&\IndII \leq \frac{\epsilon}{6k} + \mathbb{E}_{\pi \in \X^c}\left[ S_{\pi^*,0} + \Pr_r [\Gamma_V^{(g)}(D(x^*,r)) = 1 \mid \pi = \pi^*]((1 - \frac{1}{2k})\epsilon - S_{\pi^*,0}) \right] \\
&\IndII \leq \frac{\epsilon}{6k} + 1 - \frac{\epsilon}{2k} = 1 - \frac{\epsilon}{3k}
\end{align}
Finally, we insert this result into equation (\ref{eq:pr_d_succ_1}) and make use of the fact
\begin{align*}
\underset{}{\Pr}[g(u) = 1] &= \underset{}{\Pr}[(g(0, \mu_2, \dots, \mu_k) = 1) \lor ( g(1,\mu_2, \dots, \mu_k) = 1 \land g(0, \mu_2, \dots, \mu_k) = 0 \land \mu_1 = 1)] \notag\\
&= \Pr[g(0,\mu_2, \dots,\mu_k) = 1] + \Pr[g(1,\mu_2,\dots,\mu_k) = 1 \land g(0, \mu_2, \dots, \mu_k) = 0] \Pr[\mu_1 = 1]
\end{align*}
which yields
\begin{align*}
  \underset{r,\pi}{\Pr}[D(x,r) = 1]
&\geq \mathbb{E}_{\pi}\left[\frac{{\Pr}_{\pi^{(k)}}[g(c) = 1 \mid \pi_1 = \pi^*] -
{\Pr}_{\mu_{\delta}^{(k)}}[g(0, \mu_2, \dots, \mu_k) = 0] - (1 - \frac{1}{6k})\epsilon} {{\Pr}_{\mu_{\delta}^{k}}[G_{\mu}]}\right] \notag
 \end{align*}
 Using the assumptions of Lemma \ref{lemma:sec_amp_for_p_hash}, we get
 \begin{align*}
   \underset{r,\pi}{\Pr}[D(x,r) = 1]
 &\geq \frac{ {\Pr}_{\mu_{\delta}^{(k)}}[g(\mu) = 1] + \epsilon +
 \Pr_{\mu_{\delta}^{(k)}}[g(0, \mu_2, \dots, \mu_k) = 0] - (1 - \frac{1}{6k})\epsilon}
 {\Pr_{\mu_{\delta}^{k}}[G_{\mu}]} \notag\\
 &\geq \frac{\epsilon +
\delta\Pr_{\mu_{\delta}^{(k)}}[G_{\mu}] - (1 - \frac{1}{6k})\epsilon}
{\Pr_{\mu_{\delta}^{k}}[G_{\mu}]} \geq \delta + \frac{\epsilon}{6k}
\end{align*}

%%% Local Variables:
%%% mode: latex
%%% TeX-master: "../master"
%%% End:


%%% Local Variables:
%%% mode: latex
%%% TeX-master: "../master"
%%% End:


\end{document}

