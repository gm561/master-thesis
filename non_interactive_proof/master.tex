
% \documentclass[11commpt]{article}

% %% We use the memoir class because it offers a many easy to use features.
\documentclass[11pt,a4paper,article,oneside]{memoir}

\counterwithout{section}{chapter}
\usepackage[margin=1in]{geometry}

%% Packages
%% ========

%% LaTeX Font encoding -- DO NOT CHANGE
%\usepackage[OT1]{fontenc}

%% Babel provides support for languages.  'english' uses British
%% English hyphenation and text snippets like "Figure" and
%% "Theorem". Use the option 'ngerman' if your document is in German.
%% Use 'american' for American English.  Note that if you change this,
%% the next LaTeX run may show spurious errors.  Simply run it again.
%% If they persist, remove the .aux file and try again.
\usepackage[english]{babel}

%% Input encoding 'utf8'. In some cases you might need 'utf8x' for
%% extra symbols. Not all editors, especially on Windows, are UTF-8
%% capable, so you may want to use 'latin1' instead.
\usepackage[utf8]{inputenc}

%% This changes default fonts for both text and math mode to use Herman Zapfs
%% excellent Palatino font.  Do not change this.
%\usepackage[sc]{mathpazo}

%% The AMS-LaTeX extensions for mathematical typesetting.  Do not
%% remove.
\usepackage{amsmath,amssymb,amsfonts}

%% NTheorem is a reimplementation of the AMS Theorem package. This
%% will allow us to typeset theorems like examples, proofs and
%% similar.  Do not remove.
%% NOTE: Must be loaded AFTER amsmath, or the \qed placement will
%% break
\usepackage[amsmath,thmmarks]{ntheorem}

%% LaTeX' own graphics handling
\usepackage{graphicx}

%% We unfortunately need this for the Rules chapter.  Remove it
%% afterwards; or at least NEVER use its underlining features.
\usepackage{soul}

%% For code snippets
\usepackage[framemethod=tikz]{mdframed}
\newdimen\linenumbersep

\newcommand{\linenumber}[1]{%
  \linenumbersep 4pt%
  \advance\linenumbersep\mdflength{innerleftmargin}%
  \advance\linenumbersep\mdflength{innerlinewidth}%
  \advance\linenumbersep\mdflength{middlelinewidth}%
  \advance\linenumbersep\mdflength{outerlinewidth}%
  \advance\linenumbersep\mdflength{linewidth}%
  \makebox[0pt][r]{{\rmfamily\tiny#1}\hspace*{\linenumbersep}}}

%\usepackage{mdframed}
\newenvironment{codeblock}%
   {\medskip\begin{mdframed}\setlength{\parindent}{0cm}}%
   {\end{mdframed}\medskip}
\newcommand{\Ind}{\mbox{}}
\newcommand{\IndI}{\mbox\qquad}
\newcommand{\IndII}{\mbox\qquad\qquad}
\newcommand{\IndIII}{\mbox\qquad\qquad\qquad}
\newcommand{\IndIIII}{\mbox\qquad\qquad\qquad\qquad}

%% Some more packages that you may want to use.  Have a look at the
%% file, and consult the package docs for each.
%% See the TeXed file for more explanations

%% [OPT] Multi-rowed cells in tabulars
%\usepackage{multirow}

%% [REC] Intelligent cross reference package. This allows for nice
%% combined references that include the reference and a hint to where
%% to look for it.
\usepackage{varioref}

%% [OPT] Easily changeable quotes with \enquote{Text}
%\usepackage[german=swiss]{csquotes}

%% [REC] Format dates and time depending on locale
\usepackage{datetime}

%% [OPT] Provides a \cancel{} command to stroke through mathematics.
%\usepackage{cancel}

%% [NEED] This allows for additional typesetting tools in mathmode.
%% See its excellent documentation.
\usepackage{mathtools}

%% [ADV] Conditional commands
%\usepackage{ifthen}

%% [OPT] Manual large braces or other delimiters.
%\usepackage{bigdelim, bigstrut}

%% [REC] Alternate vector arrows. Use the command \vv{} to get scaled
%% vector arrows.
%\usepackage[h]{esvect}

%% [NEED] Some extensions to tabulars and array environments.
\usepackage{array}

%% [OPT] Postscript support via pstricks graphics package. Very
%% diverse applications.
%\usepackage{pstricks,pst-all}

%% [?] This seems to allow us to define some additional counters.
%\usepackage{etex}

%% [ADV] XY-Pic to typeset some matrix-style graphics
%\usepackage[all]{xy}

%% [OPT] This is needed to generate an index at the end of the
%% document.
%\usepackage{makeidx}

%% [OPT] Fancy package for source code listings.  The template text
%% needs it for some LaTeX snippets; remove/adapt the \lstset when you
%% remove the template content.
\usepackage{listings}
\lstset{language=TeX,basicstyle={\normalfont\ttfamily}}

%% [REC] Fancy character protrusion.  Must be loaded after all fonts.
\usepackage[activate]{pdfcprot}

%% [REC] Nicer tables.  Read the excellent documentation.
\usepackage{booktabs}

%%pseudocode and algorithms
\usepackage{algpseudocode}
\usepackage{algorithm}

%% define comments in single line
\algnewcommand{\LineComment}[1]{\State \(\triangleright\) #1}

%%placing in the right place
\usepackage{float}

\usepackage{caption}

\DeclareCaptionFormat{algor}{
  \hrulefill\par\offinterlineskip\vskip1pt
    \textbf{#1#2}#3\offinterlineskip\hrulefill}
\DeclareCaptionStyle{algori}{singlelinecheck=off,format=algor,labelsep=space}
\captionsetup[algorithm]{style=algori}

\algnewcommand\algorithmicinput{\textbf{Input:}}
\algnewcommand\Input{\item[\algorithmicinput]}

\algnewcommand\algorithmicauxinput{\textbf{Auxiliary input:}}
\algnewcommand\AuxInput{\item[\algorithmicauxinput]}

%indention in algorithms
\newcommand{\pushcode}[1][1]{\hskip\dimexpr#1\algorithmicindent\relax}

% hyphen in math mode
\def\hyph{\text{-}}



%% Our layout configuration.  DO NOT CHANGE.
%%% Memoir layout setup

%% NOTE: You are strongly advised not to change any of them unless you
%% know what you are doing.  These settings strongly interact in the
%% final look of the document.

% Dependencies
\usepackage{ETHlogo}

% Turn extra space before chapter headings off.
\setlength{\beforechapskip}{0pt}

\nonzeroparskip
\parindent=0pt
\defaultlists

% Chapter style redefinition
\makeatletter

\if@twoside
  \pagestyle{Ruled}
  \copypagestyle{chapter}{Ruled}
\else
  \pagestyle{ruled}
  \copypagestyle{chapter}{ruled}
\fi
\makeoddhead{chapter}{}{}{}
\makeevenhead{chapter}{}{}{}
\makeheadrule{chapter}{\textwidth}{0pt}
\copypagestyle{abstract}{empty}

\makechapterstyle{bianchimod}{%
  \chapterstyle{default}
  \renewcommand*{\chapnamefont}{\normalfont\Large\sffamily}
  \renewcommand*{\chapnumfont}{\normalfont\Large\sffamily}
  \renewcommand*{\printchaptername}{%
    \chapnamefont\centering\@chapapp}
  \renewcommand*{\printchapternum}{\chapnumfont {\thechapter}}
  \renewcommand*{\chaptitlefont}{\normalfont\huge\sffamily}
  \renewcommand*{\printchaptertitle}[1]{%
    \hrule\vskip\onelineskip \centering \chaptitlefont\textbf{\vphantom{gyM}##1}\par}
  \renewcommand*{\afterchaptertitle}{\vskip\onelineskip \hrule\vskip
    \afterchapskip}
  \renewcommand*{\printchapternonum}{%
    \vphantom{\chapnumfont {9}}\afterchapternum}}

% Use the newly defined style
\chapterstyle{bianchimod}

\setsecheadstyle{\Large\bfseries\sffamily}
\setsubsecheadstyle{\large\bfseries\sffamily}
\setsubsubsecheadstyle{\bfseries\sffamily}
\setparaheadstyle{\normalsize\bfseries\sffamily}
\setsubparaheadstyle{\normalsize\itshape\sffamily}
\setsubparaindent{0pt}

% Set captions to a more separated style for clearness
\captionnamefont{\sffamily\bfseries\footnotesize}
\captiontitlefont{\sffamily\footnotesize}
\setlength{\intextsep}{16pt}
\setlength{\belowcaptionskip}{1pt}

% Set section and TOC numbering depth to subsection
\setsecnumdepth{subsection}
\settocdepth{subsection}

%% Titlepage adjustments
\pretitle{\vspace{0pt plus 0.7fill}\begin{center}\HUGE\sffamily\bfseries}
\posttitle{\end{center}\par}
\preauthor{\par\begin{center}\let\and\\\Large\sffamily}
\postauthor{\end{center}}
\predate{\par\begin{center}\Large\sffamily}
\postdate{\end{center}}

\def\@advisors{}
\newcommand{\advisors}[1]{\def\@advisors{#1}}
\def\@department{}
\newcommand{\department}[1]{\def\@department{#1}}
\def\@thesistype{}
\newcommand{\thesistype}[1]{\def\@thesistype{#1}}

\renewcommand{\maketitlehooka}{\noindent\ETHlogo[2in]}

\renewcommand{\maketitlehookb}{\vspace{1in}%
  \par\begin{center}\Large\sffamily\@thesistype\end{center}}

\renewcommand{\maketitlehookd}{%
  \vfill\par
  \begin{flushright}
    \sffamily
    \@advisors\par
    \@department, ETH Z\"urich
  \end{flushright}
}

\checkandfixthelayout

\setlength{\droptitle}{-48pt}

\makeatother

% This defines how theorems should look. Best leave as is.
\theoremstyle{plain}
\setlength\theorempostskipamount{0pt}

\usepackage[framemethod=tikz]{mdframed}
\newdimen\linenumbersep

\newcommand{\linenumber}[1]{%
  \linenumbersep 4pt%
  \advance\linenumbersep\mdflength{innerleftmargin}%
  \advance\linenumbersep\mdflength{innerlinewidth}%
  \advance\linenumbersep\mdflength{middlelinewidth}%
  \advance\linenumbersep\mdflength{outerlinewidth}%
  \advance\linenumbersep\mdflength{linewidth}%
  \makebox[0pt][r]{{\rmfamily\tiny#1}\hspace*{\linenumbersep}}}

\newenvironment{codeblock}%
   {\medskip\begin{mdframed}\setlength{\parindent}{0cm}}%
   {\end{mdframed}\medskip}
\newcommand{\Ind}{\mbox{}}
\newcommand{\IndI}{\mbox\qquad}
\newcommand{\IndII}{\mbox\qquad\qquad}
\newcommand{\IndIII}{\mbox\qquad\qquad\qquad}
\newcommand{\IndIIII}{\mbox\qquad\qquad\qquad\qquad}

\newenvironment{todo}%
   {\medskip\begin{mdframed}\setlength{\parindent}{0cm}}%
   {\end{mdframed}\medskip}


%%% Local Variables:
%%% mode: latex
%%% TeX-master: "thesis"
%%% End:


%% Theorem environments.
%% thesis.
%% Theorem-like environments

%% This can be changed according to language. You can comment out the ones you
%% don't need.

\numberwithin{equation}{chapter}

% declare theorem style
\declaretheoremstyle[
spaceabove=6pt, spacebelow=6pt,
headfont=\normalfont\bfseries,
notefont=\mdseries, notebraces={(}{)},
bodyfont=\normalfont\itshape,
postheadspace=1em,
%qed=\qedsymbol
]{thm_sty}

% declare definition style
\declaretheoremstyle[
spaceabove=6pt, spacebelow=6pt,
headfont=\normalfont\bfseries,
notefont=\mdseries, notebraces={(}{)},
bodyfont=\normalfont,
postheadspace=1em,
qed=\ensuremath{\lozenge}
]{def_sty}


%% English variants
\declaretheorem[style=thm_sty, name=Theorem, numberwithin=chapter]{theorem}
\declaretheorem[style=thm_sty, name=Observation, sibling=theorem]{observation}
\declaretheorem[style=thm_sty, name=Lemma, sibling=theorem]{lemma}
\declaretheorem[style=def_sty, name=Definition, sibling=theorem]{definition}
%\declaretheorem[style=def_sty, name=Proof]{proof}

%\newtheorem{theorem}{Theorem}[chapter]
% \newtheorem{example}[theorem]{Example}
% \newtheorem{remark}[theorem]{Remark}
% \newtheorem{corollary}[theorem]{Corollary}
% \newtheorem{lemma}[theorem]{Lemma}
% \newtheorem{proposition}[theorem]{Proposition}
% \newtheorem{observation}[theorem]{Observation}

%\theoremstyle{definition}
%\theorembodyfont{\normalfont}
%% end def with blacksquare symbol
%\theoremsymbol{\ensuremath{\lozenge}}
%\newtheorem{definition}[theorem]{Definition}

%%Proof environment with a small square as a "qed" symbol
%\theoremstyle{nonumberplain}
%\theorembodyfont{\normalfont}
%\theoremsymbol{\ensuremath{\square}}
%\theoremseparator{.}
%\newtheorem{proof}{Proof}



%% Helpful macros.
%% Custom commands
%% ===============

%% Special characters for number sets, e.g. real or complex numbers.
\newcommand{\C}{\mathbb{C}}
\newcommand{\D}{\mathbb{D}}
\newcommand{\K}{\mathbb{K}}
\newcommand{\N}{\mathbb{N}}
\newcommand{\M}{\mathbb{M}}
\newcommand{\Q}{\mathbb{Q}}
\newcommand{\R}{\mathbb{R}}
\newcommand{\T}{\mathbb{T}}
\newcommand{\X}{\mathbb{X}}
\newcommand{\Z}{\mathbb{Z}}


\newcommand{\cA}{\mathcal{A}}
\newcommand{\cB}{\mathcal{B}}
\newcommand{\cX}{\mathcal{X}}
\newcommand{\cH}{\mathcal{H}}
\newcommand{\cW}{\mathcal{W}}
\newcommand{\cG}{\mathcal{G}}
\newcommand{\cP}{\mathcal{P}}
\newcommand{\cR}{\mathcal{R}}
\newcommand{\cD}{\mathcal{D}}
\newcommand{\cF}{\mathcal{F}}
\newcommand{\cM}{\mathcal{M}}
\newcommand{\cK}{\mathcal{K}}
\newcommand{\cT}{\mathcal{T}}
\newcommand{\cS}{\mathcal{S}}
\newcommand{\cQ}{\mathcal{Q}}
\newcommand{\cV}{\mathcal{V}}
\newcommand{\cI}{\mathcal{I}}

%define our own code commands
%use capital latters as most of these commands is already defined
\renewcommand{\For}{\textbf{for }}
\renewcommand{\If}{\textbf{if }}
\renewcommand{\Else}{\textbf{else }}
\renewcommand{\Return}{\textbf{return }}
\newcommand{\Then}{\textbf{then }}
\newcommand{\Do}{\textbf{do: }}
\renewcommand{\And}{\textbf{and }}
\newcommand{\Or}{\textbf{or }}
\newcommand{\Run}{\textbf{run }}
\newcommand{\To}{\textbf{to }}
\renewcommand{\Repeat}{\textbf{Repeat }}

%% Fixed/scaling delimiter examples (see mathtools documentation)
\DeclarePairedDelimiter\abs{\lvert}{\rvert}
\DeclarePairedDelimiter\norm{\lVert}{\rVert}

%% Use the alternative epsilon per default and define the old one as \oldepsilon
\let\oldepsilon\epsilon
\renewcommand{\epsilon}{\ensuremath\varepsilon}

%% Also set the alternate phi as default.
\let\oldphi\phi
\renewcommand{\phi}{\ensuremath{\varphi}}

% New command that introduces a tab
\newcommand{\itab}[1]{\hspace{0em}\rlap{#1}}
\newcommand{\tab}[1]{\hspace{.2\textwidth}\rlap{#1}}

\DeclareMathOperator{\la0}{\leftarrow}
\DeclareMathOperator{\ra0}{\rightarrow}

\DeclareMathOperator{\hash}{\mathit{hash}}
\DeclareMathOperator{\CanonicalSuccess}{\mathit{CanonicalSuccess}}
\DeclareMathOperator{\Success}{\mathit{Success}}
\DeclareMathOperator{\poly}{\mathit{poly}}
\DeclareMathOperator{\p}{\mathit{p}}
\DeclareMathOperator{\Time}{\mathit{Time}}
\DeclareMathOperator{\Gen}{\text{Gen}}
\DeclareMathOperator{\FindHash}{\text{FindHash}}
\DeclareMathOperator{\Bad}{\mathit{Bad}}
\DeclareMathOperator{\Good}{\mathit{Good}}
\DeclareMathOperator{\trans}{\mathit{trans}}
\DeclareMathOperator{\Canonical}{\mathit{Canonical}}


%the DWVP for the permutation
\DeclareMathOperator{\PiDWVP}{\Pi_{DWVP}}
%the DWPV for the k-wise product of permutations
\DeclareMathOperator{\kPiDWVP}{\Pi_{DWVP}^{(k)}}

\usepackage{enumerate}

\begin{document}

%\title{Latex template}
%\author{Grzegorz Makosa}
%\maketitle

\setcounter {myc}{1}
%
% Define high probability
%
\noindent
We write $\mu_{\delta}$ to denote the Bernoulli distribution, where outcome $1$ occurs with
probability $\delta$ and $0$ with probability $1-\delta$ where $0 \leq \delta \leq 1$.
Moreover, we use $\mu_{\delta}^k$ to denote a probability distribution over $k$-tuples,
where each bit of a $k$-tuple is drawn independently according to $\mu_{\delta}$.
Finally, let $u \leftarrow \mu_{\delta}^k$ denote that a $k$-tuple $u$ is chosen according to $\mu_{\delta}^k$.

The protocol execution between two probabilistic algorithms $A$ and $B$ is denoted by $\langle A, B \rangle$.
The output of $A$ in such a protocol execution is denoted by $\langle A, B \rangle_A$ and of $B$ by $\langle A, B \rangle_B$.
Finally, let $\langle A, B \rangle_{\text{trans}}$ denote the transcript of communication between $A$ and $B$.

We define a \textit{two phase algorithm} $A := (A_1, A_2)$ as an algorithm where in the first phase an algorithm $A_1$
is executed and in the second phase an algorithm $A_2$.

We say that an event happens \textit{almost surely} or with \textit{high probability} if
it occurs with probability at least $1 - 2^{-n} poly(n)$.

\begin{definition}[Dynamic weakly verifiable puzzle.]
  \label{def:dwvp}
  A dynamic weakly verifiable puzzle (DWVP) is defined by a probabilistic algorithm $P$
  called a problem poser.
  A problem solver $S := (S_1, S_2)$ for $P$ is a probabilistic two phase algorithm.
  We write $P_n(\pi)$ to denote the execution of $P$ with the randomness fixed to $\pi \in \{0,1\}^n$, and $(S_1,S_2)(\rho)$
  to denote the execution of both $S_1$ and $S_2$ with the randomness fixed to $\rho \in \{0,1\}^{*}$.

  In the first phase, the poser $P_n(\pi)$ and the solver $S_1(\rho)$ interact.
  As the result of the interaction $P_n(\pi)$ outputs a verification circuit $\Gamma_{V}$ and a hint circuit $\Gamma_{H}$.
  The algorithm $S_1(\rho)$ produces no output.
  The circuit $\Gamma_{V}$ takes as input $q \in Q$, an answer $y \in \{0,1\}^*$,
  and outputs a bit. We say that an answer $(q,y)$ is a correct solution if and only if $\Gamma_V(q,y) = 1$.
  The circuit $\Gamma_H$ on input $q \in Q$ outputs a hint such that $\Gamma_V(q,\Gamma_H(q)) = 1$.

  In the second phase, $S_2$ takes as input $x := \langle P_n(\pi), S_1(\rho) \rangle_{\text{trans}}$,
  and has oracle access to $\Gamma_V$ and $\Gamma_H$.
  The execution of $S_2$ with the input $x$ and the randomness fixed to $\rho$
  is denoted by $S_2(x, \rho)$. The queries of $S_2$ to $\Gamma_V$ and $\Gamma_H$ are called verification queries and hint queries respectively.
  The algorithm $S_2$ asks at most $h$ hint queries, $v$ verification queries, and succeeds if and only if
  it makes a verification query $(q,y)$ such that $\Gamma_V(q,y) = 1$, and it has not previously asked for a hint query on $q$.
\end{definition}
%
\begin{definition}[$k$-wise direct-product of DWVPs.]
  Let $g: \{0,1\}^{k} \rightarrow \{0,1\}$ be a monotone function and $P^{(1)}$ a problem poser as in Definition \ref{def:dwvp}.
  The $k$-wise direct product of $P^{(1)}$ is a DWVP defined by a probabilistic algorithm $P^{(g)}$.
  We write $P_{kn}^{(g)}(\pi^{(k)})$ to denote the execution of $P^{(g)}$ with the randomness fixed to $\pi^{(k)} := (\pi_1, \dots, \pi_k)$
  where for each $1 \leq i \leq n : \pi_i \in \{0,1\}^n.$
  Let $(S_1, S_2)(\rho)$ be a solver for $P^{(g)}$ as in Definition \ref{def:dwvp}.
  In the first phase, the algorithm $S_1(\rho)$ sequentially interacts in $k$ rounds with $P_{kn}^{(g)}(\pi^{(k)})$.
  In the $i$-th round $S_1(\rho)$ interacts with $P_n^{(1)}(\pi_i)$,
  and as the result $P_{n}^{(1)}(\pi_i)$ generates circuits $\Gamma_V^i, \Gamma_H^i$.
  Finally, after $k$ rounds $P_{kn}^{(g)}(\pi^{(k)})$ outputs a verification circuit
\begin{align*}
  \Gamma_V^{(g)} (q, y_1, \dots, y_k) := g(\Gamma_V^{1}(q, y_1), \dots, \Gamma_V^{k}(q, y_k))
\end{align*}
and a hint circuit
\begin{align*}
  \Gamma_H^{(k)} (q) := (\Gamma_H^{1}(q), \dots, \Gamma_H^{k}(q)).
\end{align*}
\end{definition}
%
If it is clear from a context, we omit the parameter $n$, and write $P(\pi)$ instead of $P_n(\pi)$ where $\pi \in \{0,1\}^{n}$.

A verification query $(q,y)$ of a solver $S$ for which a hint query on this $q$ has been asked before can not be a verification query that succeeds.
Therefore, without loss of generality, we make the assumption that $S$ does not ask verification queries on $q$
for which a hint query has been asked before. Furthermore, we assume that once $S$ asked a verification query that succeeds,
it does not ask any further hint or verification queries.

Let $C$ be a circuit that corresponds to a solver $S$ as in Definition \ref{def:dwvp}.
Similarly as for a two phase algorithm, we write $C(\rho) := (C_1, C_2)(\rho)$ to denote that $C$
in the first phase uses a circuit $C_1$ and in the second phase a circuit $C_2$.
Additionally, the randomness in both phases is fixed to $\rho \in \{0,1\}^{*}$.

%
\begin{codeblock}
  \textbf{Experiment $Success^{P, C}(\pi, \rho) $}
  \medskip
  \hrule
  \medskip
  \textbf{Oracle:} A problem poser $P$, a solver circuit $C = (C_1, C_2)$.\\
  \textbf{Input:}  Bitstrings $\pi \in \{0,1\}^n$, $\rho \in \{0,1\}^*$.\\
  \textbf{Output:} A bit $b \in \{0,1\}$.
  \medskip\hrule\medskip
  \Run $\langle P(\pi), C_1(\rho) \rangle$ \\
  \IndI $(\Gamma_V, \Gamma_H) := \langle P(\pi), C_1(\rho) \rangle_{P}$ \\
  \IndI $x := \langle P(\pi), C_1(\rho) \rangle_{\text{trans}}$ \\ \\
  \Run $C_2^{\Gamma_V,\Gamma_H}(x, \rho)$ \\
  \IndI \If $C_2^{\Gamma_V, \Gamma_H}(x, \rho)$ asks a verification query $(q, y)$ such that $\Gamma_V(q, y) = 1$ \Then \\
  \IndII \Return $1$ \\
  \Return $0$ \\
\end{codeblock}
%
We define the \textit{success probability} of $C$ in solving a puzzle defined by $P$ as
\begin{align}
 \underset{\pi, \rho}{\Pr}[Success^{P,C}(\pi, \rho) = 1].
\end{align}
Furthermore, we say that $C$ succeeds for $\pi$, $\rho$ if $Success^{P,C}(\pi, \rho) = 1$.
%
\begin{theorem}[Security amplification for a dynamic weakly verifiable puzzle.]
\label{th:sec_amp_for_dwvp}
Let $P^{(1)}$ be a fixed problem poser as in Definition \ref{def:dwvp}, and $P^{(g)}$ be a poser for the $k$-wise direct product of $P^{(1)}$.
There exists a probabilistic algorithm $Gen$ with oracle access to: a solver circuit $C$ for $P^{(g)}$,
a monotone function $g:\{0,1\}^k \rightarrow \{0,1\}$ and problem posers $P^{(1)}$, $P^{(g)}$.
Additionally, $Gen$ takes as input parameters $\varepsilon$, $\delta$, $n$, $k$
the number of verification queries $v$ and hint queries $h$ asked by $C$, and outputs a solver circuit $D$ for $P^{(1)}$ as in Definition \ref{def:dwvp}
such that the following holds: \\
If $C$ is such that
  \begin{align*}
    \underset{\substack{\pi^{(k)} \in \{0,1\}^{kn} \\ \rho \in \{0,1\}^{*}}}{\Pr}\left[Success^{P^{(g)}, C}(\pi^{(k)}, \rho) = 1\right]
    \geq 16(h+v)\left(\underset{u \leftarrow \mu_\delta^k}{\Pr}\left[g(u) = 1\right] + \varepsilon\right)
  \end{align*}
then $D$ satisfies almost surely
  \begin{align*}
    \underset{\substack{\pi \in \{0,1\}^{n} \\ \rho \in \{0,1\}^{*}}}
    {\Pr}\left[Success^{P^{(1)},D}(\pi, \rho) = 1\right] \geq (\delta + \frac{\varepsilon}{6k}).
  \end{align*}
Additionally, $D$ requires oracle access to $g$, $P^{(1)}$, $C$,
and asks at most $\frac{6k}{\epsilon}\log\left(\frac{6k}{\epsilon}\right) h$ hint queries and one verification query.
Finally, $\text{Size}(D) \leq \text{Size}(C) \cdot \frac{6k}{\varepsilon}$ and $\text{Time}(\text{Gen}) = \text{poly}(k, \frac{1}{\varepsilon}, n, v, h)$.
\end{theorem}
%
% The Theorem \ref{th:sec_amp_for_dwvp} implies that if there is no good solver for a puzzle defined by $P^{(1)}$, then a good solver for
% a $k$-wise direct product of $P^{(1)}$ does not exist.

% The idea of the algorithm $Gen$ is to output a circuit $D$ that solves the input puzzle often.
% We know that $C$ has good success probability for a $k$-wise product of $P^{(1)}$.
% The algorithm $Gen$ tries to find a puzzle such that when $C$ runs with this puzzle fixed
% on the first position, and disregards whether this puzzle is correctly solved
% then the assumptions of Theorem \ref{th:sec_amp_for_dwvp} are true for a $k-1$-wise direct product.
% If it is possible to find such a puzzle then $Gen$ could recurse and solve a smaller problem.
% In the optimistic case we can reach $k=1$, which means that we found a good circuit for solving a single
% puzzle by just fixing the initial puzzles of $C$.

% Otherwise, when the first position is disregarded then the success probability of $C$ is not substantially better.
% This is remarkable, as we know that $C$ performs good for $k$-wise product, it means that the first position is important,
% in the sense that $C$ solves the puzzle on that position unusually often.
% Therefore, it is reasonable to construct the circuit $D$ using $C$ by placing the input puzzle of $D$ on that position, and then
% finding remaining $k-1$ puzzles. These $k-1$ remaining puzzles are generated by the circuit $D$, hence it is possible to check
% whether they are correctly solved by the circuit $C$. We know that circuit $C$ has good success probability, and the puzzle on the first
% position is important. Therefore, if we are able to find a $k-1$ puzzles such that the fact whether the $k$-wise direct product is correctly
% solved depends on whether the puzzle on the first position is correctly solved then we can assume that $C$ is often correct on this first position.

% There are some problems with this approach, first we have to ensure that we can make a decision when the algorithm $Gen$ should recurse and when not
% correctly with high probability. Then, we have to show that it is possible to find a puzzles such that $C$ is often correct on the first position.
% Finally, we also have to be sure that we do not ask a hint query, on the final verification query to the oracle.
% To satisfy the last requirement we split $Q$.

%%% Local Variables:
%%% mode: latex
%%% TeX-master: "../master"
%%% End:


% Let $hash:Q\rightarrow\{0,1,\dots, 2(h+v)-1\}$, then a set $P_{hash} \subseteq Q$,
% defined with respect to $hash$, is the set of preimages of $0$ for $hash$.
Let $hash:Q\rightarrow\{0,1,\dots, 2(h+v)-1\}$,
the idea is to partition $Q$ such that the set of preimages of $0$ for $hash$ contains $q \in Q$ on which $C$ is not allowed to ask hint queries,
and the first successful verification query $(q,y)$ of $C$ is such that $hash(q) = 0$.
Therefore, if $C$ makes a verification query $(q,y)$ such that $hash(q) = 0$, then we know that no hint query is ever asked on this $q$.

In the experiment $CanonicalSuccess$ we denote the $i$-th query of $C$ by $q_i$ if it is a hint query, and by $(q_i, y_i)$ if it is a verification query.
A solver circuit $C$ succeeds in the experiment $CanonicalSuccess$ if it asks a successful verification query $(q_j, y_j)$ such that $hash(q_j) = 0$,
and no hint query $q_i$ is asked before $(q_j, y_j)$ such that $hash(q_i) = 0$.
%
\begin{codeblock}
  \textbf{Experiment $CanonicalSuccess^{P, C, hash}(\pi, \rho)$}
  \medskip

  \hrule

  \medskip
  \textbf{Oracle:} A problem poser $P$, a solver circuit $C = (C_1, C_2)$.\\
  \IndII A function $hash: Q \rightarrow \{0, \dots, 2(h+v) - 1\}$.\\
  \textbf{Input:}  Bitstrings $\pi \in \{0,1\}^n$ and $\rho \in \{0,1\}^*$. \\
  \textbf{Output:} A bit $b \in \{0,1\}$.

  \medskip\hrule\medskip
  Run $\langle P(\pi), C_1(\rho) \rangle$ \\
  \IndI $(\Gamma_V, \Gamma_H) := \langle P(\pi), C_1(\rho) \rangle_{P}$ \\
  \IndI $x := \langle P(\pi), C_1(\rho) \rangle_{\text{trans}}$ \\ \\
  Run $C_2^{\Gamma_V, \Gamma_H} (x, \rho)$ \\
  \IndI Let $(q_j,y_j)$ be the first verification query of $C_2$ such that $\Gamma_v(q_j, y_j) = 1$.\\
  \IndI If $C_2$ does not succeed let $(q_j, y_j)$ be an arbitrary verification query.\\
  \\
  \textbf{If} $(\forall i < j :  hash(q_i) = 0)$ \And $(hash(q_j) = 1)$ \And $(\Gamma_V(q_j, y_j) = 1)$ \then \\
  \IndI \textbf{return} 1\\
  \textbf{else}\\
  \IndI \textbf{return} 0
\end{codeblock}
%
We define the \textit{canonical success probability} of a solver $C$ for $P$ with respect to a function $hash$ as
\begin{align}
 \underset{\pi, \rho}{\Pr}[CanonicalSuccess^{P, C, hash}(\pi, \rho) = 1].
\end{align}
%
For fixed $hash$ and a problem poser $P$ a \textit{canonical success} of $C$ for $\pi, \rho$ is a situation where $CanonicalSuccess^{P, C, hash}(\pi, \rho) = 1$.

We show that if a solver circuit $C$ for $P^{(g)}$ often succeeds in the experiment $Success$, then it is
also often successful in the experiment $CanonicalSuccess$.

\begin{lemma}\textbf{(\boldmath{Success probability in solving a $k$-wise direct product of $P^{(1)}$ with respect to a function $hash$.)}}
\label{lemma:hash_function_probability}
For fixed $P^{(g)}$ let $C$ be a solver for $P^{(g)}$ with the success probability at least $\gamma$,
asking at most $h$ hint queries and $v$ verification queries.
There exists a probabilistic algorithm \textbf{FindHash} that takes as input:
parameters $\gamma$, $n$, $k$, the number of verification queries $v$ and hint queries $h$, and has
oracle access to $C$ and $P^{(g)}$. Furthermore, \textbf{FindHash} runs in time $O((h+v)^4/\gamma^4)$,
and with high probability outputs a function $hash \in \cH$
such that the canonical success probability of $C$ with respect to $hash$ is at least $\frac{\gamma}{8(h+v)}$.
\end{lemma}
%
\begin{proof}
We fix $P^{(g)}$ and a solver $C$ for $P^{(g)}$ in the whole proof of Lemma \ref{lemma:hash_function_probability}.
Let $\cH$ be a family of pairwise independent hash functions $Q \rightarrow \{0,1, \dots,2(h+v)-1\}$.
For all $m,n \in \{1, \dots, (h+v)\}$ and $k,l \in \{0,1,\dots,2(h+v)-1\}$ by the pairwise independence property of $\cH$, we have
\begin{align}
  \label{eq:hash_pr}
 \forall q_m,q_n \in Q, q_m \neq q_n : \underset{\textit{hash} \leftarrow \cH}{\Pr}[hash(q_m) = k \mid hash(q_n) = l] = \underset{\textit{hash} \leftarrow \cH}{\Pr}[hash(q_m) = k] = \frac{1}{2(h+v)}.
\end{align}
%
Let $\cP_{Success}$ be a set containing all $(\pi^{(k)},\rho)$ for which $Success^{P^{(g)}, C}(\pi^{(k)}, \rho) = 1$.
We choose uniformly at random $hash \leftarrow \cH$, and consider the experiment $CanonicalSuccess^{P^{(g)}, C, hash}(\pi^{(k)}, \rho)$.
We are interested in the probability of the event that for a fixed $(\pi, rho) \in \cP_{Success}$ the solver $C$ succeeds canonically.
Let $(q_j, y_j)$ denote the first query such that $\Gamma_V(q_j, y_j) = 1$.
We have
\begin{align*}
  &\underset{\textit{hash} \leftarrow \cH}{\Pr}[hash(q_j) = 0 \land (\forall i < j : hash(q_i) \neq 0)]\\
  &\IndII = \underset{\textit{hash} \leftarrow \cH}{\Pr}[\forall i < j : hash(q_i) \neq 0 \mid hash(q_j) = 0] \underset{\textit{hash} \leftarrow \cH}{\Pr}[hash(q_j) = 0] \\
  &\IndII \stackrel{(\ref{eq:hash_pr})}{=} \frac{1}{2(h+v)}\left(1 -\underset{\textit{hash} \leftarrow \cH}{\Pr}[\exists i < j : hash(q_i) = 0 \mid hash(q_j) = 0] \right) \\
  &\IndII \stackrel{(\ref{eq:hash_pr})}{=} \frac{1}{2(h+v)} \left( 1 -\underset{\textit{hash} \leftarrow \cH}{\Pr}[\exists i < j : hash(q_i) = 0] \right) \\
  &\IndII \stackrel{(\text{u.b})}{\geq} \frac{1}{2(h+v)} \left( 1 - \sum_{i < j} \underset{\textit{hash} \leftarrow \cH}{\Pr}[hash(q_i) = 0] \right) \\
  &\IndII \stackrel{(\ref{eq:hash_pr})}{\geq} \frac{1}{4(h+v)}.
\end{align*}

We denote the set of those $(\pi^{(k)},\rho)$ for which $CanonicalSuccess^{P^{(g)}, C, hash}(\pi^{(k)}, \rho) = 1$ by $\cP_{Canonical}$.
For $(\pi^{(k)}, \rho)$ for which $C$ succeeds canonically, we have $Success^{P^{(g)}, C}(\pi^{(k)}, \rho) = 1$.
Hence, $\cP_{Canonical} \subseteq \cP_{Success}$, and we conclude
\begin{align}
  \label{ineq:hash_high_prob}
\underset{\substack{\textit{hash} \leftarrow \cH \\ \pi^{(k)}, \rho}}{\Pr}\left[CanonicalSuccess^{P^{(g)}, C, hash}(\pi^{(k)}, \rho) = 1\right] &=
\underset{{(\pi^{(k)},\rho) \in \cP_{Success}}}{\mathbb{E}}\left[\underset{\substack{\textit{hash} \leftarrow \cH}}{\Pr}[X = 1]\right] \notag\\
&\geq \frac{\gamma}{4(h+v)}.
\end{align}
%
\begin{codeblock}
  \textbf{Algorithm: FindHash}$(\gamma, n, k, h, v)$
  \medskip
  \hrule
  \medskip
  \textbf{Oracle:} A problem poser $P^{(g)}$, a solver circuit $C$ for $P^{(g)}$.\\
  \textbf{Input:} Parameters $\gamma, n, k, h,v $\\
  \textbf{Output:} A function $hash:Q \rightarrow \{0,1, \dots, 2(h+v)-1 \}$.
  \medskip\hrule\medskip
  Let $\cH$ be a family of pairwise independent hash functions $Q \rightarrow \{0,1,\dots, 2(h+v)-1\}$\\
  \For $i = 1$ \To $32(h+v)^2/\gamma^2$ \Do \\
  \IndI $hash \xleftarrow{\$} \cH$ \\
  \IndI $count := 0$ \\
  \IndI \For $j := 1$ to $32(h+v)^2/\gamma^2$ \Do \\
  \IndII $\pi^{(k)} \xleftarrow{\$} \{0,1\}^{kn} $\\
  \IndII $\rho \xleftarrow{\$} \{0,1\}^*$ \\
  \IndII \If $CanonicalSuccess^{P^{(g)}, C, hash}(\pi^{(k)}, \rho) = 1$ \then \\
  \IndIII $count := count + 1$\\
  \IndI \If $\frac{\gamma^2}{32(h+v)^2} count \geq \frac{\gamma}{6(h+v)}$ \then \\
  \IndII \return $hash$\\
  \return $\bot$
\end{codeblock}
We show that \textbf{FindHash} chooses $hash$ such that the canonical success probability of $C$
with respect to $hash$ is at least $\frac{\gamma}{4(h+v)}$ almost surely.
Let $\cH_{Good}$ denote a family of functions $hash \in \cH$ for which
\begin{align}
  \label{eq:hash_good}
\underset{\pi^{(k)}, \rho}{\Pr}\left[CanonicalSuccess^{P^{(g)}, C, hash}(\pi^{(k)}, \rho) = 1\right] \geq \frac{\gamma}{8(h+v)},
\end{align}
and $\cH_{Bad}$ be the family of functions $hash \in \cH$ such that
\begin{align}
  \label{eq:hash_bad}
\underset{\pi^{(k)}, \rho}{\Pr}\left[CanonicalSuccess^{P^{(g)}, C^{(\cdot, \cdot)}, hash}(\pi^{(k)}, \rho) = 1\right] \leq \frac{\gamma}{16(h+v)}.
\end{align}
%
Let $N$ denote the number of iterations of the inner loop of \textbf{FindHash}.
For a fixed $hash$, we define binary random variables $X_1, \dots, X_{N}$ such that
\begin{align*}
  X_i =
  \begin{cases}
    1 & \text{if in the $i$-th iteration of the inner loop $count$ is increased}\\
    0 & \text{otherwise.}
  \end{cases}
\end{align*}
We show now that \textbf{FindHash} is unlikely to return $hash \in \cH_{Bad}$.
For $hash \in \cH_{Bad}$ by (\ref{eq:hash_bad}) we have $\mathbb{E}_{\pi^{(k)},\rho}[X_i] \leq \frac{\gamma}{16(h+v)}$.
Therefore, for any fixed $hash \in \cH_{Bad}$ using the Chernoff bound we get
\footnote{For $X = \sum_{i=1}^N X_i$ and $0 < \delta \leq 1$ we use the Chernoff bounds in the form
$\Pr[X \geq (1+\delta) \mathbb{E}[X]] \leq e^{- \mathbb{E}[X] \delta^2/3}$ and
$\Pr[X \leq (1-\delta) \mathbb{E}[X]] \leq e^{- \mathbb{E}[X] \delta^2/2}$.}
\begin{align*}
  \underset{\pi^{(k)},\rho}{\Pr} \left[\frac{1}{N} \sum_{i=1}^{N} X_i \geq \frac{\gamma}{12(h+v)} \right] \leq
  \underset{\pi^{(k)}, \rho}{\Pr}\left[\frac{1}{N} \sum_{i=1}^{N} X_i \geq (1 + \frac{1}{4}) \mathbb{E}[X_i]\right] \leq
  e^{-{\frac{\gamma}{16(h+v)}} N /48} \leq e^{-\frac{1}{24}\frac{(h+v)}{\gamma}}.
\end{align*}
%
The probability that $hash \in \cH_{Good}$, when picked, is not returned amounts
\begin{align*}
  \underset{\pi^{(k)}, \rho}{\Pr}\left[\frac{1}{N} \sum_{i=1}^{N} X_i \leq \frac{\gamma}{12(h+v)}\right] \leq
  \underset{\pi^{(k)}, \rho}{\Pr}\left[\frac{1}{N} \sum_{i=1}^{N} X_i \leq (1 - \frac{1}{3})\mathbb{E}[X_i]\right]
  \leq e^{-{\frac{\gamma}{8(h+v)}} N / 18} \leq e^{-\frac{2}{9} \frac{(h+v)}{\gamma}},
\end{align*}
where we once more used the Chernoff bound.
Now we show that the probability of picking a $hash \in \cH_{Good}$ is at least $\frac{\gamma}{8(h+v)}$.
We proof this statement by contradiction. We assume otherwise, namely that
$\underset{hash \leftarrow \cH}{\Pr}[hash \in \cH_{Good}] < \frac{\gamma}{8(g+v)}$.
We have
\begin{align*}
  &\underset{\substack{hash \leftarrow \cH \\ \pi, \rho}}{\Pr}[CanonicalSuccess^{P,C,hash}(\pi, \rho) = 1] \\
  &\IndI = \underset{\substack{hash \leftarrow \cH \\ \pi, \rho}}{\Pr}[CanonicalSuccess^{P,C,hash}(\pi, \rho) = 1 \mid hash \in \cH_{Good}]
  \underset{hash \leftarrow \cH}{\Pr}[hash \in \cH_{Good}] \\
  & \IndII + \underset{\substack{hash \leftarrow \cH \\ \pi, \rho}}{\Pr}[CanonicalSuccess^{P,C,hash}(\pi, \rho) = 1 \mid hash \notin \cH_{Good}]
  \underset{hash \leftarrow \cH}{\Pr}[hash \notin \cH_{Good}] \\
  & \IndI \leq \underset{hash \leftarrow \cH}{\Pr}[hash \in \cH_{Good}] +
  \underset{\substack{hash \leftarrow \cH \\ \pi, \rho}}{\Pr}[CanonicalSuccess^{P,C,hash}(\pi, \rho) = 1 \mid hash \notin \cH_{Good}] \\
  & \IndI < \frac{\gamma}{8(h+v)} + \frac{\gamma}{8(h+v)} = \frac{\gamma}{4(h+v)}.
\end{align*}
But this contradicts (\ref{ineq:hash_high_prob}).
Finally, we show that \textbf{FindHash} picks in one of its iteration $hash \in \cH_{Good}$ almost surely.
Let $K$ be the number of iterations of the outer loop of \textbf{FindHash}.
Let $Y_i$ be a random variable for the event
that in the $i$-th iteration of the outer loop $hash \in \cH_{Good}$ is picked.
Using $\underset{hash \leftarrow \cH}{\Pr}[hash \in \cH_{Good}] < \frac{\gamma}{8(g+v)}$ and  $K \leq \frac{32(h+v)^2}{\gamma^2}$ we conclude
\begin{align*}
  \underset{hash \leftarrow \cH}{\Pr}[ \bigcap_{1 \leq i \leq K} Y_i ] \leq \left(1 - \frac{\gamma}{8(h+v)}\right)^{K}
    \leq e^{-\frac{\gamma}{8(h+v)} K}
    \leq e^{-\frac{4(h+v)}{\gamma}}.
\end{align*}
\end{proof}
%%% Local Variables:
%%% mode: latex
%%% TeX-master: "../master"
%%% End:

%
\subsection{The hardness amplification proof for partitioned domains}
\label{st:amplification_proof}
\begin{todo}
  \textbf{TODO:} Add short introduction
\end{todo}

Let $C := (C_1, C_2)$ be a two-phase solver circuit as in Definition \ref{def:dwvp}.
We write $C_2^{(\cdot, \cdot)}$ to emphasize that $C_2$ does not obtain direct access to the hint and verification oracles.
Instead, whenever $C_2$ asks a hint or verification query, it is answered explicitly
as in the following code excerpt of the circuit $\widetilde{C}_2$.

\begin{codeblock}
  \textbf{Circuit} $\widetilde{C}_2^{\Gamma_H, C_2, \hash} (x, \rho)$
  \medskip \hrule
  \textbf{Oracle:} A hint circuit $\Gamma_H$, a circuit $C_2$, \\ \IndII a function $\hash : \cQ \rightarrow \{0,1,\dots, 2(h+v)-1\}$. \\
  \textbf{Input:} Bitstrings $x \in \{0,1\}^{*}$, $\rho \in \{0,1\}^{*}$. \\
  \textbf{Output:} A pair $(q, y)$ where $q \in \cQ$ and $y \in \{0,1\}^{*}$.
  \medskip\hrule
  \Run $C_2^{(\cdot, \cdot)}(x, \rho)$ \\
  \IndI \If $C_2^{(\cdot, \cdot)}(x, \rho)$ asks a hint query on $q$ \Then\\
  \IndII \If $\hash(q) = 0$ \Then\\
  \IndIII \Return $\bot$\\
  \IndII \textbf{else}\\
  \IndIII answer the query of $C_2^{(\cdot, \cdot)}(x, \rho)$ using $\Gamma_H(q)$\\
  \\
  \IndI \If $C_2^{(\cdot, \cdot)}(x, \rho)$ asks a verification query $(q, y)$ \Then \\
  \IndII \If $\hash(q) = 0 $ \textbf{then} \\
  \IndIII \Return $(q, y)$ \\
  \IndII \textbf{else} \\
  \IndIII answer the verification query of $C_2^{(\cdot, \cdot)}(x, \rho)$ with 0 \\
  \Return $\bot$
\end{codeblock}
%
Given $C := (C_1, C_2)$ we define a circuit $\widetilde{C} := (C_1, \widetilde{C}_2)$.
Every hint query $q$ asked by $\widetilde{C}$ is such that $hash(q) \neq 0$.
Furthermore, $\widetilde{C}$ asks no verification queries, instead it returns $(q,y)$ such that $hash(q) = 0$ or $\bot$.

For fixed $\pi$, $\rho$, and $hash$ we say that the circuit $\widetilde{C}$ \textit{succeeds} if
for $x := \langle P(\pi), C_1(\rho) \rangle_{\mathit{trans}}$,
$(\Gamma_V, \Gamma_H) := \langle P(\pi), C_1(\rho) \rangle_{P}$, we have
\begin{align*}
\Gamma_V(\widetilde{C}_2^{\Gamma_H, C_2, \hash}(x, \rho)) = 1.
\end{align*}
%
\begin{todo}
  \textbf{TODO:} Show the intuitive meaning of this lemma
\end{todo}

\begin{lemma}
  \label{lemma:ctilda_c}
  For fixed $P$, $C := (C_1, C_2)$, and $hash$ it holds
  \begin{align*}
    \underset{\pi, \rho}{\Pr}[\CanonicalSuccess^{P, C, \hash}(\pi, \rho) = 1]
    \leq
    \mkern13mu
    \underset{
      \mathclap{
      \substack{
        \pi, \rho \\
        x := \langle P(\pi), C_1(\rho) \rangle_{\mathit{trans}} \\
        (\Gamma_V, \Gamma_H) := \langle P(\pi), C_1(\rho) \rangle_{P}}}}
  {\Pr}[\Gamma_V(\widetilde{C}_2^{\Gamma_H, C_2, hash}(x, \rho)) = 1].
  \end{align*}
\end{lemma}
%
\begin{todo}
  \textbf{TODO:} Give an overview of this Lemma
\end{todo}
\begin{proof}[Proof of Lemma \ref{lemma:ctilda_c}]
If for some $\pi$, $\rho$, and $\hash$ the circuit $C := (C_1, C_2)$ succeeds canonically,
then for the same $\pi$, $\rho$, and $\hash$ the circuit $\widetilde{C} := (C_1, \widetilde{C}_2)$ also succeeds.
Using this observation, we conclude that
\begin{align*}
  &\underset{\pi, \rho}{\Pr}\left[\CanonicalSuccess^{P, C, \hash}(\pi, \rho) = 1\right] \\
  &\IndII \leq
  \mkern33mu
    \underset{
      \mathclap{
        \substack{\pi, \rho \\
        x := \langle P(\pi), C_1(\rho) \rangle_{\mathit{trans}} \\
        (\Gamma_V, \Gamma_H) := \langle P(\pi), C_1(\rho) \rangle_{P}
      }}}
    {\mathbb{E}}\mkern13mu\big[\Gamma_V(\widetilde{C}_2^{\Gamma_H, C_2, \hash}(x, \rho)) = 1\big] \\
  &\IndII =
  \mkern33mu
    \underset{
      \mathclap{
        \substack{\pi, \rho \\
        x := \langle P(\pi), C_1(\rho) \rangle_{\mathit{trans}} \\
        (\Gamma_V, \Gamma_H) := \langle P(\pi), C_1(\rho) \rangle_{P}
      }}}
    {\Pr}\mkern13mu\big[\Gamma_V(\widetilde{C}_2^{\Gamma_H, C_2, \hash}(x, \rho)) = 1\big]
\end{align*}
\end{proof}
%
%
\begin{todo}
  \textbf{TODO:} intuition behind the lemma \\
  \textbf{TODO:} bases on \cite{holenstein2011general}
\end{todo}
The following Lemma is analogous to Theorem 10 from \cite{holenstein2011general}.
\begin{lemma}[Hardness amplification for a dynamic interactive weakly verifiable puzzle with respect to $\hash$]
  \label{lemma:sec_amp_for_p_hash}
  Let $g: \{0,1\}^{k} \rightarrow \{0,1\}$ be a monotone function, $P_n^{(1)}$ a fixed
  problem poser and $\widetilde{C} := (C_1, \widetilde{C}_2)$ a probabilistic two-phase circuit
  with oracle access to a function $\hash: \cQ \rightarrow \{0,1,\dots, 2(h+v)-1\}$
  and a solver $C := (C_1, C_2)$ for $P_{kn}^{(g)}$ that asks at most $h$ hint queries and $v$ verification queries.
  There exists an algorithm Gen that takes as input parameters $\varepsilon$, $\delta$, $n$, $k$,
  has oracle access to $P_n^{(1)}$,  $\widetilde{C}$, $\hash$, $g$,
  and outputs a probabilistic two-phase circuit $D := (D_1, D_2)$ such that the following holds: \\
  If $\widetilde{C}$ is such that
  \begin{align*}
    \underset{\mathclap{\substack{
          \pi^{(k)} \in \{0,1\}^{kn}, \rho \in \{0,1\}^{*} \\
          x:= \langle P^{(g)}(\pi^{(k)}), {C}_1(\rho) \rangle_{\mathit{trans}} \\
          (\Gamma_H^{(k)}, \Gamma_V^{(g)}) := \langle P^{(g)}(\pi^{(k)}), C_1(\rho) \rangle_{P^{(g)}}}}}
    {\Pr}[\Gamma_V^{(g)}(\widetilde{C}_2^{\Gamma_H^{(k)}, C_2, \hash}(x,\rho)) = 1]
    \geq \underset{u \leftarrow \mu_\delta^k}{\Pr}[g(u) = 1] + \varepsilon,
  \end{align*}
  then $D$ satisfies almost surely over the randomness of Gen
  \begin{align*}
    \underset{
      \mathclap{
      \substack{
        \pi \in \{0,1\}^{n}, \rho \in \{0,1\}^{*} \\
        x := \langle P^{(1)}(\pi), D_1^{\widetilde{C}}(\rho) \rangle_{\mathit{trans}} \\
        (\Gamma_H, \Gamma_V) := \langle P^{(1)}(\pi), D_1^{\widetilde{C}}(\rho) \rangle_{P^{(1)}}}}}
    {\Pr}[\Gamma_V(D_2^{P^{(1)}, \widetilde{C}, \hash, g, \Gamma_H}(x, \rho)) = 1] \geq \delta + \frac{\varepsilon}{6k}.
  \end{align*}
  Furthermore, $D$
  asks at most $\frac{6k}{\epsilon}\log\left(\frac{6k}{\epsilon}\right) h$ hint queries and no verification queries.
  Finally, the running time of $\mathit{Gen}$ is polynomial in $k, \frac{1}{\varepsilon}, n$ with oracle calls.
\end{lemma}
We note that the circuit $D$ from Lemma \ref{lemma:sec_amp_for_p_hash} does not ask any verification queries,
instead it outputs a pair $(q, y)$ such that $\hash(q) = 0$ or $\bot$.

Before we give the proof of Lemma \ref{lemma:sec_amp_for_p_hash} we define additional algorithms.
First, in the following code listing the algorithm $\Gen$ from Lemma \ref{lemma:sec_amp_for_p_hash} is defined.
The procedures and circuits used by $\Gen$ are presented on the succeeding code listings.
\begin{codeblock}
  \textbf{Algorithm} $\Gen^{P^{(1)}, \widetilde{C}, g, \mathit{hash}}(\epsilon, \delta, n, k)$
  \medskip \hrule
  \textbf{Oracle:} A poser $P^{(1)}$, a solver $\widetilde{C}$ for $P^{(g)}$, functions $g: \{0,1\}^{k} \rightarrow \{0,1\}$, $hash:\cQ \rightarrow \{0,1, \dots, 2(h + v) - 1\}$. \\
  \textbf{Input:}  Parameters $\epsilon$, $\delta$, $n$, $k$.\\
  \textbf{Output:} A circuit $D$.
  \medskip\hrule
  \For $i:=1$ \To $\frac{6k}{\epsilon}n$ \Do \\
  \IndI $\pi^* \xleftarrow{\$} \{0,1\}^{n}$\\
  \IndI $\widetilde{S}_{\pi^*,0} := \text{EstimateSurplus}^{P^{(1)},  \widetilde{C}, g, hash}(\pi^*, 0, k, \epsilon, \delta,n)$\\
  \IndI $\widetilde{S}_{\pi^*,1} := \text{EstimateSurplus}^{P^{(1)},  \widetilde{C}, g, hash}(\pi^*, 1, k, \epsilon, \delta,n)$\\
  \IndI \If $ \exists b \in \{0,1\}: \widetilde{S}_{\pi^*,b} \geq (1 - \frac{3}{4k}) \epsilon$ \Then \\
  \IndII Let $C_1'$ have oracle access to $\widetilde{C}$, and have hard-coded $\pi^*$. \\
  \IndII Let $\widetilde{C}_2'$ have oracle access to $\widetilde{C}$, and have hard-coded $\pi^*$. \\
  \IndII $\widetilde{C}' := (C_1', \widetilde{C}_2')$ \\
  \IndII $g'(b_2, \dots, b_k) := g(b, b_2, \dots, b_k)$\\
  \IndII\Return $Gen^{P^{(1)}, \widetilde{C}', g', hash}(\epsilon, \delta, n, k-1)$ \\
  \textit{// all estimates are lower than $(1-\frac{3}{4k})\varepsilon$}\\
  \Return $D^{P^{(1)}, \widetilde{C}, hash, g}$
\end{codeblock}
We are interested in the probability that for $u \leftarrow \mu_{\delta}^k$ and a bit $b$ we have $g(b,u_2, \dotsc, u_k) = 1$.
The estimate of this probability is calculated by the algorithm EstimateFunctionProbability.
%
\begin{codeblock}
  \textbf{Algorithm} $\text{EstimateFunctionProbability}^{g}(b, k, \epsilon, \delta, n)$
  \medskip\hrule
  \textbf{Oracle:} A function $g : \{0,1\}^{k} \rightarrow \{0,1\}$.\\
  \textbf{Input:} A bit $b \in \{0,1\}$, parameters $k$, $\epsilon$, $\delta$, $n$. \\
  \textbf{Output:} An estimate $\widetilde{g}_b$ of $\Pr_{u \leftarrow \mu_{\delta}^{k}}[g(b,u_2, \dotsc, u_k) = 1]$.
  \medskip\hrule
  \For $i:=1$ \To $N := \frac{64k^2}{\epsilon^2} n$ \Do \\
  \IndI $u \leftarrow \mu_{\delta}^{k}$ \\
  \IndI $g_i := g(b, u_2, \dotsc, u_k)$ \\
  \Return $\frac{1}{N} \sum_{i=1}^{N} g_i$
\end{codeblock}
%
For fixed $\pi^{(k)}$, $\rho$, and $hash$ we say that the circuit $\widetilde{C} := (C_1, \widetilde{C}_2)$ \textit{succeeds on the $i$-th coordinate}
if for $x := \langle P^{(g)}(\pi^{(k)}), C_1(\rho) \rangle_{\mathit{trans}}$, $(\Gamma_V^{(g)}, \Gamma_H^{(k)}) := \langle P^{(g)}(\pi), C_1(\rho) \rangle_{P^{(g)}}$ and
$(q, y^{(k)}) := \widetilde{C}_2(x, \rho)$ we have
\begin{align*}
  \Gamma_V^i(q, y_i) = 1.
\end{align*}
%
\begin{lemma}
  \label{lemma:estimate_of_g}
  The algorithm $\text{EstimateFunctionProbability}^{g}(b, k, \epsilon, \delta, n)$ outputs an estimate $\widetilde{g}_b$
  such that $| \widetilde{g}_b - \Pr_{u \leftarrow \mu_{\delta}^{k}}\left[g(b,u_2, \dots, u_k) = 1\right] | \leq \frac{\epsilon}{8k}$ almost surely.
\end{lemma}
%
\begin{proof}
We fix notation as in the code excerpt of the algorithm EstimateFunctionProbability.
Let us define independent and identically distributed binary random variables $K_1, K_2, \dots, K_N$
such that for each $1 \leq i \leq N$ the random variable $K_i$ takes value $g_i$. We use the Chernoff bound to obtain
\begin{align*}
  &\Pr \Bigl[ \Bigl| \widetilde{g}_b - \Pr_{u \leftarrow \mu_{\delta}^{k}}\left[g(b,u_2, \dots, u_k) = 1\right] \Bigr| \geq \frac{\epsilon}{8k} \Bigr]\\
  &\IndII = \Pr \Bigl[\Bigl|\Bigl(\frac{1}{N} \sum_{i=1}^N K_i \Bigr) - \mathbb{E}_{u \leftarrow \mu_{\delta}^k}[g(b,u_2, \dots, u_k)]\Bigr|
    \geq \frac{\epsilon}{8k} \Bigr] \leq 2 \cdot e^{-n/3}.
\end{align*}
\end{proof}
%
The algorithm $\text{EvalutePuzzles}^{P^{(1)}, \widetilde{C}, \hash}(\pi^{(k)}, \rho, n, k)$
evaluates which of the $k$ puzzles of the $k$-wise direct product of $P^{(1)}$ are solved successfully by $\widetilde{C}(\rho) := (C_1,\widetilde{C}_2)(\rho)$.
To decide whether the $i$-th puzzle of the $k$-wise direct product is solved successfully we need to gain access to the verification circuit
for the puzzle generated in the $i$-th round of the interaction between $P^{(g)}$ and $\widetilde{C}$.
Therefore, the algorithm EvalutePuzzles runs $k$ times $P^{(1)}(\pi_i)$ to simulate the interaction with
$C_1(\rho)$ where in each round of interaction a fresh random bitstring $\pi_i \in \{0,1\}^{n}$ is used.

Let us introduce the additional notation.
We write $\langle P^{(1)}(\pi_i), C_1(\rho)\rangle^i$ to denote the $i$-th round of the sequential interaction.
Let $\langle P^{(1)}(\pi_i), C_1(\rho)\rangle^i_{P^{(1)}}$ be the output of $P^{(1)}(\pi_i)$ in the $i$-th round.
Finally, we write $\langle P^{(1)}(\pi_i), C_1(\rho)\rangle^i_{\mathit{trans}}$ to denote the transcript of communication in the $i$-th round.
We note that the $i$-th round of the interaction between $P^{(1)}$ and $C_1$ is well defined only if all previous rounds have been executed before.

For simplicity of the notation in the code excerpts of circuits $C_2$, $D_2$, and EvalutePuzzles we omit superscripts of some oracles.

Exemplary, for $\widetilde{C}_2^{\Gamma_H^{(k)}, C, \hash}$ we omit the superscript $C$ and instead write $\widetilde{C}_2^{\Gamma_H^{(k)}, \hash}$.
We make sure that it is clear from the context which oracles are used.

\begin{todo}
  \textbf{TODO:} Introduce this algorithm
\end{todo}

\begin{codeblock}
  \textbf{Algorithm} $\text{EvaluatePuzzles}^{P^{(1)}, \widetilde{C}, \hash}(\pi^{(k)}, \rho, n, k)$
  \medskip \hrule
  \textbf{Oracle:}  A problem poser $P^{(1)}$, a solver circuit $\widetilde{C} = (C_1, \widetilde{C}_2)$ for $P^{(g)}$,\\
  \IndII a function $hash : \cQ \rightarrow \{0,1,\dots, 2(h+v)-1\}$.\\
  \textbf{Input:} Bitstrings $\pi^{(k)} \in \{0,1\}^{kn}$, $\rho \in \{0,1\}^{*}$, parameters $n$, $k$.\\
  \textbf{Output}: A tuple $(c_1, \dots, c_k) \in \{0,1\}^{k}$.
  \medskip\hrule
  %
  \For $i:=1$ \To $k$ \Do \IndII \textit{//simulate $k$ rounds of interaction} \\
  \IndI $(\Gamma_V^{i}, \Gamma_H^{i}) := \langle P^{(1)}(\pi_i), C_1(\rho) \rangle_{P^{(1)}}^i$\\
  \IndI $x_i := \langle P^{(1)}(\pi_i), C_1(\rho) \rangle^i_{\mathit{trans}}$ \\
  $x := (x_1, \dots, x_k)$ \\
  $\Gamma_H^{(k)} := (\Gamma_H^1, \dotsc, \Gamma_H^k)$ \\
  $(q, y_1, \dots, y_k) := \widetilde{C}_2^{\Gamma_H^{(k)}, hash} (x, \rho)$ \\
  \If $(q, y_1, \dots, y_k) = \bot$ \Then \\
  \IndI \Return $(0, \dotsc, 0)$ \\
  $(c_1, \dotsc, c_k) := (\Gamma_V^{1}(q, y_1), \dotsc, \Gamma_V^{k}(q, y_k))$\\
  \Return $(c_1, \dotsc, c_k)$
\end{codeblock}
%
All puzzles used by EvalutePuzzles are generated internally. Thus, the algorithm can answer all queries of $\widetilde{C}_2$ itself.

We are interested in the success probability of $\widetilde{C}$ with the bitstring $\pi_1$ fixed to $\pi^*$ where
the fact whether $\widetilde{C}$ succeeds in solving the input puzzle defined by $P^{(1)}(\pi_1)$ placed on the first position is neglected,
and instead a bit $b$ is used. More formally, we define the surplus $S_{\pi^*, b}$ as
\begin{align}
  \label{eq:s_pi_b}
S_{\pi^*, b} = \underset{\pi^{(k)}, \rho}{\Pr}\left[g(b, c_2, \dots, c_k) = 1 \mid \pi_1 = \pi^*\right] - \underset{u \leftarrow \mu^{k}_{\delta}}{\Pr}\left[g(b, u_2, \dots, u_k) = 1\right],
\end{align}
where $(c_2, c_3, \dotsc, c_k)$ is obtained as in EvalutePuzzles.

The algorithm EstimateSurplus returns an estimate $\widetilde{S}_{\pi^*, b}$ for $S_{\pi^*, b}$.
%
\begin{codeblock}
  \textbf{Algorithm} $\text{EstimateSurplus}^{P^{(1)}, \widetilde{C}, g, \hash}(\pi^*, b, k, \epsilon, \delta, n)$
  \medskip\hrule
  \textbf{Oracle:} A problem poser $P^{(1)}$, a circuit $\widetilde{C}$ for $P^{(g)}$, functions \\
  \IndII $g: \{0,1\}^{k} \rightarrow \{0,1\}$ and  $\hash : \cQ \rightarrow \{0,1,\dots, 2(h+v)-1\}$.\\
  \textbf{Input:} A bistring $\pi^* \in \{0,1\}^{n}$, a bit $b \in \{0,1\}$, parameters $k$, $\epsilon$, $\delta$, $n$.\\
  \textbf{Output:} An estimate $\widetilde{S}_{\pi^*, b}$ for $S_{\pi^*, b}$.
  \medskip\hrule
  \For $i:=1$ \To $N := \frac{64k^2}{\epsilon^2}n$ \Do \\
  \IndI $(\pi_{2}, \dots, \pi_k) \xleftarrow{\$} \{0,1\}^{(k-1)n}$\\
  \IndI $\rho \xleftarrow{\$} \{0,1\}^{*}$\\
  \IndI $(c_1, \dots, c_k) := \text{EvalutePuzzles}^{P^{(1)}, \widetilde{C}, \hash}((\pi^*, \pi_2, \dots, \pi_k), \rho, n, k)$\\
  \IndI $\widetilde{s}_{\pi^*,b}^i := g(b, c_{2}, \dots, c_k)$\\
  $\widetilde{g}_b := \text{EstimateFunctionProbability}^{g}(b, k, \epsilon, \delta, n)$ \\
  \textbf{return} $\Bigl(\frac{1}{N} \sum_{i=1}^N \widetilde{s}_{\pi^*,b}^i \Bigr) - \widetilde{g}_b$
\end{codeblock}
%
\begin{lemma}
  \label{lemma:surplus_estimate}
The estimate $\widetilde{S}_{\pi^*,b}$ returned by EstimateSurplus differs from $S_{\pi^*, b}$ by at most $\frac{\epsilon}{4k}$ almost surely.
\end{lemma}

\begin{proof}
We use the union bound and similar argument as in Lemma \ref{lemma:estimate_of_g}
which yields that $\frac{1}{N} \sum_{i=1}^{N} \widetilde{s}_{\pi^*,b}^i$ differs from
$\mathbb{E}[g(b, c_2, \dots, c_k)]$ by at most $\frac{\epsilon}{8k}$ almost surely. Together, with Lemma $\ref{lemma:estimate_of_g}$ we conclude that the surplus estimate
returned by EstimateSurplus differs from $S_{\pi^*,b}$ by at most $\frac{\epsilon}{4k}$ with probability at least $1 - 2e^{-n}$.
\end{proof}
%
We define the following solver circuit $C' = (C_1', C_2')$ for the $(k-1)$--wise direct product of $P^{(1)}$.
\begin{todo}
  \textbf{TODO:} Give more intuition why we need this circuit and where it is used
\end{todo}
\begin{codeblock}
  \textbf{Circuit} $C_1'^{\widetilde{C}, P^{(1)}}(\rho)$
  \medskip \hrule
  \textbf{Oracle:} A solver circuit $\widetilde{C} = (C_1, \widetilde{C}_2)$ for $P^{(g)}$, a poser $P^{(1)}$. \\
  \textbf{Input:}  A bitstring $\rho \in \{0,1\}^{*}$. \\
  \textbf{Hard-coded:} A bitstring $\pi^* \in \{0,1\}^{n}$.
  \medskip\hrule
  Simulate $\langle P^{(1)}(\pi^*), C_1(\rho)\rangle^1$ \\
  Use $C_1(\rho)$ for the remaining $k-1$ rounds of interaction.
\end{codeblock}
%
\begin{codeblock}
  \textbf{Circuit} $\widetilde{C}_2'^{\Gamma_H^{(k-1)}, \widetilde{C}, \hash}(x^{(k-1)}, \rho)$
  \medskip \hrule
  \textbf{Oracle:} A hint oracle $\Gamma_H^{(k-1)} := (\Gamma_H^{2}, \dots, \Gamma_H^{k})$,\\
  \IndII a solver circuit $\widetilde{C} = (C_1, \widetilde{C}_2)$ for $P^{(g)}$, \\
  \IndII a function $\hash: \cQ \rightarrow \{0,1,\dots, 2(h+v)-1\}$. \\
  \textbf{Input:}  A transcript of $k-1$ rounds of interaction \\
  \IndII $x^{(k-1)} := (x_2, \dotsc, x_{k}) \in \{0,1\}^{*}$, a bitstring $\rho \in \{0,1\}^{*}$.\\
  \textbf{Hard-coded:} A bitstring $\pi^* \in \{0,1\}^{n}$. \\
  \textbf{Output:} A tuple $(q, y_2, \dots, y_k)$.
  \medskip\hrule
  Simulate $\langle P^{(1)}(\pi^*), C_1(\rho) \rangle^{1}$ \\
  \IndI $(\Gamma_H^*, \Gamma_V^*) := \langle P^{(1)}(\pi^*), C_1(\rho) \rangle^{1}_{P^{(1)}}$ \\
  \IndI $x^* := \langle P^{(1)}(\pi^*), C_1(\rho) \rangle^{1}_{\mathit{trans}}$ \\
  $\Gamma_H^{(k)} := (\Gamma_H^*, \Gamma_H^{2}, \dots, \Gamma_H^{k})$ \\
  $x^{(k)} := (x^*, x_2, \dots, x_{k})$ \\
  $(q, y_1, \dots, y_k) := \widetilde{C}_2^{\Gamma_H^{(k)}, \mathit{hash}}(x^{(k)}, \rho)$ \\
  \Return $(q, y_2, \dots, y_k)$
\end{codeblock}
%
We are ready to define the solver circuit $D = (D_1, D_2)$ for $P^{(1)}$ output by $\Gen$.
%
\begin{codeblock}
  \textbf{Circuit} $D_1^{\widetilde{C}}(r)$
  \medskip \hrule
  \textbf{Oracle:} A solver circuit $\widetilde{C} = (C_1, \widetilde{C}_2)$ for $P^{(g)}$.\\
  \textbf{Input:} A pair $r := (\rho, \sigma)$ where $ \rho \in \{0,1\}^{*}$ and $\sigma \in \{0,1\}^{*}$.
  \medskip\hrule
  Interact with the problem poser $\langle P^{(1)}, C_1(\rho) \rangle^1$. \\
  Let $x^* := \langle P^{(1)}, C_1(\rho) \rangle^1_{\mathit{trans}}$.
\end{codeblock}
%
\begin{codeblock}
  \textbf{Circuit} $D_2^{P^{(1)}, \widetilde{C}, \mathit{hash}, g,  \Gamma_H}(x^*, r)$
  \medskip \hrule
  \textbf{Oracle:} A poser $P^{(1)}$, a solver circuit $\widetilde{C} = (C_1, \widetilde{C}_2)$ for $P^{(g)}$, \\
  \IndII functions $hash : \cQ \rightarrow \{0,1, \dots, 2(h+v)-1\}$, $g:\{0,1\}^k \rightarrow \{0,1\}$, \\
  \IndII a hint circuit $\Gamma_H$ for $P^{(1)}$. \\
  \textbf{Input:} A communiation transcript $x^* \in \{0,1\}^{*}$, a bitstring $r := (\rho, \sigma)$ \\
  \IndII where $\rho \in \{0,1\}^{*}$ and $\sigma \in \{0,1\}^{*}$\\
  \textbf{Output}: A pair $(q, y^*)$.
  \medskip \hrule
  \For at most $\frac{6k}{\epsilon} \log(\frac{6k}{\epsilon})$ iterations \Do \\
  \IndI $(\pi_2, \dots, \pi_k) \leftarrow$ read next $(k-1)\cdot n$ bits from $\sigma$ \\
  \IndI Use $x^*$ to simulate the first round of interaction of $C_1(\rho)$ \\
  \IndI with the problem poser $P^{(1)}$.\\
  \IndI \For $i:=2$ \To $k$ \Do \\
  \IndII \Run $\langle P^{(1)}(\pi_i), C_1(\rho)\rangle^i$ \\
  \IndIII $(\Gamma_V^{i}, \Gamma_H^{i}) := \langle P^{(1)}(\pi_i), C_1(\rho) \rangle^i_{P^{(1)}}$ \\
  \IndIII $x_i := \langle P^{(1)}(\pi_i), C_1(\rho) \rangle^i_{\mathit{trans}}$ \\
  \IndI $\Gamma_H^{(k)}(q) := (\Gamma_H(q), \Gamma_H^{2}(q), \dots, \Gamma_H^{k}(q))$ \\
  \IndI $(q, y^*, y_2, \dots, y_k) := \widetilde{C}_2^{\Gamma_H^{(k)}, \hash}((x^*, x_2, \dotsc, x_k), \rho)$\\
  \IndI $(c_2, \dots, c_k) := (\Gamma_V^2(q, y_2), \dotsc, \Gamma_V^{k}(q, y_k))$ \\
  \IndI \If $g(1, c_{2}, \dots, c_k) = 1$ \And $g(0,c_{2}, \dots, c_k) = 0$ \Then \\
  \IndII \Return $(q, y^*)$ \\
  \Return $\bot$
%
\end{codeblock}
%
%
\begin{proof}[of Lemma \ref{lemma:sec_amp_for_p_hash}]
First, let us consider the case where $k=1$. The function $g: \{0,1\} \rightarrow \{0,1\}$ is either the identity or a constant function.
In the latter case, when $g$ is a constant function, Lemma \ref{lemma:sec_amp_for_p_hash} is vacuously true.
If $g$ is the identity function, then the circuit $D$ returned by Gen directly uses $\widetilde{C}$ to find a solution.
From the assumptions of Lemma \ref{lemma:sec_amp_for_p_hash} it follows that $\widetilde{C}$ succeeds with probability at least
$\delta + \epsilon$. Hence, $D$ trivially satisfies Lemma~\ref{lemma:sec_amp_for_p_hash}.

For the general case, we consider two possibilities.
Namely, either Gen in one of the iterations finds an estimate with high surplus such that $\widetilde{S}_{\pi, b} \geq (1-\frac{3}{4k})\epsilon$ and recurses,
or in all iterations it fails and outputs the circuit~$D$.

If it is possible to find an estimate with high surplus, then we introduce a new monotone function $g': \{0,1\}^{k-1} \rightarrow \{0,1\}$
such that $g'(b_2, \dots, b_k) := g(b, b_2, \dots, b_k)$ and a new circuit $\widetilde{C}' = (C_1', \widetilde{C}_2')$
with oracle access to $\widetilde{C} := (C_1, \widetilde{C}_2)$.
W apply Lemma \ref{lemma:surplus_estimate} and conclude that almost surely it holds
\begin{align*}
S_{\pi^*,b} \geq \widetilde{S}_{\pi^*, b} - \frac{\epsilon}{4k} \geq \Bigl(1 - \frac{1}{k}\Bigr)\epsilon.
\end{align*}
It follows that $\widetilde{C}'$ succeeds in solving the $(k\!-\!1)$--wise direct product of puzzles with probability at least
\begin{align*}
\Pr_{u \leftarrow \mu^{(k-1)}_{\delta}}[g'(u_1,\dots, u_{k-1} )] + \Bigl(1 - \frac{1}{k}\Bigr)\epsilon.
\end{align*}
We see that in this case $\widetilde{C}'$ satisfies the conditions of Lemma \ref{lemma:sec_amp_for_p_hash} for the $(k\!-\!1)$--wise direct product of puzzles.
Therefore, the recursive call to Gen with access to $g'$ and $\widetilde{C}$ returns $D = (D_1, D_2)$ that with high probability satisfies
\begin{align}
  \underset{
    \mathclap{
      \substack{
        \pi, \rho \\
        x := \langle P^{(1)}(\pi), D_1^{\widetilde{C}}(\rho) \rangle_{\mathit{trans}} \\
        (\Gamma_H, \Gamma_V) := \langle P^{(1)}(\pi), D_1^{\widetilde{C}}(\rho) \rangle_{P^{(1)}}}}}
  {\Pr}\Big[\Gamma_V\big(D_2^{P^{(1)}, \widetilde{C}, \hash, g, \Gamma_H}(x, \rho)\big) = 1\Big]
  &\geq \delta + \Bigl(1 - \frac{1}{k}\Bigr)\frac{\epsilon}{6(k-1)} \notag\\
  &= \delta + \frac{\epsilon}{6k}.
\end{align}
%
Let us bring our attention to the case where none of the estimates is greater than $(1-\frac{3}{4k})\epsilon$.
If all surpluses $S_{\pi,0}$ and $S_{\pi,1}$ were lower than $(1-\frac{1}{k})\epsilon$, then it would mean that $\widetilde{C}$
does not succeed on the remaining $k-1$ puzzles with much higher probability than an algorithm that solves each puzzle
independently with success probability $\delta$. However, from the assumptions of Lemma~\ref{lemma:sec_amp_for_p_hash}
we know that on all $k$ puzzles the success probability of $\widetilde{C}$ is higher.
Hence, we suspect that the first puzzle is correctly solved unusually often.
It remains to show that the fact that $\Gen$ fails to find a surplus estimate that is large implies that
with high probability there are only few surpluses greater than $(1-\frac{1}{k})\epsilon$ and their influence
is can be neglected. Additionally, we have to show that the circuit $D$ finds outputs an answer almost surely.

We fix notation as in the code listing of the circuit $D_2$.
Let us consider a single iteration of the outer loop of $D_2$ where values $\pi_2, \dotsc, \pi_k$ are fixed.
We write $\pi_1$ to denote randomness used by the problem poser to generate the input puzzle.
Furthermore, we define $c_1 := \Gamma_V(q,y_1)$ where $\Gamma_V$ is the verification circuit generated
by $P^{(1)}(\pi_1)$ in the first phase when interacting with $D_1(r)$.
We write $c := (c_1, c_2, \dotsc, c_k)$, and for $b \in \{0,1\}$ we define a set
\begin{align*}
\cG_{b}~:=~\big\{(b_1, b_2, \dots, b_k) : g(b, b_2, \dots, b_k) = 1 \big\}.
\end{align*}
Using this notation we express
\begin{align}
  \label{eqs:set_g}
  \underset{u \leftarrow \mu_{\delta}^k}{\Pr}[u \in \cG_b] = \underset{u \leftarrow \mu_{\delta}^k}{\Pr}[g(b, u_2, \dots, u_k) = 1]\notag\\
 \underset{\pi^{(k)}, \rho}{\Pr}[c \in \cG_b] = \underset{\pi^{(k)}, \rho}{\Pr}[g(b, c_2, \dots, c_k) = 1].
\end{align}
Let us fix randomness $\pi_1$ used by the problem poser to generate the input puzzle to $\pi^*$.
We use \eqref{eq:s_pi_b} and \eqref{eqs:set_g} to obtain
\begin{multline}
\label{eq:diff_s01}
\underset{u \leftarrow \mu_{\delta}^k}{\Pr}[u \in \cG_1] - \underset{u \leftarrow \mu_{\delta}^k}{\Pr}[u \in \cG_0] \\
 = \underset{\pi^{(k)}, \rho}{\Pr}[c \in \cG_1 \mid \pi_1 = \pi^*] - \underset{\pi^{(k)}, \rho}{\Pr}[c \in \cG_0 \mid \pi_1 = \pi^*] - (S_{\pi^*, 1} - S_{\pi^*,0})
\end{multline}
By monotonicity of $g$ it holds $\cG_0 \subseteq \cG_1$, and we write \eqref{eq:diff_s01} as
\begin{align}
  \label{eq:diff_s01_next}
  \underset{u \leftarrow \mu_{\delta}^k}{\Pr}[u \in \cG_1 \setminus \cG_0] = \underset{\pi^{(k)}, \rho}{\Pr}[c \in \cG_1 \setminus \cG_0 \mid \pi_1 = \pi^*] - (S_{\pi^*,1} - S_{\pi^*,0}).
\end{align}
Let us multiply both sides of \eqref{eq:diff_s01_next} by
\begin{align*}
\underset{
  \mathclap{
    \substack{r \\ x^* := \langle P^{(1)}(\pi^*), D_1(r) \rangle_{\mathit{trans}}
    \\ (\Gamma_V, \Gamma_H) := \langle P^{(1)}(\pi^*), D_1(r) \rangle_{P^{(1)}} }}}
{\Pr}\mkern13mu [\Gamma_V(D_2(x^*, r)) = 1]
 \mkern11mu / \underset{u \leftarrow \mu_{\delta}^k}{\Pr}[ u \in \cG_1 \setminus\cG_0],
\end{align*}
%
which yields
\begin{align}
\label{eq:pr_d_succ_0}
&\IndII\underset{
  \mathclap{
    \substack{r \\ x^* := \langle P^{(1)}(\pi^*), D_1(r) \rangle_{\mathit{trans}} \\ (\Gamma_V, \Gamma_H) := \langle P^{(1)}(\pi^*), D_1(r) \rangle_{P^{(1)}} }}}
{\Pr}\mkern13mu[\Gamma_V(D_2(x^*, r)) = 1] \notag\\
%
&\IndIII = \mkern13mu
  \underset{
    \mathclap{
      \substack{r \\ x^* := \langle P^{(1)}(\pi^*), D_1 (r) \rangle_{\mathit{trans}} \\ (\Gamma_V, \Gamma_H) := \langle P^{(1)}(\pi^*), D_1 (r) \rangle_{P^{(1)}} }}}
  {\Pr}\mkern13mu[\Gamma_V(D_2(x^*, r)) = 1]
  \underset{\pi^{(k)},\rho}{\Pr}[c \in \cG_1 \setminus \cG_0 \mid \pi_1 = \pi^*]
\frac{1}{\underset{u \leftarrow \mu_{\delta}^k}{\Pr}[ u \in \cG_1 \setminus \cG_0]}\notag\\
%
&\IndIIII - \mkern13mu
\underset{
  \mathclap{
  \substack{r \\ x^* := \langle P^{(1)}(\pi^*), D_1(r) \rangle_{\mathit{trans}} \\ (\Gamma_V, \Gamma_H) := \langle P^{(1)}(\pi^*), D_1(r) \rangle_{P^{(1)}} }}}
{\Pr}\mkern13mu[\Gamma_V(D_2(x^*, r)) = 1](S_{\pi^*,1} - S_{\pi^*,0})
\frac{1}{\underset{u \leftarrow \mu_{\delta}^k}{\Pr}[ u \in \cG_1 \setminus\cG_0]}.
\end{align}
Let us study the first summand of \eqref{eq:pr_d_succ_0}. First, we have
\begin{align}
  \label{eq:pr_gamma_v_0}
  \IndII &\underset{
    \mathclap{
      \substack{r \\
        x^* := \langle P^{(1)}(\pi^*), D_1 (r) \rangle_{\mathit{trans}} \\
        (\Gamma_V, \Gamma_H) := \langle P^{(1)}(\pi^*), D_1(r) \rangle_{P^{(1)}} }}}
  {\Pr}\mkern13mu[\Gamma_V(D_2(x^*, r)) = 1] \notag\\
  &\IndI = \underset{
    \mathclap{
      \substack{r \\
        x^* := \langle P^{(1)}(\pi^*), D_1 (r) \rangle_{\mathit{trans}} \\
        (\Gamma_V, \Gamma_H) := \langle P^{(1)}(\pi^*), D_1(r) \rangle_{P^{(1)}} }}}
  {\Pr}[\Gamma_V(D_2(x^*, r)) = 1 | D_2(x^*,r) \neq \bot]
  \underset{\mathclap{\substack{r \\ x^* = \langle P^{(1)}(\pi^*), D_1(r) \rangle_{\mathit{trans}}}}}{\Pr}[D_2(x^*,r) \neq \bot] \notag\\
  &\IndI \stackrel{(*)}{=}
  \underset{\pi^{(k)}, \rho}{\Pr}[c_1 = 1 \mid c \in \cG_1 \setminus \cG_0, \pi_1 = \pi^*]
  \underset{\mathclap{\substack{r \\ x^* = \langle P^{(1)}(\pi^*), D_1(r) \rangle_{\mathit{trans}}}}} {\Pr}[D_2(x^*,r) \neq \bot]
\end{align}
where in $(*)$ we use the observation that $D_2(x^*, r) \neq \bot$ occurs if and only if $D_2(x^*, r)$ finds $\pi^{(k)}$ such that $c \in \cG_1 \setminus \cG_0$.
Furthermore, conditioned on $c \in \cG_1 \setminus \cG_0$ we have that $\Gamma_V(D_2(x^*,r)) = 1$ happens if and only if $c_1 = 1$.
Inserting \eqref{eq:pr_gamma_v_0} to the numerator of the first summand of (\ref{eq:pr_d_succ_0}) yields
\begin{align}
  \label{ineq:start_for_case}
\IndI &\underset{
  \mathclap{
  \substack{r \\
    x^* := \langle P^{(1)}(\pi^*), D_1 (r) \rangle_{\mathit{trans}} \\
    (\Gamma_V, \Gamma_H) := \langle P^{(1)}(\pi^*), D_1(r) \rangle_{P^{(1)}} }}}
{\Pr}\mkern13mu[\Gamma_V(D_2(x^*, r)) = 1]
\underset{\pi^{(k)},\rho}{\Pr}[c \in \cG_1 \setminus \cG_0 \mid \pi_1 = \pi^*] \notag\\
  &\IndI = \underset{\mathclap{\substack{r
      \\ x^* = \langle P^{(1)}(\pi^*), D_1(r) \rangle_{\mathit{trans}}}}}
  {\Pr}\mkern13mu[D_2(x^*,r) \neq \bot]
  \underset{\pi^{(k)}, \rho}{\Pr}[c_1 = 1 \mid c \in \cG_1 \setminus \cG_0, \pi_1 = \pi^*]
  \underset{\pi^{(k)}, \rho}{\Pr}[c \in \cG_1 \setminus \cG_0 \mid \pi_1 = \pi^*].
\end{align}
We consider the following two cases. First, if $\Pr_{\pi^{(k)}, \rho}[ c \in \cG_1 \setminus \cG_0 \mid \pi_1 = \pi^*] \leq \frac{\epsilon}{6k}$ then
\begin{align}
  \label{ineq:case_0}
  \underset{\pi^{(k)}, \rho}{\Pr}[c_1 = 1 \mid c \in \cG_1 \setminus \cG_0, \pi_1 = \pi^*] \underset{\pi^{(k)}, \rho}{\Pr}[c \in \cG_1 \setminus \cG_0 \mid \pi_1 = \pi^*] \leq \frac{\epsilon}{6k}.
\end{align}
In case when $\Pr_{\pi^{(k)}, \rho}[c \in \cG_1 \setminus \cG_0 \mid \pi_1 = \pi^*] > \frac{\epsilon}{6k}$ the circuit $D_2$ outputs $\bot$
if and only if it fails in all $\frac{6k}{\epsilon} \log(\frac{6k}{\epsilon})$ iterations to find $\pi^{(k)}$ such that $c \in \cG_1 \setminus \cG_0$
which happens with probability
\begin{align}
  \label{ineq:case_1}
\underset{
  \mathclap{
    \substack{
      r \\
      x^* := \langle P^{(1)}(\pi^*), D_1(r) \rangle_{\mathit{trans}}}}}
{\Pr}[D_2(x^*,r) = \bot]
\leq \Big(1 - \frac{\epsilon}{6k}\Big)^{\frac{6k}{\epsilon}\log(\frac{6k}{\epsilon})} \leq \frac{\epsilon}{6k}.
\end{align}
We conclude that in both aforementioned cases using \eqref{ineq:start_for_case}, \eqref{ineq:case_0} and \eqref{ineq:case_1} the following holds
\begin{align}
  \label{ineq:first_part}
  &\underset{
    \mathclap{
    \substack{r \\
      x^* := \langle P^{(1)}(\pi^*), D_1(r) \rangle_{\mathit{trans}}}}}
  {\Pr}\mkern13mu[D_2(x^*,r) \neq \bot]
  \underset{\pi^{(k)}, \rho}{\Pr}[c_1 = 1 \mid c \in \cG_1 \setminus \cG_0, \pi_1 = \pi^*]
  \underset{\pi^{(k)}, \rho}{\Pr}[c \in \cG_1 \setminus \cG_0 \mid \pi_1 = \pi^*] \notag\\
  &\IndII \stackrel{\hphantom{(\ref{eq:s_pi_b})}}{\geq}
  \underset{\pi^{(k)}, \rho}{\Pr}[c_1 = 1 \mid c \in \cG_1 \setminus \cG_0, \pi_1 = \pi^*]\underset{\pi^{(k)}, \rho}
  {\Pr}[c \in \cG_1 \setminus \cG_0 \mid \pi_1 = \pi^*] - \frac{\epsilon}{6k} \notag\\
  &\IndII \stackrel{\hphantom{(\ref{eq:s_pi_b})}}{=}
  \underset{\pi^{(k)}, \rho}{\Pr}[c_1 = 1 \land c \in \cG_1 \setminus \cG_0 \mid \pi_1 = \pi^*] - \frac{\epsilon}{6k} \notag\\
  &\IndII \stackrel{\hphantom{(\ref{eq:s_pi_b})}}{=}
  \underset{\pi^{(k)}, \rho}{\Pr}[g(c) = 1 \mid \pi_1 = \pi^*] -  \underset{\pi^{(k)}, \rho}{\Pr}[c \in \cG_0 \mid \pi_1 = \pi^*] - \frac{\epsilon}{6k} \notag\\
  &\IndII \stackrel{(\ref{eq:s_pi_b})}{=}
   \underset{\pi^{(k)}, \rho}{\Pr}[g(c) = 1 \mid \pi_1 = \pi^*] -  \underset{u \leftarrow \mu_{\delta}^{(k)}}{\Pr}[u \in \cG_0]  - S_{\pi^*,0} - \frac{\epsilon}{6k}.
\end{align}
We insert \eqref{ineq:first_part} into \eqref{eq:pr_d_succ_0} and calculate the expected value over $\pi^*$ which yields
\begin{align}
  \label{ineq:ex_pr_gamma}
\underset{
  \mathclap{
    \substack{\pi, r \\ x := \langle P^{(1)}(\pi), D_1(r) \rangle_{\mathit{trans}} \\ (\Gamma_V, \Gamma_H) := \langle P^{(1)}(\pi), D_1(r) \rangle_{P^{(1)}} }}}
{\Pr}[\Gamma_V(D_2(x, r)) = 1]
&\geq \mathbb{E}_{\pi^*}\left[\frac{\Pr_{\pi^{(k)}, \rho}[g(c) = 1 | \pi_1 = \pi^*]
  - \Pr_{u \leftarrow \mu_{\delta}^{(k)}}[u \in \cG_0] - \frac{\epsilon}{6k}}{\Pr_{u \leftarrow \mu_{\delta}^{(k)}}[u \in \cG_1 \setminus \cG_0]}\right] \notag\\
&- \mathbb{E}_{\pi^*}\Bigl[\Bigl(
\underset{\mathclap{
  \substack{r \\ x^* := \langle P^{(1)}(\pi^*), D_1(r) \rangle_{\mathit{trans}} \\ (\Gamma_V, \Gamma_H) := \langle P^{(1)}(\pi^*), D_1(r) \rangle_{P^{(1)}} }}}
{\Pr}[\Gamma_V(D_2(x^*, r)) = 1](S_{\pi^*,1} - S_{\pi^*,0})
 + S_{\pi^*,0}\Bigr)
\frac{1}{\underset{u \leftarrow \mu_{\delta}^k}{\Pr}[ u \in \cG_1 \setminus\cG_0]}\Bigr].
\end{align}
We now show that if Gen does not recurse, then the majority of estimates is low almost surely.
Let us assume that
\begin{align}
\underset{\pi}{\Pr}\left[\left(S_{\pi,0} \leq (1 - \frac{1}{2k})\epsilon\right) \land \left( S_{\pi,1} \leq (1-\frac{1}{2k})\epsilon\right)\right] < 1 - \frac{\epsilon}{6k},
\end{align}
then Gen recurses almost surely, because the probability that
Gen does not find $\widetilde{S}_{\pi, b} \geq (1-\frac{3}{4k})\epsilon$ in all of the $\frac{6k}{\epsilon}n$ iterations is at most
\begin{align*}
  \Bigl(1 - \frac{\epsilon}{6k}\Bigr)^{\frac{6k}{\epsilon}n} \leq e^{-n}
\end{align*}
almost surely, where we used Lemma \ref{lemma:surplus_estimate}.
Therefore, under the assumption that Gen does not recurse with high probability it holds
\begin{align}
\underset{\pi, \rho}{\Pr}\left[\left(S_{\pi,0} \leq (1 - \frac{1}{2k})\epsilon\right) \land \left( S_{\pi,1} \leq (1-\frac{1}{2k})\epsilon\right)\right] \geq 1 - \frac{\epsilon}{6k}.
\end{align}
Let us define a set
\begin{align}
  \cW = \left\{ \pi :  \left(S_{\pi,0} \leq (1 - \frac{1}{2k})\epsilon\right) \land \left( S_{\pi,1} \leq (1-\frac{1}{2k})\epsilon \right) \right\}.
\end{align}
Additionally, let $\overline{\cW}$ denote the complement of $\cW$.
We bound the numerator of the second summand in (\ref{ineq:ex_pr_gamma})
\begin{align}
  \label{ineq:second_eq}
&\mathbb{E}_{\pi^*}\Big[ S_{\pi^*,0}
\mkern23mu
+
\mkern23mu
\underset{
  \mathclap{
  \substack{r \\ x^* := \langle P^{(1)}(\pi^*), D_1(r) \rangle_{\mathit{trans}}
    \\ (\Gamma_V, \Gamma_H) := \langle P^{(1)}(\pi^*), D_1 (r) \rangle_{P^{(1)}} }}}
{\Pr}\mkern13mu[\Gamma_V(D_2(x^*, r)) = 1]
(S_{\pi^*,1} - S_{\pi^*,0})\Big] \notag\\
%
&\IndII = \mathbb{E}_{\pi^*}\Bigl[ S_{\pi^*,0}
\mkern23mu + \mkern23mu
\underset{
  \mathclap{
  \substack{r \\ x^* := \langle P^{(1)}(\pi^*), D_1(r) \rangle_{\mathit{trans}}
    \\ (\Gamma_V, \Gamma_H) := \langle P^{(1)}(\pi^*), D_1 (r) \rangle_{P^{(1)}} }}}
{\Pr}\mkern13mu[\Gamma_V(D_2(x^*, r) = 1]
  (S_{\pi^*,1} - S_{\pi^*,0}) \bigm| \pi^* \in \overline{\cW}\Bigr] \Pr_{\pi^*}[\pi^* \in \overline{\cW}]\notag\\
&\IndIII +  \mathbb{E}_{\pi^*}\Bigl[ S_{\pi^*,0} \mkern13mu + \mkern13mu
\underset{
  \mathclap{
  \substack{r \\ x^* := \langle P^{(1)}(\pi^*), D_1(r) \rangle_{\mathit{trans}}
    \\ (\Gamma_V, \Gamma_H) := \langle P^{(1)}(\pi^*), D_1 (r) \rangle_{P^{(1)}} }}}
{\Pr}\mkern13mu[\Gamma_V(D_2(x^*, r)) = 1]
(S_{\pi^*,1} - S_{\pi^*,0})  \bigm| \pi^* \in \cW\Bigr] \Pr_{\pi^*}[\pi^* \in \cW] \notag\\
&\IndII \leq \frac{\epsilon}{6k} + \mathbb{E}_{\pi^*}\Bigl[ S_{\pi^*,0} \mkern23mu + \mkern23mu
\underset{
  \mathclap{
  \substack{r \\ x := \langle P^{(1)}(\pi^*), D_1(r) \rangle_{\mathit{trans}}
    \\ (\Gamma_V, \Gamma_H) := \langle P^{(1)}(\pi^*), D_1 (r) \rangle_{P^{(1)}} }}}
{\Pr}\mkern13mu\big[\Gamma_V(D_2^{\widetilde{C}}(x^*, r)) = 1\big]
\big(\bigl(1 - \frac{1}{2k}\bigr)\epsilon - S_{\pi^*,0}\big)  \bigm| \pi^* \in \cW \Bigr] \notag\\
% \Pr_{\pi^*}[\pi^* \in \cW] \notag\\
&\IndII \leq \frac{\epsilon}{6k} + (1 - \frac{1}{2k})\epsilon = (1 - \frac{1}{3k}) \epsilon.
\end{align}
Finally, we insert \eqref{ineq:second_eq} into \eqref{ineq:ex_pr_gamma} which yields
\begin{align*}
  \IndI
\underset{
  \mathclap{
  \substack{\pi, \rho \\ x := \langle P^{(1)}(\pi), D_1(\rho) \rangle_{\text{trans}}
    \\ (\Gamma_V, \Gamma_H) := \langle P^{(1)}(\pi), D_1 (\rho) \rangle_{P^{(1)}} }}}
{\Pr}\big[\Gamma_V(D_2(x, \rho)) = 1\big]
&\geq \underset{\pi^*}{\mathbb{E}}\left[\frac{{\Pr}_{\pi^{(k)}, \rho}[g(c) = 1 \mid \pi_1 = \pi^*] -
{\Pr}_{u \leftarrow \mu_{\delta}^{k}}[u \in G_0] - (1 - \frac{1}{6k})\epsilon} {\Pr_{u \leftarrow \mu_{\delta}^{k}}[u \in \cG_1 \setminus \cG_0]}\right] \notag.
 \end{align*}
 From the assumptions of Lemma \ref{lemma:sec_amp_for_p_hash} it follows that
 \begin{align}
   \label{eq:lemma_assum}
   \Pr_{\pi^{(k)}, \rho} [g(c) = 1] \geq \Pr_{u \leftarrow \mu_{\delta}^{(k)}}[g(u) = 1] + \epsilon.
 \end{align}
We observe that
\begin{align}
  \label{eq:gu_rel}
\underset{u \leftarrow \mu_{\delta}^k}{\Pr}[g(u) = 1]
&= \Pr[u \in \cG_0 \lor ( u \in \cG_1 \setminus \cG_0 \land u_1 = 1)] \notag\\
&= \Pr[u \in \cG_0] + \delta \Pr[u \in \cG_1 \setminus \cG_0].
\end{align}
 Using \eqref{eq:gu_rel} and \eqref{eq:lemma_assum} we obtain
 \begin{align}
   \label{eq:proof_final}
   \IndI
\underset{
  \mathclap{
  \substack{\pi, \rho \\ x := \langle P^{(1)}(\pi), D_1(\rho) \rangle_{\text{trans}}
    \\ (\Gamma_V, \Gamma_H) := \langle P^{(1)}(\pi), D_1 (\rho) \rangle_{P^{(1)}} }}}
{\Pr}\mkern13mu[\Gamma_V(D_2(x, \rho)) = 1]
 &\stackrel{\hphantom{\eqref{eq:gu_rel}}}{\geq} \frac{ {\Pr}_{u \leftarrow \mu_{\delta}^{k}}[g(u) = 1] + \epsilon -
 \Pr_{u \leftarrow \mu_{\delta}^{k}}[u \in \cG_0] - (1 - \frac{1}{6k})\epsilon} {\Pr_{u \leftarrow \mu_{\delta}^{k}}[u \in \cG_1 \setminus \cG_0]} \notag\\
 &\stackrel{\eqref{eq:gu_rel}}{\geq} \frac{\epsilon + \delta\Pr_{u \leftarrow \mu_{\delta}^{k}}[u \in \cG_1 \setminus \cG_0] - (1 - \frac{1}{6k})\epsilon}
{\Pr_{u \leftarrow \mu_{\delta}^{k}}[u \in \cG_1 \setminus \cG_0]} \geq \delta + \frac{\epsilon}{6k}.
\end{align}
Clearly, the running time of $\Gen$ is bounded by some polynomial $p(k, \frac{1}{\epsilon}, n)$ with oracle calls.
Furthermore, the algorithm $\Gen$ outputs a circuit $D$ such that it satisfies with probability at least $1 - \p(k, \frac{1}{\epsilon}, n) \cdot 2^n$
the statement of Lemma \ref{lemma:sec_amp_for_p_hash}. This concludes the proof of Lemma~\ref{lemma:sec_amp_for_p_hash}.
\end{proof}
%
%%% Local Variables:
%%% mode: latex
%%% TeX-master: "../master"
%%% End:


%%% Local Variables:
%%% mode: latex
%%% TeX-master: "../master"
%%% End:


\end{document}

