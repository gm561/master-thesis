%% (Master) Thesis template
% Template version used: v1.4
%
%% We use the memoir class because it offers a many easy to use features.
%\documentclass[11pt,a4paper,titlepage]{memoir}
\documentclass[11pt,a4paper,article,oneside]{memoir}

\usepackage[margin=1in]{geometry}
%% Packages
%% ========
%% LaTeX Font encoding -- DO NOT CHANGE
\usepackage[OT1]{fontenc}
%% Babel provides support for languages.  'english' uses British
%% English hyphenation and text snippets like "Figure" and
%% "Theorem". Use the option 'ngerman' if your document is in German.
%% Use 'american' for American English.  Note that if you change this,
%% the next LaTeX run may show spurious errors.  Simply run it again.
%% If they persist, remove the .aux file and try again.
\usepackage[english]{babel}
%% Input encoding 'utf8'. In some cases you might need 'utf8x' for
%% extra symbols. Not all editors, especially on Windows, are UTF-8
%% capable, so you may want to use 'latin1' instead.
\usepackage[utf8]{inputenc}
%% This changes default fonts for both text and math mode to use Herman Zapfs
%% excellent Palatino font.  Do not change this.
% \usepackage[sc]{mathpazo}
%% The AMS-LaTeX extensions for mathematical typesetting.  Do not
%% remove.
\usepackage{amsmath,amssymb,amsfonts,mathrsfs}
%% NTheorem is a reimplementation of the AMS Theorem package. This
%% will allow us to typeset theorems like examples, proofs and
%% similar.  Do not remove.
%% NOTE: Must be loaded AFTER amsmath, or the \qed placement will
%% break
\usepackage[amsmath,thmmarks]{ntheorem}
%% LaTeX' own graphics handling
\usepackage{graphicx}
%% We unfortunately need this for the Rules chapter.  Remove it
%% afterwards; or at least NEVER use its underlining features.
\usepackage{soul}
%% Some more packages that you may want to use.  Have a look at the
%% file, and consult the package docs for each.
\input{extrapackages}
%% Our layout configuration.  DO NOT CHANGE.
%\input{layoutsetup}
%% Theorem environments.  You will have to adapt this for a German
%% thesis.
%% Theorem-like environments

%% This can be changed according to language. You can comment out the ones you
%% don't need.

\numberwithin{equation}{chapter}

%% German theorems
%\newtheorem{satz}{Satz}[chapter]
%\newtheorem{beispiel}[satz]{Beispiel}
%\newtheorem{bemerkung}[satz]{Bemerkung}
%\newtheorem{korrolar}[satz]{Korrolar}
%\newtheorem{definition}[satz]{Definition}
%\newtheorem{lemma}[satz]{Lemma}
%\newtheorem{proposition}[satz]{Proposition}

%% English variants
\newtheorem{theorem}{Theorem}[chapter]
\newtheorem{example}[theorem]{Example}
\newtheorem{remark}[theorem]{Remark}
\newtheorem{corollary}[theorem]{Corollary}
\newtheorem{lemma}[theorem]{Lemma}
\newtheorem{proposition}[theorem]{Proposition}
\newtheorem{observation}[theorem]{Observation}

\theoremstyle{definition}
\theorembodyfont{\normalfont}
%% end def with blacksquare symbol
\theoremsymbol{\ensuremath{\lozenge}}
\newtheorem{definition}[theorem]{Definition}

%% Proof environment with a small square as a "qed" symbol
\theoremstyle{nonumberplain}
\theorembodyfont{\normalfont}
\theoremsymbol{\ensuremath{\square}}
\theoremseparator{.}
\newtheorem{proof}{Proof}

%\newtheorem{beweis}{Beweis}

\declaretheorem[name=Theorem, numberwithin=chapter]{thm}

%% Helpful macros.
%% Custom commands
%% ===============

%% Special characters for number sets, e.g. real or complex numbers.
\newcommand{\C}{\mathbb{C}}
\newcommand{\K}{\mathbb{K}}
\newcommand{\N}{\mathbb{N}}
\newcommand{\Q}{\mathbb{Q}}
\newcommand{\R}{\mathbb{R}}
\newcommand{\Z}{\mathbb{Z}}
\newcommand{\X}{\mathbb{X}}

\newcommand{\cX}{\mathcal{X}}
\newcommand{\cH}{\mathcal{H}}
\newcommand{\cW}{\mathcal{W}}
\newcommand{\cG}{\mathcal{G}}
\newcommand{\cB}{\mathcal{B}}
\newcommand{\cP}{\mathcal{P}}
\newcommand{\cR}{\mathcal{R}}
\newcommand{\cD}{\mathcal{D}}

%define our own code commands
%use capital latters as most of these commands is already defined
\renewcommand{\For}{\textbf{for }}
\renewcommand{\If}{\textbf{if }}
\renewcommand{\Else}{\textbf{else }}
\renewcommand{\Return}{\textbf{return }}
\newcommand{\Then}{\textbf{then }}
\newcommand{\Do}{\textbf{do: }}
\renewcommand{\And}{\textbf{and }}
\newcommand{\Or}{\textbf{or }}
\newcommand{\Run}{\textbf{run }}
\newcommand{\To}{\textbf{to }}

%% Fixed/scaling delimiter examples (see mathtools documentation)
\DeclarePairedDelimiter\abs{\lvert}{\rvert}
\DeclarePairedDelimiter\norm{\lVert}{\rVert}

%% Use the alternative epsilon per default and define the old one as \oldepsilon
\let\oldepsilon\epsilon
\renewcommand{\epsilon}{\ensuremath\varepsilon}

%% Also set the alternate phi as default.
\let\oldphi\phi
\renewcommand{\phi}{\ensuremath{\varphi}}

% New command that introduces a tab
\newcommand{\itab}[1]{\hspace{0em}\rlap{#1}}
\newcommand{\tab}[1]{\hspace{.2\textwidth}\rlap{#1}}

\DeclareMathOperator{\la0}{\leftarrow}
\DeclareMathOperator{\ra0}{\rightarrow}

\DeclareMathOperator{\hash}{\mathit{hash}}
\DeclareMathOperator{\CanonicalSuccess}{\mathit{CanonicalSuccess}}
\DeclareMathOperator{\Success}{\mathit{Success}}

%the DWVP for the permutation
\DeclareMathOperator{\PiDWVP}{\Pi_{DWVP}}
%the DWPV for the k-wise product of permutations
\DeclareMathOperator{\kPiDWVP}{\Pi_{DWVP}^{(k)}}
%% Make document internal hyperlinks wherever possible. (TOC, references)
%% This MUST be loaded after varioref, which is loaded in 'extrapackages'
%% above.  We just load it last to be safe.
\usepackage[linkcolor=black,colorlinks=true,citecolor=black,filecolor=black]{hyperref}

\newcommand{\IndI}{\mbox\qquad}
\newcommand{\IndII}{\mbox\qquad\qquad}
\newcommand{\IndIII}{\mbox\qquad\qquad\qquad}

\newcommand{\schapter}[1]{{\noindent\Large{\textbf{#1}}\\}}
\newcommand{\ssection}[1]{{\noindent\large{\textbf{#1}} \\}}
\newcommand{\ssubsection}[1]{{\noindent\textbf{#1} \\}}

\begin{document}

{\LARGE{\textbf{Contents}}} \\ \\

\schapter{1 Introduction}
\ssection{1.1 Security Amplification Theorems}
Why security amplification is interesting topic in cryptography.
Examples of the classical results -- a weak one way function implying the strong ones.
Hardness implication statements. Problems captured by weakly verifiable puzzles.
Contribution of this thesis.

\ssection{1.2 Organization of the Thesis}
Overview of the content of the following chapters.
\\
\\
\\
\schapter{2 Preliminaries}
\ssection{2.1 Notation}
Set up notation and terminology used in the Thesis.\\
\ssection{2.2 Pairwise independent family of hash functions}
Define efficient pairwise independent family of hash functions.\\
\ssection{2.3 Oracle machines}
Describe the way in which algorithms access the oracle, and the oracle queries are answered.
\\
\\
\\
\schapter{3 Weakly Verifiable Cryptographic Primitives}
% In this chapter we introduce the notion of weakly verifiable puzzles. In section \ref{def:dwvp} we provide a formal definitions that
% is followed by a series of cryptographic primitives that are captured by this notion.
% Finally, in Section \ref{st:previous_results} we give an overview of the earlier research in this area
% that is primarily covered in \cite{canetti2004hardness}, \cite{Dodis:2009:SAI:1530441.1530450}, and \cite{DBLP:journals/corr/abs-1002-3534}.
\ssection{3.1 Weakly Verifiable Puzzles}
Give a definition of dynamic, interactive weakly verifiable puzzles.
\\
\ssection{3.2 Examples}
\IndI \ssubsection{3.2.1 Message Authentication Codes}
\IndI \ssubsection{3.2.2 Public Key Encryption}
\IndI \ssubsection{3.2.3 Bit Commitments}
\IndI \ssubsection{3.2.4 CAPTACHs}
\\
\ssection{3.3 Previous results}
Give an overview of the previous works considering WVP. Present the approach of the authors,
the contribution of the paper, give a sketch of the proof (e.g. the algorithm without the formal proof). \\
\IndI \ssubsection{3.3.1 Results of R.Canetti, S.Halevi, and M.Steiner}
\IndI \ssubsection{3.3.2 Results of Y.Dodis, R.Impagliazzo, R.Jaiswal, V.Kabanets}
\IndI \ssubsection{3.3.3 Results of T.Holenstein and G.Scheonebeck}
\\
\\
\schapter{4 Security amplification for dynamic weakly verifiable puzzles}
In this chapter I present my main result.\\
\ssection{4.1 Main theorem}
\IndI \ssubsection{4.1.1 The $k$-wise direct product of weakly verifiable puzzle}
\IndI\ssubsection{4.1.2 Intuition}
\IndI\ssubsection{4.1.3 Domain partitioning}
\IndI\ssubsection{4.1.4 Amplification proof for the partitioned domain}
\IndI\ssubsection{4.1.5 Putting it together}
\ssection{4.2 Discussion}
Discuss optimality of the result.

% appendix

\end{document}
