%TODO how many verification queries might be asked
%TODO after the interaction with S
%TODO what is the relation between $r$ and $q$
%TODO in which random experiment?
%TODO define the hint function
%TODO define verification and hint queries
%TODO runtime bound
\begin{definition} {\textbf{(Dynamic weakly verifiable puzzle)}}
  A dynamic weakly verifiable puzzle (DWVP) is defined by a protocol between probabilistic algorithms $(P,S)$.
  The algorithm $P$ is called a problem poser and $S$ a problem solver.
  The problem poser $P$ produces as output a circuit $\Gamma_{V}$ and a circuit $\Gamma_{H}$.
  The problem solver $S$ does not produce any output.
  The circuit $\Gamma_{V}$ takes as its input $q \in Q$ and an answer $r \in R$.
  An answer $r$ is a correct solution for $q$ if and only if the circuit $\Gamma_V$ on input $(q,r)$ evaluates to true.
  The circuit $\Gamma_H$ on input $q$ provides a hint $r \in R$ such that $\Gamma_V(q,r) = 1$.
  The solver $S$ has oracle access to both circuits $\Gamma_V$ and $\Gamma_H$.
  The calls to the circuit $\Gamma_V$ are called verification queries. The calls to the circuit $\Gamma_H$ are hint queries.
  The solver $S$ asks at most $h$ hint queries, $v$ verification queries, and successfully solves a DWVP $\Pi$ if and only if
  it makes a verification query $(q,r)$ such that $\Gamma_V(q,r) = 1$, when it has not previously asked for a hint query on this $q$.
\end{definition}
%
% DONE should we not exclude verification queries - it seems that no
%
Suppose that $g: \{0,1\}^k \rightarrow \{0,1\}$ is a monotone function, and $\left( P^{(1)}, S^{(1)} \right)$ is a dynamic weakly verifiable puzzle.
Then $(P^{(g)}, S^{(g)})$ is a dynamic weakly verifiable puzzle $\Pi^{(g)}$, for which in the first phase the problem poser $P^{(g)}$ and solver $S^{(g)}$
sequentially create $k$ instances of a puzzle $\left( P^{(1)}, S^{(1)}\right)$. The problem poser $P^{(g)}$ produces as it output circuits $\Gamma_V^{(g)}$ and $\Gamma_H^{(g)}$.
The hint queries for a puzzle $\Pi^{(g)}$ are answered by a circuit $\Gamma_H^{(g)}(q) = \left( \Gamma_H^{(1)}(q), \dots, \Gamma_H^{(k)} \right)$, and the verification queries by a circuit $\Gamma_V^{(g)}(q, r_1, \dots, r_k) = g \left( \Gamma_V^{1}(q, r_1), \dots, \Gamma_V^{k}(q, r_k) \right)$.

Let $hash : Q \rightarrow \{0, 2(h+v)-1 \}$ be a function and $P_{hash}$ a set that contains elements $q \in Q$ for which $hash(q) = 0$.
A \textit{canonical success}, with respect to a set $P_{hash}$ in
a random experiment defined by the protocol between $P^{(g)}$ and $S^{(g)}$,
is a situation when a first successful verification query made by $S^{(g)}$ is in $P_{hash}$,
and all previous hint or verification queries are not in $P_{hash}$.
%
% TODO size of the circuit
% TODO oracle calls ? non-rewinding?
% TODO is GEN probabilistic?
% TODO running time of Gen size of D
% TODO hash function does it has to be pairwise independent?
% TODO what is a random in hash function
%
\begin{theorem} {\textbf{(Security amplification for DWVP (non unifrom version)).}}
  Let $g: \{0,1\}^k \rightarrow \{0,1\}$ be a monotone function, and $hash:Q \rightarrow \{0,2(h+v)-1\}$ a function such that
  the probability of a \textit{canonical success}, with respect to $P_{hash}$ in a random experiment defined by a protocol $(P^{(g)},S^{(g)})$, is at least $\frac{\varepsilon}{8\left(v + h\right)}$.
  If there exists a circuit $C$ that makes at most $v$ verification queries, $h$ hint queries,
  and succeeds with probability
  \begin{align}
    \Pr_{}[\Gamma_{V}^{(g)}( \langle P^{(g)},C \rangle_C ) = 1] \geq \Pr_{\mu \leftarrow \mu_{\delta}^{k}}[g(u) = 1] + \varepsilon ,
  \end{align}
  where the probability is over randomness of $P^{(g)}$,
  then there exists a probabilistic algorithm $Gen(C, g, \varepsilon, \delta, n, hash)$ which takes as input: a circuit $C$, the function $g$, the function $hash$,
  parameters $\varepsilon, \delta, n$, and produces a circuit $D$ of size at most $ size(C) \frac{6k}{\varepsilon} \log(\frac{6k}{\varepsilon}) $
  such that with high probability it satisfies
  \begin{align}
    \Pr_{}[\Gamma_V^{(1)} \left( \langle P^{(1)} ,D \rangle_D \right) = 1] \geq \frac{1}{8(h+v)} \left( \delta + \frac{\varepsilon}{6k} \right)
  \end{align}
  where the probability is taken over random coins of $P$.
  Additionally, the circuit D and the algorithm $Gen$ only require oracle access to functions $g$ and $hash$, and the running time of
  the algorithm $Gen$ is $\text{poly}\left(k, \frac{1}{\varepsilon}, n, v+h \right)$
\end{theorem}
% DONE would it make more sense to consider situations when one have just a single verification query (if the proof works out also in this situation then it is a stronger result).


%%% Local Variables:
%%% mode: latex
%%% TeX-master: "master.tex"
%%% End:
