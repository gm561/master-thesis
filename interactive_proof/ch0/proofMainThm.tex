%
\begin{proof}[Theorem \ref{th:sec_amp_for_dwvp}]
We define the following circuits.
%
\begin{codeblock}
  \textbf{Circuit} $\widetilde{D}_2^{D, P^{(1)}, \hash, g, \Gamma_V, \Gamma_H}(x, \rho)$
  \medskip
  \hrule
  \medskip
  \textbf{Oracle:} A solver circuit $D$ for $P^{(1)}$, a problem poser $P^{(1)}$, \\
  \IndII functions $\hash: Q \rightarrow \{0,1, \dots, 2(h+v) - 1\}$, $g: \{0,1\}^{k} \rightarrow \{0,1\}$ \\
  \IndII a verification oracle $\Gamma_V$, a hint oracle $\Gamma_H$.\\
  \textbf{Input:}  Bitstrings $x \in \{0,1\}^{*}$, $\rho \in \{0,1\}^{*}$.
  \medskip\hrule\medskip
  %
  $(q, y) := D_2^{P^{(1)}, \widetilde{C}, \hash, g, \Gamma_H}(x, \rho)$ \\
  Make a verification query to $\Gamma_V$ using $(q,y)$
\end{codeblock}
%
\begin{codeblock}
  \textbf{Algorithm} $\widetilde{\text{Gen}}^{P^{(1)}, g, C}(n, \epsilon, \delta, k, h, v)$
  \medskip \hrule \medskip
  \textbf{Oracle:} A problem poser $P^{(1)}$, a function $g: \{0,1\}^{k} \rightarrow \{0,1\}$, \\
  \IndII a solver circuit $C$ for $P^{(g)}$.  \\
  \textbf{Input:} Parameters $n$, $\epsilon$, $\delta$, $k$, $h$, $v$.
  \medskip\hrule\medskip
  %
  $hash := \text{FindHash}((h+v)\epsilon, n, h, v)$ \\
  Let $\widetilde{C} := (C_1, \widetilde{C}_2)$ be as in Lemma \ref{lemma:ctilda_c} with oracle access to $C$, $\hash$. \\
  $D := Gen^{P^{(1)},  \widetilde{C},  g, \hash}(\epsilon, \delta, n, k)$ \\
  \Return $\widetilde{D} := (D_1, \widetilde{D}_2)$
\end{codeblock}
%
We show that Theorem \ref{th:sec_amp_for_dwvp} follows from Lemma \ref{lemma:hash_function_probability} and Lemma \ref{lemma:sec_amp_for_p_hash}.
We fix $P^{(1)}$, $g$, $P^{(g)}$. Given a solver circuit $C = (C_1, C_2)$, asking $h$ hint queries and $v$ verification queries, such that
\begin{align*}
    \underset{\pi^{(k)}, \rho}{\Pr}\left[\Success^{P^{(g)}, C}(\pi^{(k)}, \rho) = 1\right] \geq 16(h+v)\left(\underset{u \leftarrow \mu_\delta^k}{\Pr}\left[g(u) = 1\right] + \varepsilon\right)
\end{align*}
we satisfy conditions of Lemma \ref{lemma:hash_function_probability}. Therefore, $\widetilde{\text{Gen}}$ can use the algorithm FindHash to find $\hash$ such that
\begin{align*}
    \underset{\pi^{(k)}, \rho}{\Pr}\left[\CanonicalSuccess^{P^{(g)}, C, \hash}(\pi^{(k)}, \rho) = 1\right] \geq \underset{u \leftarrow \mu_\delta^k}{\Pr}\left[g(u) = 1\right] + \varepsilon
\end{align*}
almost surely.
By Lemma \ref{lemma:ctilda_c} we know that it is possible to build $\widetilde{C} = (C_1, \widetilde{C}_2)$ such that
\begin{align*}
    \underset{
      \mathclap {
      \substack{\pi^{(k)}, \rho \\
        x := \langle P^{(g)}(\pi^{(k)}), C_1(\rho) \rangle_{\mathit{trans}} \\
        (\Gamma_V^{(g)}, \Gamma_H^{(k)}) := \langle P^{(g)}(\pi^{(k)}), C_1(\rho) \rangle_{P^{(g)}}
      }}}
    {\Pr}\mkern13mu[\Gamma_V^{(g)}(\widetilde{C}_2^{\Gamma_H^{(k)}, C_2, \hash}(x, \rho)) = 1]
    \geq
\underset{u \leftarrow \mu_\delta^k}{\Pr}\left[g(u) = 1\right] + \varepsilon.
\end{align*}
Now, we use Gen to obtain a circuit $D = (D_1, D_2)$, which by Lemma \ref{lemma:sec_amp_for_p_hash} satisfies
\begin{align}
  \label{eq:succ_prob_d}
    \underset{
      \mathclap{
      \substack{\pi, \rho \\ x := \langle P^{(1)}(\pi), D_1^{\widetilde{C}}(\rho) \rangle_{\mathit{trans}} \\
        (\Gamma_H, \Gamma_V) := \langle P^{(1)}(\pi), D_1^{\widetilde{C}}(\rho) \rangle_{P^{(1)}}}}}
    {\Pr}\mkern13mu[\Gamma_V(D_2^{P^{(1)}, \widetilde{C}, hash, g, \Gamma_H}(x, \rho)) = 1] \geq (\delta + \frac{\varepsilon}{6k})
\end{align}
almost surely.
Finally, $\widetilde{\text{Gen}}$ outputs $\widetilde{D} = (D_1, \widetilde{D}_2)$ with oracle access to $D$, $P^{(1)}$, $hash$, $g$ that with high probability satisfies
\begin{align*}
    \underset{\pi, \rho}{\Pr}\left[\Success^{P^{(1)},\widetilde{D}}(\pi, \rho) = 1\right] \geq (\delta + \frac{\varepsilon}{6k}).
\end{align*}
The running time of FindHash is $\mathit{poly(h,v,\frac{1}{\epsilon},n,t)}$  and of Gen $\mathit{poly(k, \frac{1}{\epsilon}, n, t)}$.
Thus, the overall running time of $\widetilde{\mathit{Gen}}$ is  $\mathit{poly(k,\frac{1}{\epsilon},h,v,n,t)}$.
Furthermore, the circuit $\widetilde{D}$ asks at most $\frac{6k}{\epsilon} \log(\frac{6k}{\epsilon})h$ hint queries and one verification query.
Finally, we have $\mathit{Size}(\widetilde{D}) \leq \mathit{Size}(C) \cdot \frac{6k}{\epsilon}$.
This finishes the proof of Theorem \ref{th:sec_amp_for_dwvp}.
\end{proof}

%%% Local Variables:
%%% mode: latex
%%% TeX-master: "../master"
%%% End:
