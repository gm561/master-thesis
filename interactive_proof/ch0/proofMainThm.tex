%
\begin{proof}[Theorem \ref{th:sec_amp_for_dwvp}]
Let use first define following circuits.
%
\begin{codeblock}
  \textbf{Circuit} $\widetilde{D}_2^{D, P^{(1)}, P^{(g)}, \hash, g, \Gamma_v}(x, \rho)$
  \medskip
  \hrule
  \medskip
  \textbf{Oracle:} A circuit $D$, problem posers $P^{(1)}$, $P^{(g)}$, \\
  \IndII function $\hash: Q \rightarrow \{0,1, \dots, 2(h+v) - 1\}$, $g: \{0,1\}^{k} \rightarrow \{0,1\}$ \\
  \IndII a verification oracle $\Gamma_V$, a hint oracle $\Gamma_H$\\
  \textbf{Input:}  Bitstrings $x \in \{0,1\}^{*}$, $\rho \in \{0,1\}^{*}$
  \medskip\hrule\medskip
  %
  $(q, y) := D_2^{P^{(1)}, P^{(g)}, \widetilde{C}, \hash, g, \Gamma_H}(x, \rho)$ \\
  Make a verification query to $\Gamma_V$ using $(q,y)$
\end{codeblock}
%
\begin{codeblock}
  \textbf{Algorithm} $\widetilde{\text{Gen}}^{P^{(1)}, P^{(g)}, g, C}(n, \epsilon, \delta, k, h, v)$
  \medskip \hrule \medskip
  \textbf{Oracle:} Problem posers $P^{(1)}$, $P^{(g)}$, a function $g: \{0,1\}^{k} \rightarrow \{0,1\}$, \\
  \IndII a solver circuit $C$ for $P^{(g)}$.  \\
  \textbf{Input:} Parameters $n$, $\epsilon$, $\delta$, $k$, $h$, $v$.
  \medskip\hrule\medskip
  %
  $hash := \text{FindHash}((h+v)\epsilon, n, h, v)$ \\
  Let $\widetilde{C} = (C_1, \widetilde{C}_2)$ be as in Lemma \ref{lemma:ctilda_c} with oracle access to $C$, $\hash$. \\
  $D := Gen^{P^{(1)}, P^{(g)}, \widetilde{C}, \hash, g}(\epsilon, \delta, n, k)$ \\
  \Return $\widetilde{D} := (D_1, \widetilde{D}_2)$
\end{codeblock}
%
We show that Theorem \ref{th:sec_amp_for_dwvp} follows from Lemma \ref{lemma:hash_function_probability} and Lemma \ref{lemma:sec_amp_for_p_hash}.
We use the algorithm $\widetilde{Gen}$ to obtain a circuit $D := (D_1, D_2)$ and a function $\hash$. Then, the circuit $D$ uses $\widetilde{D}$ to find $(q,y)$.
We will show that
\begin{align*}
    \underset{\pi, \rho}{\Pr}\left[\Success^{P^{(1)},\widetilde{D}}(\pi, \rho) = 1\right] \geq (\delta + \frac{\varepsilon}{6k}).
\end{align*}
%
For fixed $P^{(1)}$, $g$, $P^{(g)}$ given a solver circuit $C = (C_1, C_2)$ such that
\begin{align*}
    \underset{\pi^{(k)}, \rho}{\Pr}\left[\Success^{P^{(g)}, C}(\pi^{(k)}, \rho) = 1\right] \geq 16(h+v)\left(\underset{u \leftarrow \mu_\delta^k}{\Pr}\left[g(u) = 1\right] + \varepsilon\right)
\end{align*}
we apply Lemma \ref{lemma:hash_function_probability} to find a function $\hash$ such that
\begin{align*}
    \underset{\pi^{(k)}, \rho}{\Pr}\left[\CanonicalSuccess^{P^{(g)}, C, \hash}(\pi^{(k)}, \rho) = 1\right] \geq \underset{u \leftarrow \mu_\delta^k}{\Pr}\left[g(u) = 1\right] + \varepsilon.
\end{align*}
By Lemma \ref{lemma:ctilda_c} we know that it is possible to create a circuit $\widetilde{C} = (C_1, \widetilde{C}_2)$ with oracle access to $\hash$ and $C$ such that
\begin{align*}
    \underset{
      \mathclap {
      \substack{\pi, \rho \\
        x := \langle P(\pi), C_1(\rho) \rangle_{\mathit{trans}} \\
        (\Gamma_V^{(g)}, \Gamma_H^{(k)}) := \langle P(\pi), C_1(\rho) \rangle_{P}
      }}}
    {\Pr}\mkern13mu[\Gamma_V^{(g)}(\widetilde{C}_2^{\Gamma_H^{(k)}, C_2, \hash}(x, \rho)) = 1]
    \geq
\underset{u \leftarrow \mu_\delta^k}{\Pr}\left[g(u) = 1\right] + \varepsilon.
\end{align*}
Now, we apply Lemma \ref{lemma:sec_amp_for_p_hash} for the function $\hash$ and $\widetilde{C} = (C_1, \widetilde{C}_2)$ and obtain a circuit $D = (D_1, D_2)$ such that
\begin{align}
  \label{eq:succ_prob_d}
    \underset{
      \mathclap{
      \substack{\pi, \rho \\ x := \langle P^{(1)}(\pi), D_1(\rho) \rangle_{\mathit{trans}} \\
        (\Gamma_H, \Gamma_V) := \langle P^{(1)}(\pi), D_1(\rho) \rangle_{P^{(1)}}}}}
    {\Pr}\mkern13mu[\Gamma_V(D_2(x, \rho)) = 1] \geq (\delta + \frac{\varepsilon}{6k}).
\end{align}
%
% We know that the probability that the verification query $(q,y)$ succeeds
% amounts at least $(\delta + \frac{\epsilon}{6k})$.
% Therefore, we have
% \begin{align*}
%     \underset{\pi, \rho}{\Pr}\left[Success^{P^{(1)},\widetilde{D}}(\pi, \rho) = 1\right] \geq (\delta + \frac{\varepsilon}{6k}).
% \end{align*}
\end{proof}
% %
% %
% Finally, we conclude that the Theorem \ref{th:sec_amp_for_dwvp} holds with the algorithm $\widetilde{Gen}$ used to generate the solver circuit $\widetilde{D}$,
%TODO
%algo Gen
%\widetilde{D} more formally

%%% Local Variables:
%%% mode: latex
%%% TeX-master: "../master"
%%% End:
