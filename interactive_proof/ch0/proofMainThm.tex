\begin{proof}[Theorem \ref{th:sec_amp_for_dwvp}]
We show that Theorem \ref{th:sec_amp_for_dwvp} follows by Lemmas: \ref{lemma:sec_amp_for_p_hash}, \ref{lemma:hash_function_probability}.
First given a solver circuit $C$ such that
\begin{align*}
    \underset{\pi^{(k)}, \rho}{\Pr}\left[Success^{P^{(g)}, C}(\pi^{(k)}, \rho) = 1\right] \geq 8(h+v)\left(\underset{u \leftarrow \mu_\delta^k}{\Pr}\left[g(u) = 1\right] + \varepsilon\right)
\end{align*}
we apply Lemma \ref{lemma:hash_function_probability} to find a function $hash$ such that
\begin{align*}
    \underset{\pi^{(k)}, \rho}{\Pr}\left[CanonicalSuccess^{P^{(g)}, C, hash}(\pi^{(k)}, \rho) = 1\right] \geq \underset{u \leftarrow \mu_\delta^k}{\Pr}\left[g(u) = 1\right] + \varepsilon.
\end{align*}
By Lemma (\ref{lemma:ctilda_c}) we know that it is possible to create a circuit $\widetilde{C}$ with oracle access to $hash$ and $C$ such that
\begin{align*}
    \underset{\substack{\pi, \rho \\
        x := \langle P(\pi), C_1(\rho) \rangle_{\text{trans}} \\
        (\Gamma_V^{(g)}, \Gamma_H^{(k)}) := \langle P(\pi), C_1(\rho) \rangle_{P}
      }}
    {\Pr}[\Gamma_V^{(g)}(\widetilde{C}_2^{\Gamma_H^{(k)}, C_2, hash}(x, \rho)) = 1]
    \geq
\underset{u \leftarrow \mu_\delta^k}{\Pr}\left[g(u) = 1\right] + \varepsilon
\end{align*}
Now, we apply Lemma \ref{lemma:sec_amp_for_p_hash} for the function $hash$ and the circuit $\widetilde{C}$ and obtain a circuit $D$ such that
\begin{align}
  \label{eq:succ_prob_d}
    \underset{\substack{\pi, \rho \\ x := \langle P^{(1)}(\pi), D_1(\rho) \rangle_{\text{trans}} \\
        (\Gamma_H, \Gamma_V) := \langle P^{(1)}(\pi), D_1(\rho) \rangle_{P^{(1)}}}}
    {\Pr}[\Gamma_V(D_2(x, \rho)) = 1] \geq (\delta + \frac{\varepsilon}{6k}).
\end{align}
Finally, we use the circuit $\widetilde{D}$ that first runs the circuit $D$, and make a verification
query using $(q,y)$ returned by $D$. From, we know that probability that this query is successful amounts at least $(\delta + \frac{\epsilon}{6k})$.
Therefore, we have
\begin{align*}
    \underset{\pi, \rho}{\Pr}\left[Success^{P^{(1)},\widetilde{D}}(\pi, \rho) = 1\right] \geq (\delta + \frac{\varepsilon}{6k}).
\end{align*}
\end{proof}

%%% Local Variables:
%%% mode: latex
%%% TeX-master: "../master"
%%% End:
