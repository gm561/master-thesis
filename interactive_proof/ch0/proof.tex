%
We write $C_2^{(\cdot, \cdot)}$ to emphasize that $C_2$ does not obtain direct access to hint and verification circuits.
Instead, all hint and verification queries are answered explicitly as in the following code excerpt of the circuit $\widetilde{C}_2$.
%
\begin{codeblock}
  \textbf{Circuit} $\widetilde{C}_2^{\Gamma_H, C_2, \hash} (x, \rho)$
  \medskip \hrule \medskip
  \textbf{Oracle:} A hint circuit $\Gamma_H$, a circuit $C_2$, a function $\hash : Q \rightarrow \{0,1,\dots, 2(h+v)-1\}$. \\
  \textbf{Input:} Bitstrings $x \in \{0,1\}^{*}$, $\rho \in \{0,1\}^{*}$. \\
  \textbf{Output:} A pair $(q, y)$.
  \medskip\hrule\medskip
  \Run $C_2^{(\cdot, \cdot)}(x, \rho)$ \\
  \IndI \If $C_2^{(\cdot, \cdot)}(x, \rho)$ asks a hint query on $q$ \Then\\
  \IndII \If $\hash(q) = 0$ \Then\\
  \IndIII \Return $\bot$\\
  \IndII \textbf{else}\\
  \IndIII answer the query of $C_2^{(\cdot, \cdot)}(x, \rho)$ using $\Gamma_H(q)$\\
  \\
  \IndI \If $C_2^{(\cdot, \cdot)}(x, \rho)$ asks a verification query $(q, y)$ \Then \\
  \IndII \If $\hash(q) = 0 $ \textbf{then} \\
  \IndIII \Return $(q, y)$ \\
  \IndII \textbf{else} \\
  \IndIII answer the verification query of $C_2^{(\cdot, \cdot)}(x, \rho)$ with 0 \\
  \Return $\bot$
\end{codeblock}
%
Given $C = (C_1, C_2)$ we define a circuit $\widetilde{C} = (C_1, \widetilde{C}_2)$.
Every hint query $q$ asked by $\widetilde{C}$ is such that $hash(q) \neq 0$.
Furthermore, $\widetilde{C}$ asks no verification queries, and returns $\bot$ or $(q,y)$ such that
$hash(q) = 0$.

We say that for a fixed $\pi$, $\rho$, $hash$ the circuit $\widetilde{C}$ \textit{succeeds} if
for $x := \langle P(\pi), C_1(\rho) \rangle_{\mathit{trans}}$,
$(\Gamma_V, \Gamma_H) := \langle P(\pi), C_1(\rho) \rangle_{P}$, we have
\begin{align*}
\Gamma_V(\widetilde{C}_2^{\Gamma_H, C_2, \hash}(x, \rho)) = 1.
\end{align*}
%
\begin{lemma}
  \label{lemma:ctilda_c}
  For fixed $P, C$ and $hash$ the following statement is true
  \begin{align*}
    \underset{\pi, \rho}{\Pr}[CanonicalSuccess^{P, C, \hash}(\pi, \rho) = 1]
    \leq
    \mkern13mu
    \underset{
      \mathclap{
      \substack{
        \pi, \rho \\
        x := \langle P(\pi), C_1(\rho) \rangle_{\mathit{trans}} \\
        (\Gamma_V, \Gamma_H) := \langle P(\pi), C_1(\rho) \rangle_{P}}}}
  {\Pr}[\Gamma_V(\widetilde{C}_2^{\Gamma_H, C_2, hash}(x, \rho)) = 1]
  \end{align*}
\end{lemma}
%
\begin{proof}
If for some fixed $\pi$, $\rho$ and $\hash$ the circuit $C$ succeeds canonically, then for the same $\pi$, $\rho$ and $\hash$ also $\widetilde{C}$ succeeds.
Using this observation, we conclude that
\begin{align*}
  &\underset{\pi, \rho}{\Pr}\left[\CanonicalSuccess^{P, C, \hash}(\pi, \rho) = 1\right] \\
  &\IndII = \underset{\pi, \rho}{\mathbb{E}}\left[\CanonicalSuccess^{P, C, \hash}(\pi, \rho) = 1 \right] \\
  &\IndII \leq
  \mkern33mu
    \underset{
      \mathclap{
        \substack{\pi, \rho \\
        x := \langle P(\pi), C_1(\rho) \rangle_{\mathit{trans}} \\
        (\Gamma_V, \Gamma_H) := \langle P(\pi), C_1(\rho) \rangle_{P}
      }}}
    {\mathbb{E}}\mkern13mu[\Gamma_V(\widetilde{C}_2^{\Gamma_H, C_2, \hash}(x, \rho)) = 1] \\
  &\IndII =
  \mkern33mu
    \underset{
      \mathclap{
        \substack{\pi, \rho \\
        x := \langle P(\pi), C_1(\rho) \rangle_{\mathit{trans}} \\
        (\Gamma_V, \Gamma_H) := \langle P(\pi), C_1(\rho) \rangle_{P}
      }}}
    {\Pr}\mkern13mu[\Gamma_V(\widetilde{C}_2^{\Gamma_H, C_2, \hash}(x, \rho)) = 1]
\end{align*}
\end{proof}
%
%
\begin{lemma}\textbf{(Security amplification for dynamic weakly verifiable puzzles with respect to \boldmath{$\hash$}.)}
  \label{lemma:sec_amp_for_p_hash}
  Let $P_n^{(1)}$ be a fixed problem poser as in Definition \ref{def:dwvp} and $\widetilde{C} := (C_1, \widetilde{C}_2)$ a circuit
  with oracle access to a function $\hash: Q \rightarrow \{0,1,\dots, 2(h+v-1)\}$
  and a solver circuit $C := (C_1, C_2)$ for $P_{kn}^{(g)}$ which asks at most $h$ hint queries and $v$ verification queries.
  There exists an algorithm Gen that takes as input parameters $\varepsilon$, $\delta$, $n$, $k$,
  has oracle access to $P_n^{(1)}$,  $\widetilde{C}$, $\hash$, $g: \{0,1\}^{k} \rightarrow \{0,1\}$,
  and outputs a solver circuit $D := (D_1, D_2)$ for $P^{(1)}$ such that the following holds: \\
  If $\widetilde{C}$ is such that
  \begin{align*}
    \underset{\mathclap{\substack{
          \pi^{(k)} \in \{0,1\}^{kn}, \rho \in \{0,1\}^{*} \\
          x:= \langle P^{(g)}(\pi^{(k)}), {C}_1(\rho) \rangle_{\mathit{trans}} \\
          (\Gamma_H^{(k)}, \Gamma_V^{(g)}) := \langle P^{(g)}(\pi^{(k)}), C_1(\rho) \rangle_{P^{(g)}}}}}
    {\Pr}[\Gamma_V^{(g)}(\widetilde{C}_2^{\Gamma_H^{(k)}, C_2, \hash}(x,\rho)) = 1]
    \geq \underset{u \leftarrow \mu_\delta^k}{\Pr}[g(u) = 1] + \varepsilon,
  \end{align*}
  then $D$ satisfies almost surely
  \begin{align*}
    \underset{
      \mathclap{
      \substack{
        \pi \in \{0,1\}^{n}, \rho \in \{0,1\}^{*} \\
        x := \langle P^{(1)}(\pi), D_1^{\widetilde{C}}(\rho) \rangle_{\mathit{trans}} \\
        (\Gamma_H, \Gamma_V) := \langle P^{(1)}(\pi), D_1^{\widetilde{C}}(\rho) \rangle_{P^{(1)}}}}}
    {\Pr}[\Gamma_V(D_2^{P^{(1)}, \widetilde{C}, \hash, g, \Gamma_H}(x, \rho)) = 1] \geq (\delta + \frac{\varepsilon}{6k}).
  \end{align*}
  Furthermore, $D$
  %has oracle access to a hint circuit, $P^{(1)}$, $\widetilde{C}$, $\hash$, $g$, and
  asks at most $\frac{6k}{\epsilon}\log\left(\frac{6k}{\epsilon}\right) h$ hint queries and no verification queries.
  Finally, $\mathit{Size}(D) \leq \mathit{Size}(C)\frac{6k}{\varepsilon}$ and $\mathit{Time}(\mathit{Gen}) = \mathit{poly}(k, \frac{1}{\varepsilon}, n, t)$.
\end{lemma}
%
%

Before we give a proof of Lemma \ref{lemma:sec_amp_for_p_hash} we define some additional algorithms.
First, we are interested in the probability that for $u \leftarrow \mu_{\delta}^k$ and a bit $b$ we have $g(b,u_2, \dotsc, u_k) = 1$.
The estimate of this probability is calculated by EstimateFunctionProbability.
%
\begin{codeblock}
  \textbf{Algorithm} $\text{EstimateFunctionProbability}^{g}(b, k, \epsilon, \delta, n)$
  \medskip
  \hrule
  \medskip
  \textbf{Oracle:} A function $g : \{0,1\}^{k} \rightarrow \{0,1\}$.\\
  \textbf{Input:} A bit $b \in \{0,1\}$, parameters $k$, $\epsilon$, $\delta$, $n$. \\
  \textbf{Output:} An estimate $\widetilde{g}_b$ of $\Pr_{u \leftarrow \mu_{\delta}^{k}}[g(b,u_2, \dotsc, u_k) = 1]$.
  \medskip\hrule\medskip
  \For $i:=1$ \To $\frac{64k^2}{\epsilon^2} n$ \Do \\
  \IndI $u \leftarrow \mu_{\delta}^{k}$ \\
  \IndI $g_i := g(b, u_2, \dotsc, u_k)$ \\
  \Return $\frac{\epsilon^2}{64k^2n} \sum_{i=1}^{\frac{64k^2}{\epsilon^2} n} g_i$
\end{codeblock}
%
\begin{lemma}
  \label{lemma:estimate_of_g}
  The algorithm $\text{EstimateFunctionProbability}^{g}(b, k, \epsilon, \delta)$ outputs an estimate $\widetilde{g}_b$
  such that $| \widetilde{g}_b - \Pr_{u \leftarrow \mu_{\delta}^{k}}\left[g(b,u_2, \dots, u_k) = 1\right] | \leq \frac{\epsilon}{8k}$ almost surely.
\end{lemma}
%
\begin{proof}
We define independent, identically distributed binary random variables $K_1, K_2, \dots, K_{64k^2n/\epsilon^2}$
such that for each $1 \leq i \leq \frac{64k^2}{\epsilon^2} n$
the random variable $K_i$ takes value $g_i$. We use the Chernoff bound to obtain\footnote{For independent Bernoulli distributed
random variables $X_1, \dots, X_n$ with $X := \sum_{i=1}^n X_i$ and $0 \leq \delta \leq 1$
we use the Chernoff bound in the form $\Pr[|X - \mathbb{E}[X]| \geq \delta \mathbb{E}[X]] \leq 2 e^{- \mathbb{E}[X] \delta^2 / 3}$.}
%
\begin{align*}
  &\Pr \Bigl[ \Bigl| \widetilde{g}_b - \Pr_{u \leftarrow \mu_{\delta}^{k}}\left[g(b,u_2, \dots, u_k) = 1\right] \Bigr| \geq \frac{\epsilon}{8k} \Bigr]\\
  &\IndII = \Pr \Bigl[\Bigl|\Bigl(\frac{\epsilon^2}{64k^2n} \sum_{i=1}^{64k^2n / \epsilon^2 } K_i \Bigr) - \mathbb{E}_{u \leftarrow \mu_{\delta}^k}[g(b,u_2, \dots, u_k)\Bigr]\Bigr|
    \geq \frac{\epsilon}{8k} \Bigr] \leq 2 \cdot e^{-n/3}.
\end{align*}
\end{proof}
%
%
The algorithm $\text{EvalutePuzzles}^{P^{(1)}, \widetilde{C}, \hash}(\pi^{(k)}, \rho, n, k)$
evaluates which of $k$ puzzles of the $k$-wise direct product defined by $P^{(g)}$ are solved successfully by $\widetilde{C}(\rho) := (C_1,\widetilde{C}_2)(\rho)$.
To decide whether the $i$-th puzzle of the $k$-wise direct product is solved successfully we need to gain access to the verification circuit
for the puzzle generated in the $i$-th round of the interaction between $P^{(g)}$ and $\widetilde{C}$.
Therefore, the algorithm EvalutePuzzles runs $k$ times $P^{(1)}$ to simulate the interaction with
$C_1(\rho)$ each time with a fresh random bitstring $\pi_i \in \{0,1\}^{n}$ where $1 \leq i \leq k$.
Let us introduce some additional notation.
We denote by $\langle P^{(1)}(\pi_i), C_1(\rho)\rangle^i$ the execution of the $i$-th round of the sequential interaction.
We use $\langle P^{(1)}(\pi_i), C_1(\rho)\rangle^i_{P^{(1)}}$ to denote the output of $P^{(1)}(\pi_i)$ in the $i$-th round.
Finally, we write $\langle P^{(1)}(\pi_i), C_1(\rho)\rangle^i_{\mathit{trans}}$ to denote the transcript of communication in the $i$-th round.
Finally, we note that the $i$-th round of the interaction between $P^{(1)}$ and $C_1$ is well defined only if all previous rounds have been executed before.

To make the notation easier in the code excerpts of circuits $C_2$, $D_2$ and EvalutePuzzles we
omit some oracle superscripts. Exemplary, we write $\widetilde{C}_2^{\Gamma_H^{(k)}, \hash}$ instead of $\widetilde{C}_2^{\Gamma_H^{(k)}, C, \hash}$ where
the superscript of the oracle circuit $C$ is omitted. We make sure that it is clear from a context which oracles are used.

\begin{codeblock}
  \textbf{Algorithm} $\text{EvaluatePuzzles}^{P^{(1)}, \widetilde{C}, \hash}(\pi^{(k)}, \rho, n, k)$
  \medskip \hrule \medskip
  \textbf{Oracle:}  A problem poser $P^{(1)}$, a solver circuit $\widetilde{C} = (C_1, \widetilde{C}_2)$ for $P^{(g)}$,\\
  \IndII a function $hash : Q \rightarrow \{0,1,\dots, 2(h+v)-1\}$.\\
  \textbf{Input:} Bitstrings $\pi^{(k)} \in \{0,1\}^{kn}$, $\rho \in \{0,1\}^{*}$, parameters $n$, $k$.\\
  \textbf{Output}: A tuple $(c_1, \dots, c_k) \in \{0,1\}^{k}$.
  \medskip\hrule\medskip
  %
  \For $i:=1$ \To $k$ \Do \IndII \textit{//simulate $k$ rounds of interaction} \\
  \IndI $(\Gamma_V^{i}, \Gamma_H^{i}) := \langle P^{(1)}(\pi_i), C_1(\rho) \rangle_{P^{(1)}}^i$\\
  \IndI $x_i := \langle P^{(1)}(\pi_i), C_1(\rho) \rangle^i_{\mathit{trans}}$ \\
  $x := (x_1, \dots, x_k)$ \\
  $\Gamma_H^{(k)} := (\Gamma_H^1, \dotsc, \Gamma_H^k)$ \\
  $(q, y_1, \dots, y_k) := \widetilde{C}_2^{\Gamma_H^{(k)}, hash} (x, \rho)$ \\
  \If $(q, y_1, \dots, y_k) = \bot$ \Then \\
  \IndI \Return $(0, \dotsc, 0)$ \\
  $(c_1, \dotsc, c_k) := (\Gamma_V^{1}(q, y_1), \dotsc, \Gamma_V^{k}(q, y_k))$\\
  \Return $(c_1, \dotsc, c_k)$
\end{codeblock}
%
All puzzles used by EvalutePuzzles are generated internally. Thus the algorithm
has access to hint circuit, and can answer itself all queries of $\widetilde{C}_2$.

% For fixed $\pi^{(k)}, \rho$ let $(\Gamma_V^{(g)}, \Gamma_H^{(k)}) := \langle P^{(g)}(\pi^{(k)}), C_1(\rho) \rangle_{P^{(g)}}$
% and $x := \langle P^{(g)}(\pi^{(k)}), C_1(\rho) \rangle_{\text{trans}}$.
% Additionally, we denote by $(\Gamma_V^{i}, \Gamma_H^{i})$ the verification and hint circuits generated by $P^{(1)}(\pi_i)$ in the $i$-th
% round of the simulated interaction with $C_1(\rho)$.
% For $(q,y_1, \dots, y_k) := \widetilde{C}_2(x, \rho)$ we denote the output of $\Gamma_V^i(q,y_i)$ by $c_i$.
We are interested in the success probability of $\widetilde{C}$ with the bitstring $\pi_1$ fixed to $\pi^*$ when
the fact whether $\widetilde{C}$ succeeds in solving the first puzzle defined by $P^{(1)}(\pi_1)$ is neglected, and
instead a bit $b$ is used. More formally, we define the surplus $S_{\pi^*, b}$ as
\begin{align}
  \label{eq:s_pi_b}
S_{\pi^*, b} = \underset{\pi^{(k)}, \rho}{\Pr}\left[g(b, c_2, \dots, c_k) = 1 \mid \pi_1 = \pi^*\right] - \underset{u \leftarrow \mu^{k}_{\delta}}{\Pr}\left[g(b, u_2, \dots, u_k) = 1\right].
\end{align}
%
The algorithm EstimateSurplus returns an estimate $\widetilde{S}_{\pi^*, b}$ for $S_{\pi^*, b}$.
%
\begin{codeblock}
  \textbf{Algorithm} $\text{EstimateSurplus}^{P^{(1)}, \widetilde{C}, g, \hash}(\pi^*, b, k, \epsilon, \delta, n)$
  \medskip
  \hrule
  \medskip
  \textbf{Oracle:} A problem poser $P^{(1)}$, a circuit $\widetilde{C}$ for $P^{(g)}$, a function $g: \{0,1\}^{k} \rightarrow \{0,1\}$ \\
  \IndII a function $\hash : Q \rightarrow \{0,1,\dots, 2(h+v)-1\}$.\\
  \textbf{Input:} A bistring $\pi^* \in \{0,1\}^{n}$, a bit $b \in \{0,1\}$, parameters $k$, $\epsilon$, $\delta$, $n$.\\
  \textbf{Output:} An estimate $\widetilde{S}_{\pi^*, b}$ for $S_{\pi^*, b}$.
  \medskip\hrule\medskip
  \For $i:=1$ \To $\frac{64k^2}{\epsilon^2}n$ \Do \\
  \IndI $(\pi_{2}, \dots, \pi_k) \xleftarrow{\$} \{0,1\}^{(k-1)n}$\\
  \IndI $\rho \xleftarrow{\$} \{0,1\}^{*}$\\
  \IndI $(c_1, \dots, c_k) := \text{EvalutePuzzles}^{P^{(1)}, \widetilde{C}, \hash}((\pi^*, \pi_2, \dots, \pi_k), \rho, n, k)$\\
  \IndI $\widetilde{s}_{\pi^*,b}^i := g(b, c_{2}, \dots, c_k)$\\
  $\widetilde{g}_b := \text{EstimateFunctionProbability}^{g}(b, k, \epsilon, \delta, n)$ \\
  \textbf{return} $\Bigl(\frac{\epsilon^2}{64k^2n} \sum_{i=1}^{\frac{64k^2}{\epsilon^2} n} \widetilde{s}_{\pi^*,b}^i \Bigr) - \widetilde{g}_b$\\
\end{codeblock}
%
\begin{lemma}
  \label{lemma:surplus_estimate}
The estimate $\widetilde{S}_{\pi^*,b}$ returned by EstimateSurplus differs from $S_{\pi^*, b}$ by at most $\frac{\epsilon}{4k}$ almost surely.
\end{lemma}

\begin{proof}
We use the union bound and similar argument as in Lemma \ref{lemma:estimate_of_g}
which yields that \\$\frac{\epsilon^2}{64k^2n} \sum_{i=1}^{\frac{64k^2}{\epsilon^2}n} \widetilde{s}_{\pi^*,b}^i$ differs from
$\mathbb{E}[g(b, c_2, \dots, c_k)]$ by at most $\frac{\epsilon}{8k}$ almost surely. Together, with Lemma $\ref{lemma:estimate_of_g}$ we conclude that the surplus estimate
returned by EstimateSurplus differs from $S_{\pi^*,b}$ by at most $\frac{\epsilon}{4k}$ almost surely.
\end{proof}
%
%
We define now the following circuit $C' = (C_1', C_2')$, which is a solver for a $(k-1)$-wise direct product of $P^{(1)}$.
\begin{codeblock}
  \textbf{Circuit} $C_1'^{\widetilde{C}, P^{(1)}}(\rho)$
  \medskip \hrule \medskip
  \textbf{Oracle:} A solver circuit $\widetilde{C} = (C_1, \widetilde{C}_2)$ for $P^{(g)}$, a poser $P^{(1)}$. \\
  \textbf{Input:}  A bitstring $\rho \in \{0,1\}^{*}$ \\
  \textbf{Hard-coded:} A bitstring $\pi^* \in \{0,1\}^{n}$
  \medskip\hrule\medskip
  Simulate $\langle P^{(1)}(\pi^*), C_1(\rho)\rangle^1$ \\
  Use $C_1(\rho)$ for the remaining $k-1$ rounds of interaction.
\end{codeblock}
%
%
\begin{codeblock}
  \textbf{Circuit} $\widetilde{C}_2'^{\Gamma_H^{(k-1)}, \widetilde{C}, \hash}(x^{(k-1)}, \rho)$
  \medskip \hrule \medskip
  \textbf{Oracle:} A hint oracle $\Gamma_H^{(k-1)} := (\Gamma_H^{2}, \dots, \Gamma_H^{k})$, a solver circuit $\widetilde{C} = (C_1, \widetilde{C}_2)$ for $P^{(g)}$, \\
  \IndII a function $\hash: Q \rightarrow \{0,1,\dots, 2(h+v)-1\}$ \\
  \textbf{Input:}  A transcript of $k-1$ rounds of interaction $x^{(k-1)} := (x_2, \dotsc, x_{k}) \in \{0,1\}^{*}$, \\
  \IndII a bitstring $\rho \in \{0,1\}^{*}$\\
  \textbf{Hard-coded:} A bitstring $\pi^* \in \{0,1\}^{n}$
  \medskip\hrule\medskip
  Simulate $\langle P^{(1)}(\pi^*), C_1(\rho) \rangle^{1}$ \\
  \IndI $(\Gamma_H^*, \Gamma_V^*) := \langle P^{(1)}(\pi^*), C_1(\rho) \rangle^{1}_{P^{(1)}}$ \\
  \IndI $x^* := \langle P^{(1)}(\pi^*), C_1(\rho) \rangle^{1}_{\mathit{trans}}$ \\
  $\Gamma_H^{(k)} := (\Gamma_H^*, \Gamma_H^{2}, \dots, \Gamma_H^{k})$ \\
  $x^{(k)} := (x^*, x_2, \dots, x_{k})$ \\
  $(q, y_1, \dots, y_k) := \widetilde{C}_2^{\Gamma_H^{(k)}, \mathit{hash}}(x^{(k)}, \rho)$ \\
  \Return $(q, y_2, \dots, y_k)$
\end{codeblock}
%
We are ready to define the solver circuit $D = (D_1, D_2)$ for $P^{(1)}$ and the algorithm Gen.
%
\begin{codeblock}
  \textbf{Circuit} $D_1^{\widetilde{C}}(r)$
  \medskip \hrule \medskip
  \textbf{Oracle:} A solver circuit $\widetilde{C} = (C_1, \widetilde{C}_2)$ for $P^{(g)}$.\\
  \textbf{Input:} A pair $r := (\rho, \sigma)$ where $ \rho \in \{0,1\}^{*}$ and $\sigma \in \{0,1\}^{*}$.
  \medskip\hrule\medskip
  Interact with the problem poser $\langle P^{(1)}, C_1(\rho) \rangle^1$. \\
  Let $x^* := \langle P^{(1)}, C_1(\rho) \rangle^1_{\mathit{trans}}$.
\end{codeblock}
%
\begin{codeblock}
  \textbf{Circuit} $D_2^{P^{(1)}, \widetilde{C}, \mathit{hash}, g,  \Gamma_H}(x^*, r)$
  \medskip \hrule \medskip
  \textbf{Oracle:} A poser $P^{(1)}$, a solver circuit $\widetilde{C} = (C_1, \widetilde{C}_2)$ for $P^{(g)}$, \\
  \IndII functions $hash : Q \rightarrow \{0,1, \dots, 2(h+v)-1\}$, $g:\{0,1\}^k \rightarrow \{0,1\}$, \\
  \IndII a hint circuit $\Gamma_H^*$ for $P^{(1)}$. \\
  \textbf{Input:} A communiation transcript $x^* \in \{0,1\}^{*}$, a bitstring $r := (\rho, \sigma)$ \\
  \IndII where $\rho \in \{0,1\}^{*}$ and $\sigma \in \{0,1\}^{*}$\\
  \textbf{Output}: A pair $(q, y^*)$.
  \medskip \hrule \medskip
  \For at most $\frac{6k}{\epsilon} \log(\frac{6k}{\epsilon})$ iterations \Do \\
  \IndI $(\pi_2, \dots, \pi_k) \leftarrow$ read next $(k-1)\cdot n$ bits from $\sigma$ \\
  \IndI Use $x^*$ to simulate the first round of interiaction of $C_1(\rho)$ with the problem poser $P^{(1)}$\\
  \IndI \For $i:=2$ \To $k$ \Do \\
  \IndII \Run $\langle P^{(1)}(\pi_i), C_1(\rho)\rangle^i$ \\
  \IndIII $(\Gamma_V^{i}, \Gamma_H^{i}) := \langle P^{(1)}(\pi_i), C_1(\rho) \rangle^i_{P^{(1)}}$ \\
  \IndIII $x_i := \langle P^{(1)}(\pi_i), C_1(\rho) \rangle^i_{\mathit{trans}}$ \\
  \IndI $\Gamma_H^{(k)}(q) := (\Gamma_H(q), \Gamma_H^{2}(q), \dots, \Gamma_H^{k}(q))$ \\
  \IndI $(q, y^*, y_2, \dots, y_k) := \widetilde{C}_2^{\Gamma_H^{(k)}, \hash}(x^*, x_2, \dotsc, x_k, \rho)$\\
  \IndI $(c_2, \dots, c_k) := (\Gamma_V^2(q, y_2), \dotsc, \Gamma_V^{k}(q, y_k))$ \\
  \IndI \If $g(1, c_{2}, \dots, c_k) = 1$ \And $g(0,c_{2}, \dots, c_k) = 0$ \Then \\
  \IndII \Return $(q, y^*)$ \\
  \Return $\bot$
%
\end{codeblock}
%
%
\begin{codeblock}
  \textbf{Algorithm} $\text{Gen}^{P^{(1)}, \widetilde{C}, g, \mathit{hash}}(\epsilon, \delta, n, k)$
  \medskip \hrule \medskip
  \textbf{Oracle:} A poser $P^{(1)}$, a solver circuit $\widetilde{C}$ for $P^{(g)}$, functions $g: \{0,1\}^{k} \rightarrow \{0,1\}$, \\
  \IndII  $hash:Q \rightarrow \{0,1, \dots, 2(h + v) - 1\}$. \\
  \textbf{Input:}  Parameters $\epsilon$, $\delta$, $n$, $k$.\\
  \textbf{Output:} A circuit $D$.
  \medskip\hrule\medskip
  \For $i:=1$ \To $\frac{6k}{\epsilon}n$ \Do \\
  \IndI $\pi^* \xleftarrow{\$} \{0,1\}^{n}$\\
  \IndI $\widetilde{S}_{\pi^*,0} := \text{EstimateSurplus}^{P^{(1)},  \widetilde{C}, g, hash}(\pi^*, 0, k, \epsilon, \delta,n)$\\
  \IndI $\widetilde{S}_{\pi^*,1} := \text{EstimateSurplus}^{P^{(1)},  \widetilde{C}, g, hash}(\pi^*, 1, k, \epsilon, \delta,n)$\\
  \IndI \If $ \exists b \in \{0,1\}: \widetilde{S}_{\pi^*,b} \geq (1 - \frac{3}{4k}) \epsilon$ \Then \\
  \IndII Let $C_1'$ have oracle access to $\widetilde{C}$, and have hard-coded $\pi^*$ \\
  \IndII Let $\widetilde{C}_2'$ have oracle access to $\widetilde{C}$, and have hard-coded $\pi^*$. \\
  \IndII $\widetilde{C}' := (C_1', \widetilde{C}_2')$ \\
  \IndII $g'(b_2, \dots, b_k) := g(b, b_2, \dots, b_k)$\\
  \IndII\Return $Gen^{P^{(1)}, \widetilde{C}', g', hash}(\epsilon, \delta, n, v, h, k-1)$ \\
  // \textit{all estimates are lower than $(1-\frac{3}{4k})\varepsilon$}\\
  \Return $D^{P^{(1)}, \widetilde{C}, hash, g}$
\end{codeblock}

\begin{proof}[Lemma \ref{lemma:sec_amp_for_p_hash}]
First let us consider the case where $k=1$. The function $g: \{0,1\} \rightarrow \{0,1\}$ is either the identity or a constant function.
If $g$ is the identity function, then the circuit $D$ returned by Gen directly uses $\widetilde{C}$ to find a solution.
From the assumptions of Lemma \ref{lemma:sec_amp_for_p_hash} we know that $\widetilde{C}$ succeeds with probability at least
$\delta + \epsilon$. Hence, $D$ trivially satisfies the statement of Lemma \ref{lemma:sec_amp_for_p_hash}.
If $g$ is a constant function the statement is vacuously true.

The general case is more involved. We distinguish two possibilities.
If Gen manages to find in one of the iterations $\pi^*$ such that an estimate
$\widetilde{S}_{\pi^*,b} \geq (1-\frac{3}{4k})\epsilon$, then we define a new monotone function
$g'(b_2, \dots, b_k) := g(b, b_2, \dots, b_k)$ and a circuit $\widetilde{C}' = (C_1', \widetilde{C}_2')$ with oracle access to $\widetilde{C} := (C_1, \widetilde{C}_2)$.
We know that the surplus estimate satisfies $\widetilde{S}_{\pi^*, b} \geq (1 - \frac{3}{4k})\epsilon$, thus by Lemma \ref{lemma:surplus_estimate}
we conclude that $S_{\pi^*,b} \geq \widetilde{S}_{\pi^*, b} - \frac{\epsilon}{4k} \geq (1 - \frac{1}{k})\epsilon$ almost surely.
Therefore, the circuit $\widetilde{C}'$ succeeds in solving the $(k-1)$-wise direct product of puzzles with probability
at least $\Pr_{u \leftarrow \mu^{(k-1)}_{\delta}}[g'(u_1,\dots, u_{k-1} )] + (1 - \frac{1}{k})\epsilon$.
We see that in this case $\widetilde{C}'$ satisfies the conditions of Lemma \ref{lemma:sec_amp_for_p_hash}
for the $(k-1)$-wise direct product of puzzles, and we can call Gen recursively.

If all estimates are less than $(1-\frac{3}{4k})\epsilon$, then intuitively $C$
does not succeed on the remaining $k-1$ puzzles with much higher probability than
an algorithm that correctly solves each puzzle with probability $\delta$.
However, from the assumptions of Lemma \ref{lemma:sec_amp_for_p_hash} we know that on all
$k$ puzzles the success probability of $\widetilde{C}$ is higher.
Therefore, it is likely that the first puzzle is correctly solved unusually often.
It remains to prove that this intuition is indeed correct.

We fix the notation used in the code excerpt of the circuit $D_2$.
Additionally, we define $c_1 := \Gamma_V(q,y_1)$, where $\Gamma_V$ is the verification circuit generated
by $P^{(1)}(\pi_1)$ in the first phase after the interaction with $D_1(r)$.
Let $\cG_{b} := \{ b_1, b_2, \dots, b_k : g(b, b_2, \dots, b_k) = 1 \}$ and $c = (c_1, c_2, \dotsc, c_k)$.
We note that these are equivalent
\begin{align}
  \label{eqs:set_g}
  \underset{u \leftarrow \mu_{\delta}^k}{\Pr}[u \in \cG_b] = \underset{u \leftarrow \mu_{\delta}^k}{\Pr}[g(b, u_2, \dots, u_k) = 1]\notag\\
 \underset{\pi^{(k)}, \rho}{\Pr}[c \in \cG_b] = \underset{\pi^{(k)}, \rho}{\Pr}[g(b, c_2, \dots, c_k) = 1].
\end{align}
We fix the randomness of the problem poser $P^{(1)}$ to $\pi^*$ and use \eqref{eq:s_pi_b}, \eqref{eqs:set_g} to obtain
\begin{align}
\label{eq:diff_s01}
\underset{u \leftarrow \mu_{\delta}^k}{\Pr}[u \in \cG_1] - \underset{u \leftarrow \mu_{\delta}^k}{\Pr}[u \in \cG_0] =
\underset{\pi^{(k)}, \rho}{\Pr}[c \in \cG_1 \mid \pi_1 = \pi^*] - \underset{\pi^{(k)}, \rho}{\Pr}[c \in \cG_0 \mid \pi_1 = \pi^*] - (S_{\pi^*, 1} - S_{\pi^*,0})
\end{align}
Since $g$ is a monotone function we have $\cG_0 \subseteq \cG_1$. Therefore, we can write \eqref{eq:diff_s01} as
\begin{align}
  \label{eq:diff_s01_next}
  \underset{u \leftarrow \mu_{\delta}^k}{\Pr}[u \in \cG_1 \setminus \cG_0] = \underset{\pi^{(k)}, \rho}{\Pr}[c \in \cG_1 \setminus \cG_0 \mid \pi_1 = \pi^*] - (S_{\pi^*,1} - S_{\pi^*,0}).
\end{align}
Still fixing $\pi_1 = \pi^*$ we multiply both sides of (\ref{eq:diff_s01_next}) by
\begin{align*}
\underset{
  \mathclap{
    \substack{r \\ x^* := \langle P^{(1)}(\pi^*), D_1(r) \rangle_{\mathit{trans}}
    \\ (\Gamma_V, \Gamma_H) := \langle P^{(1)}(\pi^*), D_1(r) \rangle_{P^{(1)}} }}}
{\Pr}\mkern13mu [\Gamma_V(D_2(x^*, r)) = 1]
/ \underset{u \leftarrow \mu_{\delta}^k}{\Pr}[ u \in \cG_1 \setminus\cG_0].
\end{align*}
%
which yields
\begin{align}
\label{eq:pr_d_succ_0}
&\underset{
  \mathclap{
    \substack{r \\ x^* := \langle P^{(1)}(\pi^*), D_1(r) \rangle_{\mathit{trans}} \\ (\Gamma_V, \Gamma_H) := \langle P^{(1)}(\pi^*), D_1(r) \rangle_{P^{(1)}} }}}
{\Pr}\mkern13mu[\Gamma_V(D_2(x^*, r)) = 1] \notag\\
%
&\IndII = \mkern13mu
  \underset{
    \mathclap{
      \substack{r \\ x^* := \langle P^{(1)}(\pi^*), D_1 (r) \rangle_{\mathit{trans}} \\ (\Gamma_V, \Gamma_H) := \langle P^{(1)}(\pi^*), D_1 (r) \rangle_{P^{(1)}} }}}
  {\Pr}\mkern13mu[\Gamma_V(D_2(x^*, r)) = 1]
  \underset{\pi^{(k)},\rho}{\Pr}[c \in \cG_1 \setminus \cG_0 \mid \pi_1 = \pi^*]
\frac{1}{\underset{u \leftarrow \mu_{\delta}^k}{\Pr}[ u \in \cG_1 \setminus \cG_0]}\notag\\
%
&\IndIII - \mkern13mu
\underset{
  \mathclap{
  \substack{r \\ x^* := \langle P^{(1)}(\pi^*), D_1(r) \rangle_{\mathit{trans}} \\ (\Gamma_V, \Gamma_H) := \langle P^{(1)}(\pi^*), D_1(r) \rangle_{P^{(1)}} }}}
{\Pr}\mkern13mu[\Gamma_V(D_2(x^*, r)) = 1](S_{\pi^*,1} - S_{\pi^*,0})
\frac{1}{\underset{u \leftarrow \mu_{\delta}^k}{\Pr}[ u \in \cG_1 \setminus\cG_0]}.
\end{align}
We analyze the first summand of \eqref{eq:pr_d_succ_0}. First, we have
\begin{align}
  \label{eq:pr_gamma_v_0}
  \IndII &\underset{
    \mathclap{
      \substack{r \\
        x^* := \langle P^{(1)}(\pi^*), D_1 (r) \rangle_{\mathit{trans}} \\
        (\Gamma_V, \Gamma_H) := \langle P^{(1)}(\pi^*), D_1(r) \rangle_{P^{(1)}} }}}
  {\Pr}\mkern13mu[\Gamma_V(D_2(x^*, r)) = 1] \notag\\
  &\IndI = \underset{
    \mathclap{
      \substack{r \\
        x^* := \langle P^{(1)}(\pi^*), D_1 (r) \rangle_{\mathit{trans}} \\
        (\Gamma_V, \Gamma_H) := \langle P^{(1)}(\pi^*), D_1(r) \rangle_{P^{(1)}} }}}
  {\Pr}[\Gamma_V(D_2(x^*, r)) = 1 | D_2(x^*,r) \neq \bot]
  \underset{\mathclap{\substack{r \\ x^* = \langle P^{(1)}(\pi^*), D_1(r) \rangle_{\mathit{trans}}}}}{\Pr}[D_2(x^*,r) \neq \bot] \notag\\
  &\IndI \stackrel{(*)}{=}
  \underset{\pi^{(k)}, \rho}{\Pr}[c_1 = 1 \mid c \in \cG_1 \setminus \cG_0, \pi_1 = \pi^*]
  \underset{\mathclap{\substack{r \\ x^* = \langle P^{(1)}(\pi^*), D_1(r) \rangle_{\mathit{trans}}}}} {\Pr}[D_2(x^*,r) \neq \bot],
\end{align}
where in $(*)$ we use the observation that $D_2(x^*, r) \neq \bot$ implies that the circuit $D_2(x^*, r)$ finds $\pi^{(k)}$
such that $c \in \cG_1 \setminus \cG_0$, and if $\Gamma_V(D_2(x^*,r)) = 1$ then $c_1 = 1$.
Inserting \eqref{eq:pr_gamma_v_0} to the numerator of the first summand of (\ref{eq:pr_d_succ_0}) yields
\begin{align}
\IndII &\underset{
  \mathclap{
  \substack{r \\
    x^* := \langle P^{(1)}(\pi^*), D_1 (r) \rangle_{\mathit{trans}} \\
    (\Gamma_V, \Gamma_H) := \langle P^{(1)}(\pi^*), D_1(r) \rangle_{P^{(1)}} }}}
{\Pr}\mkern13mu[\Gamma_V(D_2(x^*, r)) = 1]
\underset{\pi^{(k)},\rho}{\Pr}[c \in \cG_1 \setminus \cG_0 \mid \pi_1 = \pi^*] \notag\\
  &\IndII = \underset{\mathclap{\substack{r
      \\ x^* = \langle P^{(1)}(\pi^*), D_1(r) \rangle_{\mathit{trans}}}}}
  {\Pr}\mkern13mu[D_2(x^*,r) \neq \bot]
  \underset{\pi^{(k)}, \rho}{\Pr}[c_1 = 1 \mid c \in \cG_1 \setminus \cG_0, \pi_1 = \pi^*]
  \underset{\pi^{(k)}, \rho}{\Pr}[c \in \cG_1 \setminus \cG_0 \mid \pi_1 = \pi^*].
\end{align}
We consider the following two cases. If $\Pr_{\pi^{(k)}, \rho}[ c \in \cG_1 \setminus \cG_0 \mid \pi_1 = \pi^*] \leq \frac{\epsilon}{6k}$ then
\begin{align}
  \label{ineq:case_0}
  \underset{\pi^{(k)}, \rho}{\Pr}[c_1 = 1 \mid c \in \cG_1 \setminus \cG_0, \pi_1 = \pi^*] \underset{\pi^{(k)}, \rho}{\Pr}[c \in \cG_1 \setminus \cG_0 \mid \pi_1 = \pi^*] \leq \frac{\epsilon}{6k}.
\end{align}
When $\Pr_{\pi^{(k)}, \rho}[c \in \cG_1 \setminus \cG_0 \mid \pi_1 = \pi^*] > \frac{\epsilon}{6k}$ the circuit $D_2$ outputs $\bot$
if and only if it fails in all $\frac{6k}{\epsilon} \log(\frac{6k}{\epsilon})$ iterations to find $\pi^{(k)}$ such that $c \in \cG_1 \setminus \cG_0$
which happens with probability
\begin{align}
  \label{ineq:case_1}
\underset{
  \mathclap{
    \substack{
      r \\
      x^* := \langle P^{(1)}(\pi^*), D_1(r) \rangle_{\mathit{trans}}}}}
{\Pr}[D_2(x^*,\rho) = \bot]
\leq (1 - \frac{\epsilon}{6k})^{\frac{6k}{\epsilon}\log(\frac{6k}{\epsilon})} \leq \frac{\epsilon}{6k}.
\end{align}
We conclude that in both cases by \eqref{ineq:case_0} and \eqref{ineq:case_1} we have
\begin{align}
  \label{ineq:first_part}
  &\underset{
    \mathclap{
    \substack{r \\
      x^* := \langle P^{(1)}(\pi^*), D_1(r) \rangle_{\mathit{trans}}}}}
  {\Pr}\mkern13mu[D_2(x^*,r) \neq \bot]
  \underset{\pi^{(k)}, \rho}{\Pr}[c_1 = 1 \mid c \in \cG_1 \setminus \cG_0, \pi_1 = \pi^*]
  \underset{\pi^{(k)}, \rho}{\Pr}[c \in \cG_1 \setminus \cG_0 \mid \pi_1 = \pi^*] \notag\\
  &\IndII \geq \underset{\pi^{(k)}, \rho}{\Pr}[c_1 = 1 \mid c \in \cG_1 \setminus \cG_0, \pi_1 = \pi^*]\underset{\pi^{(k)}, \rho}
  {\Pr}[c \in \cG_1 \setminus \cG_0 \mid \pi_1 = \pi^*] - \frac{\epsilon}{6k} \notag\\
  &\IndII = \underset{\pi^{(k)}, \rho}{\Pr}[c_1 = 1 \land c \in \cG_1 \setminus \cG_0 \mid \pi_1 = \pi^*] - \frac{\epsilon}{6k} \notag\\
  &\IndII = \underset{\pi^{(k)}, \rho}{\Pr}[g(c) = 1 \mid \pi_1 = \pi^*] -  \underset{\pi^{(k)}, \rho}{\Pr}[c \in \cG_0 \mid \pi_1 = \pi^*] - \frac{\epsilon}{6k} \notag\\
  &\IndII \stackrel{(\ref{eq:s_pi_b})}{=}
   \underset{\pi^{(k)}, \rho}{\Pr}[g(c) = 1 \mid \pi_1 = \pi^*] -  \underset{u \leftarrow \mu_{\delta}^{(k)}}{\Pr}[u \in \cG_0]  - S_{\pi^*,0} - \frac{\epsilon}{6k}.
\end{align}
We take the expected value of \eqref{eq:pr_d_succ_0} over $\pi^*$ and insert \eqref{ineq:first_part} to obtain
\begin{align}
  \label{ineq:ex_pr_gamma}
\underset{
  \mathclap{
    \substack{r \\ x := \langle P^{(1)}(\pi), D_1(r) \rangle_{\mathit{trans}} \\ (\Gamma_V, \Gamma_H) := \langle P^{(1)}(\pi), D_1(r) \rangle_{P^{(1)}} }}}
{\Pr}\mkern13mu[\Gamma_V(D_2(x, r)) = 1]
&\geq \mathbb{E}_{\pi^*}\left[\frac{\Pr_{\pi^{(k)}, \rho}[g(c) = 1 | \pi_1 = \pi^*]
  - \Pr_{u \leftarrow \mu_{\delta}^{(k)}}[u \in \cG_0] - \frac{\epsilon}{6k}}{\Pr_{u \leftarrow \mu_{\delta}^{(k)}}[u \in \cG_1 \setminus \cG_0]}\right] \notag\\
&\IndI - \mathbb{E}_{\pi^*}\Bigl[\Bigl(S_{\pi^*,0} \mkern13mu + \mkern13mu
\underset{\mathclap{
  \substack{r \\ x^* := \langle P^{(1)}(\pi^*), D_1(r) \rangle_{\mathit{trans}} \\ (\Gamma_V, \Gamma_H) := \langle P^{(1)}(\pi^*), D_1(r) \rangle_{P^{(1)}} }}}
{\Pr}[\Gamma_V(D_2(x^*, r)) = 1](S_{\pi^*,1} - S_{\pi^*,0})\Bigr)
\frac{1}{\underset{u \leftarrow \mu_{\delta}^k}{\Pr}[ u \in \cG_1 \setminus\cG_0]}\Bigr].
\end{align}
For the second summand of \eqref{ineq:ex_pr_gamma} we show that if Gen does not recurse, then the majority of estimates is low almost surely.
Let us assume that
\begin{align}
\underset{\pi, \rho}{\Pr}\left[\left(S_{\pi,0} \leq (1 - \frac{1}{2k})\epsilon\right) \land \left( S_{\pi,1} \leq (1-\frac{1}{2k})\epsilon\right)\right] < 1 - \frac{\epsilon}{6k},
\end{align}
then the algorithm Gen recurses with almost surely, because the probability that
Gen does not find $\widetilde{S}_{\pi, b} \geq (1-\frac{3}{4k})$ in none of the $\frac{6k}{\epsilon}n$ iterations is at most
\begin{align*}
  (1 - \frac{\epsilon}{6k})^{\frac{6k}{\epsilon}n} \leq e^{-n}.
\end{align*}
Therefore, under the assumption that Gen does not recurse, we have with high probability
\begin{align}
\underset{\pi, \rho}{\Pr}\left[\left(S_{\pi,0} \leq (1 - \frac{1}{2k})\epsilon\right) \land \left( S_{\pi,1} \leq (1-\frac{1}{2k})\epsilon\right)\right] \geq 1 - \frac{\epsilon}{6k}.
\end{align}
Let us define a set
\begin{align}
  \cW = \left\{ \pi :  \left(S_{\pi,0} \leq (1 - \frac{1}{2k})\epsilon\right) \land \left( S_{\pi,1} \leq (1-\frac{1}{2k})\epsilon \right) \right\}
\end{align}
and use $\cW^c$ to denote the complement of $\cW$.
We bound the numerator of the second summand in (\ref{ineq:ex_pr_gamma})
\begin{align}
  \label{ineq:second_eq}
&\mathbb{E}_{\pi^*}[ S_{\pi^*,0}
\mkern23mu
+
\mkern23mu
\underset{
  \mathclap{
  \substack{r \\ x^* := \langle P^{(1)}(\pi^*), D_1(r) \rangle_{\mathit{trans}}
    \\ (\Gamma_V, \Gamma_H) := \langle P^{(1)}(\pi^*), D_1 (r) \rangle_{P^{(1)}} }}}
{\Pr}\mkern13mu[\Gamma_V(D_2(x^*, r)) = 1]
(S_{\pi^*,1} - S_{\pi^*,0})] \notag\\
%
&\IndII = \mathbb{E}_{\pi^* \in \cW^c}[ S_{\pi^*,0}
\mkern23mu + \mkern23mu
\underset{
  \mathclap{
  \substack{r \\ x^* := \langle P^{(1)}(\pi^*), D_1(r) \rangle_{\mathit{trans}}
    \\ (\Gamma_V, \Gamma_H) := \langle P^{(1)}(\pi^*), D_1 (r) \rangle_{P^{(1)}} }}}
{\Pr}\mkern13mu[\Gamma_V(D_2(x^*, r) = 1]
  (S_{\pi^*,1} - S_{\pi^*,0})] \notag\\
&\IndIII +  \mathbb{E}_{\pi^* \in \cW}[ S_{\pi^*,0} \mkern13mu + \mkern13mu
\underset{
  \mathclap{
  \substack{r \\ x^* := \langle P^{(1)}(\pi^*), D_1(r) \rangle_{\mathit{trans}}
    \\ (\Gamma_V, \Gamma_H) := \langle P^{(1)}(\pi^*), D_1 (r) \rangle_{P^{(1)}} }}}
{\Pr}\mkern13mu[\Gamma_V(D_2(x^*, r)) = 1]
(S_{\pi^*,1} - S_{\pi^*,0}) ] \notag\\
&\IndII \leq \frac{\epsilon}{6k} + \mathbb{E}_{\pi^* \in \cW}[ S_{\pi^*,0} \mkern23mu + \mkern23mu
\underset{
  \mathclap{
  \substack{r \\ x := \langle P^{(1)}(\pi^*), D_1(r) \rangle_{\mathit{trans}}
    \\ (\Gamma_V, \Gamma_H) := \langle P^{(1)}(\pi^*), D_1 (r) \rangle_{P^{(1)}} }}}
{\Pr}\mkern13mu[\Gamma_V(D_2^{\widetilde{C}}(x^*, r)) = 1]
((1 - \frac{1}{2k})\epsilon - S_{\pi^*,0})] \notag\\
&\IndII \leq \frac{\epsilon}{6k} + 1 - \frac{\epsilon}{2k} = 1 - \frac{\epsilon}{3k}.
\end{align}
We observe that
\begin{align}
  \label{eq:gu_rel}
\underset{u \leftarrow \mu_{\delta}^k}{\Pr}[g(u) = 1]
&= \Pr[u \in \cG_0 \lor ( u \in \cG_1 \setminus \cG_0 \land u_1 = 1)] \notag\\
&= \Pr[u \in \cG_0] + \Pr[u \in \cG_1 \setminus \cG_0] \Pr[u_1 = 1].
\end{align}
Finally, we insert (\ref{ineq:first_part}) and (\ref{ineq:second_eq}) into equation \eqref{ineq:ex_pr_gamma}, and use (\ref{eq:gu_rel}) to obtain
\begin{align*}
  \IndII
\underset{
  \mathclap{
  \substack{\pi^*, \rho \\ x := \langle P^{(1)}(\pi^*), D_1(\rho) \rangle_{\text{trans}}
    \\ (\Gamma_V, \Gamma_H) := \langle P^{(1)}(\pi^*), D_1 (\rho) \rangle_{P^{(1)}} }}}
{\Pr}\mkern13mu[\Gamma_V(D_2(x, \rho)) = 1]
&\geq \mathbb{E}_{\pi^*}\left[\frac{{\Pr}_{\pi^{(k)}, \rho}[g(c) = 1 \mid \pi_1 = \pi^*] -
{\Pr}_{u \leftarrow \mu_{\delta}^{k}}[u \in G_0] - (1 - \frac{1}{6k})\epsilon} {\Pr_{u \leftarrow \mu_{\delta}^{k}}[u \in \cG_1 \setminus \cG_0]}\right] \notag.
 \end{align*}
 From the assumptions of Lemma \ref{lemma:sec_amp_for_p_hash} we know that $\Pr_{\pi^{(k)}, \rho} [g(c) = 1] \geq \Pr_{u \leftarrow \mu_{\delta}^{(k)}}[g(u) = 1] + \epsilon$,
 thus we get
 \begin{align}
   \label{eq:proof_final}
\underset{
  \mathclap{
  \substack{\pi^*, \rho \\ x := \langle P^{(1)}(\pi^*), D_1(\rho) \rangle_{\text{trans}}
    \\ (\Gamma_V, \Gamma_H) := \langle P^{(1)}(\pi^*), D_1 (\rho) \rangle_{P^{(1)}} }}}
{\Pr}\mkern13mu[\Gamma_V(D_2(x, \rho)) = 1]
 &\geq \frac{ {\Pr}_{u \leftarrow \mu_{\delta}^{k}}[g(u) = 1] + \epsilon -
 \Pr_{u \leftarrow \mu_{\delta}^{k}}[u \in \cG_0] - (1 - \frac{1}{6k})\epsilon} {\Pr_{u \leftarrow \mu_{\delta}^{k}}[u \in \cG_1 \setminus \cG_0]} \notag\\
 &\stackrel{\eqref{eq:gu_rel}}{\geq} \frac{\epsilon + \delta\Pr_{u \leftarrow \mu_{\delta}^{k}}[u \in \cG_1 \setminus \cG_0] - (1 - \frac{1}{6k})\epsilon}
{\Pr_{u \leftarrow \mu_{\delta}^{k}}[u \in \cG_1 \setminus \cG_0]} \geq \delta + \frac{\epsilon}{6k}
\end{align}
Clearly, the running time of Gen is $poly(k, \frac{1}{\epsilon}, n, t)$.
\end{proof}

%
%%% Local Variables:
%%% mode: latex
%%% TeX-master: "../master"
%%% End:
