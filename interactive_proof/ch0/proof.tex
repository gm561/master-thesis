%
\newcommand{\prGx}{(\Gamma_V,\Gamma_H) := \langle P^{(1)}(\pi^*), D_1(\rho) \rangle_{P^{(1)}} \\ x^* := \langle P^{(1)}(\pi^*), D_1(\rho) \rangle_{\text{trans}}}
%
We define the following circuit $\widetilde{C}_2$:
\begin{codeblock}
  \textbf{Circuit $\widetilde{C}_2^{\Gamma_V^{(g)}, \Gamma_H^{(k)}, C_2, hash} (x, \rho)$}
  \medskip \hrule \medskip
  \textbf{Oracle:} $\Gamma_V^{(g)}, \Gamma_H^{(k)}, hash, C_2$ \\
  \textbf{Input:} A transcript $x$, a bitstring $\rho$. \\
  \textbf{Output:} A tuple $(q, y_1, \dots, y_k)$ or $\bot$.
  \medskip\hrule\medskip
  Run $C_2^{\Gamma_V^{(g)},\Gamma_H^{(k)}}(x, \rho)$ \\
  \IndI \If $C_2$ asks a hint query on $q$ \then\\
  \IndII \If $q \in P_{hash}$ \then\\
  \IndIII \return $\bot$\\
  \IndII \textbf{else}\\
  \IndIII answer the query using $\Gamma_H^{(k)}(q)$\\
  \\
  \IndI \If $C_2$ asks a verification query $(q, y_1, \dots, y_k)$ \then \\
  \IndII \If $q \in P_{hash}$ \textbf{then} \\
  \IndIII ask a verification query $(q, y_1, \dots, y_k)$ \\
  \IndIII \return $(q, y_1, \dots, y_k)$ \\
  \IndII \textbf{else} \\
  \IndIII answer the verification query with 0 \\
  \return $\bot$
\end{codeblock}
%
We define a new solver circuit $\widetilde{C} = (C_1, \widetilde{C}_2)$
that in the first phase uses $C_1$ and in the second phase $\widetilde{C}_2$.
%
From a circuit $C$ we can build a circuit $\widetilde{C}$ that asks at most one verification query
$(q, y_1, \dots, y_k)$ such that $q \in P_{hash}$, and every hint query on $q$ is such that $q \notin P_{hash}$.
Furthermore, we write $(q, y_1, \dots, y_k) := \widetilde{C}_2(x, \rho)$ to denote
the verification query $(q, y_1, \dots, y_k)$ asked by $\widetilde{C}_2$.
If $\widetilde{C}_2$ does not ask a verification query we write $\bot := \widetilde{C}_2(x, \rho)$.
%
\begin{lemma}\textbf{(Security amplification of a dynamic weakly verifiable puzzle with respect to $P_{hash}$.)}
  \label{lemma:sec_amp_for_p_hash}
  For fixed $P^{(1)}$ there exists an algorithm $Gen$, with oracle access to:
  $P^{(1)}$, a monotone function $g:\{0,1\}^{(k)} \rightarrow \{0,1\}$,
  a solver circuit $C$ for $P^{(g)}$ and a function $hash : Q \rightarrow \{0, \dots, 2(h+v)-1\}$.
  Additionally, $Gen$ takes as input parameters $\varepsilon, \delta, n$,
  the number of verification queries $v$ and hint queries $h$ asked by $C$, the number of puzzles to solve $k$,
  and outputs a solver circuit $D$ for $P^{(1)}$ as in Definition \ref{def:dwvp}
  such that the following holds: \\
  If $C$ is such that \\
  \begin{align*}
    \underset{\pi^{(k)}, \rho}{\Pr}\left[CanonicalSuccess^{P^{(g)}, C, hash}(\pi^{(k)}, \rho)=1\right] \geq \underset{u \leftarrow \mu_\delta^k}{\Pr}[g(u) = 1] + \varepsilon,
  \end{align*}
  then $D$ satisfies almost surely
  \begin{align*}
    \underset{\pi, \rho}{\Pr}\left[CanonicalSuccess^{P^{(1)}, D, hash}(\pi, \rho)=1\right] \geq (\delta + \frac{\varepsilon}{6k}).
  \end{align*}
  Additionally, $D$ requires oracle access to $g$, $P^{(1)}$, $C$.
  Furthermore, $D$ asks at most $\frac{6k}{\epsilon}\log\left(\frac{6k}{\epsilon}\right) h$ hint queries and at most one verification query.
  Finally, $Size(D) \leq Size(C)\frac{6k}{\varepsilon}$ and $Time(Gen) = poly(k, \frac{1}{\varepsilon}, n, v, h)$.
\end{lemma}
%
\begin{proof}
%
First we define the following procedure that takes as input $b \in \{0,1\}$, and returns
an estimate for $\underset{(u_2, \dots, u_k) \leftarrow \mu_{\delta}^{k-1}}{\Pr}[g(b,u_2, \dots, u_k )=1]$.
%
\begin{codeblock}
  $\textbf{EstimateFunctionProbability}^{g}(b, \epsilon, \delta)$
  \medskip
  \hrule
  \medskip
  \textbf{Oracle:} A function $g$.\\
  \textbf{Input:} A bit $b \in \{0,1\}$, parameters $k$, $\epsilon$ \\
  \textbf{Output:} An estimate $\widetilde{g} \in \left[0,1\right]$.
  \medskip\hrule\medskip
  \textbf{For} $i:=1$ \To $\frac{16k^2}{\epsilon^2}\log(n)$ \Do \\
  \IndI $(u_2, \dots, u_k) \leftarrow \mu_{\delta}^{(k-1)}$ \\
  \IndI $g_i := g(b, u_2, \dots, u_k)$ \\
  \textbf{return} $\frac{\epsilon^2}{16k^2\log(n)} \sum_{i=1}^{\frac{16k^2}{\epsilon^2}\log(n)} g_i$
\end{codeblock}
%
\begin{lemma}[Estimate for the function $g$.]
  \label{lemma:estimate_of_g}
  The procedure $\textbf{EstimateFunctionProbability}^g(b)$ outputs an estimate $\widetilde{g}$
  for the function $g: \{0,1\}^{n} \rightarrow \{0,1\}$ with the first bit fixed to $b \in \{0,1\}$
  such that $| \widetilde{g} - \underset{(u_2,\dots,u_k) \leftarrow \mu_{\delta}^{k}}{\Pr}\left[g(b,u_2, \dots, u_k) = 1\right] | \leq \frac{\epsilon}{4k}$ almost surely.
\end{lemma}
%
\begin{proof}
We define a binary random variable $K_i$ for the event $g_i = 1$. By Chernoff bound we get
\begin{align*}
  \underset{}{\Pr}\left[\left|\frac{\epsilon^2 \log(n)}{16k^2} \sum_{i=1}^{\frac{16k^2}{\epsilon^2}\log(n)} \widetilde{g}_i - \mathbb{E}[K_i]\right| \geq \frac{\epsilon}{4k}  \right]
  \leq 2 \cdot e^{-\log(n)/3}.
\end{align*}
\end{proof}
%
Next we define a procedure $\textbf{EvalutePuzzles}^{C, P^{(1)}, hash}(\pi^{(k)}, \rho)$.
%
\begin{codeblock}
  $\textbf{EvaluatePuzzles}^{P^{(1)}, P^{(g)}, C, hash}(\pi^{(k)}, \rho)$
  \medskip \hrule \medskip
  \textbf{Oracle:}  A circuit $C$, posers $P^{(1)}$, $P^{(g)}$, a function $hash$.\\
  \textbf{Input:} Bitstrings $\pi^{(k)}$, $\rho$.\\
  \textbf{Output}: A tuple $(c_1, \dots, c_k)$.
  %
  \medskip\hrule\medskip
  %
  \Run $\langle P^{(g)}(\pi^{(k)}), C_1(\rho) \rangle$ \\
  \IndI $(\Gamma_V^{(g)}, \Gamma_H^{(k)}) := \langle P(\pi^{(k)}), C_1(\rho) \rangle_{P^{(g)}}$ \\
  \IndI $x := \langle P^{(g)}(\pi^{(k)}), C_1(\rho) \rangle_{\text{trans}}$ \\ \\
  %
  $(q, y_1, \dots, y_k) := \widetilde{C}_2^{\Gamma_V^{(g)}, \Gamma_H^{(k)}} (x, \rho)$ \\
  \For $i:=1$ \To $k$ \Do \IndII //simulate $k$ rounds of sequential interaction \\
  \IndI $(\Gamma_V^{i}, \Gamma_H^{i}) := \langle P^{(1)}(\pi_i), C_1(\rho) \rangle_{P^{(1)}} $\\
  \For $i:=1$ \To $k$ \Do \\
  \IndI $c_i := \Gamma_v^{i}(q, y_i)$\\
  \return $(c_1, \dots, c_k)$
\end{codeblock}
%
All puzzles used by the procedure are generated internally.
Therefore, it is possible to answer all hint and verification queries without calls to hint and verification oracles.
For fixed $\pi^{(k)}, \rho$ let $(\Gamma_V^{(g)}, \Gamma_H^{(k)}) := \langle P^{(g)}(\pi^{(k)}), C_1(\rho) \rangle_{P^{(g)}}$
and $x := \langle P^{(g)}(\pi^{(k)}), C_1(\rho) \rangle_{\text{trans}}$.
Additionally, we denote by $(\Gamma_V^{i}, \Gamma_H^{i})$ the verification and hint circuits generated in the $i$-th
round of the interaction between $P^{(g)}(\pi^{(k)})$ and $C_1(\rho)$.
Finally, for $(q,y_1, \dots, y_k) := \widetilde{C}_2(x^{(k)}, \rho)$ we denote the output of $\Gamma_V^i(q,y_i)$ by $c_i$.
For $b \in \{0,1\}$ we define the surplus
\begin{align}
  \label{eq:s_pi_b}
S_{\pi^*, b} = \underset{\pi^{(k)}, \rho}{\Pr}\left[g(b, c_2, \dots, c_k) = 1 \mid \pi_1 = \pi^*\right] - \underset{(u_2, \dots, u_k) \leftarrow \mu^{(k)}}{\Pr}\left[g(b, u_2, \dots, u_k) = 1\right]
\end{align}
%
The surplus $S_{\pi^*, b}$ tells us how good $\widetilde{C}$ performs when the bitstring $\pi_1$ is fixed to $\pi^*$,
and the fact whether $\widetilde{C}$ succeeds in solving the first puzzle defined by $P^{(1)}(\pi_1)$ is neglected.
Instead, the bit $b$ is used as the input on the first position of the function $g$.

The procedure \textbf{EstimateSurplus} returns an estimate $\widetilde{S}_{\pi^*, b}$ for $S_{\pi^*, b}$.
%
\begin{codeblock}
  $\textbf{EstimateSurplus}^{P^{(1)}, C, hash}(\pi^*, b)$
  \medskip
  \hrule
  \medskip
  \textbf{Oracle:} An algorithm $P^{(1)}$, a circuit $C$, a function $hash$, a function $g$.\\
  \textbf{Input:} A bistring $\pi^*$, a bit $b$, an integer $k$.\\
  \textbf{Output:} A circuit $D$.
  \medskip\hrule\medskip
  $\widetilde{g}_b := \textbf{EvaluteFunctionProbability}^{g}(b, \epsilon, \delta)$ \\
  \textbf{For} $i:=1$ to $\frac{16k^2}{\epsilon^2}\log(n)$ \Do \\
  \IndI $(\pi_{2}, \dots, \pi_k) \xleftarrow{\$} \{0,1\}^{(k-1)n}$\\
  \IndI $\rho \xleftarrow{\$} \{0,1\}^{*}$\\
  \IndI $(c_1, \dots, c_k) := \textbf{EvalutePuzzles}^{P^{(1)}, C, hash}(\pi^*, \pi_2, \dots, \pi_k, \rho)$\\
  \IndI $\widetilde{s}_{\pi^*,b}^i := g(b, c_{2}, \dots, c_k)$\\
  \textbf{return} $\frac{\epsilon^2\log(n)}{16k^2} \sum_{i=1}^{\frac{16k^2}{\epsilon^2}\log(n)} \widetilde{s}_{\pi^*,b}^i - \widetilde{g}_b$\\
\end{codeblock}
%
\begin{lemma}
  \label{lemma:surplus_estimate}
The estimate $\widetilde{S}_{\pi^*,b}$ returned by \textbf{EstimateSurplus} differs from $S_{\pi^*, b}$ by at most $\frac{\epsilon}{2k}$ almost surely.
\end{lemma}

\begin{proof}
We use union bound and similar argument as in Lemma \ref{lemma:estimate_of_g}
which yields that \\$\frac{\epsilon^2\log(n)}{16k^2} \sum_{i=1}^{\frac{16k^2}{\epsilon^2}\log(n)} \widetilde{s}_{\pi^*,b}^i$ differs from
$\mathbb{E}[g(b, c_2, \dots, c_k)]$ by at most $\frac{\epsilon}{4k}$ almost surely. Together, with Lemma $\ref{lemma:estimate_of_g}$ we conclude that the surplus estimate
returned by \textbf{EstimateSurplus} differs from $S_{\pi^*,b}$ by at most $\frac{\epsilon}{2k}$ almost surely.
\end{proof}
From Lemma \ref{lemma:surplus_estimate} we conclude that if $\widetilde{S}_{\pi^*,b} \geq (1-\frac{3}{4k})\epsilon$, then $S_{\pi^*,b} \geq (1-\frac{1}{k})\epsilon$ almost surely.
%
\begin{codeblock}
  \textbf{Circuit $D = (D_1, D_2) (\sigma)$}
  \medskip \hrule \medskip
  \textbf{Phase I $D_1^{C}(\sigma)$}
  \medskip \hrule \medskip
  \textbf{Oracle:} A circuit $C$.\\
  \textbf{Input:} A bitstring $\sigma \in \{0,1\}^{*}$.
  \medskip\hrule\medskip

  Interact with the problem poser $P^{(1)}$ using $C_1(\rho)$. \\
  \IndI Let $x^*$ be the transcript of any internal simulations of $C_1$ and the interaction with the  \\ \IndI problem poser $P^{(1)}$.\\
  \IndI Let $\Gamma_V^*, \Gamma_H^*$ be the verification and hint circuits output by the problem poser $P^{(1)}$.

  \medskip \hrule \medskip
  \textbf{Phase II $D_2^{P^{(1)}, C}(x^*, \sigma)$}
  \medskip \hrule \medskip
  \textbf{Oracle:} $P^{(1)}$, $C$, $hash$, $g$, $\Gamma_V^{*}$, $\Gamma_H^{*}$. \\
  \textbf{Input:} A transcript $x^*$, a bitstring $\sigma \in \{0,1\}^{*}$.\\
  \textbf{Output}: A verification query $(q, y^*)$.
  \medskip \hrule \medskip
  \For at most $\frac{6k}{\epsilon} \log(\frac{6k}{\epsilon})$ iterations \Do \\
  \IndI $\pi^{(k-1)} \leftarrow$ read $(k-1)\cdot n$ bits from $\sigma$ \\
  \IndI \For $i:=2$ \To $k$ \Do \IndIII // Finish remaining $k-1$ interactions. \\
  \IndII Simulate $\langle P^{(1)}(\pi_i), C_1(\rho)\rangle$ \\
  \IndIII $ x_i := \langle P^{(1)}(\pi_i), C_1(\rho)\rangle_{\text{trans}}$ \\
  \IndIII $(\Gamma_V^{i}, \Gamma_H^{i}) := \langle P^{(1)}(\pi_i), C_1(\rho) \rangle_{P^{(1)}}$ \\
  \IndI $\Gamma_V^{(g)} := g(\Gamma_V^{*}, \Gamma_V^{2}, \dots, \Gamma_V^{k})$  \\
  \IndI  $\Gamma_H^{(k)} := (\Gamma_H^{*}, \Gamma_H^{2}, \dots, \Gamma_H^{k})$ \\
  \IndI $(q, y^*, y_2, \dots, y_k) := \widetilde{C}^{\Gamma_V^{(g)}, \Gamma_H^{(k)}, C, hash}((x^*, x_2, \dots, x_k), \rho)$\\
  \IndI $(c^*, c_2, \dots, c_k) := (\Gamma_V^*(q, y^*), \Gamma_V^2(q, y_2), \dots, \Gamma_V^{k}(q, y_k))$ \\
  \IndI \If $g(1, c_{2}, \dots, c_k) = 1 \land g(0,c_{2}, \dots, c_k) = 0$ \then \\
  \IndII Make a verification query $(q, y^*)$ \\
  \IndII \return $(q, y^*)$ \\
  \return $\bot$
%
\end{codeblock}
%
\begin{codeblock}
  \textbf{Algorithm $Gen^{C,P^{(1)},g,hash}(\epsilon, \delta, n, v, h, k)$}
  \medskip
  \hrule
  \medskip
  \textbf{Oracle:} $ P^{(1)}, C, g, hash$ \\
  \textbf{Input:}  $\epsilon, \delta, n, v, h, k$\\
  \textbf{Output:} $D$
  \medskip\hrule\medskip
  \For $i:=1$ to $\frac{6k}{\epsilon}\log(n)$ \Do \\
  \IndI $\pi^* \xleftarrow{\$} \{0,1\}^{n}$\\
  \IndI $\widetilde{S}_{\pi^*,0} := \textbf{EstimateSurplus}^{P^{(1)}, C, hash}(\pi^*, 0)$\\
  \IndI $\widetilde{S}_{\pi^*,1} := \textbf{EstimateSurplus}^{P^{(1)}, C, hash}(\pi^*, 1)$\\
  \IndI \If $ \exists b \in \{0,1\}: \widetilde{S}_{\pi^*,b} \geq (1 - \frac{3}{4k}) \epsilon$ \then \\
  \IndII Let $C_1'$ be as $C_1$ except the first round of interaction between $C_1$ and $P^{(g)}$ which \\
  \IndII is simulated internally by using $P^{(1)}(\pi^*)$\\
  \IndII Let $C_2'$ be as $C_2$ except the solution for the first puzzle which is discarded. \\
  \IndII $C' := (C_1', C_2')$ \\
  \IndII $g'(b_2, \dots, b_k) := g(b, b_2, \dots, b_k)$\\
  \IndII\return $Gen^{C', P^{(1)}, g', hash}(\epsilon, \delta, n, v, h, k-1)$ \\
  // all estimates are lower than $(1-\frac{3}{4k})\varepsilon$\\
  \return $D^{C}$
\end{codeblock}
For $k=1$ the function $g: \{0,1\} \rightarrow \{0,1\}$ is either the identity or a constant function.
If $g$ is the identity function then the success probability of $C$ in the random experiment $CanonicalSuccess$ is as least $\delta + \epsilon$,
and $D$ simply uses the circuit $\widetilde{C}$. In case $g$ is a constant function the statement is vacuously true.

In case $Gen$ manages to find an estimate that satisfies $\widetilde{S}_{\pi^*,b} \geq (1-\frac{3}{4k})\epsilon$
we define a monotone function $g'(b_2, \dots, b_k) := g(b, b_2, \dots, b_k)$,
and a circuit $\widetilde{C}' = (C_1', C_2')$, where
$C_1'$ first internally simulates the interaction between $C_1$ and $P^{(1)}(\pi^*)$, and then interacts with $P^{(g')}$.
The circuit $C_2'$ is defined as $C_2$ with the solution for the first puzzle discarded.
The surplus estimate is greater than $1 - \frac{3}{4k}\epsilon$.
Therefore, the canonical success probability for the $(k-1)$-wise direct product of puzzles is at least $\Pr_{u \leftarrow \mu^{k-1}_{\delta}}[g'(u_1,\dots, u_{k-1} )] + \epsilon$.
Hence, the circuit $C'$ satisfies the conditions of Lemma \ref{lemma:sec_amp_for_p_hash} for $k-1$ puzzles and we recurse using $g'$ and $C'$.

If all estimates are less than $(1-\frac{3}{4k})\epsilon$, then intuitively $C$
does not perform much better on the remaining $k-1$ puzzles than an algorithm that solves each puzzle independently with probability $\delta$.
However, from the assumption we know that on all $k$ puzzles $\widetilde{C}$ has higher success probability.
Therefore, it is likely that the first puzzle is correctly solved with the probability higher than $\delta$.
%
We now show that this intuition is indeed correct. For a fixed $\pi^*$ using (\ref{eq:s_pi_b}), we get
\begin{align}
\label{eq:diff_s01}
  &\underset{u \leftarrow \mu_{\delta}^k}{\Pr}[g(1, u_2, \dots, u_k)=1] - \underset{u \leftarrow \mu_{\delta}^k}{\Pr}[g(0, u_2, \dots, u_k)=1] = \notag\\
&\IndI  \underset{\pi^{(k)}, \rho}{\Pr}[g(1, c_2, \dots, c_k) =1 \mid \pi_1 = \pi^*] - \underset{\pi^{(k)}, \rho}{\Pr}[g(0, c_2, \dots, c_k) = 1 \mid \pi_1 = \pi^*] - (S_{\pi^*,1} - S_{\pi^*,0}).
\end{align}
Let $\cG_{b} := \{ b_1, b_2, \dots, b_k : g(b, b_2, \dots, b_k) = 1 \}$.
From the monotonicity of $g$ we know that $\cG_0 \subseteq \cG_1$.
Using $\cG_0 \subseteq \cG_1$ and (\ref{eq:diff_s01}) we get:
\begin{align}
  \label{eq:diff_s01_next}
  \underset{u \leftarrow \mu_{\delta}^k}{\Pr}[u \in \cG_1 \setminus \cG_0] = \underset{\pi^{(k)}, \rho}{\Pr}[c \in \cG_1 \setminus \cG_0 \mid \pi_1 = \pi^*] - (S_{\pi^*,1} - S_{\pi^*,0}).
\end{align}
From (\ref{eq:diff_s01_next}) fixing $\pi_1 = \pi^*$ we obtain
\begin{align}
\label{eq:pr_d_succ_0}
&\underset{\rho}{\Pr}[CanonicalSuccess^{P^{(1)}, D, hash}(\pi^*, \rho) = 1] = \notag\\
&\IndII\frac{\underset{\rho}{\Pr}[CanonicalSuccess^{P^{(1)}, D, hash}(\pi^*, \rho) = 1] \underset{\pi^{(k)}, \rho}{\Pr}[c \in \cG_1 \setminus \cG_2]}
{\underset{u \leftarrow \mu_{\delta}^k}{\Pr}[u \in \cG_1 \setminus \cG_0]} \notag\\
&\IndIII-\frac{\underset{\rho}{\Pr}[CanonicalSuccess^{P^{(1)}, D, hash}(\pi^*, \rho) = 1] (S_{\pi^*,1} - S_{\pi^*,0})}
{\underset{u \leftarrow \mu_{\delta}^k}{\Pr}[u \in \cG_1 \setminus \cG_0]}
\end{align}
We make use of the fact that the event $c \in \cG_1 \setminus \cG_0$ implies $D(x^*, r) \neq \bot$, and write the first summand of (\ref{eq:pr_d_succ_0}) as
\begin{align}
  &\underset{\rho}{\Pr}[CanonicalSuccess^{P^(1), D, hash}(\pi^*,\rho) = 1] \underset{\pi^{(k)}, \rho}{\Pr}[c \in \cG_1 \setminus \cG_0 \mid \pi_1 = \pi^*] = \notag\\
  &\IndI \underset{\substack{\rho \\ x^* = \langle P^{(1)}(\pi^*), D_1(\rho) \rangle_{\text{trans}}}}{\Pr}[D_2(x^*,\rho) \neq \bot]
  \underset{\pi^{(k)}, \rho}{\Pr}[c_1 = 1 \mid c \in \cG_1 \setminus \cG_0, \pi_1 = \pi^*]
  \underset{\pi^{(k)}, \rho}{\Pr}[c \in \cG_1 \setminus \cG_0 \mid \pi_1 = \pi^*]
\end{align}
Now we consider two cases:
if $\underset{\pi^{(k)}, \rho}{\Pr}[ c \in \cG_1 \setminus \cG_0 \mid \pi_1 = \pi^*] \leq \frac{\epsilon}{6k}$ then
\begin{align}
  \underset{\pi^{(k)}}{\Pr}[c_1 = 1 \mid c \in \cG_1 \setminus \cG_0, \pi_1 = \pi^*] \underset{\pi^{(k)}}{\Pr}[c \in \cG_1 \setminus \cG_0 \mid \pi_1 = \pi^*] \leq \frac{\epsilon}{6k},
\end{align}
for $\underset{\pi^{(k)}, \rho}{\Pr}[c \in \cG_1 \setminus \cG_0] > \frac{\epsilon}{6k}$ the circuit $D_2$ outputs $\bot$
if and only if it fails in all $\frac{6k}{\epsilon} \log(\frac{6k}{\epsilon})$ iterations to find $\pi^{(k)}$ such that $g(1, c_2, \dots, c_k) = 1 \land g(0, c_2, \dots, c_k) = 0$
(i.e. in none of the iterations $c \in \cG_1 \setminus \cG_0$) which happens with probability
\begin{align}
\underset{\substack{\rho \\ x^* := \langle P^{(1)}(\pi^*), D_1(\rho) \rangle_{\text{trans}}}}{\Pr}[D_2(x^*,\rho) = \bot]
\leq (1 - \frac{\epsilon}{6k})^{\frac{6k}{\epsilon}\log(\frac{\epsilon}{6k})} \leq \frac{\epsilon}{6k}.
\end{align}
We conclude that in both cases:
\begin{align}
  &\underset{\substack{\rho \\ x^* := \langle P^{(1)}(\pi^*), D_1(\rho) \rangle_{\text{trans}}}}{\Pr}[D_2(x^*,\rho) \neq \bot]
  \underset{\pi^{(k)}, \rho}{\Pr}[c_1 = 1 \mid c \in \cG_1 \setminus \cG_0, \pi_1 = \pi^*]
  \underset{\pi^{(k)}, \rho}{\Pr}[c \in \cG_1 \setminus \cG_0 \mid \pi_1 = \pi^*] \notag\\
  &\IndIII \geq \underset{\pi^{(k)}, \rho}{\Pr}[c_1 = 1 \mid c \in \cG_1 \setminus \cG_0, \pi_1 = \pi^*]\underset{\pi^{(k)}, \rho}{\Pr}[c \in \cG_1 \setminus \cG_0 \mid \pi_1 = \pi^*] - \frac{\epsilon}{6k}.
\end{align}
Therefore, we have
\begin{align*}
  &\underset{\substack{\rho \\ x^* := \langle P^{(1)}(\pi^*), D_1(\rho) \rangle_{\text{trans}}}}{\Pr}[D_2(x^*,\rho) \neq \bot]
  \underset{\pi^{(k)}, \rho}{\Pr}[c_1 = 1 \mid c \in \cG_1 \setminus \cG_0, \pi_1 = \pi^*]
  \underset{\pi^{(k)}, \rho}{\Pr}[c \in \cG_1 \setminus \cG_0 \mid \pi_1 = \pi^*] \notag\\
  &\IndII \geq \underset{\pi^{(k)}, \rho}{\Pr}[c_1 = 1 \land c \in \cG_0 \setminus \cG_1 \mid \pi_1 = \pi^*] - \frac{\epsilon}{6k} \\
  &\IndII = \underset{\pi^{(k)}, \rho}{\Pr}[g(c_1, c_2,\dots, c_k) = 1 \land g(0, c_2, \dots, c_k) = 0 \mid \pi_1 = \pi^*] - \frac{\epsilon}{6k} \\
  &\IndII = \underset{\pi^{(k)}, \rho}{\Pr}[g(c_1, c_2,\dots, c_k) = 1 \mid \pi_1 = \pi^*] -  \underset{\pi^{(k)}, \rho}{\Pr}[c \in \cG_0 \mid \pi_1 = \pi^*] - \frac{\epsilon}{6k},
\end{align*}
and finally by (\ref{eq:s_pi_b})
\begin{align}
  &\underset{\substack{\rho \\ x^* := \langle P^{(1)}(\pi^*), D_1(\rho) \rangle_{\text{trans}}}}{\Pr}[D_2(x^*,\rho) \neq \bot]
  \underset{\pi^{(k)}, \rho}{\Pr}[c_1 = 1 \mid c \in \cG_1 \setminus \cG_0, \pi_1 = \pi^*]
  \underset{\pi^{(k)}, \rho}{\Pr}[c \in \cG_1 \setminus \cG_0 \mid \pi_1 = \pi^*] \notag\\
  &\IndIII = \underset{\pi^{(k)}, \rho}{\Pr}[g(c_1, c_2,\dots, c_k) = 1 \mid \pi_1 = \pi^*] -  \underset{u \leftarrow \mu_{\delta}^{(k)}}{\Pr}[u \in \cG_0]  - S_{\pi^*,0} - \frac{\epsilon}{6k}.
\end{align}
Inserting this result into the equation (\ref{eq:pr_d_succ_0}) yields
\begin{align}
\label{eq:pr_d_succ_1}
  &\underset{\pi, \rho}{\Pr}[CanonicalSuccess^{P^{(1)}, D, hash}] = \notag\\
&\IndI = \mathbb{E}_{\pi^*}\left[\frac{{\Pr}_{\pi^{(k)}}[g(c) = 1 \mid \pi_1 = \pi^*]
- {\Pr}_{u \leftarrow \mu_{\delta}^{k}}[u \in \cG_0] - \frac{\epsilon}{6k}}
{\Pr_{u \leftarrow \mu_{\delta}^{k}}[u \in \cG_1 \setminus \cG_0]}\right] \notag\\
&\IndII - \mathbb{E}_{\pi^*}\left[\frac{S_{\pi^*,0} + \Pr_{\rho} [CanonicalSuccess^{P^{(1)}, D, hash}(\pi^*, \rho) = 1](S_{\pi^*,1} - S_{\pi^*,0})}
{{\Pr}_{u \leftarrow \mu_{\delta}^{k}}[u \in \cG_1 \setminus \cG_0]}\right]
\end{align}
For the second summand we show that if we do not recurse, then almost surely majority of estimates is low.
Let assume
\begin{align}
\underset{\pi, \rho}{\Pr}\left[\left(S_{\pi,0} \leq (1 - \frac{1}{2k})\epsilon\right) \land \left( S_{\pi,1} \leq (1-\frac{1}{2k})\epsilon\right)\right] < 1 - \frac{\epsilon}{6k},
\end{align}
then the algorithm recurses almost surely.
Therefore, under the assumption that $Gen$ does not recurse, we have almost surely
\begin{align}
\underset{\pi, \rho}{\Pr}\left[\left(S_{\pi,0} \leq (1 - \frac{1}{2k})\epsilon\right) \land \left( S_{\pi,1} \leq (1-\frac{1}{2k})\epsilon\right)\right] \geq 1 - \frac{\epsilon}{6k}.
\end{align}
Let us define a set
\begin{align}
  \cW = \left\{ \pi :  \left(S_{\pi,0} \leq (1 - \frac{1}{2k})\epsilon\right) \land \left( S_{\pi,1} \leq (1-\frac{1}{2k})\epsilon \right) \right\}
\end{align}
and use $\cW^c$ to denote the complement of $\cW$.
We bound the second summand in (\ref{eq:pr_d_succ_1})
\begin{align}
&\mathbb{E}_{\pi^*}\left[ S_{\pi^*,0} + \Pr_{\rho} [CanonicalSuccess^{P^{(1)}, D, hash} (\pi^*, \rho) = 1](S_{\pi^*,1} - S_{\pi^*,0}) \right] \notag\\
&\IndII = \mathbb{E}_{\pi^* \in \cW^c}\left[ S_{\pi^*,0} + \Pr_{\rho} [CanonicalSuccess^{P^{(1)}, D, hash}(\pi^*, \rho) = 1](S_{\pi^*,1} - S_{\pi^*,0}) \right] \notag\\
&\IndIII +  \mathbb{E}_{\pi^* \in \cW}\left[ S_{\pi^*,0} + \Pr_{\rho} [CanonicalSuccess^{P^{(1)}, D, hash}(\pi^*, \rho) = 1](S_{\pi^*,1} - S_{\pi^*,0}) \right] \notag\\
&\IndII \leq \frac{\epsilon}{6k} + \mathbb{E}_{\pi^* \in \cW^c}\left[ S_{\pi^*,0} + \Pr_{\rho} [CanonicalSuccess^{P^{(1)}, D, hash}(\pi^*, \rho) = 1]((1 - \frac{1}{2k})\epsilon - S_{\pi^*,0}) \right] \notag\\
&\IndII \leq \frac{\epsilon}{6k} + 1 - \frac{\epsilon}{2k} = 1 - \frac{\epsilon}{3k}
\end{align}
Finally, we insert this result into equation (\ref{eq:pr_d_succ_1}) and make use of the fact
\begin{align*}
\underset{u \leftarrow \mu_{\delta}^k}{\Pr}[g(u) = 1]
&= \Pr[u \in \cG_0 \lor ( u \in \cG_1 \setminus \cG_0 \land u_1 = 1)] \notag\\
&= \Pr[u \in \cG_0] + \Pr[u \in \cG_1 \setminus \cG_0] \Pr[u_1 = 1]
\end{align*}
which yields
\begin{align*}
  \underset{\pi, \rho}{\Pr}[CanonicalSuccess^{P^{(1)}, D, hash}]
&\geq \mathbb{E}_{\pi^*}\left[\frac{{\Pr}_{\pi^{(k)}}[g(c) = 1 \mid \pi_1 = \pi^*] -
{\Pr}_{u \leftarrow \mu_{\delta}^{k}}[u \in G_0] - (1 - \frac{1}{6k})\epsilon} {\Pr_{u \leftarrow \mu_{\delta}^{k}}[u \in \cG_1 \setminus \cG_0]}\right] \notag
 \end{align*}
 Using the assumptions of Lemma \ref{lemma:sec_amp_for_p_hash}, we get
 \begin{align}
   \label{eq:proof_final}
   \underset{\pi, \rho}{\Pr}[CanonicalSuccess^{P^{(1)}, D, hash} = 1]
 &\geq \frac{ {\Pr}_{u \leftarrow \mu_{\delta}^{k}}[g(u) = 1] + \epsilon +
 \Pr_{u \leftarrow \mu_{\delta}^{k}}[u \in \cG_0] - (1 - \frac{1}{6k})\epsilon} {\Pr_{u \leftarrow \mu_{\delta}^{k}}[u \in \cG_1 \setminus \cG_0]} \notag\\
 &\geq \frac{\epsilon + \delta\Pr_{u \leftarrow \mu_{\delta}^{k}}[u \in \cG_1 \setminus \cG_0] - (1 - \frac{1}{6k})\epsilon}
{\Pr_{u \leftarrow \mu_{\delta}^{k}}[u \in \cG_1 \setminus \cG_0]} \geq \delta + \frac{\epsilon}{6k}
\end{align}
\end{proof}
\begin{lemma}
  \label{lemma:ctilda_c}
  For fixed $P, C$ and $hash$ the following statement is true
  \begin{align*}
    \underset{\pi, \rho}{\Pr}[CanonicalSuccess^{P, C, hash}(\pi, \rho) = 1]
    \leq \underset{\pi, \rho}{\Pr}[CanonicalSuccess^{P, \widetilde{C}, hash}(\pi, \rho) = 1]
  \end{align*}
\end{lemma}
%
\begin{proof}
For some $\pi, \rho$ if $C$ succeeds canonically then also $\widetilde{C}$ succeeds canonically.
Using this observation, we conclude that
\begin{align*}
  &\underset{\pi, \rho}{\Pr}\left[CanonicalSuccess^{P, C, hash}(\pi, \rho) = 1\right] \\
  &\IndII = \underset{\pi, \rho}{\mathbb{E}}\left[CanonicalSuccess^{P, C, hash}(\pi, \rho) = 1 \right] \\
  &\IndII \leq \underset{\pi, \rho}{\Pr}\left[CanonicalSuccess^{P, \widetilde{C}, hash}(\pi, \rho) = 1\right]
\end{align*}
% hack to have a square at the end of the proof
\\\text{  }
\end{proof}
%
\begin{proof}[Theorem \ref{th:sec_amp_for_dwvp}]
We show that Theorem \ref{th:sec_amp_for_dwvp} follows by Lemmas: \ref{lemma:sec_amp_for_p_hash}, \ref{lemma:hash_function_probability}, \ref{lemma:ctilda_c}.
First given a solver circuit $C$ such that
\begin{align*}
    \underset{\pi^{(k)}, \rho}{\Pr}\left[Success^{P^{(g)}, C}(\pi^{(k)}, \rho) = 1\right] \geq 8(h+v)\left(\underset{u \leftarrow \mu_\delta^k}{\Pr}\left[g(u) = 1\right] + \varepsilon\right)
\end{align*}
by Lemma \ref{lemma:hash_function_probability} we can find a function $hash$ such that
\begin{align*}
    \underset{\pi^{(k)}, \rho}{\Pr}\left[CanonicalSuccess^{P^{(g)}, C, hash}(\pi^{(k)}, \rho) = 1\right] \geq \underset{u \leftarrow \mu_\delta^k}{\Pr}\left[g(u) = 1\right] + \varepsilon.
\end{align*}
By Lemma \ref{lemma:ctilda_c} we know that we can find a circuit $\widetilde{C}$ such that
\begin{align*}
    \underset{\pi^{(k)}, \rho}{\Pr}\left[CanonicalSuccess^{P^{(g)}, \widetilde{C}, hash}(\pi^{(k)}, \rho) = 1\right] \geq \underset{u \leftarrow \mu_\delta^k}{\Pr}\left[g(u) = 1\right] + \varepsilon.
\end{align*}
Finally, we apply Lemma \ref{lemma:sec_amp_for_p_hash} with the function $hash$ and the circuit $\widetilde{C}$ to obtain a circuit $D$ such that
\begin{align*}
    \underset{\pi, \rho}{\Pr}\left[CanonicalSuccess^{P^{(1)}, D, hash}(\pi, \rho) = 1\right] \geq \delta + \frac{\epsilon}{6k}.
\end{align*}
If $D$ succeeds in the experiment $CanonicalSuccess$ then it also succeeds in the experiment $Succees$.
\end{proof}
%%% Local Variables:
%%% mode: latex
%%% TeX-master: "../master"
%%% End:
