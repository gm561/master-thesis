%
\begin{codeblock}
  \textbf{Circuit $\widetilde{C}_2^{\Gamma_H, C_2, hash} (x, \rho)$}
  \medskip \hrule \medskip
  \textbf{Oracle:} A hint circuit $\Gamma_H$, a circuit $C_2$, \\
  \IndII a function $hash : Q \rightarrow \{0,1,\dots, 2(h+v)-1\}$. \\
  \textbf{Input:} Bitstrings $x \in \{0,1\}^{*}$, $\rho \in \{0,1\}^{*}$. \\
  \textbf{Output:} A tuple $(q, y)$.
  \medskip\hrule\medskip
  \Run $C_2^{(\cdot, \cdot)}(x, \rho)$ \\
  \IndI \If $C_2^{(\cdot, \cdot)}(x, \rho)$ asks a hint query on $q$ \Then\\
  \IndII \If $hash(q) = 0$ \Then\\
  \IndIII \Return $\bot$\\
  \IndII \textbf{else}\\
  \IndIII answer the query of $C_2^{(\cdot, \cdot)}(x, \rho)$ using $\Gamma_H(q)$\\
  \\
  \IndI \If $C_2^{(\cdot, \cdot)}(x, \rho)$ asks a verification query $(q, y)$ \Then \\
  \IndII \If $hash(q) = 0 $ \textbf{then} \\
  \IndIII \Return $(q, y)$ \\
  \IndII \textbf{else} \\
  \IndIII answer the verification query of $C_2^{(\cdot, \cdot)}(x, \rho)$ with 0 \\
  \Return $\bot$
\end{codeblock}
%
Given $C = (C_1, C_2)$ we define a circuit $\widetilde{C} = (C_1, \widetilde{C}_2)$.
Every hint query $q$ asked by $\widetilde{C}$ is such that $hash(q) \neq 0$.
Furthermore, $\widetilde{C}$ asks no verification queries, instead it returns $\bot$ or $(q,y)$ such that
$hash(q) = 0$.

We say that for a fixed $\pi$, $\rho$, $hash$ the circuit $\widetilde{C}$ \textit{succeeds} if
for $x := \langle P(\pi), C_1(\rho) \rangle_{\text{trans}}$,
$(\Gamma_V, \Gamma_H) := \langle P(\pi), C_1(\rho) \rangle_{P}$, we have
\begin{align*}
\Gamma_V(\widetilde{C}_2^{\Gamma_H, C_2, hash}(x, \rho)) = 1.
\end{align*}
%
\begin{lemma}
  \label{lemma:ctilda_c}
  For fixed $P, C$ and $hash$ the following statement is true
  \begin{align*}
    \underset{\pi, \rho}{\Pr}[CanonicalSuccess^{P, C, hash}(\pi, \rho) = 1]
    \leq
    \mkern13mu
    \underset{
      \mathclap{
      \substack{
        \pi, \rho \\
        x := \langle P(\pi), C_1(\rho) \rangle_{\text{trans}} \\
        (\Gamma_V, \Gamma_H) := \langle P(\pi), C_1(\rho) \rangle_{P}
      }}} {\Pr}\mkern13mu[\Gamma_V(\widetilde{C}_2^{\Gamma_H, C_2, hash}(x, \rho)) = 1]
  \end{align*}
\end{lemma}
%
\begin{proof}
If for some fixed $\pi$, $\rho$ and $hash$ the circuit $C$ succeeds canonically, then for the same $\pi$, $\rho$ and $hash$ also $\widetilde{C}$ succeeds.
Using this observation, we conclude that
\begin{align*}
  &\underset{\pi, \rho}{\Pr}\left[CanonicalSuccess^{P, C, hash}(\pi, \rho) = 1\right] \\
  &\IndII = \underset{\pi, \rho}{\mathbb{E}}\left[CanonicalSuccess^{P, C, hash}(\pi, \rho) = 1 \right] \\
  &\IndII \leq
  \mkern33mu
    \underset{
      \mathclap{
        \substack{\pi, \rho \\
        x := \langle P(\pi), C_1(\rho) \rangle_{\text{trans}} \\
        (\Gamma_V, \Gamma_H) := \langle P(\pi), C_1(\rho) \rangle_{P}
      }}}
    {\mathbb{E}}\mkern13mu[\Gamma_V(\widetilde{C}_2^{\Gamma_H, C_2, hash}(x, \rho)) = 1] \\
  &\IndII =
  \mkern33mu
    \underset{
      \mathclap{
        \substack{\pi, \rho \\
        x := \langle P(\pi), C_1(\rho) \rangle_{\text{trans}} \\
        (\Gamma_V, \Gamma_H) := \langle P(\pi), C_1(\rho) \rangle_{P}
      }}}
    {\Pr}\mkern13mu[\Gamma_V(\widetilde{C}_2^{\Gamma_H, C_2, hash}(x, \rho)) = 1]
\end{align*}
% hack to have a square at the end of the proof
\\\text{  }
\end{proof}
%
%
\begin{lemma}\textbf{(Security amplification of a dynamic weakly verifiable puzzle with respect to \boldmath{$hash$}.)}
  \label{lemma:sec_amp_for_p_hash}
  For fixed $P^{(1)}$ there exists an algorithm $Gen$ that takes as input parameters $\varepsilon, \delta, n, k$
  has oracle access to $P^{(1)}$, $P^{(g)}$, $\widetilde{C}$, functions $hash : Q \rightarrow \{0,1,\dots, 2(h+v-1)\}$, $g: \{0,1\}^{k} \rightarrow \{0,1\}$,
  and outputs a circuit $D := (D_1, D_2)$ such that the following holds: \\
  If $\widetilde{C} := (C_1, \widetilde{C}_2)$ has oracle access to $hash$ and a solver circuit $C := (C_1, C_2)$ for $P^{(g)}$,
  which asks at most $h$ hint queries and $v$ verification queries, is such that
  \begin{align*}
    \underset{\mathclap{\substack{
          \pi^{(k)} \in \{0,1\}^{kn}, \rho \in \{0,1\}^{*} \\
          x:= \langle P^{(g)}(\pi^{(k)}), {C}_1(\rho) \rangle_{\text{trans}} \\
          (\Gamma_H^{(k)}, \Gamma_V^{(g)}) := \langle P^{(g)}(\pi^{(k)}), C_1(\rho) \rangle_{P^{(g)}}}}}
    {\Pr}\mkern13mu[\Gamma_V^{(g)}(\widetilde{C}_2^{\Gamma_H^{(k)}, C_2, hash}(x,\rho)) = 1]
    \geq \underset{u \leftarrow \mu_\delta^k}{\Pr}[g(u) = 1] + \varepsilon,
  \end{align*}
  then $D$ satisfies almost surely
  \begin{align*}
    \underset{
      \mathclap{
      \substack{
        \pi \in \{0,1\}^{n}, \rho \in \{0,1\}^{*} \\
        x := \langle P^{(1)}(\pi), D_1^{\widetilde{C}}(\rho) \rangle_{\text{trans}} \\
        (\Gamma_H, \Gamma_V) := \langle P^{(1)}(\pi), D_1^{P^{(1)}, \widetilde{C}}(\rho) \rangle_{P^{(1)}}}}}
    {\Pr}\mkern13mu[\Gamma_V(D_2^{P^{(1)}, P^{(g)}, \widetilde{C}, hash, g, \Gamma_H}(x, \rho)) = 1] \geq (\delta + \frac{\varepsilon}{6k}).
  \end{align*}
  Furthermore, $D$ has oracle access to a hint circuit, $P^{(1)}$, $P^{(g)}$,  $\widetilde{C}$, $hash$, $g$, and
  asks at most $\frac{6k}{\epsilon}\log\left(\frac{6k}{\epsilon}\right) h$ hint queries and no verification queries.
  Finally, $Size(D) \leq Size(C)\frac{6k}{\varepsilon}$ and $Time(Gen) = poly(k, \frac{1}{\varepsilon}, n, v, h)$.
\end{lemma}
%
%
Before proving Lemma \ref{lemma:sec_amp_for_p_hash} we define additional algorithms that are later used by $Gen$.
First, we are interested in the probability that for $u \leftarrow \mu_{\delta}^k$ and a bit $b \in \{0,1\}$ the function $g$
with the first input bit set to $b$ takes value $1$. The estimate of this probability is calculated by the algorithm
$EstimateFunctionProbability$.
%
\begin{codeblock}
  $\textbf{Algorithm: EstimateFunctionProbability}^{g}(b, k, \epsilon, \delta, n)$
  \medskip
  \hrule
  \medskip
  \textbf{Oracle:} A function $g : \{0,1\}^{k} \rightarrow \{0,1\}$.\\
  \textbf{Input:} A bit $b \in \{0,1\}$, parameters $k$, $\epsilon$, $\delta$, $n$. \\
  \textbf{Output:} An estimate $\widetilde{g}$ of $\Pr_{u \leftarrow \mu_{\delta}^{k}}[g(b,u_2, \dots, u_k) = 1]$.
  \medskip\hrule\medskip
  \For $i:=1$ \To $\frac{64^2}{\epsilon^2} n$ \Do \\
  \IndI $u \leftarrow \mu_{\delta}^{k}$ \\
  \IndI $g_i := g(b, u_2, \dots, u_k)$ \\
  \textbf{return} $\frac{\epsilon^2}{64^2n} \sum_{i=1}^{\frac{64^2}{\epsilon^2} n} g_i$
\end{codeblock}
%
\begin{lemma}
  \label{lemma:estimate_of_g}
  The algorithm $\textbf{EstimateFunctionProbability}^{g}(b, k, \epsilon, \delta)$ outputs an estimate $\widetilde{g}$
  of $\Pr_{u \leftarrow \mu_{\delta}^k}[g(b, u_2, \dots, u_k) = 1]$ where $b \in \{0,1\}$
  such that $| \widetilde{g} - \Pr_{u \leftarrow \mu_{\delta}^{k}}\left[g(b,u_2, \dots, u_k) = 1\right] | \leq \frac{\epsilon}{8k}$ almost surely.
\end{lemma}
%
\begin{proof}
We define independent, identically distributed binary random variables $K_1, K_2, \dots, K_{\frac{64k^2}{\epsilon^2} n}$
such that for each $1 \leq i \leq \frac{64k^2}{\epsilon^2} n$
the random variable $K_i$ takes value $g_i$. We use the Chernoff bound to obtain
\footnote{For independent Bernoulli distributed random variables $X_1, \dots, X_n$ with $X := \sum_{i=1}^n X_i$ and $0 \leq \delta \leq 1$
  we use the Chernoff bound in the form $\Pr[|X - \mathbb{E}[X]| \geq \delta \mathbb{E}[X]] \leq 2 e^{- \mathbb{E}[X] \delta^2 / 3}$.}
\begin{align*}
  \Pr \left[\left|\left(\frac{\epsilon^2}{64k^2n} \sum_{i=1}^{\frac{64k^2}{\epsilon^2}n } K_i \right) - \mathbb{E}[K_i]\right|
    \geq \frac{\epsilon}{8k} \right] \leq 2 \cdot e^{-n/3}.
\end{align*}
\end{proof}
%
%
The next algorithm $\textbf{EvalutePuzzles}^{P^{(1)}, P^{(g)}, \widetilde{C}, hash}(\pi^{(k)}, \rho, n, k)$
evaluates which of $k$ puzzles of the $k$-wise direct product defined by $P^{(g)}$ are solved successfully by $\widetilde{C}(\rho) := (C_1,\widetilde{C}_2)(\rho)$.
To decide whether the $i$-th puzzle of the $k$-wise direct product is solved successfully we need to gain access to the verification oracle
for the puzzle generated in the $i$-th round of the interaction between $P^{(g)}$ and $\widetilde{C}$.
Therefore, in the algorithm \textbf{EvalutePuzzles}, we use $P^{(1)}$, and invoke it $k$ times to simulate the interaction with $C_1(\rho)$.
Let us introduce some additional notation. We denote by $\langle P^{(1)}(\pi_i), C_1(\rho)\rangle^i$ the execution of
the $i$-th round of the simulation, and by $\langle P^{(1)}(\pi_i), C_1(\rho)\rangle^i_{P^{(1)}}$ the output of $P^{(1)}(\pi_i)$ in the $i$-th round.
Furthermore, we write $\langle P^{(1)}(\pi_i), C_1(\rho)\rangle^i_{trans}$ to denote a transcript of communication in the $i$-th round.
%

To make the notation easier in the code excerpts of circuits $C_2$, $D_2$ and \textit{EvalutePuzzles} we
omit some oracle signatures writing for example $\widetilde{C}_2^{\Gamma_H^{(k)}, hash}$ instead of $\widetilde{C}_2^{\Gamma_H^{(k)}, C, hash}$ where
the access to oracle circuit $C$ is omitted. We make sure that it is clear from a context which circuit $C$ is used by $\widetilde{C}_2$.

\begin{codeblock}
  $\textbf{Algorithm: EvaluatePuzzles}^{P^{(1)}, P^{(g)}, \widetilde{C}, hash}(\pi^{(k)}, \rho, n, k)$
  \medskip \hrule \medskip
  \textbf{Oracle:}  Problem posers $P^{(1)}$, $P^{(g)}$, a circuit $\widetilde{C} = (C_1, \widetilde{C}_2)$,\\
  \IndII a function $hash : Q \rightarrow \{0,1,\dots, 2(h+v)-1\}$.\\
  \textbf{Input:} Bitstrings $\pi^{(k)} \in \{0,1\}^{kn}$, $\rho \in \{0,1\}^{*}$, parameters $n$, $k$.\\
  \textbf{Output}: A tuple $(c_1, \dots, c_k) \in \{0,1\}^{k}$.
  \medskip\hrule\medskip
  %
  \For $i:=1$ \To $k$ \Do \IndII //simulate $k$ rounds of interaction \\
  \IndI $(\Gamma_V^{i}, \Gamma_H^{i}) := \langle P^{(1)}(\pi_i), C_1(\rho) \rangle_{P^{(1)}}^i$\\
  \IndI $x_i := \langle P^{(1)}(\pi_i), C_1(\rho) \rangle^i_{\text{trans}}$ \\
  $x := (x_1, \dots, x_k)$ \\
  $\Gamma_H^{(k)} := (\Gamma_H^1, \dots, \Gamma_H^k)$ \\
  $(q, y_1, \dots, y_k) := \widetilde{C}_2^{\Gamma_H^{(k)}, hash} (x, \rho)$ \\
  $(c_1, \dots, c_k) := (\Gamma_V^{1}(q, y_1), \dots, \Gamma_V^{k}(q, y_k))$\\
  \Return $(c_1, \dots, c_k)$
\end{codeblock}
%
All puzzles used by the algorithm \textbf{EvalutePuzzles} are generated internally.
Thus, the algorithm can answer itself all queries of $\widetilde{C}_2$ to the hint oracle.

% For fixed $\pi^{(k)}, \rho$ let $(\Gamma_V^{(g)}, \Gamma_H^{(k)}) := \langle P^{(g)}(\pi^{(k)}), C_1(\rho) \rangle_{P^{(g)}}$
% and $x := \langle P^{(g)}(\pi^{(k)}), C_1(\rho) \rangle_{\text{trans}}$.
% Additionally, we denote by $(\Gamma_V^{i}, \Gamma_H^{i})$ the verification and hint circuits generated by $P^{(1)}(\pi_i)$ in the $i$-th
% round of the simulated interaction with $C_1(\rho)$.
% For $(q,y_1, \dots, y_k) := \widetilde{C}_2(x, \rho)$ we denote the output of $\Gamma_V^i(q,y_i)$ by $c_i$.
We are interested in the success probability of $\widetilde{C}$ with the bitstring $\pi_1$ fixed to $\pi^*$ when
the fact whether $\widetilde{C}$ succeeds in solving the first puzzle defined by $P^{(1)}(\pi_1)$ is neglected, and
instead a bit $b \in \{0,1\}$ is used. More formally, we define the surplus $S_{\pi^*, b}$ as
\begin{align}
  \label{eq:s_pi_b}
S_{\pi^*, b} = \underset{\pi^{(k)}, \rho}{\Pr}\left[g(b, c_2, \dots, c_k) = 1 \mid \pi_1 = \pi^*\right] - \underset{u \leftarrow \mu^{k}_{\delta}}{\Pr}\left[g(b, u_2, \dots, u_k) = 1\right].
\end{align}
%
The algorithm \textbf{EstimateSurplus} returns an estimate $\widetilde{S}_{\pi^*, b}$ for $S_{\pi^*, b}$.
%
\begin{codeblock}
  $\textbf{Algorithm: EstimateSurplus}^{P^{(1)}, P^{(g)}, \widetilde{C}, g, hash}(\pi^*, b, k, \epsilon, \delta, n)$
  \medskip
  \hrule
  \medskip
  \textbf{Oracle:} Problem posers $P^{(1)}$, $P^{(g)}$, a circuit $\widetilde{C}$, a function $g: \{0,1\}^{k} \rightarrow \{0,1\}$ \\
  \IndII a function $hash : Q \rightarrow \{0,1,\dots, 2(h+v)-1\}$.\\
  \textbf{Input:} A bistring $\pi^* \in \{0,1\}^{n}$, a bit $b \in \{0,1\}$, parameters $k$, $\epsilon$, $\delta$, $n$.\\
  \textbf{Output:} An estimate $\widetilde{S}_{\pi^*, b}$ for $S_{\pi^*, b}$.
  \medskip\hrule\medskip
  \For $i:=1$ \To $\frac{64k^2}{\epsilon^2}n$ \Do \\
  \IndI $(\pi_{2}, \dots, \pi_k) \xleftarrow{\$} \{0,1\}^{(k-1)n}$\\
  \IndI $\rho \xleftarrow{\$} \{0,1\}^{*}$\\
  \IndI $(c_1, \dots, c_k) := \textbf{EvalutePuzzles}^{P^{(1)}, P^{(g)}, \widetilde{C}, hash}((\pi^*, \pi_2, \dots, \pi_k), \rho)$\\
  \IndI $\widetilde{s}_{\pi^*,b}^i := g(b, c_{2}, \dots, c_k)$\\
  $\widetilde{g}_b := \textbf{EstimateFunctionProbability}^{g}(b, k, \epsilon, \delta, n)$ \\
  \textbf{return} $\left(\frac{\epsilon^2}{64k^2n} \sum_{i=1}^{\frac{64k^2}{\epsilon^2} n} \widetilde{s}_{\pi^*,b}^i \right) - \widetilde{g}_b$\\
\end{codeblock}
%
\begin{lemma}
  \label{lemma:surplus_estimate}
The estimate $\widetilde{S}_{\pi^*,b}$ returned by \textbf{EstimateSurplus} differs from $S_{\pi^*, b}$ by at most $\frac{\epsilon}{4k}$ almost surely.
\end{lemma}

\begin{proof}
We use the union bound and similar argument as in Lemma \ref{lemma:estimate_of_g}
which yields that \\$\frac{\epsilon^2}{64k^2n} \sum_{i=1}^{\frac{64k^2}{\epsilon^2}n} \widetilde{s}_{\pi^*,b}^i$ differs from
$\mathbb{E}[g(b, c_2, \dots, c_k)]$ by at most $\frac{\epsilon}{8k}$ almost surely. Together, with Lemma $\ref{lemma:estimate_of_g}$ we conclude that the surplus estimate
returned by \textbf{EstimateSurplus} differs from $S_{\pi^*,b}$ by at most $\frac{\epsilon}{4k}$ almost surely.
\end{proof}
%
%
\begin{codeblock}
  \textbf{Circuit $C_1'^{\widetilde{C}, P^{(1)}}(\rho)$}
  \medskip \hrule \medskip
  \textbf{Oracle:} A circuit $\widetilde{C} = (C_1, \widetilde{C}_2)$, a poser $P^{(1)}$. \\
  \textbf{Input:}  A bitstring $\rho \in \{0,1\}^{*}$ \\
  \textbf{Hard-coded:} A bitstring $\pi^* \in \{0,1\}^{n}$
  \medskip\hrule\medskip
  Simulate $\langle P^{(1)}(\pi^*), C_1(\rho)\rangle^1$ \\
  Use $C_1(\rho)$ for the remaining $k-1$ rounds of interaction.
\end{codeblock}
%
%
\begin{codeblock}
  \textbf{Circuit $\widetilde{C}_2'^{\Gamma_H^{(k-1)}, \widetilde{C}, hash}(x^{(k-1)}, \rho)$}
  \medskip \hrule \medskip
  \textbf{Oracle:} A hint oracle $\Gamma_H^{(k-1)} := (\Gamma_H^{2}, \dots, \Gamma_H^{k})$, a circuit $\widetilde{C} = (C_1, \widetilde{C}_2)$, \\
  \IndII a function $hash: Q \rightarrow \{0,1,\dots, 2(h+v)-1\}$ \\
  \textbf{Input:}  A tuple $x^{(k-1)} := (x_2, \dots, x_{k}) \in \{0,1\}^{*}$, a bitstring $\rho \in \{0,1\}^{*}$\\
  \textbf{Hard-coded:} A bitstring $\pi^* \in \{0,1\}^{n}$
  \medskip\hrule\medskip
  Simulate $\langle P^{(1)}(\pi^*), C_1(\rho) \rangle^{1}$ \\
  \IndI $(\Gamma_H^*, \Gamma_V^*) := \langle P^{(1)}(\pi^*), C_1(\rho) \rangle^{1}_{P^{(1)}}$ \\
  \IndI $x^* := \langle P^{(1)}(\pi^*), C_1(\rho) \rangle^{1}_{\text{trans}}$ \\
  $\Gamma_H^{(k)} := (\Gamma_H^*, \Gamma_H^{2}, \dots, \Gamma_H^{k})$ \\
  $x^{(k)} := (x^*, x_2, \dots, x_{k})$ \\
  $(q, y_1, \dots, y_k) := \widetilde{C}_2^{\Gamma_H^{(k)}, hash}(x^{(k)}, \rho)$ \\
  \Return $(q, y_2, \dots, y_k)$
\end{codeblock}
%
We are ready to define the circuit $D = (D_1, D_2)$ and the algorithm $Gen$.
%
\begin{codeblock}
  \textbf{Circuit $D_1^{\widetilde{C}}(r)$}
  \medskip \hrule \medskip
  \textbf{Oracle:} A circuit $\widetilde{C} = (C_1, \widetilde{C}_2)$.\\
  \textbf{Input:} A tuple $r := (\rho, \sigma)$ where $ \rho \in \{0,1\}^{*}$ and $\sigma \in \{0,1\}^{*}$.
  \medskip\hrule\medskip
  Interact with the problem poser $\langle P^{(1)}, C_1(\rho) \rangle^1$.
\end{codeblock}
%
\begin{codeblock}
  \textbf{Circuit $D_2^{P^{(1)}, P^{(g)}, \widetilde{C}, hash, g,  \Gamma_H^*}(x^*, \rho)$}
  \medskip \hrule \medskip
  \textbf{Oracle:} A poser $P^{(1)}$, a solver circuit $\widetilde{C} = (C_1, \widetilde{C}_2)$, \\
  \IndII functions $hash : Q \rightarrow \{0,1, \dots, 2(h+v)-1\}$, $g:\{0,1\}^k \rightarrow \{0,1\}$, \\
  \IndII a hint circuit $\Gamma_H^*$ for $P^{(1)}$. \\
  %
  \textbf{Input:} Bitstrings $x^* \in \{0,1\}^{*}$, $\rho := (\tau, \sigma)$ such that $\tau \in \{0,1\}^{*}$ and $\sigma \in \{0,1\}^{*}$\\
  \textbf{Output}: A tuple $(q, y^*)$.
  \medskip \hrule \medskip
  \For at most $\frac{6k}{\epsilon} \log(\frac{6k}{\epsilon})$ iterations \Do \\
  \IndI $(\pi_2, \dots, \pi_k) \leftarrow$ read next $(k-1)\cdot n$ bits from $\sigma$ \\
  \IndI \For $i:=2$ \To $k$ \Do \\
  \IndII \Run $\langle P^{(1)}(\pi_i), C_1(\rho)\rangle^i$ \\
  \IndIII $(\Gamma_V^{i}, \Gamma_H^{i}) := \langle P^{(1)}(\pi_i), C_1(\tau) \rangle^i_{P^{(1)}}$ \\
  \IndIII $x_i := \langle P^{(1)}(\pi_i), C_1(\tau) \rangle^i_{\text{trans}}$ \\
  \IndI $\Gamma_H^{(k)}(q) := (\Gamma_H^{*}(q), \Gamma_H^{2}(q), \dots, \Gamma_H^{k}(q))$ \\
  \IndI $(q, y^*, y_2, \dots, y_k) := \widetilde{C}_2^{\Gamma_H^{(k)}, hash}((x^*, x_2, \dots, x_k), \tau)$\\
  \IndI $(c_2, \dots, c_k) := (\Gamma_V^2(q, y_2), \dots, \Gamma_V^{k}(q, y_k))$ \\
  \IndI \If $g(1, c_{2}, \dots, c_k) = 1$ \And $g(0,c_{2}, \dots, c_k) = 0$ \Then \\
  \IndII \Return $(q, y^*)$ \\
  \Return $\bot$
%
\end{codeblock}
%
%
\begin{codeblock}
  \textbf{Algorithm $Gen^{P^{(1)}, P^{(g)}, \widetilde{C}, g, hash}(\epsilon, \delta, n, v, h, k)$}
  \medskip
  \hrule
  \medskip
  \textbf{Oracle:} Posers $P^{(1)}$, $P^{(g)}$, circuit $\widetilde{C}$, functions $g: \{0,1\}^{k} \rightarrow \{0,1\}$, \\
  \IndII  $hash:Q \rightarrow \{0,1, \dots, 2(h + v) - 1\}$. \\
  \textbf{Input:}  Parameters $\epsilon$, $\delta$, $n$, $k$, the number of verification $v$ and hint $h$ queries.\\
  \textbf{Output:} A circuit $D$.
  \medskip\hrule\medskip
  \For $i:=1$ \To $\frac{6k}{\epsilon}n$ \Do \\
  \IndI $\pi^* \xleftarrow{\$} \{0,1\}^{n}$\\
  \IndI $\widetilde{S}_{\pi^*,0} := \textbf{EstimateSurplus}^{P^{(1)}, P^{(g)}, \widetilde{C}, g, hash}(\pi^*, 0, k, \epsilon, \delta,n)$\\
  \IndI $\widetilde{S}_{\pi^*,1} := \textbf{EstimateSurplus}^{P^{(1)}, P^{(g)}, \widetilde{C}, g, hash}(\pi^*, 1, k, \epsilon, \delta,n)$\\
  \IndI \If $ \exists b \in \{0,1\}: \widetilde{S}_{\pi^*,b} \geq (1 - \frac{3}{4k}) \epsilon$ \Then \\
  \IndII Let $C_1'$ have oracle access to $\widetilde{C}$, and have hard-coded $\pi^*$ \\
  \IndII Let $\widetilde{C}_2'$ have oracle access to $\widetilde{C}$, and have hard-coded $\pi^*$. \\
  \IndII $\widetilde{C}' := (C_1', \widetilde{C}_2')$ \\
  \IndII $g'(b_2, \dots, b_k) := g(b, b_2, \dots, b_k)$\\
  \IndII\Return $Gen^{P^{(1)},P^{(g)}, \widetilde{C}', g', hash}(\epsilon, \delta, n, v, h, k-1)$ \\
  // \textit{all estimates are lower than $(1-\frac{3}{4k})\varepsilon$}\\
  \Return $D^{P^{(1)},P^{(g)}, \widetilde{C}, hash, g}$
\end{codeblock}

\begin{proof}[Lemma \ref{lemma:sec_amp_for_p_hash}]
First let us consider the case where $k=1$. The function $g: \{0,1\} \rightarrow \{0,1\}$ is either the identity or a constant function.
If $g$ is the identity function, then the circuit $D$ returned by \textit{Gen} directly uses $\widetilde{C}$ to find a solution.
From the assumptions of Lemma \ref{lemma:sec_amp_for_p_hash} we know that $\widetilde{C}$ succeeds with probability at least
$\delta + \epsilon$. Hence, $D$ trivially satisfies the statement of Lemma \ref{lemma:sec_amp_for_p_hash}.
If $g$ is a constant function the statement is vacuously true.

The general case is more involved. We distinguish two possibilities.
If $Gen$ manages to find in one of the iterations $\pi^*$ such that an estimate
$\widetilde{S}_{\pi^*,b} \geq (1-\frac{3}{4k})\epsilon$, then we define a new monotone function
$g'(b_2, \dots, b_k) := g(b, b_2, \dots, b_k)$ and a circuit $\widetilde{C}' = (C_1', \widetilde{C}_2')$ with oracle access to $\widetilde{C} := (C_1, \widetilde{C}_2)$,
where $C_1'$ first internally simulates the interaction between $C_1$ and $P^{(1)}(\pi^*)$, and then use $C_1$ to interact with $P^{(g')}$.
The circuit $\widetilde{C}_2'$ uses $\widetilde{C}$ to obtain a solution $(q, y_1, \dots, y_k)$ for the $k$-wise direct product with $\pi_1$ fixed to $\pi^*$,
and returns $(q, y_2, \dots, y_k)$.
We know that one of the surplus estimates $\widetilde{S}_{\pi^*, b}$ is greater or equal $1 - \frac{3}{4k}\epsilon$, and using Lemma \ref{lemma:surplus_estimate}
we conclude that $S_{\pi^*,b} \geq \widetilde{S}_{\pi^*, b} - \frac{\epsilon}{4k} \geq \epsilon - \frac{\epsilon}{k}$ almost surely.
Therefore, the circuit $\widetilde{C}'$ succeeds in solving the $(k-1)$-wise direct product of puzzles with probability
at least $\Pr_{u \leftarrow \mu^{k-1}_{\delta}}[g'(u_1,\dots, u_{k-1} )] + \epsilon$.
We see that in this case $\widetilde{C}'$ satisfies the conditions of Lemma \ref{lemma:sec_amp_for_p_hash}
for the $(k-1)$-wise direct product of puzzles, and we recurse using $g'$ and $\widetilde{C}'$.

If all estimates are less than $(1-\frac{3}{4k})\epsilon$, then intuitively $C$
does not succeeds on the remaining $k-1$ puzzles with much higher probability than
an algorithm that correctly solves each puzzle with probability $\delta$.
However, from the assumption we know of Lemma \ref{lemma:sec_amp_for_p_hash} that on all $k$ puzzles the success probability of $\widetilde{C}$ is higher.
Therefore, it is likely that the first puzzle is correctly solved unusual often.
%
It remains to prove that this intuition is indeed correct.
Let $\cG_{b} := \{ b_1, b_2, \dots, b_k : g(b, b_2, \dots, b_k) = 1 \}$ then we have
\begin{align}
  \label{eqs:set_g}
  \underset{u \leftarrow \mu_{\delta}^k}{\Pr}[u \in G_b] = \underset{u \leftarrow \mu_{\delta}^k}{\Pr}[g(b, u_2, \dots, u_k) = 1]\notag\\
 \underset{\pi^{(k)}, \rho}{\Pr}[c \in G_b] = \underset{\pi^{(k)}, \rho}{\Pr}[g(b, c_2, \dots, c_k) = 1].
\end{align}
We fix $\pi^*$ and use (\ref{eq:s_pi_b}), (\ref{eqs:set_g}) to obtain
\begin{align}
\label{eq:diff_s01}
\underset{u \leftarrow \mu_{\delta}^k}{\Pr}[u \in G_1] - \underset{u \leftarrow \mu_{\delta}^k}{\Pr}[u \in G_0] =
\underset{\pi^{(k)}, \rho}{\Pr}[c \in G_1 \mid \pi_1 = \pi^*] - \underset{\pi^{(k)}, \rho}{\Pr}[c \in G_0 \mid \pi_1 = \pi^*] - (S_{\pi^*, 1} - S_{\pi^*,0})
\end{align}
Since $g$ is monotone, we have that $\cG_0 \subseteq \cG_1$, and can write (\ref{eq:diff_s01}) as
\begin{align}
  \label{eq:diff_s01_next}
  \underset{u \leftarrow \mu_{\delta}^k}{\Pr}[u \in \cG_1 \setminus \cG_0] = \underset{\pi^{(k)}, \rho}{\Pr}[c \in \cG_1 \setminus \cG_0 \mid \pi_1 = \pi^*] - (S_{\pi^*,1} - S_{\pi^*,0}).
\end{align}
Still fixing $\pi_1 = \pi^*$ we multiply both sides of (\ref{eq:diff_s01_next}) by
\begin{align*}
\underset{
  \mathclap{
    \substack{\rho \\ x := \langle P^{(1)}(\pi^*), D_1(\rho) \rangle_{\text{trans}}
    \\ (\Gamma_V, \Gamma_H) := \langle P^{(1)}(\pi^*), D_1(\rho) \rangle_{P^{(1)}} }}}
{\Pr}\mkern13mu [\Gamma_V(D_2(x, \rho)) = 1]
/ \underset{u \leftarrow \mu_{\delta}^k}{\Pr}[ u \in \cG_1 \setminus\cG_0].
\end{align*}
%
which yields
\begin{align}
\label{eq:pr_d_succ_0}
&\underset{
  \mathclap{
    \substack{\rho \\ x := \langle P^{(1)}(\pi^*), D_1(\rho) \rangle_{\text{trans}}
      \\ (\Gamma_V, \Gamma_H) := \langle P^{(1)}(\pi^*), D_1(\rho) \rangle_{P^{(1)}} }}}
{\Pr}\mkern13mu[\Gamma_V(D_2(x, \rho)) = 1] \notag\\
%
&\IndII = \mkern13mu
  \underset{
    \mathclap{
      \substack{\rho \\ x := \langle P^{(1)}(\pi^*), D_1 (\rho) \rangle_{\text{trans}}
        \\ (\Gamma_V, \Gamma_H) := \langle P^{(1)}(\pi^*), D_1 (\rho) \rangle_{P^{(1)}} }}}
  {\Pr}\mkern13mu[\Gamma_V(D_2(x, \rho)) = 1]
  \underset{\pi^{(k)},\rho}{\Pr}[c \in \cG_1 \setminus \cG_0 \mid \pi = \pi^*]
\frac{1}{
  \underset{u \leftarrow \mu_{\delta}^k}{\Pr}[ u \in \cG_1 \setminus \cG_0]
}\notag\\
%
&\IndIII - \mkern13mu
\underset{
  \mathclap{
  \substack{\rho \\ x := \langle P^{(1)}(\pi^*), D_1(\rho) \rangle_{\text{trans}}
    \\ (\Gamma_V, \Gamma_H) := \langle P^{(1)}(\pi^*), D_1(\rho) \rangle_{P^{(1)}} }}}
{\Pr}\mkern13mu[\Gamma_V(D_2(x, \rho)) = 1](S_{\pi^*,1} - S_{\pi^*,0})
\frac{1}{
\underset{u \leftarrow \mu_{\delta}^k}{\Pr}[ u \in \cG_1 \setminus\cG_0]
}
\end{align}
We make use of the facts that $\Gamma_V(D(x,\rho)) = 1$ implies that $c_1 = 1$ and $D_2(x,\rho) \neq \bot$, and that
the event $D_2(x^*, \rho) \neq \bot$ implies $c \in \cG_1 \setminus \cG_0$, which let us
write the numerator of the first summand of (\ref{eq:pr_d_succ_0}) as
\begin{align}
\IndII &\underset{
  \mathclap{
  \substack{\rho \\
    x := \langle P^{(1)}(\pi^*), D_1 (\rho) \rangle_{\text{trans}} \\
    (\Gamma_V, \Gamma_H) := \langle P^{(1)}(\pi^*), D_1(\rho) \rangle_{P^{(1)}} }}}
{\Pr}\mkern13mu[\Gamma_V(D_2(x, \rho)) = 1]
\underset{\pi^{(k)},\rho}{\Pr}[c \in \cG_1 \setminus \cG_0 \mid \pi_1 = \pi^*] \notag\\
  &\IndII = \underset{\mathclap{\substack{\rho
      \\ x = \langle P^{(1)}(\pi^*), D_1(\rho) \rangle_{\text{trans}}}}}
  {\Pr}\mkern13mu[D_2(x,\rho) \neq \bot]
  \underset{\pi^{(k)}, \rho}{\Pr}[c_1 = 1 \mid c \in \cG_1 \setminus \cG_0, \pi_1 = \pi^*]
  \underset{\pi^{(k)}, \rho}{\Pr}[c \in \cG_1 \setminus \cG_0 \mid \pi_1 = \pi^*]
\end{align}
Now we consider two cases:
if $\underset{\pi^{(k)}, \rho}{\Pr}[ c \in \cG_1 \setminus \cG_0 \mid \pi_1 = \pi^*] \leq \frac{\epsilon}{6k}$ then
\begin{align}
  \underset{\pi^{(k)}}{\Pr}[c_1 = 1 \mid c \in \cG_1 \setminus \cG_0, \pi_1 = \pi^*] \underset{\pi^{(k)}}{\Pr}[c \in \cG_1 \setminus \cG_0 \mid \pi_1 = \pi^*] \leq \frac{\epsilon}{6k},
\end{align}
for $\underset{\pi^{(k)}, \rho}{\Pr}[c \in \cG_1 \setminus \cG_0 \mid \pi_1 = \pi^*] > \frac{\epsilon}{6k}$ the circuit $D_2$ outputs $\bot$
if and only if it fails in all $\frac{6k}{\epsilon} \log(\frac{6k}{\epsilon})$ iterations to find $\pi^{(k)}$ such that $g(1, c_2, \dots, c_k) = 1 \land g(0, c_2, \dots, c_k) = 0$
(i.e. in none of the iterations $c \in \cG_1 \setminus \cG_0$) which happens with probability
\begin{align}
\underset{
  \mathclap{
    \substack{
      \rho \\
      x := \langle P^{(1)}(\pi), D_1(\rho) \rangle_{\text{trans}}}}}
{\Pr}\mkern13mu[D_2(x,\rho) = \bot]
\leq (1 - \frac{\epsilon}{6k})^{\frac{6k}{\epsilon}\log(\frac{\epsilon}{6k})} \leq \frac{\epsilon}{6k}.
\end{align}
We conclude that in both cases:
\begin{align}
  \label{ineq:first_part}
  &\underset{
    \mathclap{
    \substack{\rho \\
      x := \langle P^{(1)}(\pi^*), D_1(\rho) \rangle_{\text{trans}}}}}
  {\Pr}\mkern13mu[D_2(x,\rho) \neq \bot]
  \underset{\pi^{(k)}, \rho}{\Pr}[c_1 = 1 \mid c \in \cG_1 \setminus \cG_0, \pi_1 = \pi^*]
  \underset{\pi^{(k)}, \rho}{\Pr}[c \in \cG_1 \setminus \cG_0 \mid \pi_1 = \pi^*] \notag\\
  &\IndII \geq \underset{\pi^{(k)}, \rho}{\Pr}[c_1 = 1 \mid c \in \cG_1 \setminus \cG_0, \pi_1 = \pi^*]\underset{\pi^{(k)}, \rho}
  {\Pr}[c \in \cG_1 \setminus \cG_0 \mid \pi_1 = \pi^*] - \frac{\epsilon}{6k} \notag\\
  &\IndII = \underset{\pi^{(k)}, \rho}{\Pr}[c_1 = 1 \land c \in \cG_0 \setminus \cG_1 \mid \pi_1 = \pi^*] - \frac{\epsilon}{6k} \notag\\
  &\IndII = \underset{\pi^{(k)}, \rho}{\Pr}[g(c_1, c_2,\dots, c_k) = 1 \land g(0, c_2, \dots, c_k) = 0 \mid \pi_1 = \pi^*] - \frac{\epsilon}{6k} \notag\\
  &\IndII = \underset{\pi^{(k)}, \rho}{\Pr}[g(c) = 1 \mid \pi_1 = \pi^*] -  \underset{\pi^{(k)}, \rho}{\Pr}[c \in \cG_0 \mid \pi_1 = \pi^*] - \frac{\epsilon}{6k} \notag\\
  &\IndII \stackrel{(\ref{eq:s_pi_b})}{=}
   \underset{\pi^{(k)}, \rho}{\Pr}[g(c_1, c_2,\dots, c_k) = 1 \mid \pi_1 = \pi^*] -  \underset{u \leftarrow \mu_{\delta}^{(k)}}{\Pr}[u \in \cG_0]  - S_{\pi^*,0} - \frac{\epsilon}{6k}.
\end{align}
For the second summand of (\ref{eq:pr_d_succ_0}) we show that if we do not recurse, then the majority of the estimates is low almost surely.
Let us assume that
\begin{align}
\underset{\pi, \rho}{\Pr}\left[\left(S_{\pi,0} \leq (1 - \frac{1}{2k})\epsilon\right) \land \left( S_{\pi,1} \leq (1-\frac{1}{2k})\epsilon\right)\right] < 1 - \frac{\epsilon}{6k},
\end{align}
then the algorithm recurses almost surely.
Therefore, under the assumption that $Gen$ does not recurse, we have with high probability
\begin{align}
\underset{\pi, \rho}{\Pr}\left[\left(S_{\pi,0} \leq (1 - \frac{1}{2k})\epsilon\right) \land \left( S_{\pi,1} \leq (1-\frac{1}{2k})\epsilon\right)\right] \geq 1 - \frac{\epsilon}{6k}.
\end{align}
Let us define a set
\begin{align}
  \cW = \left\{ \pi :  \left(S_{\pi,0} \leq (1 - \frac{1}{2k})\epsilon\right) \land \left( S_{\pi,1} \leq (1-\frac{1}{2k})\epsilon \right) \right\}
\end{align}
and use $\cW^c$ to denote the complement of $\cW$.
We bound the second summand in (\ref{eq:pr_d_succ_0})
\begin{align}
  \label{ineq:second_eq}
&\mathbb{E}_{\pi^*}[ S_{\pi^*,0}
\mkern23mu
+
\mkern23mu
\underset{
  \mathclap{
  \substack{\rho \\ x := \langle P^{(1)}(\pi^*), D_1(\rho) \rangle_{\text{trans}}
    \\ (\Gamma_V, \Gamma_H) := \langle P^{(1)}(\pi^*), D_1 (\rho) \rangle_{P^{(1)}} }}}
{\Pr}\mkern13mu[\Gamma_V(D_2(x, \rho)) = 1]
(S_{\pi^*,1} - S_{\pi^*,0})] \notag\\
%
&\IndII = \mathbb{E}_{\pi^* \in \cW^c}[ S_{\pi^*,0}
\mkern23mu + \mkern23mu
\underset{
  \mathclap{
  \substack{\rho \\ x := \langle P^{(1)}(\pi^*), D_1(\rho) \rangle_{\text{trans}}
    \\ (\Gamma_V, \Gamma_H) := \langle P^{(1)}(\pi^*), D_1 (\rho) \rangle_{P^{(1)}} }}}
{\Pr}\mkern13mu[\Gamma_V(D_2(x, \rho)) = 1]
  (S_{\pi^*,1} - S_{\pi^*,0})] \notag\\
&\IndIII +  \mathbb{E}_{\pi^* \in \cW}[ S_{\pi^*,0} \mkern13mu + \mkern13mu
\underset{
  \mathclap{
  \substack{\rho \\ x := \langle P^{(1)}(\pi^*), D_1(\rho) \rangle_{\text{trans}}
    \\ (\Gamma_V, \Gamma_H) := \langle P^{(1)}(\pi^*), D_1 (\rho) \rangle_{P^{(1)}} }}}
{\Pr}\mkern13mu[\Gamma_V(D_2(x, \rho)) = 1]
(S_{\pi^*,1} - S_{\pi^*,0}) ] \notag\\
&\IndII \leq \frac{\epsilon}{6k} + \mathbb{E}_{\pi^* \in \cW^c}[ S_{\pi^*,0} \mkern23mu + \mkern23mu
\underset{
  \mathclap{
  \substack{\rho \\ x := \langle P^{(1)}(\pi^*), D_1(\rho) \rangle_{\text{trans}}
    \\ (\Gamma_V, \Gamma_H) := \langle P^{(1)}(\pi^*), D_1 (\rho) \rangle_{P^{(1)}} }}}
{\Pr}\mkern13mu[\Gamma_V(D_2^{\widetilde{C}}(x, \rho)) = 1]
((1 - \frac{1}{2k})\epsilon - S_{\pi^*,0})] \notag\\
&\IndII \leq \frac{\epsilon}{6k} + 1 - \frac{\epsilon}{2k} = 1 - \frac{\epsilon}{3k}.
\end{align}
We observe that
\begin{align}
  \label{eq:gu_rel}
\underset{u \leftarrow \mu_{\delta}^k}{\Pr}[g(u) = 1]
&= \Pr[u \in \cG_0 \lor ( u \in \cG_1 \setminus \cG_0 \land u_1 = 1)] \notag\\
&= \Pr[u \in \cG_0] + \Pr[u \in \cG_1 \setminus \cG_0] \Pr[u_1 = 1].
\end{align}
Finally, we insert (\ref{ineq:first_part}) and (\ref{ineq:second_eq}) into equation (\ref{eq:pr_d_succ_0}) and use (\ref{eq:gu_rel}) to obtain
\begin{align*}
  \IndII
\underset{
  \mathclap{
  \substack{\rho \\ x := \langle P^{(1)}(\pi^*), D_1(\rho) \rangle_{\text{trans}}
    \\ (\Gamma_V, \Gamma_H) := \langle P^{(1)}(\pi^*), D_1 (\rho) \rangle_{P^{(1)}} }}}
{\Pr}\mkern13mu[\Gamma_V(D_2(x, \rho)) = 1]
&\geq \mathbb{E}_{\pi^*}\left[\frac{{\Pr}_{\pi^{(k)}, \rho}[g(c) = 1 \mid \pi_1 = \pi^*] -
{\Pr}_{u \leftarrow \mu_{\delta}^{k}}[u \in G_0] - (1 - \frac{1}{6k})\epsilon} {\Pr_{u \leftarrow \mu_{\delta}^{k}}[u \in \cG_1 \setminus \cG_0]}\right] \notag.
 \end{align*}
 From the assumptions of Lemma \ref{lemma:sec_amp_for_p_hash} we know that $\Pr_{\pi^{(k)}, \rho} [g(c) = 1] \geq \Pr_{u \leftarrow \mu_{\delta}^{(k)}}[g(u) = 1]$,
 thus we get
 \begin{align}
   \label{eq:proof_final}
\underset{
  \mathclap{
  \substack{\rho \\ x := \langle P^{(1)}(\pi^*), D_1(\rho) \rangle_{\text{trans}}
    \\ (\Gamma_V, \Gamma_H) := \langle P^{(1)}(\pi^*), D_1 (\rho) \rangle_{P^{(1)}} }}}
{\Pr}\mkern13mu[\Gamma_V(D_2(x, \rho)) = 1]
 &\geq \frac{ {\Pr}_{u \leftarrow \mu_{\delta}^{k}}[g(u) = 1] + \epsilon +
 \Pr_{u \leftarrow \mu_{\delta}^{k}}[u \in \cG_0] - (1 - \frac{1}{6k})\epsilon} {\Pr_{u \leftarrow \mu_{\delta}^{k}}[u \in \cG_1 \setminus \cG_0]} \notag\\
 &\geq \frac{\epsilon + \delta\Pr_{u \leftarrow \mu_{\delta}^{k}}[u \in \cG_1 \setminus \cG_0] - (1 - \frac{1}{6k})\epsilon}
{\Pr_{u \leftarrow \mu_{\delta}^{k}}[u \in \cG_1 \setminus \cG_0]} \geq \delta + \frac{\epsilon}{6k}
\end{align}
\end{proof}

%
%%% Local Variables:
%%% mode: latex
%%% TeX-master: "../master"
%%% End:
