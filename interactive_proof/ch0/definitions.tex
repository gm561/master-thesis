%
%Define high probability
%
\noindent
We write $u \leftarrow \mu_{\delta}^k$ to denote a tuple $u$ of length $k$ which each element
is independently drawn from the Bernoulli distribution with parameter $\delta$.
We denote the protocol execution between probabilistic algorithms $A$ and $B$ by $\langle A, B \rangle$.
Additionally, the output of $A$ in such a protocol execution is denoted by $\langle A, B \rangle_A$.
%
%FIXME is x a puzzle?
\begin{definition}[Dynamic weakly verifiable puzzle.]
  \label{def:dwvp}
  A dynamic weakly verifiable puzzle (DWVP) is defined by a probabilistic algorithm $P$
  called a problem poser.
  A problem solver $S = (S_1, S_2)$ for $P$ is a probabilistic two phase algorithm.
  We write $P(\pi)$ to denote the execution of $P$ with the randomness fixed to $\pi \in \{0,1\}^n$, and $S_1(\rho)$
  to denote that the execution of the first phase of $S$ with the randomness fixed to $\rho \in \{0,1\}^{*}$.
  The poser $P(\pi)$ and the solver $S_1(\rho)$ interact.
  As the result of the interaction $P(\pi)$ outputs circuits $\Gamma_{V}$, $\Gamma_{H}$, and a bitstring $x \in \{0,1\}^{*}$.
  The algorithm $S_1(\rho)$ produces no output.
  The circuit $\Gamma_{V}$ takes as input $q \in Q$, an answer $y \in \{0,1\}^*$,
  and outputs a bit. An answer $(q,y)$ is a correct solution if and only if $\Gamma_V(q,y) = 1$.
  The circuit $\Gamma_H$ on input $q \in Q$ outputs a hint such that $\Gamma_V(q,\Gamma_H(q)) = 1$.

  In the second phase $S_2$ takes as input $x$, and has oracle access to $\Gamma_V$ and $\Gamma_H$.
  The execution of $S_2$ with $x$ and the randomness fixed to $\rho$
  is denoted by $S_2(x, \rho)$. The queries of $S_2$ to $\Gamma_V$ are called verification queries, and to $\Gamma_H$ hint queries.
  The algorithm $S_2$ can ask at most $h$ hint queries, $v$ verification queries, and succeeds if and only if
  it makes a verification query $(q,y)$ such that $\Gamma_V(q,y) = 1$, and it has not previously asked for a hint query on $q$.
\end{definition}
%
\begin{definition}[$k$-wise direct-product of DWVPs.]
Let $g: \{0,1\}^{k} \rightarrow \{0,1\}$ be a monotone function and $P^{(1)}$ a problem poser as in Definition \ref{def:dwvp}.
The $k$-wise direct product of $P^{(1)}$ is a DWVP defined by a probabilistic algorithm $P^{(g)}$.
We write $P^{(g)}(\pi^{(k)})$ to denote the execution of $P^{(g)}$ with the randomness fixed to $\pi^{(k)} := (\pi_1, \dots, \pi_k)$.
Let $S := (S_1, S_2)$ be a solver for $P^{(g)}$ as in Definition \ref{def:dwvp}.
The algorithm $P^{(g)}$ sequentially interacts in $k$ rounds with $S_1$ such that
in the $i$th round $(x_i, \Gamma_V^{i}, \Gamma_H^{i} ) := \left\langle P^{(1)}(\pi_i), S_1(\rho) \right\rangle_{P^{(g)}}$.
Finally, $P^{(g)}$ outputs:
a verification circuit
\begin{align*}
  \Gamma_V^{(g)} (q, y_1, \dots, y_k) := g(\Gamma_V^{1}(q, y_1), \dots, \Gamma_V^{k}(q, y_k)),
\end{align*}
a hint circuit
\begin{align*}
  \Gamma_H^{(k)} (q) := (\Gamma_H^{1}(q), \dots, \Gamma_H^{k}(q)),
\end{align*}
and a puzzle $x^{(k)} := (x_1, \dots, x_k)$.

\end{definition}
%
Let $C$ be a random circuit that corresponds to a solver $S$ in Definition \ref{def:dwvp}.
Similarly as for two phase algorithm, we write $C := (C_1, C_2)$ to denote that $C$ in the first phase uses $C_1$,
and in the second phase $C_2$.
%
A verification query $(q,y)$ of $C$ for which a hint query on this $q$ has been asked before can not be a successfully verification query.
Therefore, without loss of generality, we make an assumption that $C$ does not ask verification queries on $q \in Q$,
for which a hint query has been asked before.

%
\begin{codeblock}
  \textbf{Experiment $Success^{P, C^{(\cdot, \cdot)}}(\pi, \rho) $}
  \medskip
  \hrule
  \medskip
  \textbf{Oracle:} A problem poser $P$, a solver circuit $C^{(\cdot,\cdot)}$.\\
  \textbf{Input:}  Bitstrings $\pi$, $\rho$.\\
  \textbf{Output:} A bit $b \in \{0,1\}$.
  \medskip\hrule\medskip
  $(x, \Gamma_V, \Gamma_H) := \langle P(\pi), C_1(\rho) \rangle_{P}$ \\
  Run $C_2^{\Gamma_V,\Gamma_H}(x, \rho)$ \\
  \IndI \If $C_2^{\Gamma_V, \Gamma_H}$ asks a verification query $(q, y)$ and $\Gamma_V(q, y) = 1$ \then \\
  \IndII \return $1$ \\
%  \IndI Let $Q_{V} := \{q_i: \text{  asked a verification query $(q_i,y_i)$ and $\Gamma_V(q_i, y_i) = 1$} \}$\\
%  \IndI Let $Q_{Hint} := \{q_i: \text{$C^{\Gamma_V, \Gamma_H}$ asked a hint query on $q_i$} \}$\\
%  \If $\exists q_j \in Q_{solved} : \forall i < j,  q_i \notin Q_{Hint}$ \then \\
%  \IndI \return $1$\\
%  \textbf{else} \\
  \return $0$ \\
\end{codeblock}
%
The success probability of $C$ in solving a puzzle defined by $P$ in the experiment $Success$ is
\begin{align}
 \underset{\pi, \rho}{\Pr}[Success^{P,C^{(\cdot, \cdot)}}(\pi, \rho) = 1].
\end{align}
%
\begin{todo}
  % \textbf{TODO:} Is the circuit bound well defined?\\
  \textbf{TODO:} What happens when $8(h+v) \left(\underset{\mu \leftarrow \mu_\delta^k}{\Pr}[g(\mu) = 1] + \varepsilon\right) \geq 1$. The formula does not work then.\\
\end{todo}
%
\begin{theorem}[Security amplification for a dynamic weakly verifiable puzzle.]
\label{th:sec_amp_for_dwvp}
Let $P^{(1)}$ be a fixed problem poser as in Definition \ref{def:dwvp}, and $P^{(g)}$ be the poser for the $k$-wise direct product of $P^{(1)}$.
There exists a probabilistic algorithm $Gen(C, g, \varepsilon, \delta, n, v, h)$ which takes as input: a solver circuit $C$ for the puzzle posed by $P^{(g)}$,
a monotone function $g:\{0,1\}^k \rightarrow \{0,1\}$, parameters $\varepsilon, \delta,n$,
the number of verification queries $v$, and hint queries $h$ asked by $C$, and outputs a random circuit $D$
such that the following holds: \\
If $C$ is such that \\
  \begin{align*}
    \underset{\pi^{(k)}, \rho}{\Pr}[Success^{P^{(g)}, C}(\pi^{(k)}, \rho) = 1] \geq 8(h+v)\left(\underset{u \leftarrow \mu_\delta^k}{\Pr}[g(u) = 1] + \varepsilon\right)
  \end{align*}
then $D$ satisfies almost surely
  \begin{align*}
    \underset{\pi, \rho}{\Pr}[Success^{P^{(1)},D}(\pi, \rho) = 1] \geq (\delta + \frac{\varepsilon}{6k}).
  \end{align*}
Additionally, $Gen$ and $D$ require oracle access to $g$, $P^{(1)}$,$C$.
Furthermore, $D$ requires also oracle access to $\Gamma_V^{(g)}$ and $\Gamma_H^{(g)}$,
and asks at most $h$ hint queries and $v$ verification queries.
Finally, $Size(D) \leq Size(C) \cdot \frac{6k}{\varepsilon}$ and $Time(Gen) = poly(k, \frac{1}{\varepsilon}, n, v, h)$.
\end{theorem}
%
The Theorem \ref{th:sec_amp_for_dwvp} implies that if there is no good solver for a puzzle defined by $P^{(1)}$, then a good solver for
a $k$-wise direct product of $P^{(1)}$ does not exist.

%\begin{todo}
%  \textbf{TODO:} Write it more clearly, give more intuition about the function g() and then why we can approach the problem in this way.
%\end{todo}
%We know that the circuit $C$ performs better than an algorithm that would solve independently each of $k$ puzzles defined by $P^{(1)}$ with probability $\delta$.

The idea of the algorithm $Gen$ is to output a circuit $D$ that solves the input puzzle often.
We know that $C$ has good success probability for a $k$-wise product of $P^{(1)}$.
The algorithm $Gen$ tries to find a puzzle such that when $C$ runs with this puzzle fixed
on the first position, and disregards whether this puzzle is correctly solved
then the assumptions of Theorem \ref{th:sec_amp_for_dwvp} are true but for a $k-1$-wise direct product.
If it is possible to find such a puzzle then $Gen$ recurses.
Otherwise, when the first position is disregarded then the success probability of $C$ is not substantially better.
As we know that $C$ performs good for $k$-wise product, it means that the first position is important, in the sense that
it is solved unusually often.
Therefore, it is reasonable to construct the circuit $D$ using $C$ by placing the input puzzle of $D$ on that position, and then
finding remaining $k-1$ puzzles. These $k-1$ remaining puzzles are generated by the circuit $D$, hence it is possible to check
whether they are correctly solved by the circuit $C$. We know that circuit $C$ has good success probability, and the puzzle on the first
position is important. Therefore, if we are able to find a $k-1$ puzzles such that the fact whether the $k$-wise direct product is correctly
solved depends on whether the puzzle on the first position is correctly solved then we can assume that $C$ is often correct on this first position.

There are some problems with this approach, first we have to ensure that we can make a decision when the algorithm $Gen$ should recurse and when not
correctly with high probability. Then, we have to show that it is possible to find a puzzles such that $C$ is often correct on the first position.
Finally, we also have to be sure that we do not ask a hint query, on the final verification query to the oracle.
To satisfy the last requirement we split $Q$.

%%% Local Variables:
%%% mode: latex
%%% TeX-master: "../master"
%%% End:
