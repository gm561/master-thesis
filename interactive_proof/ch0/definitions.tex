%
% Define high probability
%
\noindent
We write $\mu_{\delta}$ to denote the Bernoulli distribution, where outcome $1$ occurs with
probability $\delta$ and $0$ with probability $1-\delta$ where $0 \leq \delta \leq 1$.
Moreover, we use $\mu_{\delta}^k$ to denote a probability distribution over $k$-tuples,
where each bit of a $k$-tuple is drawn independently according to $\mu_{\delta}$.
Finally, let $u \leftarrow \mu_{\delta}^k$ denote that a $k$-tuple $u$ is chosen according to $\mu_{\delta}^k$.

The protocol execution between two probabilistic algorithms $A$ and $B$ is denoted by $\langle A, B \rangle$.
The output of $A$ in such a protocol execution is denoted by $\langle A, B \rangle_A$ and of $B$ by $\langle A, B \rangle_B$.
Finally, let $\langle A, B \rangle_{\text{trans}}$ denote the transcript of communication between $A$ and $B$.

We define a \textit{two phase algorithm} $A := (A_1, A_2)$ as an algorithm where in the first phase an algorithm $A_1$
is executed and in the second phase an algorithm $A_2$.

We say that an event happens \textit{almost surely} or with \textit{high probability} if
it occurs with probability at least $1 - 2^{-n} poly(n)$.

\begin{definition}[Dynamic weakly verifiable puzzle.]
  \label{def:dwvp}
  A dynamic weakly verifiable puzzle (DWVP) is defined by a probabilistic algorithm $P$
  called a problem poser.
  A problem solver $S := (S_1, S_2)$ for $P$ is a probabilistic two phase algorithm.
  We write $P_n(\pi)$ to denote the execution of $P$ with the randomness fixed to $\pi \in \{0,1\}^n$, and $(S_1,S_2)(\rho)$
  to denote the execution of both $S_1$ and $S_2$ with the randomness fixed to $\rho \in \{0,1\}^{*}$.

  In the first phase, the poser $P_n(\pi)$ and the solver $S_1(\rho)$ interact.
  As the result of the interaction $P_n(\pi)$ outputs a verification circuit $\Gamma_{V}$ and a hint circuit $\Gamma_{H}$.
  The algorithm $S_1(\rho)$ produces no output.
  The circuit $\Gamma_{V}$ takes as input $q \in Q$, an answer $y \in \{0,1\}^*$,
  and outputs a bit. We say that an answer $(q,y)$ is a correct solution if and only if $\Gamma_V(q,y) = 1$.
  The circuit $\Gamma_H$ on input $q \in Q$ outputs a hint such that $\Gamma_V(q,\Gamma_H(q)) = 1$.

  In the second phase, $S_2$ takes as input $x := \langle P_n(\pi), S_1(\rho) \rangle_{\text{trans}}$,
  and has oracle access to $\Gamma_V$ and $\Gamma_H$.
  The execution of $S_2$ with the input $x$ and the randomness fixed to $\rho$
  is denoted by $S_2(x, \rho)$. The queries of $S_2$ to $\Gamma_V$ and $\Gamma_H$ are called verification queries and hint queries respectively.
  The algorithm $S_2$ asks at most $h$ hint queries, $v$ verification queries, and succeeds if and only if
  it makes a verification query $(q,y)$ such that $\Gamma_V(q,y) = 1$, and it has not previously asked for a hint query on $q$.
\end{definition}
%
\begin{definition}[$k$-wise direct-product of DWVPs.]
  Let $g: \{0,1\}^{k} \rightarrow \{0,1\}$ be a monotone function and $P^{(1)}$ a problem poser as in Definition \ref{def:dwvp}.
  The $k$-wise direct product of $P^{(1)}$ is a DWVP defined by a probabilistic algorithm $P^{(g)}$.
  We write $P_{kn}^{(g)}(\pi^{(k)})$ to denote the execution of $P^{(g)}$ with the randomness fixed to $\pi^{(k)} := (\pi_1, \dots, \pi_k)$
  where for each $1 \leq i \leq n : \pi_i \in \{0,1\}^n.$
  Let $(S_1, S_2)(\rho)$ be a solver for $P^{(g)}$ as in Definition \ref{def:dwvp}.
  In the first phase, the algorithm $S_1(\rho)$ sequentially interacts in $k$ rounds with $P_{kn}^{(g)}(\pi^{(k)})$.
  In the $i$-th round $S_1(\rho)$ interacts with $P_n^{(1)}(\pi_i)$,
  and as the result $P_{n}^{(1)}(\pi_i)$ generates circuits $\Gamma_V^i, \Gamma_H^i$.
  Finally, after $k$ rounds $P_{kn}^{(g)}(\pi^{(k)})$ outputs a verification circuit
\begin{align*}
  \Gamma_V^{(g)} (q, y_1, \dots, y_k) := g(\Gamma_V^{1}(q, y_1), \dots, \Gamma_V^{k}(q, y_k))
\end{align*}
and a hint circuit
\begin{align*}
  \Gamma_H^{(k)} (q) := (\Gamma_H^{1}(q), \dots, \Gamma_H^{k}(q)).
\end{align*}
\end{definition}
%
If it is clear from a context, we omit the parameter $n$, and write $P(\pi)$ instead of $P_n(\pi)$ where $\pi \in \{0,1\}^{n}$.

A verification query $(q,y)$ of a solver $S$ for which a hint query on this $q$ has been asked before can not be a verification query that succeeds.
Therefore, without loss of generality, we make the assumption that $S$ does not ask verification queries on $q$
for which a hint query has been asked before. Furthermore, we assume that once $S$ asked a verification query that succeeds,
it does not ask any further hint or verification queries.

Let $C$ be a circuit that corresponds to a solver $S$ as in Definition \ref{def:dwvp}.
Similarly as for a two phase algorithm, we write $C(\rho) := (C_1, C_2)(\rho)$ to denote that $C$
in the first phase uses a circuit $C_1$ and in the second phase a circuit $C_2$.
Additionally, the randomness in both phases is fixed to $\rho \in \{0,1\}^{*}$.

%
\begin{codeblock}
  \textbf{Experiment $Success^{P, C}(\pi, \rho) $}
  \medskip
  \hrule
  \medskip
  \textbf{Oracle:} A problem poser $P$, a solver circuit $C = (C_1, C_2)$.\\
  \textbf{Input:}  Bitstrings $\pi \in \{0,1\}^n$, $\rho \in \{0,1\}^*$.\\
  \textbf{Output:} A bit $b \in \{0,1\}$.
  \medskip\hrule\medskip
  \Run $\langle P(\pi), C_1(\rho) \rangle$ \\
  \IndI $(\Gamma_V, \Gamma_H) := \langle P(\pi), C_1(\rho) \rangle_{P}$ \\
  \IndI $x := \langle P(\pi), C_1(\rho) \rangle_{\text{trans}}$ \\ \\
  \Run $C_2^{\Gamma_V,\Gamma_H}(x, \rho)$ \\
  \IndI \If $C_2^{\Gamma_V, \Gamma_H}(x, \rho)$ asks a verification query $(q, y)$ such that $\Gamma_V(q, y) = 1$ \Then \\
  \IndII \Return $1$ \\
  \Return $0$ \\
\end{codeblock}
%
We define the \textit{success probability} of $C$ in solving a puzzle defined by $P$ as
\begin{align}
 \underset{\pi, \rho}{\Pr}[Success^{P,C}(\pi, \rho) = 1].
\end{align}
Furthermore, we say that $C$ succeeds for $\pi$, $\rho$ if $Success^{P,C}(\pi, \rho) = 1$.
%
\begin{theorem}[Security amplification for a dynamic weakly verifiable puzzle.]
\label{th:sec_amp_for_dwvp}
Let $P^{(1)}$ be a fixed problem poser as in Definition \ref{def:dwvp}, and $P^{(g)}$ be a poser for the $k$-wise direct product of $P^{(1)}$.
There exists a probabilistic algorithm $Gen$ with oracle access to: a solver circuit $C$ for $P^{(g)}$,
a monotone function $g:\{0,1\}^k \rightarrow \{0,1\}$ and problem posers $P^{(1)}$, $P^{(g)}$.
Additionally, $Gen$ takes as input parameters $\varepsilon$, $\delta$, $n$, $k$
the number of verification queries $v$ and hint queries $h$ asked by $C$, and outputs a solver circuit $D$ for $P^{(1)}$ as in Definition \ref{def:dwvp}
such that the following holds: \\
If $C$ is such that
  \begin{align*}
    \underset{\substack{\pi^{(k)} \in \{0,1\}^{kn} \\ \rho \in \{0,1\}^{*}}}{\Pr}\left[Success^{P^{(g)}, C}(\pi^{(k)}, \rho) = 1\right]
    \geq 16(h+v)\left(\underset{u \leftarrow \mu_\delta^k}{\Pr}\left[g(u) = 1\right] + \varepsilon\right)
  \end{align*}
then $D$ satisfies almost surely
  \begin{align*}
    \underset{\substack{\pi \in \{0,1\}^{n} \\ \rho \in \{0,1\}^{*}}}
    {\Pr}\left[Success^{P^{(1)},D}(\pi, \rho) = 1\right] \geq (\delta + \frac{\varepsilon}{6k}).
  \end{align*}
Additionally, $D$ requires oracle access to $g$, $P^{(1)}$, $C$,
and asks at most $\frac{6k}{\epsilon}\log\left(\frac{6k}{\epsilon}\right) h$ hint queries and one verification query.
Finally, $\text{Size}(D) \leq \text{Size}(C) \cdot \frac{6k}{\varepsilon}$ and $\text{Time}(\text{Gen}) = \text{poly}(k, \frac{1}{\varepsilon}, n, v, h)$.
\end{theorem}
%
% The Theorem \ref{th:sec_amp_for_dwvp} implies that if there is no good solver for a puzzle defined by $P^{(1)}$, then a good solver for
% a $k$-wise direct product of $P^{(1)}$ does not exist.

% The idea of the algorithm $Gen$ is to output a circuit $D$ that solves the input puzzle often.
% We know that $C$ has good success probability for a $k$-wise product of $P^{(1)}$.
% The algorithm $Gen$ tries to find a puzzle such that when $C$ runs with this puzzle fixed
% on the first position, and disregards whether this puzzle is correctly solved
% then the assumptions of Theorem \ref{th:sec_amp_for_dwvp} are true for a $k-1$-wise direct product.
% If it is possible to find such a puzzle then $Gen$ could recurse and solve a smaller problem.
% In the optimistic case we can reach $k=1$, which means that we found a good circuit for solving a single
% puzzle by just fixing the initial puzzles of $C$.

% Otherwise, when the first position is disregarded then the success probability of $C$ is not substantially better.
% This is remarkable, as we know that $C$ performs good for $k$-wise product, it means that the first position is important,
% in the sense that $C$ solves the puzzle on that position unusually often.
% Therefore, it is reasonable to construct the circuit $D$ using $C$ by placing the input puzzle of $D$ on that position, and then
% finding remaining $k-1$ puzzles. These $k-1$ remaining puzzles are generated by the circuit $D$, hence it is possible to check
% whether they are correctly solved by the circuit $C$. We know that circuit $C$ has good success probability, and the puzzle on the first
% position is important. Therefore, if we are able to find a $k-1$ puzzles such that the fact whether the $k$-wise direct product is correctly
% solved depends on whether the puzzle on the first position is correctly solved then we can assume that $C$ is often correct on this first position.

% There are some problems with this approach, first we have to ensure that we can make a decision when the algorithm $Gen$ should recurse and when not
% correctly with high probability. Then, we have to show that it is possible to find a puzzles such that $C$ is often correct on the first position.
% Finally, we also have to be sure that we do not ask a hint query, on the final verification query to the oracle.
% To satisfy the last requirement we split $Q$.

%%% Local Variables:
%%% mode: latex
%%% TeX-master: "../master"
%%% End:
