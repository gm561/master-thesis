\begin{definition} {\textbf{(Weakly verifiable puzzle.)}}
A system for weakly verifiable puzzles consists of algorithms for generating random puzzles and for verifying solutions to these puzzles. The pair of algorithms $Z = (G,V)$ where
\begin{itemize}
\item
  The puzzle generator algorithm $G$, on security parameter $k$, outputs a random puzzle $p$
  along with some check information $c$, $(p,c) \leftarrow G(1^k)$

\item
  The puzzle verifier $V$ is a deterministic efficient algorithm that on input a puzzle $p$,
  check information $c$, and answer  $a$, outputs either zero or one, $V(p,c,a) \in \{0,1\}$
\end{itemize}

A \textit{solver} for the above puzzle system is an efficient algorithm \textit{S} that gets a puzzle \textit{p} as input and outputs an answer \textit{a}, outputs either zero or one,
$V(p,c,a) \in \{0,1\}$
\end{definition}

\begin{definition} {Dynamic weakly verifiable puzzle}
A dynamic weakly verifiable puzzle consists of two algorithms \textit{P} and \textit{S}. Where \textit{S} is a problem solver and \textit{P} is a problem poser.
The poser \textit{P} outputs circuits $\Gamma^{V}(q,r)$ and $\Gamma^{H}(q,r)$ where $q \in Q$ (for some well defined set $Q$). The circuit $\Gamma^{V}(q,r)$ is used to verify
correctness of the solutions $r$. Additionally, $\Gamma^{H}(q)$ is a circuit that evaluates a hint function.
A solver can make a number of verification and hint queries.
A solver successfully solves a DWVP if it makes a successfully verification query for a $q$ when
it has not previously asked for verification or hint query on $q$.
%TODO should we not exclude verification queries?


\end{definition}

\begin{definition} {Interactive weakly verifiable puzzle}

\end{definition}

\begin{definition} {Dynamic interactive weakly verifiable puzzle}

\end{definition}

% \definition{}

% \definition{}

% \definition{weakly verifiable puzzle}

%%% Local Variables:
%%% mode: latex
%%% TeX-master: "master.tex"
%%% End:
