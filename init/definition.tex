%TODO how many verification queries might be asked
\begin{definition} {\textbf{(Dynamic weakly verifiable puzzle)}}
  A dynamic weakly verifiable puzzle (DWVP) is defined by a protocol between probabilistic algorithms $(P,S)$.
  The algorithm $P$ is a problem poser, and $S$ is a problem solver.
  The problem poser $P$
  %TODO after the interaction with S
  \textit{P} outputs circuit $\Gamma_{V}$, which takes as input $q \in Q$ and a solution $r \in R$,
  and a circuit $\Gamma_{H}$, which takes as input $q \in Q$.
  The circuit $\Gamma_{V}(q,r)$ is used to verify correctness of the solution $r$ for $q$.
  %TODO make it a bit more clean
  %TODO what is the relation between $r$ and $q$
  %TODO in which random experiment?
  Additionally, $\Gamma_{H}(q)$ is a circuit that evaluates a hint function.
  %TODO define the hint function
  The solver $S$ has oracle access to both circuits $\Gamma_V$ and $\Gamma_H$.
  The calls to the circuit $\Gamma_V$ are called verification queries, and the calls to the circuit $\Gamma_H$ are hint queries.
  %TODO define verification and hint queries
  The solver $S$ successfully solves a DWVP $\Pi$ if it makes a successfully verification query for a $q \in Q$ when
  it has not previously asked for a hint query on this $q$.
  % TODO runtime bound
\end{definition}
%
% DONE should we not exclude verification queries
Let $hash$ be a function $Q \rightarrow \{0, 2(h+v)-1 \}$ that partitions a query space $Q$ into two sets $P_{hash}$ and $Q \setminus P_{hash}$.
The set $P_{hash}$ contains elements $q \in Q$ such that $hash(q) = 0$.
The elements of $P_{hash}$ are called attacking queries.
A \textit{canonical success} with respect to set $P_{hash}$ is a situation when the first successfully verification query is in $P_{hash}$
and all previous hint or verification queries are not in $P_{hash}$.
%
% TODO make this version of the theorem non uniform i.e. assume a hash functio
% TODO size of the circuit
% TODO oracle calls ? non-rewinding?
% TODO is GEN probabilitic?
% TODO running time of Gen size of D
% TODO hash function does it has to be pairwise independent?
% TODO what is random in hash function
% TODO move definitions slightly up
% TODO describe hash function in more details
% TODO most probably we have to lessen the theorem because of the hash function
% TODO would it make more sense to consider situations when one have just a single verification query (if the proof works out also in this situation then it is a stronger result).
\begin{theorem} {\textbf{(Security amplification for DWVP (non unifrom version)).}}
  Let $g: \{0,1\}^k \rightarrow \{0,1\}$ be a monotone function, and $hash$ a function such that
  the probability of a \textit{canonical success} of a problem solver $S$ with respect to $P_{hash}$ is at least $\frac{\varepsilon}{8\left(v + h\right)}$.
  If there exists a circuit $C$ that makes at most $v$ verification queries and $h$ hint queries
  and succeeds with probability
  \begin{align}
    \Pr_{}[\Gamma_{V}^{(g)}( \langle P^{(g)},C \rangle_C ) = 1] \geq \Pr_{\mu \leftarrow \mu_{\delta}^{k}}[g(u) = 1] + \varepsilon ,
  \end{align}
  where the probability is over random coins of $P$ and $C$,
  then there exists a probabilistic algorithm $Gen(C, g, \varepsilon, \delta, n, hash)$ which takes as input a circuit $C$, a function $g$,
  parameters $\varepsilon, \delta, n$ , a function $hash$, and
  produce a circuit $D$ such that with hight probability
  it satisfies
  \begin{align}
    \Pr_{}[\Gamma^{(1)} \left( \langle P^{1} ,D \rangle_D \right) = 1] \geq \frac{1}{8(h+v)} \left( \delta + \frac{\varepsilon}{6k} \right)
  \end{align}
  where the probability is taken over random coins of $D$ and uniformly chosen random input puzzle.
\end{theorem}



%%% Local Variables:
%%% mode: latex
%%% TeX-master: "master.tex"
%%% End:
