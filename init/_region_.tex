\message{ !name(master.tex)}
% \documentclass[11commpt]{article}

% %% We use the memoir class because it offers a many easy to use features.
\documentclass[11pt,a4paper,article,oneside]{memoir}

\counterwithout{section}{chapter}
\usepackage[margin=1in]{geometry}

%% Packages
%% ========

%% LaTeX Font encoding -- DO NOT CHANGE
%\usepackage[OT1]{fontenc}

%% Babel provides support for languages.  'english' uses British
%% English hyphenation and text snippets like "Figure" and
%% "Theorem". Use the option 'ngerman' if your document is in German.
%% Use 'american' for American English.  Note that if you change this,
%% the next LaTeX run may show spurious errors.  Simply run it again.
%% If they persist, remove the .aux file and try again.
\usepackage[english]{babel}

%% Input encoding 'utf8'. In some cases you might need 'utf8x' for
%% extra symbols. Not all editors, especially on Windows, are UTF-8
%% capable, so you may want to use 'latin1' instead.
\usepackage[utf8]{inputenc}

%% This changes default fonts for both text and math mode to use Herman Zapfs
%% excellent Palatino font.  Do not change this.
%\usepackage[sc]{mathpazo}

%% The AMS-LaTeX extensions for mathematical typesetting.  Do not
%% remove.
\usepackage{amsmath,amssymb,amsfonts}

%% NTheorem is a reimplementation of the AMS Theorem package. This
%% will allow us to typeset theorems like examples, proofs and
%% similar.  Do not remove.
%% NOTE: Must be loaded AFTER amsmath, or the \qed placement will
%% break
\usepackage[amsmath,thmmarks]{ntheorem}

%% LaTeX' own graphics handling
\usepackage{graphicx}

%% We unfortunately need this for the Rules chapter.  Remove it
%% afterwards; or at least NEVER use its underlining features.
\usepackage{soul}

%% Some more packages that you may want to use.  Have a look at the
%% file, and consult the package docs for each.
\input{extrapackages}

%% Our layout configuration.  DO NOT CHANGE.
%\input{layoutsetup}

%% Theorem environments.
%% thesis.
%% Theorem-like environments

%% This can be changed according to language. You can comment out the ones you
%% don't need.

\numberwithin{equation}{chapter}

%% German theorems
%\newtheorem{satz}{Satz}[chapter]
%\newtheorem{beispiel}[satz]{Beispiel}
%\newtheorem{bemerkung}[satz]{Bemerkung}
%\newtheorem{korrolar}[satz]{Korrolar}
%\newtheorem{definition}[satz]{Definition}
%\newtheorem{lemma}[satz]{Lemma}
%\newtheorem{proposition}[satz]{Proposition}

%% English variants
\newtheorem{theorem}{Theorem}[chapter]
\newtheorem{example}[theorem]{Example}
\newtheorem{remark}[theorem]{Remark}
\newtheorem{corollary}[theorem]{Corollary}
\newtheorem{lemma}[theorem]{Lemma}
\newtheorem{proposition}[theorem]{Proposition}
\newtheorem{observation}[theorem]{Observation}

\theoremstyle{definition}
\theorembodyfont{\normalfont}
%% end def with blacksquare symbol
\theoremsymbol{\ensuremath{\lozenge}}
\newtheorem{definition}[theorem]{Definition}

%% Proof environment with a small square as a "qed" symbol
\theoremstyle{nonumberplain}
\theorembodyfont{\normalfont}
\theoremsymbol{\ensuremath{\square}}
\theoremseparator{.}
\newtheorem{proof}{Proof}

%\newtheorem{beweis}{Beweis}

\declaretheorem[name=Theorem, numberwithin=chapter]{thm}


%% Helpful macros.
%% Custom commands
%% ===============

%% Special characters for number sets, e.g. real or complex numbers.
\newcommand{\C}{\mathbb{C}}
\newcommand{\K}{\mathbb{K}}
\newcommand{\N}{\mathbb{N}}
\newcommand{\Q}{\mathbb{Q}}
\newcommand{\R}{\mathbb{R}}
\newcommand{\Z}{\mathbb{Z}}
\newcommand{\X}{\mathbb{X}}

\newcommand{\cX}{\mathcal{X}}
\newcommand{\cH}{\mathcal{H}}
\newcommand{\cW}{\mathcal{W}}
\newcommand{\cG}{\mathcal{G}}
\newcommand{\cB}{\mathcal{B}}
\newcommand{\cP}{\mathcal{P}}
\newcommand{\cR}{\mathcal{R}}
\newcommand{\cD}{\mathcal{D}}

%define our own code commands
%use capital latters as most of these commands is already defined
\renewcommand{\For}{\textbf{for }}
\renewcommand{\If}{\textbf{if }}
\renewcommand{\Else}{\textbf{else }}
\renewcommand{\Return}{\textbf{return }}
\newcommand{\Then}{\textbf{then }}
\newcommand{\Do}{\textbf{do: }}
\renewcommand{\And}{\textbf{and }}
\newcommand{\Or}{\textbf{or }}
\newcommand{\Run}{\textbf{run }}
\newcommand{\To}{\textbf{to }}

%% Fixed/scaling delimiter examples (see mathtools documentation)
\DeclarePairedDelimiter\abs{\lvert}{\rvert}
\DeclarePairedDelimiter\norm{\lVert}{\rVert}

%% Use the alternative epsilon per default and define the old one as \oldepsilon
\let\oldepsilon\epsilon
\renewcommand{\epsilon}{\ensuremath\varepsilon}

%% Also set the alternate phi as default.
\let\oldphi\phi
\renewcommand{\phi}{\ensuremath{\varphi}}

% New command that introduces a tab
\newcommand{\itab}[1]{\hspace{0em}\rlap{#1}}
\newcommand{\tab}[1]{\hspace{.2\textwidth}\rlap{#1}}

\DeclareMathOperator{\la0}{\leftarrow}
\DeclareMathOperator{\ra0}{\rightarrow}

\DeclareMathOperator{\hash}{\mathit{hash}}
\DeclareMathOperator{\CanonicalSuccess}{\mathit{CanonicalSuccess}}
\DeclareMathOperator{\Success}{\mathit{Success}}

%the DWVP for the permutation
\DeclareMathOperator{\PiDWVP}{\Pi_{DWVP}}
%the DWPV for the k-wise product of permutations
\DeclareMathOperator{\kPiDWVP}{\Pi_{DWVP}^{(k)}}

\usepackage{enumerate}

\begin{document}

\message{ !name(definition.tex) !offset(-69) }
%TODO how many verification queries might be asked
%TODO after the interaction with S
%TODO what is the relation between $r$ and $q$
%TODO in which random experiment?
%TODO define the hint function
%TODO define verification and hint queries
%TODO runtime bound
\begin{definition} {\textbf{(Dynamic weakly verifiable puzzle)}}
  A dynamic weakly verifiable puzzle (DWVP) is defined by a protocol between probabilistic algorithms $(P,S)$.
  The algorithm $P$ is called a problem poser and $S$ a problem solver.
  The problem poser $P$ produces as output a circuit $\Gamma_{V}$ and a circuit $\Gamma_{H}$.
  The circuit $\Gamma_{V}$ takes as its input $q \in Q$ and an answer $r \in R$.
  An answer $r$ is a correct solution for $q$ if and only if the circuit $\Gamma_V$ on input $(q,r)$ evaluates to true.
  The circuit $\Gamma_H$ on input $q$ provides a hint $r \in R$ such that the circuit $\Gamma_V(q,r) = 1$.
  The solver $S$ has oracle access to both circuits $\Gamma_V$ and $\Gamma_H$.
  The calls to the circuit $\Gamma_V$ are called verification queries. The calls to the circuit $\Gamma_H$ are hint queries.
  The solver $S$ asks at most $h$ hint queries, $v$ verification queries, and successfully solves a DWVP $\Pi$ if and only if
  it makes a verification query $(q,r)$ such that $\Gamma_V(q,r) = 1$, when it has not previously asked for a hint query on this $q$.
\end{definition}
%
% DONE should we not exclude verification queries - it seems that no
%
Suppose that $g: \{0,1\}^k \rightarrow \{0,1\}$ is a monotone function, and $\left( P^{(1)}, S^{(1)} \right)$ is a dynamic weakly verifiable puzzle.
Then $(P^{(g)}, S^{(g)})$ is a dynamic weakly verifiable puzzle $\Pi^{(g)}$, for which in the first phase the problem poser $P^{(g)}$ and solver $S^{(g)}$
sequentially create $k$ instances of a puzzle $\left( P^{(1)}, S^{(1)}\right)$. The problem poser $P^{(g)}$ produces as it output circuits $\Gamma_V^{(g)}$ and $\Gamma_H^{(g)}$.
The hint queries for a puzzle $\Pi^{(g)}$ are answered by a circuit $\Gamma_H^{(g)}(q) = \left( \Gamma_H^{(1)}(q), \dots, \Gamma_H^{(k)} \right)$, and the verification queries by a circuit $\Gamma_V^{(g)}(q, r_1, \dots, r_k) = g \left( \Gamma_V^{1}(q, r_1), \dots, \Gamma_V^{k}(q, r_k) \right)$.

Let $hash : Q \rightarrow \{0, 2(h+v)-1 \}$ be a function and $P_{hash}$ a set that contains elements $q \in Q$ for which $hash(q) = 0$.
A \textit{canonical success} with respect to a set $P_{hash}$ in
a random experiment defined by the protocol between $P^{(g)}$ and $S^{(g)}$,
is a situation when a first successfully verification query, made by $S^{(g)}$, is in $P_{hash}$,
and all previous hint or verification queries are not in $P_{hash}$.
%
% TODO size of the circuit
% TODO oracle calls ? non-rewinding?
% TODO is GEN probabilistic?
% TODO running time of Gen size of D
% TODO hash function does it has to be pairwise independent?
% TODO what is a random in hash function
%
\begin{theorem} {\textbf{(Security amplification for DWVP (non unifrom version)).}}
  Let $g: \{0,1\}^k \rightarrow \{0,1\}$ be a monotone function, and $hash:Q \rightarrow \{0,2(h+v)-1\}$ a function such that
  the probability of a \textit{canonical success} is at least $\frac{\varepsilon}{8\left(v + h\right)}$.
  If there exists a circuit $C$ that makes at most $v$ verification queries, $h$ hint queries,
  and succeeds with probability
  \begin{align}
    \Pr_{}[\Gamma_{V}^{(g)}( \langle P^{(g)},C \rangle_C ) = 1] \geq \Pr_{\mu \leftarrow \mu_{\delta}^{k}}[g(u) = 1] + \varepsilon ,
  \end{align}
  where the probability is over randomness of $P^{(g)}$,
  then there exists a probabilistic algorithm $Gen(C, g, \varepsilon, \delta, n, hash)$ which takes as input: a circuit $C$, a function $g$, a function $hash$,
  parameters $\varepsilon, \delta, n$, and produces a circuit $D$ of size at most $ size(C) \frac{6k}{\varepsilon} \log(\frac{6k}{\varepsilon}) $
  such that with high probability it satisfies
  \begin{align}
    \Pr_{}[\Gamma_V^{(1)} \left( \langle P^{(1)} ,D \rangle_D \right) = 1] \geq \frac{1}{8(h+v)} \left( \delta + \frac{\varepsilon}{6k} \right)
  \end{align}
  where the probability is taken over random coins of $P$.
  Additionally, the circuit D and the algorithm $Gen$ only require oracle access to functions $g$ and $hash$.
\end{theorem}
% DONE would it make more sense to consider situations when one have just a single verification query (if the proof works out also in this situation then it is a stronger result).


%%% Local Variables:
%%% mode: latex
%%% TeX-master: "master.tex"
%%% End:

\message{ !name(master.tex) !offset(-63) }

\end{document}

