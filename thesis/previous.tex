\label{st:previous_results}
In the last chapter we gave an overview of different types of cryptographic primitives that motivated studies of weakly verifiable puzzles.
The focus of this chapter is on giving the outline of the previous research.
We give a short overview of techniques used in the series of papers \cite{canetti2004hardness, Dodis:2009:SAI:1530441.1530450, DBLP:journals/corr/abs-1002-3534}
and provide some intuition and insight into the problem of hardness amplification for weakly verifiable puzzles.
First, we describe weakly verifiable puzzles studied in \cite{canetti2004hardness} that are neither interactive nor dynamic.
Then, in Section~\ref{section:dijk} we bring our focus on the dynamic non-interactive puzzles studied in \cite{Dodis:2009:SAI:1530441.1530450}.
Finally, in Section~\ref{section:iwvp} we give an overview of the results of Holenstein and Schoenebeck \cite{DBLP:journals/corr/abs-1002-3534}
where non-dynamic but interactive weakly verifiable puzzles are studied.
%
\section{Weakly Verifiable Puzzles}
\label{subsec:chs}
The notion of \textit{weakly verifiable puzzles} has been coined by Canetti, Halevi, and Steiner in the paper
\textit{Hardness amplification of weakly verifiable puzzles} \cite{canetti2004hardness}.
The puzzles considered there are non-dynamic and non-interactive.
Moreover, the number of verification queries is limited to one. This constitutes a special case of Definition \ref{def:dwvp}.
In this section we provide the definition of weakly verifiable puzzles (WVPs) and state the theorem of hardness amplification for WVPs
in a similar vein as in \cite{canetti2004hardness}.
Finally, we give the intuition behind the proof of this theorem. It is noteworthy that the main proof of this thesis,
presented in Chapter~\ref{ch:main_result}, uses many ideas of the work of Canetti, Halevi, and Steiner \cite{canetti2004hardness}.
%
\subsection{The definition}
We give the definition of WVP from \cite{canetti2004hardness}. However, we use the notation and terminology defined in this thesis.

\begin{definition}[Weakly Verifiable Puzzle, \cite{canetti2004hardness}]
  \label{def:wvp}
A \textit{weakly verifiable puzzle} is defined by a pair of polynomial time algorithms:
a probabilistic puzzle-generation algorithm $G$ and a deterministic verification algorithm $V$.
For a security parameter $k \in \N$ we write $G(1^k; \rho)$ to denote that $G$ takes as input $1^k$
and uses the randomness $\rho \in \{0,1\}^{*}$.
The algorithm $G$ outputs $p \in \{0,1\}^{*}$ and a check information $c \in \{0,1\}^{*}$.
The \textit{verifier} $V$ takes as input $p$, $c$, an answer $a \in \{0,1\}^{*}$ and outputs $b \in \{0,1\}$.

A \textit{solver} $S$ for $G$ is a probabilistic polynomial time algorithm that
takes as input $p$ and outputs $a$. We denote the randomness used by $S$ as $\pi \in \{0,1\}^{*}$
and define the \textit{success probability} of $S$ in solving a puzzle defined by $(V,G)$ as
\begin{align*}
  \underset{\substack{\rho \in \{0,1\}^{*}, \pi \in \{0,1\}^{*} \\ (p,c):=G(1^k; \rho) \\ a := S(p, \pi)}}{\Pr}\Big[ V(p,c,a) = 1\Big].
\end{align*}
We write $P := (G,V)$ to denote a weakly verifiable puzzle $P$ defined by algorithms $G$ and $V$.
\end{definition}

Let us argue that Definition \ref{def:wvp} is a special case of Definition \ref{def:dwvp}.
First, we note that for $G$ that takes as input $1^k$ the length of the randomness used is bounded by some polynomial $p(k)$.
For a fixed $k$, without loss of generality, we can represent $G(1^k; \rho)$ as a probabilistic circuit of polynomial size
that takes as input only the randomness $\rho$.
In Definition \ref{def:wvp} a verification algorithm $V$ takes as input $p$, $c$, $a$.
Again, without loss of generality, we can assume that bitstrings $p$ and $c$ are hard-coded
in the circuit $\Gamma_V$ from Definition \ref{def:dwvp}.
Furthermore, in Definition \ref{def:wvp} $|Q| = 1$ and thus, it is not necessary to pass $q$ as the parameter to $\Gamma_V$.
Hence, for fixed $p$ and $c$ the algorithm $V$ corresponds to $\Gamma_V$.
The puzzles considered in Definition \ref{def:wvp} are non--dynamic.
Thus, there is no element corresponding to the hint circuit $\Gamma_H$, and without loss of generality, we can assume that
$\Gamma_H$ outputs $\bot$ for every query.
Finally, we note that the puzzles described in Definition~\ref{def:wvp} are non--interactive.

\subsection{The hardness amplification}
We will now give the definition of the $n$-wise direct product of weakly verifiable puzzles.
%
\begin{definition}[$n$-wise direct product of weakly verifiable puzzles \cite{canetti2004hardness}.]
  \label{def:n-fold-rep}
  Let $n~\in~\N$ and a weakly verifiable puzzle $P = (G,V)$ be fixed.
  We define the $n$-wise direct product of $P$ as a weakly verifiable puzzle where the puzzle-generation algorithm
  $G^{(n)}$ takes as input $1^{k \cdot n}$, uses the randomness $\rho \in \{0,1\}^{*}$
  and outputs tuples $p^{(n)} := (p_1, \dotsc, p_n) \in \{0,1\}^{*}$ and $c^{(n)} := (c_1, \dotsc, c_k) \in \{0,1\}^{*}$
  where for each $1 \leq i \leq n$ a pair $(p_i, c_i)$ is an independent instance of a weakly verifiable puzzle defined by $G$ and $V$ with the security parameter $k$.
  Finally, the verification algorithm $V^{(n)}$ takes as input $p^{(n)}$, $c^{(n)}$, an answer $a^{(n)}$, and outputs $b \in \{0,1\}$
  such that $b = 1$ if and only if for all $1 \leq i \leq n$ it holds $V(p_i, c_i, a_i) = 1$.
 \end{definition}
%
Let us give additional notation and terminology. We write $P^{(n)} := (G^{(n)}, V^{(n)})$ to denote the $n$-wise direct product of $P$.
For $P^{(n) } := (G^{(n)},V^{(n)})$ we say \textit{puzzle on the $i$-th coordinate} to refer to the $i$-th puzzle of the $n$-wise direct product of $P$
(this puzzle corresponds to the one generated by $G^{(n)}_i$,  $V^{(n)}_i$).

The $n$-wise direct product of WVP is solved successfully if and only if all $n$ puzzles are solved successfully.
In contrast, in Definition~\ref{def:k_wise_direct_product} we are interested in a more general situation where a monotone function $g: \{0,1\}^{n} \rightarrow \{0,1\}$
is used to decide which coordinates of the $n$-wise direct product of puzzles have to be solved successfully.
Clearly, we can assume that $g$ is such that the puzzles on all coordinates have to be solved successfully
which matches the case considered in Definition \ref{def:n-fold-rep}.

The main theorem proved in \cite{canetti2004hardness} states that it is possible to turn a good solver for $P^{(n)}$
into a good solver for $P$.
%
\begin{theorem}[Hardness amplification for weakly verifiable puzzles \cite{canetti2004hardness}.]
  \label{thm:wvp_chs}
Let $n,q \in \N$ and $\delta: \N \rightarrow (0,1)$ be an efficiently computable function.
Moreover, let $P = (G,V)$ be a weakly verifiable puzzle. We denote the running time of the
puzzle-generation algorithm $G$ by $T_G$ and of the verification algorithm $V$ by $T_V$.
If $S^{(n)}$ is a solver for the $n$--wise direct product of $P$ that success probability is at least $\delta^{n}$
and the running time is $T$, then there exists a solver $S$ for $P$ with oracle access to $S^{(n)}$ that success
probability is at least $\delta(1-\frac{1}{q})$ and the running time is $O\Big(\frac{nq^3}{\delta^{2n-1}}(T + nT_G + nT_V)\Big)$.
\end{theorem}
%
The parameter $q$ is introduced as it is not possible to achieve
the perfect hardness amplification. We note that in the analysis of
the running time of the CHS-solver we explicitly take into account the time needed for the oracle calls to $S^{(n)}$, $V$, $G$.

Let $S^{(n)}$ be a solver for $P^{(n)}$ that success probability is at least $\delta^{n}$.
The following algorithm CHS-solver has oracle access to $S^{(n)}$ and
solves a puzzle defined by $P$ with probability at least $\delta(1  - \frac{1}{q})$.

We denote by $p \in \{0,1\}^{*}$ the output of $G$, which is also the input taken by the CHS-solver.
To make the notation shorter in the following code excerpts we do not write the randomness used by $G$ explicitly.

\begin{codeblock}
  \textbf{Algorithm:} $\text{CHS-solver}^{S^{(n)},V,G}(p, n, k, q, \delta)$
  \medskip\hrule
  \textbf{Oracle:} A solver $S^{(n)}$ for $P^{(n)}$, a verification algorithm $V$ for $P$, a puzzle-generation algorithm $G$ for $P$.\\
  \textbf{Input:}  A bistring $p \in \{0,1\}^{*}$, parameters $n, k, q, \delta$.
  \medskip\hrule
  $\mathit{prefix} := \emptyset$\\
  \For $i := 1$ \To $n\!-\!1$ \Do \\
  \IndI $p^* := \mathit{ExtendPrefix}^{S^{(n)}, V, G}(\mathit{prefix}, i, n, k, q, \delta)$\\
  \IndI \If $p^* = \bot$ \Then \Return $\mathit{OnlinePhase}^{S^{(n)}, V, G}(\mathit{prefix}, p, i, n, k, q, \delta)$ \\
  \IndI \Else $\mathit{prefix} := \mathit{prefix} \circ p^*$\\
  $ a^{(n)} := S^{(n)}(\mathit{prefix} \circ p)$ \\
  \Return $a_n$
\end{codeblock}
%
\begin{codeblock}
  \textbf{Algorithm:} $\mathit{OnlinePhase^{S^{(n)}, V, G}(\mathit{prefix}, p, i, n, k, q, \delta)}$
  \medskip \hrule
  \textbf{Oracle:} A solver algorithm $S^{(n)}$ for $P^{(n)}$, a puzzle-generation algorithm $G$ for $P$, a~verification algorithm $V$ for~$P$.\\
  \textbf{Input:} A $(v-1)$--tuple of bitstrings $\mathit{prefix}$, a bitstring $p \in \{0,1\}^{*}$, \\ parameters $v, n, k, q, \delta$.
  \medskip\hrule
  \Repeat $\Big\lceil\frac{6q \ln (6q)}{\delta^{n-i+1}}\Big\rceil$ times \\
  \IndI $((p_{i+1}, \dotsc, p_{n}),(c_{i+1}, \dots, c_n)) := G^{(n-i-1)}(1^k)$\\
  \IndI $a^{(n)} := S^{(n)}(\mathit{prefix}, p, p_{i+1}, \dotsc, p_n)$\\
  \IndI \If $\forall_{i+1 \leq j \leq n} V(p_j, c_j, a_j) = 1$ \Then \Return $a_i$\\
  \Return $\bot$
\end{codeblock}
%
\begin{codeblock}
  \textbf{Algorithm:} $\mathit{ExtendPrefix^{S^{(n)}, V, G}(prefix, i, n, k, q, \delta)}$
  \medskip \hrule
  \textbf{Oracle:} A solver algorithm $S^{(n)}$ for $P^{(n)}$, a puzzle-generation algorithm $G$ for $P$, a~verification algorithm $V$ for~$P$.\\
  \textbf{Input:} A $(i-1)$--tuple of puzzles $\mathit{prefix}$, parameters $i, n, k, q, \delta$.
  \medskip\hrule
  \Repeat $\Big\lceil \frac{6q}{\delta^{n-v+1}} \ln (\frac{18qn}{\delta}) \Big\rceil$ times \\
  \IndI $(p^*, c^*) := G(1^k) $\\
  \IndI $\bar{\nu}_i := \mathit{EstimateResSuccProb}^{S^{(n)},G,V}(\mathit{prefix} \circ p^*, i, n, k, q, \delta)$\\
  \IndI \If $\bar{\nu}_i \geq \delta^{n-i}$ \Then \Return $p^*$ \\
  \Return $\bot$
\end{codeblock}
%
\begin{codeblock}
  \textbf{Algorithm:} $\mathit{EstimateResSuccProb}^{S^{(n)},V, G}(\mathit{prefix}, i, n, k, q, \delta)$
  \medskip \hrule
  \textbf{Oracle:} A solver algorithm for $P^{(n)}$, a verification algorithm $V$ for $P$, a puzzle-generation algorithm $G$ for~$P$\\
  \textbf{Input:} A $i$--tuple of puzzles $\mathit{prefix}$, parameters $i, n, k, q, \delta$.
  \medskip\hrule
  $successes := 0$ \\
  \Repeat $M := \Big\lceil \frac{84q^2}{\delta^{n-i}} \ln \Big(\frac{18qn \cdot N_i}{\delta} \Big) \Big\rceil$ times \\
  \IndI $((p_{i+1}, \dotsc, p_n), (c_{i+1}, \dotsc, c_n)) := G^{(n-i)}(1^k)$\\
  \IndI $a^{(n)} := A(\mathit{prefix}, p_{i+1}, \dotsc, p_{n})$\\
  \IndI \If $\forall_{i + 1\leq j \leq n} : V(p_j, c_j, a_j) = 1$ \Then $\mathit{successes := successes + 1}$ \\
  \Return $successes / M$
\end{codeblock}
%
A full proof of Theorem \ref{thm:wvp_chs} is presented in \cite{canetti2004hardness}.
We limit ourselves to providing the intuition why the CHS-solver transforms a good solver
for the $n$--wise direct product of $P$ into a good solver for $P$.

Let us consider the $n$-wise direct product of $P$, and for simplicity a deterministic solver $S^{(n)}$ for $P^{(n)} := (G^{(n)}, V^{(n)})$.
Furthermore, we write $p^{(n)}, c^{(n)}$ to denote the output of $G^{(n)}$.
We define a matrix $M$ as follows. The columns of $M$ are labeled with all possible bitstrings $p_1$
whereas the rows are labeled with all possible tuples $(p_2, \dotsc, p_n)$ output by $G^{(n)}$ when executed with different randomness.
A cell of $M$ contains a binary $n$-tuple such that the $i$-th bit equals $1$ if and only if $V_i(p_i, c_i, a_i) = 1$ where
 $a^{(n)} := S^{(n)}(p^{(n)})$ and $p^{(n)}$ is a tuple of bitstrings inferred by a column and a row of the cell.
We make the following observation.
%
\begin{observation}
\label{obs:wvp_matrix}
Let $S^{(n)}$ be a deterministic polynomial time solver for $P^{(n)}$ that success probability is at least $\delta^{n}$.
Then, the matrix $M$ defined as above has either a column with a $\delta^{(n-1)}$ fraction of cells that are $1^n$ vectors, or
a conditional probability that a cell is of the form $1^n$ given that the last $(n-1)$ bits of this cell are equal $1$ is at least $\delta$.
\end{observation}
%
Let us explain, at least intuitively using Observation \ref{obs:wvp_matrix}, how the CHS-solver can be used to solve a puzzle defined by $P$
with substantial probability given oracle access to $S^{(n)}$.
The CHS-solver starts with the first position and tries to fix a bitstring $p^*$ on this position such that the success probability of $S^{(n)}$ on the remaining $(n-1)$
position is at least $\delta^{(n-1)}$. If it is possible to find $p^*$ such that this condition is satisfied, then we fix $p^*$
on this position and repeat the whole procedure again in the consecutive iteration for the next position.
If the $\mathit{CHS\text{--}solver}$ fails to find a bitstring $p^*$, then we assume that there is no column of $M$ that contains a $\delta^{(n-1)}$ fraction
of cells that are of the form $1^n$. We use Observation~\ref{obs:wvp_matrix} to conclude that the conditional probability of
solving the first puzzle given that all puzzles on the remaining position are solved successfully is at least~$\delta$.
We place the input puzzle $p$ on this position and note that the remaining puzzles are generated by the CHS-solver.
Thus, it is possible to efficiently verify whether these puzzles are successful solved by $S^{(n)}$.

Obviously, the CHS-solver can still fail. First, it may happen that it does not find a column
with a high fraction of puzzles that are solved successfully, although such a column exists.
Secondly, we cannot exclude a situation where no such column exists, but the algorithm fails to find a cell such that last $(n\!-\!1)$ bits are 1.
Finally, it is also possible that the estimate returned by $\mathit{EstimateResSuccProb}$ is incorrect.

It is possible to show that all these events happen with small probability such that
the success probability of the CHS-solver is at least $\delta(1\!-\!\frac{1}{q})$ almost surely.

In Chapter \ref{ch:main_result} we study a more general class of puzzles that are not only weakly verifiable but also dynamic and interactive.
Furthermore, we allow a more general situation where a solver successfully solves the $n$-wise direct product of puzzles
although it succeeds only on a subset of coordinates of the $n$-wise direct product of $P$.
It turns out that it is possible to use a similar technique of fixing puzzles on consecutive positions of the $n$-wise direct product of
puzzles to prove hardness amplification in this more general setting.
%
\section{Dynamic Weakly Verifiable Puzzles}
\label{section:dijk}
Some of the cryptographic primitives presented in Chapter~\ref{ch:examples_wvcp}
are not only weakly verifiable but also dynamic (MAC and SIG). This type of puzzles is defined and studied in \cite{Dodis:2009:SAI:1530441.1530450}.
We give a short overview of this work and state the definition of a \textit{dynamic weakly verifiable puzzle} (DWVP) that closely follows
the one included in \cite{Dodis:2009:SAI:1530441.1530450}.
Finally, we describe important parts of the proof of hardness amplification for DWVPs.

\subsection{The definition}
\begin{definition}[Dynamic Weakly Verifiable Puzzle.]
  \label{def:dwvp_dodis}
  A \textit{dynamic weakly verifiable puzzle} is defined by a distribution $\cD$ on pairs $(x, \alpha)$ where
  $\alpha$ is a secret information and $x$ is a bitstring.
  Furthermore, we consider a set $\cQ$ (for some set of indices $\cQ$) and a probabilistic polynomial time computable verification relation $R$ such that
  $R(\alpha, q, r) = 1$ if and only if $r \in \{0,1\}^{*}$ is a correct answer to $q \in \cQ$
  on the set determined by $\alpha$. Finally, let $H$ be a probabilistic polynomial time computable \textit{hint} function
  that on input $\alpha$, $q$ outputs a bitstring $\{0,1\}^{*}$.

  A solver $S$ takes as input $x$ and can make hint queries on $q \in \cQ$ which are answered using $H(\alpha, q)$ and verification
  queries $(q,r)$ answered by means of $R(\alpha, q, r)$.
  We say that $S$ succeeds if and only if it makes a verification query on $(q,r)$ such that
  $R(\alpha,q,r) = 1$ and it has not previously asked for a hint query on this $q$.
  We write $P := (\cD, R, H)$ to denote a DWVP with the distribution $\cD$ and $R$, $H$ being a verification and hint relations, respectively.
\end{definition}
%
The above definition is a special case of Definition \ref{def:dwvp}.
To see this we observe that instead of considering a distribution $\cD$ on pairs $(x,\alpha)$
we can use a problem poser that outputs circuits $\Gamma_H$ and $\Gamma_V$ with hard-coded $\alpha$ that correspond to $H$ and $V$, respectively.
It is clear that the solver $S$ from Definition \ref{def:dwvp_dodis} can be turned into a family of probabilistic polynomial size circuits
with oracle access to $\Gamma_H$ and $\Gamma_V$. Furthermore, in the first phase, which is non-interactive,
a bitstring $x$ is sent by the problem poser to the solver.

\subsection{The hardness amplification theorem}
\begin{definition}[$n$-wise direct product of DWVPs \cite{Dodis:2009:SAI:1530441.1530450}.]
  \label{def:n-wdp-dwvp}
For a dynamic weakly verifiable puzzle $P~:=~(\cD, R, H)$ we define the $n$-wise direct product of $P$
as a DWVP with a distribution $\cD^{(n)}$ on tuples $(x_1, \alpha_1), \dotsc, (x_n, \alpha_n)$ where
each $(x_i, \alpha_i)$ is drawn from the distribution $\cD$.
Furthermore, the hint relation is defined by $H^{(n)}(q, \alpha_1, \dotsc, \alpha_n) := (H(\alpha_1, q), \dotsc, H(\alpha_n, q))$ and
the verification relation $R^{(n)}(\alpha_1, \dotsc, \alpha_n, r_1, \dotsc, r_n, q)$ evaluates to $1$ if and only if
for at least $n - (1 - \gamma)\delta n$ of bitstrings $r_1, \dotsc, r_n $ it holds $R(\alpha_i, q, r_i)~=~1$ where $0~\leq~\gamma,\delta~\leq~1$.
%
\end{definition}
We write $P^{(n)} := (D^{(n)}, H^{(n)}, R^{(n)})$ to denote the $n$-wise direct product of $P$.

In contrast to defined in the previous section the $n$-wise direct product of WVP, Definition \ref{def:n-wdp-dwvp}
is more general in the sense that it is enough if the solver succeeds only on a fraction of coordinates.

Dynamic weakly verifiable puzzles generalize games of breaking security of message authentication codes and public key signature schemes.
In the case of MAC $x$ is an empty bitstring. For the public key signature schemes $x$ is a public key.

We write $(\cH_{\mathit{hint}}, \cV_{\mathit{verif}}) \leftarrow S(x; \delta)$ to denote
the execution of $S$ with the input $x \in \{0,1\}^{*}$ and using the randomness $\delta \in \{0,1\}^{*}$
where $\cH_{hint}$ is the set of all hint queries asked by $S$, and $\cV_{verif}$ is
the set of pairs $(q,r)$ of all verification queries asked in the execution of $S$.

With no loss of generality, we make the assumption that once $S$
made a successful verification query it does not ask any further hint and verification queries.
We define the \textit{success probability} of a solver $S$ as
\begin{align*}
  \underset{\substack{\delta \in \{0,1\}^{*} \\(x,\alpha) \leftarrow \cD \\ (\cH_{hint}, \cV_{verif}) \leftarrow S(x;\delta))}}
  {\Pr}\big[\exists (q,r) \in \cV_{verif} : q \notin \cH_{hint} \land R(\alpha, q, r) = 1 \big]
\end{align*}

\begin{theorem}\textbf{(Hardness amplification for dynamic weakly verifiable puzzles \cite{Dodis:2009:SAI:1530441.1530450}.)}
\label{lemma:dwvp}
Let $S^{(n)}$ be a probabilistic algorithm for $P^{(n)}$ that succeeds with
probability at least $\epsilon$, where $\epsilon \geq (800/\gamma\delta) \cdot (h+v) \cdot e^{-\gamma^2\delta n/40}$, and $h$ and $v$
denote the number of hint and verification queries asked by $S^{(n)}$, respectively.
There exists a probabilistic algorithm $S$ that solves a puzzle defined by $P$ with probability at least $1-\delta$.
Furthermore, $S$ makes $O(h(h+v)/\epsilon) \cdot \log(1/\gamma\delta)$ hint queries and at most one verification query.
The running time of $S$ is polynomial in $h,v,\frac{1}{\epsilon}, t, \omega, \log(1/\gamma\delta)$ where
$\omega$ is the time needed to ask a single hint query.
\end{theorem}

It is worth seeing why the approach presented in the previous section that works well for the direct product of WVPs
cannot be applied for the direct product of DWVPs (moving aside for a moment the issue of solving only a fraction of puzzles successfully).
Loosely speaking, for DWVP the algorithm CHS-solver breaks in the $\mathit{OnlinePhase}$ where the solver $S^{(n)}$ can be called multiple times.
It is possible that in one of these calls $S^{(n)}$ asks a hint query on $q$
which prevents in one of the later runs a verification query $(q,r)$ to succeed.
The fact that a hint query on $q$ has been asked before makes it impossible to make a successful verification query on this $q$.
Thus, we cannot dismiss a situation where the success probability of $S^{(n)}$ decreases with the number of iterations.

The solution proposed in \cite{Dodis:2009:SAI:1530441.1530450} is to partition the set $\cQ$ into a set of \textit{attacking queries} $\cQ_{\mathit{attack}}$
and a set of \textit{advice queries} $\cQ_{\mathit{adv}}$. The idea is to allow a solver for the direct product to ask hint
queries only on $q \in \cQ_{\mathit{adv}}$, and to halt the execution whenever a hint query is asked on $q \in \cQ_{\mathit{attack}}$.

More formally, for a function $\hash:\cQ \rightarrow \{0,1,\dotsc, 2(h+v)\!-\!1 \}$
we define $\cQ_{\mathit{attack}} := \{q \in \cQ : \hash(q) = 0 \}$ and $\cQ_{adv} := \{q \in \cQ: \hash(q) \neq 0\}$
It is possible, for a fixed solver $S$ that asks at most $h$ hint queries and $v$ verification queries,
to find a function $hash: \cQ \rightarrow \{0,1,\dotsc 2(h+v)-1\}$ such that the success probability of $S$ with respect to
$\cQ_{\mathit{attack}}$ and $\cQ_{\mathit{adv}}$ partitioned using $\hash$ is multiplied by $\frac{1}{8(h+v)}$.
If $h$ and $v$ are not too big, then the success probability of $S$ can be still substantial.
\begin{lemma}
\textbf{(Success probability in solving a dynamic weakly verifiable puzzle with respect to a function hash \cite{Dodis:2009:SAI:1530441.1530450}).}
  \label{lemma:hash_function_previous}
Let $S$ be a solver for DWVP which success probability is at least $\delta$, the running time is at most $t$,
and the number of hint and verification queries is at most $h$ and $v$, respectively.
There exists a probabilistic algorithm that runs in time polynomial in $h,v,\frac{1}{\delta},t$
that outputs a function $\hash : \cQ \rightarrow \{0,1, \dotsc, 2(h+v)-1\}$
that partitions $\cQ$ into $\cQ_{attack}$ and $\cQ_{adv}$ such that
with probability at least $\frac{\delta}{8(h+v)}$ the first successful verification query $(q',a)$ asked by $S$ is such that $q' \in \cQ_{attack}$
and all previous hint and verification queries have been asked on $q \in \cQ_{adv}$.
\end{lemma}
A function $\hash$ can be found by means of a natural sampling technique.
We follow exactly the same approach of partitioning $\cQ$ in Section \ref{st:domain_partition}.

Let $H_{\alpha}(q)$ denote a polynomial time probabilistic algorithm that takes as input $q$,
has hard-coded $\alpha$ and outputs $H(\alpha, q)$.
Similarly, we use $R_{\alpha}(q,r)$ to denote a polynomial time probabilistic algorithm that computes $R(\alpha, q, r)$ and has hard-coded bitstring $\alpha$.
The algorithm DWVP-solver has oracle oracle access to $S^{(n)}$, $R_{\alpha}$ and $H_{\alpha}$ as well as a function $\hash$ as in Lemma \ref{lemma:hash_function_previous}.
%
\begin{codeblock}
  \textbf{Algorithm:} $\text{DWVP-solver}^{S^{(n)}, \hash, H_{\alpha}^{(n)}, R_{\alpha}^{(n)}}(x)$
  \medskip
  \hrule
  \textbf{Oracle:}  A solver $S^{(n)}$ for $P^{(n)}$, algorithms $R_{\alpha}$ and $H_{\alpha}$, a function $hash : \cQ \rightarrow \{0,1, \dotsc, 2(h+v)-1\}$.\\
  \textbf{Input:} A bistring $x \in \{0,1\}^{*}$.
  \medskip\hrule
  \Repeat at most $O(\frac{h+v}{\epsilon} \cdot \log(\frac{1}{\gamma\delta}))$ times \\
  \IndI Let $i \xleftarrow{\$} \{1, \dotsc, n\}$ be a position for $x$.\\
  \IndI Generate $(x_1, \alpha_1), \dotsc, (x_{i-1}, \alpha_{i-1}), (x_{i+1}, \alpha_{i+1}), \dotsc, (x_n, \alpha_n)$ \\
  \IndI using $(n-1)$ calls to $P$ each time with fresh randomness.\\
  \IndI \Run $S^{(n)}(x_1, \dotsc, x_{i-1}, x, x_{i+1}, \dotsc, x_n)$\\
  \IndII \If $S^{(n)}$ asks a hint query on $q$ \Then \\
  \IndIII \If $hash(q) \neq 0$ \Then abort current run of $S^{(n)}$\\
  \IndIII Ask a verification query $r := H(q)$\\
  \IndIII Let $(r_1, \dotsc, r_{i-1}, r_{i+1}, \dotsc, r_{n})$ be hints for query $q$ for puzzle\\
  \IndIII sets $(x_1, \dotsc, x_{i-1}, x_{i+1}, x_n)$\\
  \IndIII Answer the hint query of $S^{(n)}$ using $(r_1, \dots, r_{i-1}, r, r_{i+1}, r_n)$\\
  \IndII \If $S^{(n)}$ asks a verification query $(q, r_1, \dots, r_n)$ \Then \\
  \IndIII \If $hash(q) = 0$ \Then answer the query with $0$\\
  \IndIII Let $m := |j: V(q,r_j) = 1, j \neq i|$\\
  \IndIII \If $m \geq n - n(1-\gamma)\delta$ \Then \\
  \IndIIII make a verification query $(q, r_i)$ and halt.\\
  \IndIII \Else with probability $\rho^{m - n(1-\gamma)\delta}$ ask a verification query \\
  \IndIIII $(q, r_i)$ and halt. \\
  \IndIII Halt the current run of $S^{(n)}$ and go to the next iteration.\\
  \Return $\bot$
\end{codeblock}

In each iteration of the DWVP-solver the position for the input puzzle is chosen uniformly at random.
The remaining $(n-1)$ puzzles are generated by the algorithm, thus it is possible to answer
all hint and verification queries for these puzzles.
We assume that $\hash$ is such that the success probability of $S^{(n)}$ with respect to $\hash$ is at least $\frac{\delta}{8(h+v)}$.
The DWVP-solver calls $S^{(n)}$ multiple times, but the function $\hash$ is used
to partition the query domain into $\cQ_{\mathit{attack}}$ and $\cQ_{\mathit{adv}}$.
If a hint query is asked on $q$ such that $hash(q) = 0$ then the current execution of $S^{(n)}$
is aborted, and the DWVP-solver goes to the next iteration.
In this way, we ensure that the DWVP-solver never asks a hint query that could prevent a verification query from succeeding.
If a verification query is asked on $q$ such that $hash(q) \neq 0$ we answer this query with $0$.

Finally, in the case where $S^{(n)}$ asks a verification query on $q$ such that $hash(q) = 0$,
a \textit{soft decision system} is used to decide whether to ask a verification query.
The idea is that if there are many puzzles among the ones generated by the algorithm that are solved successfully,
then it is likely that also the input puzzle is solved successfully.
We discount $\gamma\delta n$ to take into account that not all puzzles have to be solved successfully.
The detail calculations provided in \cite{Dodis:2009:SAI:1530441.1530450} show that this approach
yields a demanded result.

In Chapter \ref{ch:main_result} we consider a weakly verifiable puzzles that are not only dynamic but also interactive.
We use a very similar technique to partition domain $\cQ$ into advice and attacking queries.
Instead of the requirement to succeed on at least a fraction of puzzles we consider a more general approach where
an arbitrary monotone function $g : \{0,1\}^{n} \rightarrow \{0,1\}$ is used to
determine on which coordinates the solver has to succeed in order to successfully solve the $k$-wise direct product of puzzles.

\section{Interactive Weakly Verifiable Puzzles}
\label{section:iwvp}
Hardness amplification for interactive but non-dynamic weakly verifiable puzzles
has been studied by Holenstein and Schoenebeck in \cite{DBLP:journals/corr/abs-1002-3534}.
We will give now an overview of this work and compare it with our approach.

\subsection{The definition}
\begin{definition}[Interactive Weakly Verifiable Puzzle \cite{DBLP:journals/corr/abs-1002-3534}.]
  \label{def:iwvp}
An \textit{interactive weakly verifiable puzzle} is defined by a protocol given by two probabilistic algorithms $P$ and $S$.
The algorithm $P$ is called the problem poser and produces as output a verification circuit $\Gamma$.
The algorithm $S$ called the problem solver produces no output.
Furthermore, the \textit{success probability} of the algorithm $S^*$ in solving an interactive weakly verifiable puzzle defined by $(P,S)$ is:
\begin{align*}
  \underset{\substack{\rho, \pi \\ \Gamma^{(g)} := \langle P(\rho), S^*(\pi) \rangle_{P}}}{\Pr}\Big[\Gamma^{(g)}(\langle P(\rho),S^*(\pi) \rangle_{S^*}) = 1 \Big].
\end{align*}
\end{definition}
It is not hard to see that Definition \ref{def:iwvp} is a special case of Definition \ref{def:dwvp}.
The puzzles in Definition \ref{def:iwvp} are non-dynamic, thus $|Q| = 1$. Furthermore, we can assume that $\Gamma_H$ always outputs $\bot$.
The number of verification queries is limited to at most one.

Similarly as in the previous sections, we define the $k$-wise direct product of puzzles.
\begin{definition}\textbf{($\boldsymbol{k}$-wise direct product of interactive weakly verifiable puzzles)}
Let $g: \{0,1\}^{k} \rightarrow \{0,1\}$ be a monotone function and $(P,S)$ be a fixed interactive weakly verifiable puzzle.
The $k$-wise direct product of $(P,S)$ denoted by $(P^{(g)}, S^{(g)})$ is an interactive weakly verifiable puzzle where the problem poser and the solver
sequentially interact in $k$ rounds. In each round $(P,S)$ is used to produce an instance of the interactive weakly verifiable puzzle.
As the result circuits $\Gamma^{(1)}, \dotsc, \Gamma^{(k)}$ for $P$ are generated.
Finally, $P^{(g)}$ outputs the circuit $\Gamma^{(g)}(y_1, \dotsc, y_k) := g(\Gamma^{(1)}(y_1), \dotsc, \Gamma^{(k)}(y_k))$.
\end{definition}

\subsection{The hardness amplification theorem}
\begin{theorem}\textbf{(Hardness amplification for interactive weakly verifiable puzzles \cite{DBLP:journals/corr/abs-1002-3534}.)}
There exists an algorithm $\mathit{Gen}(C,g,\epsilon, \delta, n)$ which takes as input a solver circuit $C$ for the $k$-wise
direct product of $(P,S)$, a monotone function $g: \{0,1\}^{*} \rightarrow \{0,1\}$, and parameters $\epsilon,\delta,n$.
The algorithm $\mathit{Gen}$ outputs a solver circuit $D$ for $P$ such that the following holds.
If $C$ is such that
\begin{align*}
\Pr\Big[\Gamma^{(g)}(\langle P^{(g)}, C \rangle_C) = 1\Big] \geq \Pr_{u \leftarrow \mu_{\delta}^{(k)}} \Big[ g(u) = 1 \Big] + \epsilon,
\end{align*}
where the probability is over the randomness of $P^{(g)}$ and $C$, then $D$ satisfies almost surely
\begin{align*}
  \Pr\Big[ \Gamma(\langle P, D\rangle_{D}) = 1\Big] \geq \delta + \frac{\epsilon}{6k},
\end{align*}
where the probability is over the randomness of $P$ and $D$.
Additionally, $\mathit{Gen}$ and $D$ only require oracle access to $g$ and $C$.
Furthermore, $\mathit{Size}(D) \leq \mathit{Size}(C) \cdot \frac{6k}{\epsilon} \log(\frac{6k}{\epsilon})$,
and the running time of Gen is polynomial in $k, \frac{1}{\epsilon}, n$ with oracle calls.
\end{theorem}

First, we notice that the above definition does not impose any restrictions on the time complexity of the poser and the solver.
We consider a general approach where $\mathit{Gen}$ is used to define a polynomial time reduction between a solver for the $k$-wise
direct product of puzzles to a solver for a single puzzle.
Furthermore, in the previous sections we considered solvers for the $k$-wise direct product that are compared with the algorithms that either
solve all puzzles (\cite{canetti2004hardness}) or allow a fraction of puzzles to be solved incorrectly (\cite{Dodis:2009:SAI:1530441.1530450}).
In the above definition a more general approach is considered where a binary monotone function $g$ is used.

In chapter \ref{ch:main_result} we use very similar approach as Holenstein and Schoenebeck.
The difference is that we give a hardness amplification proof for puzzles that are not only interactive but also dynamic.

% in Definition \ref{def:dwvp}, the puzzles studied by Holenstein and Schoenebeck
% are non-dynamic. Thus, only a verification circuit $\Gamma$ is generated and no hint circuit is ever used.
% \begin{todo}
%   \textbf{TODO:} Compare to the work of CHS i.e. what we use there to compare the puzzles
% \end{todo}

% In order to estimate how much better a solver circuit $C$ for the $n$-wise direct product performs when
% a puzzle on the first position is fixed a notion of a surplus $S_{\pi^*, b}$ is introduced:
% \begin{align*}
% S_{\pi^*, b} := \Pr_{\pi^{(k)}} [ c \in \cG_b | \pi_1 = \pi^* ] - \Pr_{u \leftarrow \mu_{\delta}^{k}} [u \in \cG_b],
% \end{align*}
% which intuitively tells us how much better a solver $C$ performs when a first puzzle is always solved correctly (case when $b = 1$)
% or is always solved incorrectly (when $b = 0$).
% Now we observe the following fact. If there exists a puzzle which is fixed on the first position for which the surplus is bigger than
% $(1 - \frac{1}{k})\epsilon$ then we can fixed a this first puzzle and inductively solve the problem for the $(k-1)$-direct product of puzzles.
% \begin{todo}
%   \textbf{TODO:} what do it mean
% \end{todo}
% On the other hand, if we there is no such puzzle to fix on the first position it means that when we fix the first bit of $g$ then
% the performance between the solver $C$ and an algorithm that solves puzzle on each position independently with probability $\delta$
% is similar. However, we know that when the first bit of a function $g$ is not fixed then the solver $C$ is better.
% Thus, we draw a conclusion that the puzzle on the first position has to be solved unusually often.

% In a case it would be possible to fix all $(k-1)$ puzzles except the last one, the proof become trivial as we know that a function
% $g$ with the first $k-1$ bits fixed is either the identity or a constant function.
% %TODO write why it is true in this case

% Thus, it is enough to show that if it is not possible to find an estimate that is low then
% if we place an input puzzle on this position and we can find remaining $k-1$ puzzle such that
% $c \in \cG_1 \setminus \cG_0$ then this puzzle is solved with substantial probability.
% The whole proof is given in \cite{DBLP:journals/corr/abs-1002-3534}, and requires some probability manipulations.

% %cite tell more what is used in our proof which technique do we use.
% Our proof of the hardness amplification for dynamic interactive weakly verifiable puzzles closely follows the one given in \cite{DBLP:journals/corr/abs-1002-3534}.
% \begin{todo}
%   \textbf{TODO:} Why do we consider fixing 0/1 on the first position
%   \textbf{TODO:} How the technique is generalized to approach of CHS \\
%   \textbf{TODO:} Explain why we compare to such a probability i.e. why we consider with $\mu$ \\
%   \textbf{TODO:} Why the requirement for $g$ being a monotone function is interesting \\
%   \textbf{TODO:} Give the intuition behind the proof. \\
%   \textbf{TODO:} Give a proof under the simplified assumptions? \\
%   1) The algorithm always output an answer \\
%   2) For every pi the surpluses $S_{\pi^*, 0}$ and $S_{\pi^*, 1}$ re less than $(1-\frac{1}{k})\epsilon$.\\
% \end{todo}

% Local Variables:
% mode: latex
% TeX-master: "thesis"
% End:
