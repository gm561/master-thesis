%
\subsection{Domain partitioning}
\label{st:domain_partition}
In this section we use the technique from \cite{dodis2009security}.
The set $\cQ$ is partitioned into a set of advice $\cQ_{adv}$ and attacking queries $\cQ_{attack}$.
The circuit $C$ can make hint queries only on $q \in \cQ_{adv}$ and successful verification queries on $q \in \cQ_{attack}$.
The idea is to avoid collisions between hint and verification queries even in the independent runs of $C$.
If $C$ tries to ask a hint query on $q \in \cQ_{attack}$, the execution of $C$ is aborted.
Similarly, all verification queries on $q \in \cQ_{adv}$ are answered as unsuccessful.
In this section we show that it is possible to find $\cQ_{attack}$ and $\cQ_{adv}$ such that
the success probability of $C$ with $Q$ partitioned is still substantial.

Let $\hash:\cQ\rightarrow\{0,1,\dots, 2(h\!+\!v)\!-\!1\}$, the idea is to partition $\cQ$ such that the set of preimages
of $0$ for $\hash$ contains $q \in \cQ$ on which $C$ is not allowed to ask hint queries,
and the first successful verification query $(q,y)$ of $C$ is such that $\hash(q) = 0$.
Therefore, if $C$ makes a verification query $(q,y)$ such that $\hash(q) = 0$, then we know that no hint query is ever asked on this $q$.

We denote the $i$-th query of $C$ by $q_i$ if it is a hint query and by $(q_i, y_i)$ if it is a verification query.
Let us define the following experiment $\CanonicalSuccess$ in which $\cQ$ is partitioned using a function $\hash$.
We say that a solver circuit \textit{succeeds} in the experiment $\CanonicalSuccess$
if it asks a successful verification query $(q_j, y_j)$ such that $\hash(q_j) = 0$,
and no hint query $q_i$ has been asked before $(q_j, y_j)$ such that $\hash(q_i) = 0$.
%
\begin{codeblock}
  \textbf{Experiment} $\CanonicalSuccess^{P, C, \hash}(\pi, \rho)$
  \medskip \hrule
  \textbf{Oracle:} A problem poser $P$, a solver circuit $C = (C_1, C_2)$ for $P$,\\
  \IndII a function $\hash: \cQ \rightarrow \{0, 1, \dotsc, 2(h+v) - 1\}$.\\
  \textbf{Input:}  Bitstrings $\pi \in \{0,1\}^n$, $\rho \in \{0,1\}^*$. \\
  \textbf{Output:} A bit $b \in \{0,1\}$.
  \medskip\hrule
  \Run $\langle P(\pi), C_1(\rho) \rangle$ \\
  \IndI $(\Gamma_V, \Gamma_H) := \langle P(\pi), C_1(\rho) \rangle_{P}$ \\
  \IndI $x := \langle P(\pi), C_1(\rho) \rangle_{\trans}$ \\ \\
  \Run $C_2^{\Gamma_V, \Gamma_H} (x, \rho)$ \\
  \IndI \If $C_2^{\Gamma_V, \Gamma_H} (x, \rho)$ does not succeed for any verification query \Then \\
  \IndII \Return $0$ \\
  \IndI Let $(q_j,y_j)$ be the first verification query such that $\Gamma_V(q_j, y_j) = 1$.\\
  \\
  \If $(\forall i < j :  \hash(q_i) \neq 0)$ \And $(\hash(q_j) = 0)$ \Then \\
  \IndI \Return 1\\
  \Else\\
  \IndI \Return 0
\end{codeblock}
%
We define the \textit{canonical success probability} of a solver circuit $C$ for $P$ with respect to a function $\hash$ as
\begin{align}
 \underset{\pi, \rho}{\Pr}[\CanonicalSuccess^{P, C, \hash}(\pi, \rho) = 1].
\end{align}
%
For fixed $\hash$ and $P$ a \textit{canonical success} of $C$ for bistrings $\pi$, $\rho$ is a situation where
$\CanonicalSuccess^{P, C, \hash}(\pi, \rho) = 1$.

We show now that if a solver circuit $C$ for $P$ often succeeds in the experiment $\Success$,
then there exists a function $\hash$ such that $C$ also often succeeds in the experiment $\CanonicalSuccess$
provided that the number of hint and verification queries is not too large.
%
\begin{lemma}[Success probability in solving a dynamic interactive weakly verifiable puzzle with respect to a function $\hash$]
\label{lemma:hash_function_probability}
For a fixed problem poser $P_n$ let $C$ be a solver for $P_n$ with the success probability at least $\gamma$,
asking at most $h$ hint queries and $v$ verification queries.
Moreover, let $\cH$ be a family of pairwise independent efficient hash functions $\cQ \rightarrow \{0,1, \dots,2(h+v)-1\}$.
There exists a probabilistic algorithm FindHash that takes as input parameters $\gamma$, $n$, $h$, $v$, and has oracle access to $C$ and $P_n$.
Furthermore, FindHash runs in time polynomial in $(h,v,\frac{1}{\gamma},n)$ and with high probability outputs a function $\hash \in \cH$
such that the canonical success probability of $C$ with respect to $\hash$ is at least $\frac{\gamma}{16(h+v)}$.
\end{lemma}
%
\begin{proof}[Proof of Lemma \ref{lemma:hash_function_probability}]
We fix a problem poser $P$ and a solver $C$ for $P$ in the whole proof of Lemma \ref{lemma:hash_function_probability}.
For $k,l \in \{1, \dotsc, (h+v)\}$ and $\alpha,\beta \in \{0,1,\dotsc,2(h+v)-1\}$ by the pairwise independence property,
we have that for every $q_k, q_l \in \cQ$ such that $q_k \neq q_l$ the following holds
\begin{align}
 \label{eq:hash_pr}
 \underset{\hash \leftarrow \cH}{\Pr}[\hash(q_k) = \alpha \mid \hash(q_l) = \beta] &=
 \underset{\hash \leftarrow \cH}{\Pr}[\hash(q_k) = \alpha] \notag\\ &= \frac{1}{2(h+v)}.
\end{align}
%
We write $\cP_{\Success}$ to denote a set containing all $(\pi,\rho)$ for which $\Success^{P, C}(\pi, \rho)~=~1$.
Let us fix $(\pi^*, \rho^*) \in \cP_{\mathit{Success}}$. We are interested in the probability over
a choice of $\hash$ of the event $\CanonicalSuccess^{P,C,\hash}(\pi^*, \rho^*) = 1$.
Let $(q_j, y_j)$ denote the first verification query such that $\Gamma_V(q_j, y_j) = 1$, we have
\begin{align}
  \label{ineq:prob_hash_success}
  \underset{\hash \leftarrow \cH}{\Pr}&\left[\CanonicalSuccess^{P,C,\hash}(\pi^*, \rho^*) = 1\right] \notag\\
  \IndII &\overset{\hphantom{\eqref{eq:hash_pr}}}{=} \underset{\hash \leftarrow \cH}{\Pr}\big[\hash(q_j) = 0 \land (\forall i < j : \hash(q_i) \neq 0)\big] \notag\\
  \IndII &\overset{\hphantom{\eqref{eq:hash_pr}}}{=} \underset{\hash \leftarrow \cH}{\Pr}\big[
  \forall i < j : \hash(q_i) \neq 0 \mid \hash(q_j) = 0\big] \underset{\hash \leftarrow \cH}{\Pr}[\hash(q_j) = 0] \notag\\
  \IndII &\overset{(\ref{eq:hash_pr})}{=} \frac{1}{2(h+v)}\left(1 -\underset{\hash \leftarrow \cH}{\Pr}[\exists i < j : \hash(q_i) = 0 \mid \hash(q_j) = 0] \right) \notag\\
  \IndII &\stackrel{\substack{\hphantom{\eqref{eq:hash_pr}}\\(*)}}{\geq} \frac{1}{2(h+v)} \left( 1 - \sum_{i < j} \underset{\hash \leftarrow \cH}{\Pr}[\hash(q_i) = 0 | \hash(q_j) = 0] \right) \notag\\
  \IndII &\stackrel{(\ref{eq:hash_pr})}{=} \frac{1}{2(h+v)} \left( 1 -  \sum_{i < j}\underset{\hash \leftarrow \cH}{\Pr}[\hash(q_i) = 0] \right) \notag\\
  \IndII &\overset{(\ref{eq:hash_pr})}{\geq} \frac{1}{4(h+v)},
\end{align}
where in $(*)$ we used the union bound.
Let us denote the set of those $(\pi,\rho)$ for which $\CanonicalSuccess^{P, C, \hash}(\pi, \rho) = 1$ by $\cP_{\mathit{Canonical}}$.
If for $\pi$, $\rho$ the circuit $C$ succeeds canonically, then for the same $\pi$, $\rho$ we also have $\Success^{P, C}(\pi, \rho) = 1$.
Hence, $\cP_{\mathit{Canonical}} \subseteq \cP_{\mathit{Success}}$, and we conclude
\begin{align}
  \label{ineq:hash_high_prob}
&\underset{\substack{\hash \leftarrow \cH \\ \pi, \rho}}{\Pr}\left[\CanonicalSuccess^{P, C, \hash}(\pi, \rho) = 1\right] \notag\\
&\IndI = \underset{\substack{\hash \leftarrow \cH \\ \pi, \rho}}{\Pr}\left[\CanonicalSuccess^{P,C,\hash}(\pi, \rho) = 1 \mid (\pi, \rho) \in \cP_{\mathit{Success}}\right]
\underset{\pi, \rho}{\Pr}[(\pi, \rho) \in \cP_{\mathit{Success}}] \notag\\
&\IndII + \underset{\substack{\hash \leftarrow \cH \\ \pi, \rho}}{\Pr}\left[\CanonicalSuccess^{P,C,\hash}(\pi, \rho) = 1 \mid (\pi, \rho) \notin \cP_{\mathit{Success}} \right]
\underset{\pi, \rho}{\Pr}[(\pi, \rho) \notin \cP_{\mathit{Success}}] \notag\\
%
&\IndI = \underset{\substack{\hash \leftarrow \cH \\ \pi, \rho}}{\Pr}\left[\CanonicalSuccess^{P,C,\hash}(\pi, \rho) = 1 \mid (\pi, \rho) \in \cP_{\mathit{Success}}\right]
\underset{\pi, \rho}{\Pr}[(\pi, \rho) \in \cP_{\mathit{Success}}] \notag\\
%
&\IndI \geq
\underset{\substack{\hash \leftarrow \cH \\ \pi, \rho}}{\Pr}\left[\CanonicalSuccess^{P,C,\hash}(\pi, \rho) = 1 \mid (\pi, \rho) \in \cP_{\mathit{Success}}\right] \cdot \gamma \notag\\
&\IndI =
\underset{(\pi, \rho) \in \cP_{\mathit{Success}}}
{\mathbb{E}}\left[\underset{\hash \leftarrow \cH}{\Pr}[\CanonicalSuccess^{P,C,\hash}(\pi, \rho) = 1] \right] \cdot \gamma \notag\\
&\hphantom{aaa}\stackrel{(\ref{ineq:prob_hash_success})}{\geq} \frac{\gamma}{4(h+v)}. %use phantom space to align text in the equation
\end{align}
%
\begin{codeblock}
  \textbf{Algorithm} $\text{FindHash}^{P,C}(\gamma, n, h, v)$
  \medskip\hrule
  \textbf{Oracle:} A problem poser $P$, a solver circuit $C$ for $P$.\\
  \textbf{Input:} Parameters $\gamma$, $n$. The number of hint queries $h$ and of verification queries $v$. \\
  \textbf{Output:} A function $\hash:\cQ \rightarrow \{0,1, \dots, 2(h+v)-1 \}$.
  \medskip\hrule
  \For $i := 1$ \To $32n(h+v)^2/\gamma^2$ \Do \\
  \IndI $\hash \leftarrow{} \cH$ \\
  \IndI $\mathit{count} := 0$ \\
  \IndI \For $j := 1$ \To $32n(h+v)^2/\gamma^2$ \Do \\
  \IndII $\pi \leftarrow \{0,1\}^{n} $\\
  \IndII $\rho \leftarrow \{0,1\}^*$ \\
  \IndII \If $\CanonicalSuccess^{P, C, \hash}(\pi, \rho) = 1$ \Then \\
  \IndIII $\mathit{count} := \mathit{count} + 1$\\
  \IndI \If $\mathit{count} \geq \frac{\gamma}{12(h+v)} \frac{32(h+v)^2}{\gamma^2}n$ \Then \\
  \IndII \Return $\hash$\\
  \Return $\bot$
\end{codeblock}
We show that FindHash chooses $\hash \in \cH$ such that the canonical success probability of $C$
with respect to $\hash$ is at least $\frac{\gamma}{16(h+v)}$ almost surely.
Let $\cH_{\Good}$ denote a family of functions $\hash \in \cH$ for which
\begin{align}
  \label{eq:hash_good}
\underset{\pi, \rho}{\Pr}\left[\CanonicalSuccess^{P, C, \hash}(\pi, \rho) = 1\right] \geq \frac{\gamma}{8(h+v)},
\end{align}
and $\cH_{\mathit{Bad}}$ be a family of functions $\hash \in \cH$ such that
\begin{align}
  \label{eq:hash_bad}
  \underset{\pi, \rho}{\Pr}\left[\CanonicalSuccess^{P, C, \hash}(\pi, \rho) = 1\right] \leq \frac{\gamma}{16(h+v)}.
\end{align}
Let $N$ denote the number of iterations of the inner loop of FindHash.
We consider a single iteration of the outer loop of FindHash in which $\hash$ is fixed.
Let us define independent and identically distributed binary random variables $X_1, \dots, X_{N}$ such that
\begin{align*}
  X_i =
  \begin{cases}
    1 & \text{if in the $i$-th iteration of the inner loop $\mathit{count}$ is increased}\\
    0 & \text{otherwise.}
  \end{cases}
\end{align*}
We turn now to the case where $\hash \in \cH_{\Bad}$ and show that it is unlikely that $\hash \in \cH_{\Bad}$
is returned by FindHash.
From \eqref{eq:hash_bad} it follows that $\mathbb{E}_{{\pi}, \rho}[X_i]~\leq~\frac{\gamma}{16(h+v)}$.
In the following inequalities \eqref{ineq:hash_bad_upper} and \eqref{ineq:hash_good_upper} in steps denoted
with $(*)$ we use the trivial facts $h + v \geq 1$ and $\gamma \leq 1$.
For any fixed $\hash \in \cH_{\Bad}$ using the Chernoff bound we get
\begin{align}
  \label{ineq:hash_bad_upper}
  \underset{\pi,\rho}{\Pr} \left[\frac{1}{N} \sum_{i=1}^{N} X_i \geq \frac{\gamma}{12(h+v)} \right]
  &\leq \underset{\pi, \rho}{\Pr}\left[\frac{1}{N} \sum_{i=1}^{N} X_i \geq \bigl(1 + \frac{1}{3}\bigr) \mathbb{E}[X_i]\right] \notag\\
  &\leq e^{-{\frac{\gamma}{16(h+v)}} N / 27} \leq e^{-\frac{2}{27}\frac{(h+v)}{\gamma}n} \stackrel{(*)}{\leq} e^{-\frac{2}{27}n}.
\end{align}
The probability that $\hash \in \cH_{\Good}$, when picked, is not returned amounts
\begin{align}
  \label{ineq:hash_good_upper}
  \underset{\pi, \rho}{\Pr}\left[\frac{1}{N} \sum_{i=1}^{N} X_i \leq \frac{\gamma}{12(h+v)}\right]
  &\leq \underset{\pi, \rho}{\Pr}\left[\frac{1}{N} \sum_{i=1}^{N} X_i \leq \bigl(1 - \frac{1}{3}\bigr)\mathbb{E}[X_i]\right] \notag\\
  &\leq e^{-{\frac{\gamma}{8(h+v)}} N / 18} \leq e^{-\frac{2}{9} \frac{(h+v)}{\gamma}n} \stackrel{(*)}{\leq} e^{-\frac{2}{9}n}
\end{align}
where we once more used the Chernoff bound.
We show now that the probability of picking $\hash \in \cH_{\mathit{Good}}$ is at least $\frac{\gamma}{8(h+v)}$.
To obtain a contradiction suppose that
\begin{align}
  \label{ass:hash_max_prob}
\underset{\hash \leftarrow \cH}{\Pr}[\hash \in \cH_{\mathit{Good}}] < \frac{\gamma}{8(h+v)}.
\end{align}
Using \eqref{ass:hash_max_prob} we conclude that it is possible to bound the probability of the~canonical success as follows
\begin{align*}
  &\underset{\substack{\hash \leftarrow \cH \\ \pi, \rho}}{\Pr}[\CanonicalSuccess^{P,C,\hash}(\pi, \rho) = 1] \\
  &\IndI = \underset{\substack{ \hash \leftarrow \cH \\ \pi, \rho}}{\Pr}[\CanonicalSuccess^{P,C,\hash}(\pi, \rho) = 1 \mid \hash \in \cH_{\mathit{Good}}]
  \underset{\hash \leftarrow \cH}{\Pr}[\hash \in \cH_{\mathit{Good}}] \\
  & \IndII + \underset{\substack{\hash \leftarrow \cH \\ \pi, \rho}}{\Pr}[\CanonicalSuccess^{P,C,\hash}(\pi, \rho) = 1 \mid \hash \notin \cH_{\mathit{Good}}]
  \underset{\hash \leftarrow \cH}{\Pr}[\hash \notin \cH_{\mathit{Good}}] \\
  & \IndI \leq \underset{\hash \leftarrow \cH}{\Pr}[\hash \in \cH_{\mathit{Good}}] +
  \underset{\substack{\hash \leftarrow \cH \\ \pi, \rho}}{\Pr}[\CanonicalSuccess^{P,C,\hash}(\pi, \rho) = 1 \mid \hash \notin \cH_{\mathit{Good}}] \\
  & \IndI  \stackrel{\substack{ (\ref{eq:hash_good}) \\ (\ref{ass:hash_max_prob})}}{<} \frac{\gamma}{8(h+v)} + \frac{\gamma}{8(h+v)} = \frac{\gamma}{4(h+v)},
\end{align*}
which contradicts (\ref{ineq:hash_high_prob}). Hence, we conclude that the probability of choosing $\hash \in \cH_{\mathit{Good}}$ amounts at least $\frac{\gamma}{8(h+v)}$.

We show that FindHash picks in one of its iteration $\hash \in \cH_{\mathit{Good}}$ almost surely.
Let $K$ denote the random variable that takes value equal to the number of iterations of
the outer loop of FindHash and $Y_i$ be a random variable for the event
that in the $i$-th iteration of the outer loop $\hash \notin \cH_{\mathit{Good}}$ is picked.
We use $\underset{\hash \leftarrow \cH}{\Pr}[\hash \in \cH_{\textit{Good}}] \geq \frac{\gamma}{8(h+v)}$ and $K \leq \frac{32(h+v)^2}{\gamma^2}n$ to conclude
\begin{align*}
  \underset{\hash \leftarrow \cH}{\Pr}\Bigl[ \bigcap_{1 \leq i \leq K} Y_i \Bigr] \leq \left(1 - \frac{\gamma}{8(h+v)}\right)^{\frac{32(h+v)^2}{\gamma^2}n}
    &\leq e^{-\frac{\gamma}{8(h+v)} \frac{32(h+v)^2}{\gamma^2}n} \\
    &\leq e^{-\frac{4(h+v)}{\gamma}n} \leq e^{-n}.
\end{align*}
It is clear that the running time of FindHash is $\mathit{poly}(n,h,v,\gamma)$ with oracle calls.
This finishes the proof of Lemma \ref{lemma:hash_function_probability}.
\end{proof}

%%% Local Variables:
%%% mode: latex
%%% TeX-master: "../master"
%%% End:
