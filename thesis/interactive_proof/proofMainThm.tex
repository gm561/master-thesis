%
\subsection{Putting it together}
\label{st:put_together}
In the previous sections we have shown that it is possible to partition the domain $Q$ such that
if the number of hint and verification queries is small, then success probability
of a solver for the $k$-wise direct product is still substantial.
As shown in Lemma \ref{lemma:sec_amp_for_p_hash} we can build a circuit that returns a solution that is likely to be good.
It remains to use these building blocks and prove Theorem~\ref{th:sec_amp_for_dwvp}.

\begin{proof}[of Theorem \ref{th:sec_amp_for_dwvp}]
Let $\widetilde{\text{Gen}}$ be the algorithm from Theorem \ref{th:sec_amp_for_dwvp} that outputs
the solver circuit $\widetilde{D}$ for $P$. First, in the following code listing we define $\widetilde{\Gen}$.
%
\begin{codeblock}
  \textbf{Algorithm} $\widetilde{\text{Gen}}^{P^{(1)}, g, C}(n, \epsilon, \delta, k, h, v)$
  \medskip\hrule
  \textbf{Oracle:} A problem poser $P^{(1)}$, a function $g: \{0,1\}^{k} \rightarrow \{0,1\}$, \\
  \IndII a solver circuit $C$ for $P^{(g)}$.  \\
  \textbf{Input:} Parameters $n$, $\epsilon$, $\delta$, $k$, $h$, $v$.\\
  \textbf{Return:} A circuit $\widetilde{D} = (D_1, \widetilde{D}_2)$.
  \medskip\hrule
  %
  $hash := \text{FindHash}((h+v)\epsilon, n, h, v)$ \\
  Let $\widetilde{C} := (C_1, \widetilde{C}_2)$ be as in Lemma \ref{lemma:ctilda_c} with oracle access to $C$, $\hash$. \\
  $D := Gen^{P^{(1)},  \widetilde{C},  g, \hash}(\epsilon, \delta, n, k)$ \\
  \Return $\widetilde{D} := (D_1, \widetilde{D}_2)$
\end{codeblock}
%
\begin{codeblock}
  \textbf{Circuit} $\widetilde{D}_2^{D, P^{(1)}, \hash, g, \Gamma_V, \Gamma_H}(x, \rho)$
  \medskip \hrule
  \textbf{Oracle:} A circuit $D :=(D_1, \widetilde{D}_2)$ from Lemma \ref{lemma:sec_amp_for_p_hash}, a problem poser $P^{(1)}$, \\
  \IndII functions $\hash: Q \rightarrow \{0,1, \dots, 2(h+v) - 1\}$, $g: \{0,1\}^{k} \rightarrow \{0,1\}$ \\
  \IndII a verification oracle $\Gamma_V$, a hint oracle $\Gamma_H$.\\
  \textbf{Input:}  Bitstrings $x \in \{0,1\}^{*}$, $\rho \in \{0,1\}^{*}$.
  \medskip\hrule
  $(q, y) := D_2^{P^{(1)}, \widetilde{C}, \hash, g, \Gamma_H}(x, \rho)$ \\
  Ask verification query $(q,y)$ to $\Gamma_V$.
\end{codeblock}
%
We show that Theorem \ref{th:sec_amp_for_dwvp} follows from Lemma \ref{lemma:hash_function_probability} and Lemma \ref{lemma:sec_amp_for_p_hash}.
We fix $P^{(1)}$, $g$, $P^{(g)}$ in the whole proof and consider a solver circuit $C = (C_1, C_2)$, asking at most $h$ hint queries and $v$ verification queries such that
\begin{align*}
    \underset{\pi^{(k)}, \rho}{\Pr}\left[\Success^{P^{(g)}, C}(\pi^{(k)}, \rho) = 1\right] \geq 16(h+v)\Big(\underset{u \leftarrow \mu_\delta^k}{\Pr}\left[g(u) = 1\right] + \varepsilon\Big).
\end{align*}
First, we note that $C$ meets the requirements of Lemma \ref{lemma:hash_function_probability}.
Furthermore, trivially success probability of $C$ is at least $(h+v)\epsilon$.
Thus, the algorithm $\widetilde{\text{Gen}}$ can call FindHash to obtain $\hash:\cQ \rightarrow \{0,1,\dotsc, 2(h+v)-1\}$
such that with high probability it holds
\begin{align*}
    \underset{\pi^{(k)}, \rho}{\Pr}\left[\CanonicalSuccess^{P^{(g)}, C, \hash}(\pi^{(k)}, \rho) = 1\right] \geq \underset{u \leftarrow \mu_\delta^k}{\Pr}\left[g(u) = 1\right] + \varepsilon.
\end{align*}
Additionally, the running time of $\FindHash$ is $\mathit{poly}(h,v,\frac{1}{\epsilon},n)$ with oracle calls.
Applying Lemma \ref{lemma:ctilda_c} for $C=(C_1, C_2)$ we obtain a circuit $\widetilde{C} = (C_1, \widetilde{C}_2)$ that satisfies
\begin{align*}
    \underset{
      \mathclap {
      \substack{\pi^{(k)}, \rho \\
        x := \langle P^{(g)}(\pi^{(k)}), C_1(\rho) \rangle_{\mathit{trans}} \\
        (\Gamma_V^{(g)}, \Gamma_H^{(k)}) := \langle P^{(g)}(\pi^{(k)}), C_1(\rho) \rangle_{P^{(g)}}
      }}}
    {\Pr}\mkern13mu[\Gamma_V^{(g)}(\widetilde{C}_2^{\Gamma_H^{(k)}, C_2, \hash}(x, \rho)) = 1]
    \geq
\underset{u \leftarrow \mu_\delta^k}{\Pr}\left[g(u) = 1\right] + \varepsilon.
\end{align*}
Now, we use the algorithm Gen as in Lemma \ref{lemma:sec_amp_for_p_hash} that yields a circuit $D = (D_1, D_2)$ which with high probability satisfies
\begin{align}
  \label{eq:succ_prob_d}
    \underset{
      \mathclap{
      \substack{\pi, \rho \\ x := \langle P^{(1)}(\pi), D_1^{\widetilde{C}}(\rho) \rangle_{\mathit{trans}} \\
        (\Gamma_H, \Gamma_V) := \langle P^{(1)}(\pi), D_1^{\widetilde{C}}(\rho) \rangle_{P^{(1)}}}}}
    {\Pr}\mkern13mu[\Gamma_V(D_2^{P^{(1)}, \widetilde{C}, hash, g, \Gamma_H}(x, \rho)) = 1] \geq (\delta + \frac{\varepsilon}{6k}).
\end{align}
Finally, $\widetilde{\text{Gen}}$ outputs $\widetilde{D} = (D_1, \widetilde{D}_2)$ with oracle access to $D$, $P^{(1)}$, $hash$, $g$ such that with high probability it holds
\begin{align*}
    \underset{\pi, \rho}{\Pr}\left[\Success^{P^{(1)},\widetilde{D}}(\pi, \rho) = 1\right] \geq \delta + \frac{\varepsilon}{6k}.
\end{align*}
The running time of $\Gen$ is polynomial in $k, \frac{1}{\epsilon}, n)$ with oracle calls.
Thus, the overall running time of $\widetilde{\mathit{Gen}}$ is  polynomial in $k,\frac{1}{\epsilon},h,v,n$ with oracle access.
Furthermore, the circuit $\widetilde{D}$ asks at most $\frac{6k}{\epsilon} \log(\frac{6k}{\epsilon})h$ hint queries and one verification query.
Finally, we have $\mathit{Size}(\widetilde{D}) \leq \mathit{Size}(C) \cdot \frac{6k}{\epsilon}$.
This finishes the proof of Theorem~\ref{th:sec_amp_for_dwvp}.
\end{proof}

%%% Local Variables:
%%% mode: latex
%%% TeX-master: "../master"
%%% End:
