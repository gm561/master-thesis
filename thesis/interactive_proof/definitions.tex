%TODO pairwise independent hash function
\noindent
%
\begin{definition}[$k$-wise direct-product of DWVPs.]
  Let $g: \{0,1\}^{k} \rightarrow \{0,1\}$ be a monotone function and $P_n^{(1)}$ a problem poser as in Definition \ref{def:dwvp}.
  The $k$-wise direct product of $P_n^{(1)}$ is a DWVP defined by a circuit $P_{kn}^{(g)}$.
  We write $P_{kn}^{(g)}(\pi^{(k)})$ to denote the execution of $P_{kn}^{(g)}$ with the randomness fixed to $\pi^{(k)} := (\pi_1, \dots, \pi_k)$
  where for each $1 \leq i \leq n : \pi_i \in \{0,1\}^n.$
  Let $(C_1, C_2)(\rho)$ be a solver for $P_{kn}^{(g)}$ as in Definition \ref{def:dwvp}.
  In the first phase, the algorithm $C_1(\rho)$ sequentially interacts in $k$ rounds with $P_{kn}^{(g)}(\pi^{(k)})$.
  In the $i$-th round $C_1(\rho)$ interacts with $P_n^{(1)}(\pi_i)$,
  and as the result $P_{n}^{(1)}(\pi_i)$ generates circuits $\Gamma_V^i, \Gamma_H^i$.
  Finally, after $k$ rounds $P_{kn}^{(g)}(\pi^{(k)})$ outputs a verification circuit
\begin{align*}
  \Gamma_V^{(g)} (q, y_1, \dots, y_k) := g(\Gamma_V^{1}(q, y_1), \dots, \Gamma_V^{k}(q, y_k))
\end{align*}
and a hint circuit
\begin{align*}
  \Gamma_H^{(k)} (q) := (\Gamma_H^{1}(q), \dots, \Gamma_H^{k}(q)).
\end{align*}
\end{definition}
%
If it is clear from the context, we omit the subscript $n$ and write $P(\pi)$ instead of $P_n(\pi)$ where $\pi \in \{0,1\}^{n}$.

A verification query $(q,y)$ of a solver $C$ for which a hint query on this $q$ has been asked before cannot be a verification query for which $C$ succeeds.
Therefore, without loss of generality, we make the assumption that $C$ does not ask verification queries on $q$
for which a hint query has been asked before. Furthermore, we assume that once $C$ asked a verification query that succeeds,
it does not ask any further hint or verification queries.
%
\begin{codeblock}
  \textbf{Experiment} $\Success^{P, C}(\pi, \rho)$
  \medskip \hrule
  \textbf{Oracle:} A problem poser $P$, a solver $C = (C_1, C_2)$ for $P$.\\
  \textbf{Input:}  Bitstrings $\pi \in \{0,1\}^n$, $\rho \in \{0,1\}^*$.\\
  \textbf{Output:} A bit $b \in \{0,1\}$.
  \medskip\hrule\medskip
  \Run $\langle P(\pi), C_1(\rho) \rangle$ \\
  \IndI $(\Gamma_V, \Gamma_H) := \langle P(\pi), C_1(\rho) \rangle_{P}$ \\
  \IndI $x := \langle P(\pi), C_1(\rho) \rangle_{\mathit{trans}}$ \\ \\
  \Run $C_2^{\Gamma_V,\Gamma_H}(x, \rho)$ \\
  \IndI \If $C_2^{\Gamma_V, \Gamma_H}(x, \rho)$ asks a verification query $(q, y)$ s.t. $\Gamma_V(q, y) = 1$ \Then \\
  \IndII \Return $1$ \\
  \Return $0$ \\
\end{codeblock}
%
We define the \textit{success probability} of $C$ in solving a puzzle defined by $P$ as
\begin{align}
 \underset{\pi, \rho}{\Pr}[\Success^{P,C}(\pi, \rho) = 1].
\end{align}
Furthermore, we say that $C$ \textit{succeeds} for $\pi$, $\rho$ if $\Success^{P,C}(\pi, \rho) = 1$.
%
\begin{theorem}[Security amplification for dynamic weakly verifiable puzzles.]
\label{th:sec_amp_for_dwvp}
Let $P_{n}^{(1)}$ be a fixed problem poser as in Definition \ref{def:dwvp}
and $P_{kn}^{(g)}$ a problem poser for the $k$-wise direct product of $P_{n}^{(1)}$.
Additionally, let $C$ be a problem solver for $P_{kn}^{(g)}$ asking at most $h$ hint queries and $v$ verification queries.
There exists a probabilistic algorithm Gen with oracle access to a solver circuit $C$,
a monotone function $g:\{0,1\}^k \rightarrow \{0,1\}$ and problem posers $P_{n}^{(1)}$, $P_{kn}^{(g)}$.
Furthermore, $\mathit{Gen}$ takes as input parameters $\varepsilon$, $\delta$, $n$, $k$, $h$, $v$, and outputs a solver circuit $D$ for $P_{n}^{(1)}$
such that the following holds: \\
If $C$ is such that
  \begin{align*}
    \underset{\substack{\pi^{(k)} \in \{0,1\}^{kn} \\ \rho \in \{0,1\}^{*}}}{\Pr}\left[\mathit{Success}^{P_{kn}^{(g)}, C}(\pi^{(k)}, \rho) = 1\right]
    \geq 16(h+v)\Bigl(\underset{u \leftarrow \mu_\delta^k}{\Pr}\left[g(u) = 1\right] + \varepsilon\Bigr)
  \end{align*}
then $D$ is a two phase probabilistic circuit and satisfies almost surely
  \begin{align*}
    \underset{\substack{\pi \in \{0,1\}^{n} \\ \rho \in \{0,1\}^{*}}}
    {\Pr}\left[\mathit{Success}^{P_{n}^{(1)},D}(\pi, \rho) = 1\right] \geq \delta + \frac{\varepsilon}{6k}.
  \end{align*}
Additionally, $D$ requires oracle access to $g$, $P_{n}^{(1)}$, $C$, hint and verification circuits
and asks at most $\frac{6k}{\epsilon}\log\left(\frac{6k}{\epsilon}\right) h$ hint queries and one verification query.
Finally, $\mathit{Time}(\mathit{Gen}) = \mathit{poly}(k, \frac{1}{\varepsilon}, n, v, h)$ with oracle access to $C$.
\end{theorem}

% The Theorem \ref{th:sec_amp_for_dwvp} implies that if there is no good solver for a puzzle defined by $P^{(1)}$, then a good solver for
% a $k$-wise direct product of $P^{(1)}$ does not exist.

%%% Local Variables:
%%% mode: latex
%%% TeX-master: "../master"
%%% End:
