\chapter{Appendix}
\section{Basic Inequalities}
\begin{lemma}[Chernoff Bounds]
For independent, identically distributed Bernoulli random variables $X_1, \dots, X_n$ with $X := \sum_{i=1}^n X_i$
with $\Pr[X_i = 1] = p_i$ and $\Pr[X_i = 0] = 1 - p_i$ for all $ 1 \leq  i \leq n$.
we have the following inequalities for $0 \leq \delta \leq 1$ and $\mathbb{E}[X] = \sum_{i=1}^{n} p_i$:
\begin{gather}
\label{ineq:ch0}
\Pr[X \geq (1+\delta) \mathbb{E}[X]] \leq e^{- \mathbb{E}[X] \delta^2/3} \\
\label{ineq:ch1}
\Pr[X \leq (1-\delta) \mathbb{E}[X]] \leq e^{- \mathbb{E}[X] \delta^2/2} \\
\label{ineq:ch2}
\Pr[|X - \mathbb{E}[X]| \geq \delta \mathbb{E}[X]] \leq 2 e^{- \mathbb{E}[X] \delta^2 / 3}.
\end{gather}
\end{lemma}

\section{Proof of Lemma \ref{lemma:sec_amp_for_p_hash} under simplifing assumptions}
We give a proof of Lemma \ref{lemma:sec_amp_for_p_hash} where we assume that circuit $D$ never outputs $\bot$ (equivalently it finds
$c \in \cG_1 \setminus \cG_0$) and that for all surpluses we have $S_{\pi,b} \leq (1 - \frac{1}{k})\epsilon$.
The following calculations are intended to give Reader more intuition.
We have
\begin{align*}
  \underset{\pi, r}{\Pr}[\Gamma_V(D_2(x,r)) = 1] &= \underset{}{\Pr}[\Gamma_V(D_2(x^*,r)) = 1 \mid D(x,r) \neq \bot] \underset{}{\Pr}[D(x,r) \neq \bot] \\
  &= \Pr[c_1 \mid c \in \cG_1 \setminus \cG_0] \\
  &= \frac{\underset{}{\Pr}[c_1 = 1 \land c \in \cG_1 \setminus \cG_0]}{\underset{}{\Pr}[c \in \cG_1 \setminus \cG_0]} \\
  &= \mathbb{E}[\frac{\underset{}{\Pr}[c_1 = 1 \land c \in \cG_1 \setminus \cG_0] ( \underset{}{\Pr}[c \in \cG_0 \setminus \cG_1] - (S_{\pi,1} - S_{\pi,0}))}
  {\underset{}{\Pr}[c \in \cG_0 \setminus \cG_1] \underset{}{\Pr}[u \in \cG_1 \setminus \cG_0]}]  \\
  &\geq \frac{\underset{}{\Pr}[c_1 = 1 \land c \in \cG_1 \setminus \cG_0] - (S_{\pi,1} - S_{\pi,0})}{\underset{}{\Pr}[u \in \cG_1 \setminus \cG_0]} \\
  &\geq \frac{\delta \underset{}{\Pr}[u \in \cG_1 \setminus \cG_0] + \epsilon - (S_{\pi,1} - S_{\pi,0})}{\underset{}{\Pr}[u \in \cG_1 \setminus \cG_0]} \\
  &\geq \delta + \frac{\epsilon}{k}
\end{align*}
where in $(*)$ and $(**)$.