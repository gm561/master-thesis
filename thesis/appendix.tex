\chapter{Appendix}
\section{Concentration Bounds}
\begin{lemma}[Chernoff Bounds]
For independent Bernoulli distributed random variables $X_1, \dotsc, X_n$ with $X := \sum_{i=1}^n X_i$
and $\Pr[X_i = 1] = p_i$ for all $ 1 \leq  i \leq n$ the following inequalities hold
\begin{gather}
\label{ineq:ch0}
\Pr[X \geq (1+\delta) \mathbb{E}[X]] \leq e^{- \mathbb{E}[X] \delta^2/3} \\
\label{ineq:ch1}
\Pr[X \leq (1-\delta) \mathbb{E}[X]] \leq e^{- \mathbb{E}[X] \delta^2/2} \\
\label{ineq:ch2}
\Pr[|X - \mathbb{E}[X]| \geq \delta \mathbb{E}[X]] \leq 2 e^{- \mathbb{E}[X] \delta^2 / 3},
\end{gather}
where $0 \leq \delta \leq 1$.

For independent and identically distributed Bernoulli random variables $X_1, \dotsc, X_n$ with $X := \sum_{i=1}^n X_i$
where $\Pr[X_i = 1] = p$ for some $p \in (0,1)$ for all $ 1 \leq  i \leq n$ we have
\begin{gather}
\label{ineq:ch00}
\Pr[x \geq (p + \varepsilon)n] \leq e^{-\frac{\epsilon^2k}{2}}.
\end{gather}

\end{lemma}
\vspace*{\fill}
\pagebreak

\section{The Proof of Lemma \ref{lemma:sec_amp_for_p_hash} Under the Simplifying Assumptions}
\label{st:proofSimAssm}
We give a proof of Lemma \ref{lemma:sec_amp_for_p_hash} in the case where non of the estimate is large.
For the sake of simplicity we make the following assumptions
\begin{gather}
  \label{pr:always_d}
\underset{
  \mathclap{
  \substack{\pi, \rho \\ x := \langle P^{(1)}(\pi), D_1(\rho) \rangle_{\trans}}}}
{\Pr}[D_2(x,\rho) \neq \bot] = 1 \\
  \label{pr:low_surpluses}
\forall \pi \in \{0,1\}^{n} : S_{\pi, b} \leq (1 - \frac{1}{k}).
\end{gather}
Loosely speaking, in \eqref{pr:always_d} we assume that for every $\pi$ circuit $D$ outputs an answer,
and in \eqref{pr:low_surpluses} we make the assumption that all surpluses are low.
In the complete proof of Lemma \ref{lemma:sec_amp_for_p_hash} this assumptions fail only slightly such that it is still possible
to obtain the desired result. However, the calculations are fairly lengthy.
The following simplified calculations are intended to give the Reader better intuition about the full proof.
We have
\begin{align*}
\underset{
  \mathclap{
  \substack{\pi, \rho \\ x := \langle P^{(1)}(\pi), D_1(\rho) \rangle_{\trans}
    \\ (\Gamma_V, \Gamma_H) := \langle P^{(1)}(\pi), D_1 (\rho) \rangle_{P^{(1)}} }}}
{\Pr}[\Gamma_V&(D_2(x,\rho)) = 1]
  \overset{\eqref{pr:always_d}}{=}
\underset{
  \mathclap{
  \substack{\pi, \rho \\ x := \langle P^{(1)}(\pi), D_1(\rho) \rangle_{\trans}
    \\ (\Gamma_V, \Gamma_H) := \langle P^{(1)}(\pi), D_1 (\rho) \rangle_{P^{(1)}} }}}
{\Pr}[\Gamma_V(D_2(x^*,\rho)) = 1 \mid D_2(x,\rho) \neq \bot] \\
  &\overset{\hphantom{a}(*)\hphantom{a}}{=} \underset{\pi^{(k)}}{\Pr}[c_1 = 1 \mid c \in \cG_1 \setminus \cG_0] \\
  &\overset{\hphantom{\eqref{eq:diff_s01_next}}}{=} \frac{\Pr_{\pi^{(k)}}[c_1 = 1 \land c \in \cG_1 \setminus \cG_0]}{\Pr_{\pi^{(k)}}[c \in \cG_1 \setminus \cG_0]} \\
  &\overset{\eqref{eq:diff_s01_next}}{=} \underset{\pi^*}{\mathbb{E}}\left[\frac{\Pr_{\pi^{(k-1)}}[c_1^* = 1 \land c \in \cG_1 \setminus \cG_0]
    \big(\Pr_{\pi^{(k-1)}}[c \in \cG_0 \setminus \cG_1] - (S_{\pi^*,1} - S_{\pi^*,0})\big)}
  {\underset{\pi^{(k-1)}}{\Pr}[c \in \cG_0 \setminus \cG_1] \underset{u \leftarrow \mu_{\delta}^{k}}{\Pr}[u \in \cG_1 \setminus \cG_0]}\right]  \\
  &\overset{\hphantom{\eqref{eq:diff_s01_next}}}{\geq} \underset{\pi^*}{\mathbb{E}}\left[\frac{\Pr_{\pi^{(k-1)}}[c_1^* = 1 \land c \in \cG_1 \setminus \cG_0]
    \big(\Pr_{\pi^{(k-1)}}[c \in \cG_0 \setminus \cG_1] - (1 - \frac{1}{k})\epsilon\big)}
  {\underset{\pi^{(k-1)}}{\Pr}[c \in \cG_0 \setminus \cG_1] \underset{u \leftarrow \mu_{\delta}^{k}}{\Pr}[u \in \cG_1 \setminus \cG_0]}\right]  \\
  &\overset{\hphantom{\eqref{eq:diff_s01_next}}}{\geq} \frac{\Pr_{\pi^{(k)}}[c_1 = 1 \land c \in \cG_1 \setminus \cG_0] -
    (1 - \frac{1}{k})\epsilon}{\underset{u \leftarrow \mu_{\delta}^{k}}{\Pr}[u \in \cG_1 \setminus \cG_0]} \\
  &\overset{\eqref{eq:gu_rel}}{\geq} \frac{\delta \underset{}{\Pr}[u \in \cG_1 \setminus \cG_0] + \epsilon - (1-\frac{1}{k})\epsilon}{\underset{}{\Pr}[u \in \cG_1 \setminus \cG_0]} \\
  &\overset{\hphantom{\eqref{eq:diff_s01_next}}}{\geq} \delta + \frac{\epsilon}{k},
\end{align*}
where in $(*)$ we use the facts that $D_2(x,\rho) \neq \bot$ if and only if $D_2$ finds $c \in \cG_1 \setminus \cG_0$
and conditioned on $D_2(x,\rho) \neq \bot$  we have that $\Gamma_V(D_2(x,r)) = 1$ if and only if $c_1 = 1$.