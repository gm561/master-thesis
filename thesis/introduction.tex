\section{Weakly Verifiable Puzzles}
\textit{Weakly verifiable puzzles} were introduced by Cannetti et al. \cite{canetti2005hardness}.
Intuitively, a weakly verifiable puzzle is a problem generated by a party called a~poser
and solved by a solver. Nevertheless, one does not require the solver to have an efficient way to verify~a solution.
On the other hand, the poser has access to some secret information which makes the task of verifying a solution easy.

An example of a weakly verifiable puzzle is a CAPTCHA (an automated Turing test) defined in \cite{von2003captcha}.
It is easily solvable by humans but is at least mildly harder to solve for computer programs.
The poser generates an instance of a CAPTCHA together with the unique solution.
Therefore, it can trivially verify solutions.
On the other hand, the solution space is often relatively small,
thus it must be hard for a computer program solving the puzzle to verify a~solution.

\section{Hardness Amplification}
An important task in cryptography is to turn a certain problem that is only mildly hard into one that is substantially harder.
An example is the well known hardness amplification lemma for one-way function by Yao \cite{yao1982theory}.
This result implies that it is possible to build a strong one-way function from a function that is only weakly one-way.

A similar hardness amplification statement can be studied for weakly verifiable puzzles.
Given a puzzle that is solved with substantial probability, we want to construct a puzzle that is considerable harder to solve.

There are two approaches to amplify hardness for weakly verifiable puzzles.
Both base on combining several weakly-hard puzzles into a construction that is strongly-hard.
First, one can use \textit{sequential repetition} where puzzles are solved in rounds that start one after another.
Ahn et al. \cite{von2003captcha} observed that sequential repetition amplifies hardness for weakly verifiable puzzle.
However, this approach may be inefficient as it increases the number of communication rounds.
Often a better solution from the practical reasons is to amplify hardness by \textit{parallel repetition}
where several independent puzzles are sent in one round.
We note that Bellare et al. \cite{bellare1997does} show that parallel repetition may fail to amplify hardness in certain settings.

To prove hardness amplification one has to show that the following implication holds
\begin{align*}
  A \implies B
\end{align*}
where $A$ is a statement that a problem $P$ is hard, and $B$ denotes that a problem $Q$ is hard.
It turns out that it is often easier to consider the following logically equivalent implication
\begin{align*}
  \lnot B \implies \lnot A.
\end{align*}
%where under the assumption that $Q$ is easy, we try to show that $P$ is easy.
This approach is used in this thesis. More precisely, we assume existence of an algorithm that successfully
solves the parallel repetition of weakly verifiable puzzles with substantial probability.
Under this assumption we construct an algorithm that success probability in solving a single puzzle is substantial.

\section{Previous works}
% Breaking security is often defined as a game in which an adversary has to solve a certain problem.
% It turns out that for some cryptographic primitives a task of winning a game by an adversary
% is equivalent to solving a weakly verifiable puzzle.

The proof of Yao for amplifying hardness of one--way functions relies to a great extent on the fact that it is possible
for an adversary to easily verify correctness of a solution. Therefore, to show hardness amplification
for weakly verifiable puzzles a different approach has to be developed.

Cannetti et al. \cite{canetti2005hardness} define weakly verifiable puzzles and give the hardness amplification proof for these puzzles.

Hardness amplification for weakly verifiable puzzles in a situation where the solver can make several mistakes is studied by Impagliazzo et al. \cite{impagliazzo2007chernoff}.
They introduce a \textit{threshold function} that determines the minimal fraction of puzzles that must be successfully solved.

Holenstein and Schoenebeck \cite{holenstein2011general} give a more general proof for hardness amplification
of weakly verifiable puzzles. They use an arbitrary binary monotone function in the place of a threshold function.
Additionally, they introduce \textit{interactive weakly verifiable puzzles} where an puzzle is defined by
an interactive protocol between the poser and the solver. These puzzles generalize, for example, the binding property of commitment protocols.

Dodis et al. \cite{dodis2009security} extends the proof of hardness amplification of Imaglizazzo et al.
Not only, they consider threshold functions but also introduce \textit{dynamic interactive weakly verifiable puzzles} where a puzzle is defined together with a set of indices $\cQ$.
The solver must give a correct answer on any $q \in \cQ$. Furthermore, it can verify correctness of a limited number of solutions and
obtain several hints understood as correct solutions on some $q \in \cQ$. The solver must solve a puzzle on $q$ for which it did not ask a hint before.
The dynamic weakly verifiable puzzles can be seen as, for example, games of breaking security of MAC.
In this setting the set $\cQ$ is a set of messages. The task of the solver is to provide a MAC for $q \in \cQ$.
The solver can ask for tags for messages of his choice, but must provided a MAC for a message for which it did not ask a hint before.

% In some cryptographic settings it is enough to solve only a fraction of puzzles included in the parallel repetition of weakly verifiable puzzles.
% A significant example are hard artificial intelligence problems like CAPTCHAs where the goal is to distinguish a human from a computer program.
% A human has on average an advantage over computer programs in recognizing a distorted text. However, we cannot exclude a situation where he or she also makes mistakes.

% The proof of hardness amplification where a threshold function is used to decide what fraction of weakly verifiable puzzles has to be solved correctly
% in order to successfully solve the parallel repetition of weakly verifiable puzzles is given by Impagliazzo, Jaiswal, and Kabanets \cite{impagliazzo2007chernoff}.
% A similar proof by Dodis, Impagliazzo, Jaiswal, and Kabanets \cite{Dodis:2009:SAI:1530441.1530450} additionally
% takes into account settings where an adversary can ask a limited number of queries that verify correctness of a solution.
% Furthermore, an adversary can obtain limited number of hints.

\section{Contribution of the Thesis}
We introduce the notion of \textit{dynamic interactive weakly verifiable puzzles} that generalize dynamic and interactive
weakly verifiable puzzles.
We give a proof of hardness amplification for this type of puzzles by combining techniques
used by Dodis et al. \cite{dodis2009security} and Holenstein et al. \cite{holenstein2011general}.
As a result we show that it is possible to amplify hardness of dynamic interactive weakly verifiable puzzles where
to create an instance of a puzzle the solver and the poser engage in the execution of an interactive protocol.
Furthermore, the solver can verify a limited number of solutions and obtain correct solutions on some $q \in \cQ$.
Finally, the proof applies also to a setting where a monotone binary function is used to decide which puzzles of the parallel repetition of puzzles
have to be successfully solved.
%
\section{Organization of the Thesis}
In Chapter \ref{ch:preliminaries} we set up the notation and terminology used in the thesis.
We define a \textit{monotone binary function} and a \textit{pairwise independent family of hash functions}.

Next, in Chapter \ref{ch:diwvp_main_thm} we introduce \textit{dynamic interactive weakly verifiable
puzzles} and state the hardness amplification theorem for this type of puzzles.
Proving this theorem is the main focus of this thesis.

Chapter \ref{ch:examples_wvcp} is devoted to the different types of cryptographic primitives that
can be seen as weakly verifiable.
We briefly describe MACs, signature schemes, commitment protocols, and CAPTCHAs and explain why
these constructions are generalized by dynamic interactive weakly verifiable puzzles.

Then, in Chapter \ref{ch:previous_results} we give an outline of the earlier studies of weakly
verifiable puzzles and compare them with the results obtained in this thesis.

Finally, in Chapter \ref{ch:main_result} we give the proof of hardness amplification
for dynamic interactive weakly verifiable puzzles and discuss this result.

% Local Variables:
% mode: latex
% TeX-master: "thesis"
% End: