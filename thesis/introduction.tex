\section{Hardness Amplification Theorems}
An important question in cryptography is whether it is possible to turn a certain problem
that is only mildly hard into a problem that solving is substantially harder.
An example is the well known hardness amplification result by Yao \cite{Goldreich:2000:FCB:519078}
which states that it is possible to build a strong one--way function from a function that is only weakly one-way.

In this Thesis we study hardness amplification of \textit{weakly verifiable puzzles}.
This notion generalize numerous basic cryptographic primitives like signature schemes,
message authentication codes, or bit commitment protocols, just to name a few.
The characteristic property of a weakly verifiable puzzle is that one does not require that verifying the correctness of a solution
by a puzzle solver is efficient\footnote{We note that the word \textit{weakly} in the context of weakly verifiable
puzzles is used in a different meaning than for a weak one--way function.}.
On the other hand, we assume that an algorithm that generates an instance of a weakly verifiable puzzle
has access to secret information which makes the task of verifying correctness of a solution easy.
An example is CAPTCHA where a problem of checking whether a solution is correct by a solver is comparably hard to finding a correct solution.
Whereas, the poser knows the text from which the CAPTCHA is generated and can trivially verify whether a solution is correct.

There are two main approaches to amplify hardness that base on combining several weakly hard problems into
a construction that is strongly hard. First, one can try to use \textit{sequential repetition} where a protocol is repeated
in rounds that start one after another. It has been observed that sequential repetition amplifies hardness
of weakly verifiable puzzles \cite{von2003captcha}. However, this approach may be inefficient as it increases
the number of rounds creating additional communication burden.

A more useful technique both from the practical and theoretical point is to amplify hardness using \textit{parallel repetition}
where several independent weakly hard problems are sent in one round.
However, it has been shown that in some settings parallel repetition may fail to amplify hardness \cite{bellare1997does}.
Therefore, some studies are required to prove that parallel repetition amplifies hardness
of weakly verifiable puzzles.

Proving hardness amplification requires showing that the following implication holds
\begin{align*}
  A \implies B
\end{align*}
where $A$ is a statement that a problem $P$ is hard, and $B$ denotes that a problem $Q$ is hard.
It turns out that it is often sensible to consider a logically equivalent statement
\begin{align*}
  \lnot B \implies \lnot A.
\end{align*}
Thus, under the assumption that $Q$ is easy, we try to prove that $P$ is easy.
This approach is used in the Thesis. More precisely, we assume existence of an algorithm that successfully
solves the parallel repetition of weakly verifiable puzzles with substantial probability, and
under this assumption we construct an algorithm that solves a single puzzle with substantial probability.

\section{Weakly Verifiable Puzzles}
Breaking security is often defined as a game in which an adversary has to solve a certain problem.
It turns out that for some cryptographic primitives a task of winning a game by an adversary
is equivalent to solving a weakly verifiable puzzle.

The proof of Yao for amplifying hardness of one--way functions relies to the great extent on the fact that it is possible
for an adversary to easily verify correctness of a solution. Therefore, to show hardness amplification
for weakly verifiable puzzles a different approach has to be developed.

Weakly verifiable puzzles have been introduced and studied by Cannetti, Halevi, and Steiner \cite{canetti2004hardness}.
They show how to amplify hardness of the parallel repetition of weakly verifiable puzzles.

In some cryptographic settings it is enough to solve only a fraction of puzzles included in the parallel repetition of weakly verifiable puzzles.
A significant example are hard artificial intelligence problems like CAPTCHAs where the goal is to distinguish a human from a computer program.
A human has on average an advantage over computer programs in recognizing a distorted text. However, we can not exclude a situation where he or she also makes mistakes.

The proof of hardness amplification where a threshold function is used to decide what fraction of weakly verifiable puzzles has to be solved correctly
in order to successfully solve the parallel repetition of weakly verifiable puzzles is given by Impagliazzo, Jaiswal, and Kabanets \cite{impagliazzo2007chernoff}.
A similar proof by Dodis, Impagliazzo, Jaiswal, and Kabanets \cite{Dodis:2009:SAI:1530441.1530450} additionally
takes into account settings where an adversary can ask a limited number of queries that verify correctness of a solution.
Furthermore, an adversary can obtain limited number of hints. The puzzles defined in this way captures
some standard security definitions of cryptographic primitives like message authentication codes and signature schemes.

Holenstein and Schoenebeck \cite{DBLP:journals/corr/abs-1002-3534} give a more natural proof for hardness amplification
of weakly verifiable puzzles where only a fraction of puzzles has to be solved correctly.
Furthermore, the puzzles considered by them generalize to games such as breaking the binding property of bit commitment protocols
where an instance of a puzzle is generated in an interactive phase.

\section{Contribution of the Thesis}
In this Thesis we apply the proof technique presented in \cite{DBLP:journals/corr/abs-1002-3534}
in the context of weakly verifiable puzzles as in \cite{Dodis:2009:SAI:1530441.1530450}.
As a result we prove that it is possible to amplify hardness of weakly verifiable puzzles where an adversary
can ask a limited number of hint and verification queries, an instance of a puzzle is created in an interactive protocol,
and a monotone binary function is used to decide which puzzles of the parallel repetition of weakly verifiable puzzles
have to be successfully solved (this generalize a case where only a fraction of puzzles has to be solved successfully).
%
\section{Organization of the Thesis}
In Chapter \ref{ch:preliminaries} we lay down notation and terminology used in the Thesis.

Next, in Chapter \ref{ch:intro_weakly} we define a \textit{dynamic interactive weakly verifiable puzzle} and
give an overview of cryptographic primitives that are generalized by this notion.
Furthermore, we give an outline of earlier studies of weakly verifiable puzzles and compare
it with the results contained in this Thesis.

Finally, in Chapter \ref{ch:main_result} we formulate and prove the main theorem of this Thesis.
Namely, we show that it is possible to amplify hardness of dynamic interactive weakly verifiable puzzles.
%
