%
\subsection{The proof of hardness amplification theorem}
\label{st:mainTheoremProof}
In this section we prove Theorem \ref{th:sec_amp_for_dwvp} and state two additional lemmas that help us to prove this theorem.
Informally speaking, Lemma \ref{lemma:hash_function_probability} states that it is possible to partition $\cQ$ such that
the solver $C$ has still substantial success probability and there are no conflicting hint and verification queries.
In Lemma~\ref{lemma:sec_amp_for_p_hash} we state that it is possible to amplify hardness for dynamic interactive puzzles
under the assumption that there are no conflicting hint and verification queries.

\paragraph{Domain partitioning} Let $\hash:\cQ\rightarrow\{0,1,\dots, 2(h\!+\!v)\!-\!1\}$, the idea is to partition $\cQ$ such that the set of preimages
of $0$ for $\hash$ contains $q \in \cQ$ on which $C$ is not allowed to ask hint queries,
and the first successful verification query $(q,y)$ of $C$ is such that $\hash(q) = 0$.
Therefore, if $C$ makes a verification query $(q,y)$ such that $\hash(q) = 0$, then we know that no hint query is ever asked on this $q$.

We denote the $i$-th query of $C$ by $q_i$ if it is a hint query and by $(q_i, y_i)$ if it is a verification query.
Let us define the following experiment $\CanonicalSuccess$ in which $\cQ$ is partitioned using a function $\hash$.
We say that a solver circuit \textit{succeeds} in the experiment $\CanonicalSuccess$
if it asks a successful verification query $(q_j, y_j)$ such that $\hash(q_j) = 0$,
and no hint query $q_i$ has been asked before $(q_j, y_j)$ such that $\hash(q_i) = 0$.
% The result of the experiment $\CanonicalSuccess$ is $1$ if and only if the solver $C$ makes
% a successful verification query with respect to the domain partitioned using $\hash$.
%
\begin{codeblock}
  \textbf{Experiment} $\CanonicalSuccess^{P, C, \hash}(\pi, \rho)$
  \medskip \hrule
  \textbf{Oracle:} A problem poser $P$, a solver circuit $C = (C_1, C_2)$ for $P$,\\
  \IndII a function $\hash: \cQ \rightarrow \{0, 1, \dotsc, 2(h+v) - 1\}$.\\
  \textbf{Input:}  Bitstrings $\pi \in \{0,1\}^n$, $\rho \in \{0,1\}^*$. \\
  \textbf{Output:} A bit $b \in \{0,1\}$.
  \medskip\hrule
  \Run $\langle P(\pi), C_1(\rho) \rangle$ \\
  \IndI $(\Gamma_V, \Gamma_H) := \langle P(\pi), C_1(\rho) \rangle_{P}$ \\
  \IndI $x := \langle P(\pi), C_1(\rho) \rangle_{\trans}$ \\ \\
  \Run $C_2^{\Gamma_V, \Gamma_H} (x, \rho)$ \\
  \IndI \If $C_2^{\Gamma_V, \Gamma_H} (x, \rho)$ does not succeed for any verification query \Then \\
  \IndII \Return $0$ \\
  \IndI Let $(q_j,y_j)$ be the first verification query such that $\Gamma_V(q_j, y_j) = 1$.\\
  \\
  \If $(\forall i < j :  \hash(q_i) \neq 0)$ \And $(\hash(q_j) = 0)$ \Then \\
  \IndI \Return 1\\
  \Else\\
  \IndI \Return 0
\end{codeblock}
%
We define the \textit{canonical success probability} of a solver circuit $C$ for $P$ with respect to a function $\hash$ as
\begin{align}
 \underset{\pi, \rho}{\Pr}[\CanonicalSuccess^{P, C, \hash}(\pi, \rho) = 1].
\end{align}
%
For fixed $\hash$ and $P$ a \textit{canonical success} of $C$ for bistrings $\pi$, $\rho$ is a situation where
$\CanonicalSuccess^{P, C, \hash}(\pi, \rho) = 1$.

We now give Lemma \ref{lemma:hash_function_probability}. Loosely speaking, it states that if a solver circuit $C$ for $P$
often succeeds in the experiment $\Success$, then there exists a function $\hash$ such that $C$ also
often succeeds in the experiment $\CanonicalSuccess$ provided that the number of hint and verification queries is not too large.
This lemma is analogous to Lemma 4 from \cite{dodis2009security}.

\begin{lemma}[Canonical success probability with respect to a function $\hash$, \cite{dodis2009security}]
\label{lemma:hash_function_probability}
For a fixed problem poser $P_n$ let $C$ be a solver for $P_n$ with the success probability at least $\gamma$,
asking at most $h$ hint queries and $v$ verification queries.
Moreover, let $\cH$ be a family of pairwise independent efficient hash functions $\cQ \rightarrow \{0,1, \dots,2(h+v)-1\}$.
There exists a probabilistic algorithm FindHash that takes as input parameters $\gamma$, $n$, $h$, $v$, and has oracle access to $C$ and $P_n$.
Furthermore, FindHash runs in time polynomial in $(h,v,\frac{1}{\gamma},n)$ and with high probability outputs a function $\hash \in \cH$
such that the canonical success probability of $C$ with respect to $\hash$ is at least $\frac{\gamma}{16(h+v)}$.
\end{lemma}
We prove Lemma \ref{lemma:hash_function_probability} in Section \ref{st:domain_partition}.
%
\paragraph{Hardness amplification}%
Let $C := (C_1, C_2)$ be a two-phase solver circuit as in Definition \ref{def:dwvp}.
We write $C_2^{(\cdot, \cdot)}$ to emphasize that $C_2$ does not obtain direct access to the hint and verification oracles.
Instead, whenever $C_2$ asks a hint or verification query, it is answered explicitly
as in the following code excerpt of the circuit $\widetilde{C}_2$.
We define the following circuit $\widetilde{C}_2$.
\begin{codeblock}
  \textbf{Circuit} $\widetilde{C}_2^{\Gamma_H, C_2, \hash} (x, \rho)$
  \medskip \hrule
  \textbf{Oracle:} A hint circuit $\Gamma_H$, a circuit $C_2$, \\ \IndII a function $\hash : \cQ \rightarrow \{0,1,\dots, 2(h+v)-1\}$. \\
  \textbf{Input:} Bitstrings $x \in \{0,1\}^{*}$, $\rho \in \{0,1\}^{*}$. \\
  \textbf{Output:} A pair $(q, y)$ where $q \in \cQ$ and $y \in \{0,1\}^{*}$.
  \medskip\hrule
  \Run $C_2^{(\cdot, \cdot)}(x, \rho)$ \\
  \IndI \If $C_2^{(\cdot, \cdot)}(x, \rho)$ asks a hint query on $q$ \Then\\
  \IndII \If $\hash(q) = 0$ \Then\\
  \IndIII \Return $\bot$\\
  \IndII \textbf{else}\\
  \IndIII answer the query of $C_2^{(\cdot, \cdot)}(x, \rho)$ using $\Gamma_H(q)$\\
  \\
  \IndI \If $C_2^{(\cdot, \cdot)}(x, \rho)$ asks a verification query $(q, y)$ \Then \\
  \IndII \If $\hash(q) = 0 $ \textbf{then} \\
  \IndIII \Return $(q, y)$ \\
  \IndII \textbf{else} \\
  \IndIII answer the verification query of $C_2^{(\cdot, \cdot)}(x, \rho)$ with 0 \\
  \Return $\bot$
\end{codeblock}
%
Given $C := (C_1, C_2)$ we define a circuit $\widetilde{C} := (C_1, \widetilde{C}_2)$.
Every hint query $q$ asked by $\widetilde{C}$ is such that $hash(q) \neq 0$. Furthermore, $\widetilde{C}$ asks no verification queries,
instead it returns $(q,y)$ such that $hash(q) = 0$ or $\bot$.
We define the circuit $\widetilde{C}$ to use the approach of Holenstein et al. \cite{holenstein2011general} in the proof of hardness amplification.
More precisely, we state the following lemma that is analogous to Theorem 10 from \cite{holenstein2011general}.
\begin{lemma}[Hardness amplification with respect to $\hash$, \cite{holenstein2011general}]
  \label{lemma:sec_amp_for_p_hash}
  Let $g: \{0,1\}^{k} \rightarrow \{0,1\}$ be a binary monotone function, $P_n^{(1)}$ a fixed
  problem poser and $\widetilde{C} := (C_1, \widetilde{C}_2)$ a probabilistic two-phase circuit
  with oracle access to a function $\hash: \cQ \rightarrow \{0,1,\dots, 2(h+v)-1\}$
  and a solver $C := (C_1, C_2)$ for $P_{kn}^{(g)}$ that asks at most $h$ hint queries and $v$ verification queries.
  There exists an algorithm $\widetilde{\Gen}$ that takes as input parameters $\varepsilon$, $\delta$, $n$, $k$,
  has oracle access to $P_n^{(1)}$,  $\widetilde{C}$, $\hash$, $g$,
  and outputs a probabilistic two-phase circuit $\widetilde{D} := (D_1, \widetilde{D}_2)$ such that the following holds: \\
  If $\widetilde{C}$ is such that
  \begin{align*}
    \underset{\mathclap{\substack{
          \pi^{(k)} \in \{0,1\}^{kn}, \rho \in \{0,1\}^{*} \\
          x:= \langle P^{(g)}(\pi^{(k)}), {C}_1(\rho) \rangle_{\mathit{trans}} \\
          (\Gamma_H^{(k)}, \Gamma_V^{(g)}) := \langle P^{(g)}(\pi^{(k)}), C_1(\rho) \rangle_{P^{(g)}}}}}
    {\Pr}[\Gamma_V^{(g)}(\widetilde{C}_2^{\Gamma_H^{(k)}, C_2, \hash}(x,\rho)) = 1]
    \geq \underset{u \leftarrow \mu_\delta^k}{\Pr}[g(u) = 1] + \varepsilon,
  \end{align*}
  then $\widetilde{D}$ satisfies almost surely over the randomness of $\widetilde{\Gen}$
  \begin{align*}
    \underset{
      \mathclap{
      \substack{
        \pi \in \{0,1\}^{n}, \rho \in \{0,1\}^{*} \\
        x := \langle P^{(1)}(\pi), D_1^{\widetilde{C}}(\rho) \rangle_{\mathit{trans}} \\
        (\Gamma_H, \Gamma_V) := \langle P^{(1)}(\pi), D_1^{\widetilde{C}}(\rho) \rangle_{P^{(1)}}}}}
    {\Pr}[\Gamma_V(\widetilde{D}_2^{P^{(1)}, \widetilde{C}, \hash, g, \Gamma_H}(x, \rho)) = 1] \geq \delta + \frac{\varepsilon}{6k}.
  \end{align*}
  Furthermore, $\widetilde{D}$ asks at most $\frac{6k}{\epsilon}\log\left(\frac{6k}{\epsilon}\right) h$ hint queries and no verification queries
  and outputs a pair $(q, y)$ such that $\hash(q) = 0$ or $\bot$.
  Finally, the running time of $\mathit{Gen}$ is polynomial in $k, \frac{1}{\varepsilon}, n$ with oracle calls.
\end{lemma}
%
We now give the proof of Theorem \ref{th:sec_amp_for_dwvp} that combines Lemma~\ref{lemma:hash_function_probability} and Lemma~\ref{lemma:sec_amp_for_p_hash}.
\begin{proof}[Proof of Theorem \ref{th:sec_amp_for_dwvp}]
  We fix the notation as in Lemma \ref{lemma:sec_amp_for_p_hash}.
  Additionally, let $\Gen$ and $D := (D_1, D_2)$ be as in Theorem \ref{th:sec_amp_for_dwvp}.
  We define~$\Gen$ and $D_2$ in the following code listings.
%
\begin{codeblock}
  \textbf{Algorithm} $\Gen^{P^{(1)}, g, C}(n, \epsilon, \delta, k, h, v)$
  \medskip\hrule
  \textbf{Oracle:} A poser $P^{(1)}$, a function $g: \{0,1\}^{k} \rightarrow \{0,1\}$, a solver $C$ for $P^{(g)}$.  \\
  \textbf{Input:} Parameters $n$, $\epsilon$, $\delta$, $k$, $h$, $v$.\\
  \textbf{Output:} A circuit $\widetilde{D} := (D_1, \widetilde{D}_2)$.
  \medskip\hrule
  Let $\gamma$ be the success probability of $C$.\\
  $hash := \FindHash^{P^{(g)}, C}(\gamma, n, h, v)$ \\
  Let $\widetilde{C} := (C_1, \widetilde{C}_2)$ be as in Lemma \ref{lemma:ctilda_c} with oracle access to $C$ and $\hash$. \\
  $\widetilde{D} := \widetilde{\Gen}^{P^{(1)},  \widetilde{C},  g, \hash}(\epsilon, \delta, n, k)$ \\
  \Return $D := (D_1, D_2)$ \textit{// where $D$ has oracle access to $\widetilde{D}, \hash, g$}
\end{codeblock}
%
\begin{codeblock}
  \textbf{Circuit} $D_2^{\widetilde{D}, P^{(1)}, \hash, g, \Gamma_V, \Gamma_H}(x, \rho)$
  \medskip \hrule
  \textbf{Oracle:} A circuit $\widetilde{D} :=(D_1, \widetilde{D}_2)$ from Lemma \ref{lemma:sec_amp_for_p_hash}, a problem poser $P^{(1)}$, \\
  \IndII functions $\hash: Q \rightarrow \{0,1, \dots, 2(h+v) - 1\}$, $g: \{0,1\}^{k} \rightarrow \{0,1\}$ \\
  \IndII a verification oracle $\Gamma_V$, a hint oracle $\Gamma_H$.\\
  \textbf{Input:}  Bitstrings $x \in \{0,1\}^{*}$, $\rho \in \{0,1\}^{*}$.
  \medskip\hrule
  $(q, y) := D_2^{P^{(1)}, \widetilde{C}, \hash, g, \Gamma_H}(x, \rho)$ \\
  Ask verification query $(q,y)$ to $\Gamma_V$.
\end{codeblock}
%
%Let us fix $P^{(1)}$, $g$, $P^{(g)}$ in the whole proof.
We consider a solver circuit $C := (C_1, C_2)$ that asks at most $h$ hint queries and $v$ verification queries and satisfies
\begin{align*}
    \underset{\pi^{(k)}, \rho}{\Pr}\left[\Success^{P^{(g)}, C}(\pi^{(k)}, \rho) = 1\right] \geq 16(h+v)\Big(\underset{u \leftarrow \mu_\delta^k}{\Pr}\left[g(u) = 1\right] + \varepsilon\Big).
\end{align*}
%The circuit $C$ meets the requirements of Lemma \ref{lemma:hash_function_probability}.
%Clearly the success probability of $C$ is at least $(h+v)\epsilon$.
The algorithm $\Gen$ can use $\FindHash$ to obtain $\hash:\cQ \rightarrow \{0,1,\dotsc, 2(h+v)-1\}$ such that almost surely over the randomness of $\FindHash$ it holds
\begin{align*}
    \underset{\pi^{(k)}, \rho}{\Pr}\left[\CanonicalSuccess^{P^{(g)}, C, \hash}(\pi^{(k)}, \rho) = 1\right] \geq \underset{u \leftarrow \mu_\delta^k}{\Pr}\left[g(u) = 1\right] + \varepsilon.
\end{align*}
The running time of $\FindHash$ is $\mathit{poly}(h,v,\frac{1}{\epsilon},n)$ with oracle calls.
Applying Lemma \ref{lemma:ctilda_c} for $C=(C_1, C_2)$ we obtain a circuit $\widetilde{C} = (C_1, \widetilde{C}_2)$ that satisfies
\begin{align*}
    \underset{
      \mathclap {
      \substack{\pi^{(k)}, \rho \\
        x := \langle P^{(g)}(\pi^{(k)}), C_1(\rho) \rangle_{\mathit{trans}} \\
        (\Gamma_V^{(g)}, \Gamma_H^{(k)}) := \langle P^{(g)}(\pi^{(k)}), C_1(\rho) \rangle_{P^{(g)}}
      }}}
    {\Pr}\mkern13mu[\Gamma_V^{(g)}(\widetilde{C}_2^{\Gamma_H^{(k)}, C_2, \hash}(x, \rho)) = 1]
    \geq
\underset{u \leftarrow \mu_\delta^k}{\Pr}\left[g(u) = 1\right] + \varepsilon.
\end{align*}
We use the algorithm $\widetilde{\Gen}$ as in Lemma \ref{lemma:sec_amp_for_p_hash}
that yields a circuit $\widetilde{D} = (D_1, \widetilde{D}_2)$ and almost surely over the randomness $\widetilde{\Gen}$ satisfies
\begin{align}
  \label{eq:succ_prob_d}
    \underset{
      \mathclap{
      \substack{\pi, \rho \\ x := \langle P^{(1)}(\pi), D_1^{\widetilde{C}}(\rho) \rangle_{\mathit{trans}} \\
        (\Gamma_H, \Gamma_V) := \langle P^{(1)}(\pi), D_1^{\widetilde{C}}(\rho) \rangle_{P^{(1)}}}}}
    {\Pr}\mkern13mu[\Gamma_V(D_2^{P^{(1)}, \widetilde{C}, hash, g, \Gamma_H}(x, \rho)) = 1] \geq \delta + \frac{\epsilon}{6k}.
\end{align}
Finally, $\Gen$ outputs $D = (D_1, D_2)$ with oracle access to $\widetilde{D}_2$, $P^{(1)}$, $hash$, $g$ such that almost surely over
the randomness of $\Gen$ it holds
\begin{align*}
    \underset{\pi, \rho}{\Pr}\left[\Success^{P^{(1)},\widetilde{D}}(\pi, \rho) = 1\right] \geq \delta + \frac{\varepsilon}{6k}.
\end{align*}
The running time of $\widetilde{\Gen}$ is $\poly(k, \frac{1}{\epsilon}, n)$ with oracle calls.

The running time of $\Gen$ is  $poly(k,\frac{1}{\epsilon},h,v,n)$ with oracle access.
The circuit $D$ asks at most $\frac{6k}{\epsilon} \log(\frac{6k}{\epsilon})h$ hint queries and one verification query and
$\Size(D) \leq \Size(C) \cdot \frac{6k}{\epsilon}$.
This finishes the proof of Theorem~\ref{th:sec_amp_for_dwvp}.
\end{proof}

%%% Local Variables:
%%% mode: latex
%%% TeX-master: "../master"
%%% End:
