\section{Examples}
\label{section:wvp_examples}
In this section we give examples of cryptographic constructions that motivates studies of different types of weakly verifiable puzzles.

\subsection{Message Authentication Codes}
Let us consider a setting in which two parties a \textit{sender} and a \textit{receiver} communicate over an insecure channel.
Messages of the sender may be intercepted, modified, and replaced by a third party called an \textit{adversary}.
The receiver needs a way to ensure that received messages have been indeed sent by the sender and have not been modified by the adversary.
The solution is to use \textit{message authentication codes}.

Loosely speaking, the message authentication codes may be explained as follows.
Let sender, receiver, and adversary be polynomial time algorithms, and messages be represented as bitstrings.
Furthermore, we assume that the sender and the receive share a secrete key to which an adversary has no access.
The sender appends to every message a tag which is computed as a function of the key and the message.
The receiver, using the key, has a way to check whether an appended tag is valid for a received message.
The receiver accepts a message if the tag is valid, otherwise it rejects.
We require that it is hard for the adversary to find a tag and a message, not sent before, that is accepted by the receiver with non-negligible probability.
We give the following formal definition of \textit{Message Authentication Code} based on \cite{LectureNotesCrypo} and \cite{Goldreich:2004:FCV:975541}.
\begin{definition}[Message Authentication Code]
  \label{def:mac}
  Let $\cM$ be a set of messages, $\cK$ a set of keys and $\cT$ a set of tags.
  We define the \textnormal{message authentication code (MAC)} as an efficiently computable function $\cM \times \cK \rightarrow \cT$.
  Furthermore, we say that MAC is \textit{secure} if it satisfies the following condition:

  Let $k \xleftarrow{\$} \cK$ be fixed and $H$ be a polynomial size circuit that takes as input a message $m \in \cM$ and outputs a tag $t \in \cT$
  such that $f(m,k) = t$. We say that MAC is secure if there is no probabilistic polynomial time algorithm
  with oracle access to $H$ that with non-negligible probability outputs a message $m \in \cM$
  as well as a corresponding tag $t \in \cT$ such that $f(m, k) = t$, and $\Gamma_H$ has not been queried for a tag of message $m$.
\end{definition}

\begin{todo}
  \textbf{TODO:} Define as game?
\end{todo}

We show how MAC is captured by notion of dynamic weakly verifiable puzzles where at most one verification query is asked.
For fixed $f$ and $n \in \N$ the sender corresponds to the problem poser, the adversary to the problem solver,
and the key is a bitstring $\pi \in \{0,1\}^{n}$ taken as auxiliary input by the problem poser.
In the first phase, which is non interactive, the problem poser outputs a hint circuit
$\Gamma_H$ and a verification circuit $\Gamma_V$ where both circuits have hard-coded $\pi$.
The circuit $\Gamma_H$ takes as input a message and outputs a tag for this message and corresponds to the circuit $H$ from Definition \ref{def:mac}.
The circuit $\Gamma_V$ that as input $m \in \cM$ and $t \in \cT$ and outputs a bit $1$ if and only if $f(m, \pi) = t$.
In the second phase the problem solver takes no input ($x^*$ is empty string) and is given oracle access to $\Gamma_H$ and $\Gamma_V$.
We consider a case where at most on verification query is asked.
Thus, a task of finding by an adversary a valid tag $t \in \cT$ for a message $m \in \cM$ such that a hint for $m$ has not been asked before
corresponds to asking a successful verification query by a problem poser to $\Gamma_V$.
%
\subsection{Public Key Signature Scheme}
% \begin{todo}
%   \textbf{TODO:} Add introduction that gives intuition about the Public Key Signature Schemes
% \end{todo}
First, we give a definition of public key encryption scheme, and what it means for such a scheme to be secure.
These definitions are based on \cite{Goldreich:2004:FCV:975541}.

\begin{definition}[Public key signature scheme]
Let $\cQ$ be the set of messages. A \textnormal{public key signature scheme} is defined by a triple of probabilistic polynomial time algorithms:
$G$ -- the key generation algorithm,
$V$ -- the verification algorithm,
$S$ -- the signing algorithm,
such that the following conditions are satisfied:
\begin{enumerate}[-]
  \item $G(1^n)$ outputs a pair of bitstrings $k_{priv} \in \{0,1\}^{n}$ and $k_{pub} \in \{0,1\}^{n}$ where $n \in \N$ is a security parameter.
    We call $k_{priv}$ a private key and $k_{pub}$ a public key.
  \item The signing algorithm $S$ takes as input $k_{priv} \in \{0,1\}^{n}$, $q \in \cQ$ and outputs a signature $s \in S$.
  \item The verification algorithm $V$ takes as input $k_{pub} \in \{0,1\}^{n}$, $q \in \cQ$, and $s \in S$ and outputs a bit $b \in \{0,1\}$.
  \item For every $k_{priv}$, $k_{pub}$ output by $G$ and every $q \in \cQ$ it holds
    \begin{align*}
      \Pr[V(k_{pub}, q, S(k_{priv}, q))] = 1,
    \end{align*}
    where the probability is over the random coins of $V$ and $S$.
\end{enumerate}
\end{definition}
We say that $s \in S$ is a \textit{valid} signature for $q \in \cQ$ if and only if $V(k_{pub}, q, s) = 1$.
%
%TODO efficiency of the algorithms
%
\begin{definition}\textbf{(Security of public key signature scheme with respect to a chosen message attack)}
Let an \textnormal{adversary} $A$ be a probabilistic polynomial time algorithm that takes as input $k_{pub}$ and has oracle access to $S$.
We say that $A$ \textnormal{succeeds} if it finds a signature $s \in S$ for a message $q \in \cQ$ such that $V(k_{pub}, q, s) = 1$
and the oracle $S$ has not been queried for a signature of $q$.
The public key encryption scheme is \textnormal{secure} if there is no polynomial time adversary that succeeds with non-negligible probability.
\end{definition}
%
\begin{todo}
  \textbf{TODO:} define as game
\end{todo}
We will show now that a public key signature scheme defined as above can be represented as a dynamic weakly verifiable puzzle.
Let the problem poser correspond to an entity that generates $k_{pub}$,$k_{priv}$ and the problem solver to an adversary.
In the first phase, the problem poser uses algorithm $G(1^n)$ to obtain $k_{pub}$, $k_{priv}$ and sends to the adversary the public key $k_{pub}$.
Then the problem poser generates a hint circuit $\Gamma_H$ and a verification circuit $\Gamma_V$.
The hint circuit $\Gamma_H$ takes as input $q \in \cQ$ and outputs a signature for $q$. The verification circuit
$\Gamma_V$ takes as input $s \in S$ and $q \in \cQ$ and checks whether $s \in S$ is a valid signature for $q \in \cQ$.
In the second phase, the problem solver takes as input a transcript of the messages from the first round which consists solely of $k_{pub}$.
Additionally, it is given oracle access to $\Gamma_V$ and $\Gamma_H$.
It is clear that if the adversary asks a successful verification query $(q,s)$,
then it also successfully forges a signature of $q$.

Thus, a game of forging a signature of a public key signature schemes is a weakly verifiable puzzle that
is dynamic but not interactive as in the first phase only a single message is sent.
%
\subsection{Bit Commitments}
Let us consider the following \textit{bit commitment protocol} that involves two parties a \textit{sender} and a \textit{receiver}.
We suppose that the sender and the receiver are polynomial time probabilistic algorithms.
The protocol consists of a \textit{commit phase} and a \textit{reveal phase}.
In the commit phase the sender and the receiver interact, as the result the sender commits to a value $b \in \{0,1\}$.
We require that after the commit phase it is hard for the receiver to correctly guess $b$.
In the reveal phase the sender opens the commitment by sending to the receiver a pair $(y,b')$ where $y \in \{0,1\}^{*}$ is an information
that helps the receiver to verify that the sender committed to the value $b' \in \{0,1\}$.
A desirable property of bit commitment protocol is that in the reveal phase it should be hard for the sender to find $y$ such that the receiver
is convinced that the sender was committed to the value $\lnot b$.

We base the following definition of \textit{bit commitment protocol} on \cite{LectureNotesComThCrypto}.
\begin{definition}[Bit Commitment Protocol]
  \label{def:bit_commitment}
For a security parameter $n~\in~\N$ a \textnormal{bit commitment protocol} is defined by a pair $(S_n, R_n)$
where $S_n = (S_1, S_2)$ is a two phase probabilistic circuit, and $R$ is a probabilistic circuit.
We call $S_n$ the sender and $R$ the receiver. The circuit $S_1$ takes as input a pair $(b, \rho_S)$
where $b \in \{0,1\}$ is interpreted as a bit to which $S_n$
commits, and $\rho_S \in \{0,1\}^{n}$ is the randomness used by the algorithm $S_n$.
The receiver $R$ takes only auxiliary input $\rho_R \in \{0,1\}^{*}$ that is the randomness used by $R$.
The protocol consists of two phases. In the \textnormal{commit phase}, circuits $S_1$ and $R$ engage in the protocol execution.
As the result $S_1$ commits to $b$ and $R_n$ generates a verification circuit $\Gamma_V$.
The circuit $\Gamma_V$ takes as input a bit $b' \in \{0,1\}$ and a bitstring $y \in \{0,1\}^{*}$ and outputs a bit.
In the \textnormal{reveal phase} the circuit $S_2$ takes as input a communication transcript from the commitment phase
$\langle S_1, R \rangle_{\mathit{trans}}$, the bitstring $\rho_s$ and returns $(b', y)$.
We require a bit commitment protocol to have the following properties:
\begin{enumerate}[]
\item{\textnormal{\textbf{(Correctness)}}} For a fixed $b \in \{0,1\}$ we have
  \begin{align*}
    \underset{\substack{\rho_S \in \{0,1\}^{n}, \rho_R \in \{0,1\}^{*} \\
        \Gamma_V := \langle S_1(b,\rho_S), R(\rho_R) \rangle_{R} \\
        (b',y) := S_2(\langle S_1(b,\rho_s), R(\rho_R) \rangle_{\mathit{trans}},\rho_S)}}{\Pr}\Big[\Gamma_V(b',y) = 1 \Big] \geq 1 - \epsilon(n),
  \end{align*}
where $\epsilon(n)$ is a negligible function of $n$.
\item{\textnormal{\textbf{(Hiding)}}}
  \begin{todo}
    \textbf{TODO:} Describe it using equations, define somehow the guess of R? Maybe as a last message in the first phase of communication
  \end{todo}
  Probability over random coins of $S_n$ and $R_n$ that any polynomial size circuit
  can guess bit $b$ correctly after the commit phase is at most $\frac{1}{2} + \epsilon(n)$ where $\epsilon(n)$ is a negligible function of $n$.
  \begin{todo}
    \textbf{TODO:} What is $y_0$ and $y_1$
  \end{todo}
\item{\textnormal{\textbf{(Binding)}}}
  For every polynomial size circuit $S_n$ we have
  \begin{align*}
    \underset{\substack{
        \rho_S \in \{0,1\}^{n}, \rho_R \in \{0,1\}^{*} \\
        \Gamma_V := \langle S_1(b,\rho_S), R(\rho_R) \rangle_{R} \\ (b',y) := S_2(\rho_S)}}{\Pr}[\Gamma_V(0,y_0) = 1 \land \Gamma_V(1,y_1) = 1] \leq \epsilon(k),
  \end{align*}
  where $\epsilon(n)$ is a negligible function in $n$.
\end{enumerate}
\end{definition}

\begin{todo}
  \textbf{TODO:} This is not clear ... access to the oracle etc.\\
  \textbf{TODO:} Why it is not possible to verify -- i.e. the sender does not even
  know the function that is used by the receiver to validate decommitment.\\
  \textbf{TODO} $Q=1$
\end{todo}

The bit commitment protocols can be generalized as an interactive weakly verifiable puzzle as follows.
The number of hint queries amounts to zero and the number of the verification queries is at most one.
The sender corresponds to the problem solver, and the receiver is the problem poser.
Additionally, we require the problem solver to ask a verification query $(b',y)$ only on $b' := \lnot b$ where $b$
is a bit to which the problem solver is committed after the first phase.
The first phase corresponds to the commit phase.
The second phase is the reveal phase where the problem poser tries to find a bitstring $y$ such that $\Gamma_V(\lnot b, y) = 1$.

\subsection{Automated Turing Tests}
The goal of \textit{Automated Turing Tests} is to distinguish humans from computers which
is frequently used to prevent computer programs from accessing resources for humans.
An example is \textit{CAPTCHA} defined in \cite{von2003captcha}.
Loossly speaking, CAPTCHA is a test that human can solve with probability close to 1, but it is hard to write a computer program
that has a success probability comparable to the one achieved by humans.
An example of CAPTACHA is an image depicting a distorted text. Most humans guess the text which is displayed on the image correctly, but it might be hard to write
a program for which it would also be easy. We note that the definition of hardness has not been particular well defined
and bases on opinions of AI community.
\begin{todo}
  \textbf{TODO:} Is it dynamic non-interactive?
\end{todo}
CAPTCHAs based on guessing the distorted text are weakly verifiable puzzles.
In the first round the problem poser and problem solver engage in interactive protocol, such that
after the execution of the protocol the problem poser has a way to verify the solution.
The problem poser in the second round takes as input a distorted image, and try to guess the text that was used to generated it.
The standard CAPTCHAs are non-dynamic, as the problem poser does not gain access to the hint oracle and
asks only a single verification query.

Our definition captures also the above type of problems, additionally it is also applicable in the broader context for a different
AI problems.

As it is not know how good the possible algorithm can be to recognize CAPTCHA it is likely that the gap between human
performance and a performance of computer programs may be small. Therefore, it is of interest to find a way to amplify this gap.
It turns out that it is indeed possible which for not dynamic puzzles was proved in \cite{DBLP:journals/corr/abs-1002-3534}.
The proof we give in Chapter \ref{ch:main_result} applies also to the dynamic context.

\begin{todo}
  \textbf{TODO:} Give an optimization problem for gap amplification
\end{todo}

% why is it hard to automatically check the solution for CAPTCHAs
%todo define information theoretic security
% \subsection{Information theoretically secure constructions}
% The definition presented in \ref{def:dwvp} applies also to information theoretic secure constructions.

%%% Local Variables:
%%% mode: latex
%%% TeX-master: "thesis"
%%% End:
