%
In the previous chapter we formally defined dynamic interactive weakly verifiable puzzles
and stated the hardness amplification theorem for this type of puzzles.
In this chapter we give an overview of cryptographic primitives that can be seen as various types of weakly verifiable puzzles.
First, in Section \ref{section:mac} we describe message authentication codes which the game based security definition can be seen being a dynamic
non-interactive weakly verifiable puzzle.
In Section \ref{section:pks} we give the chosen-message attack security definition
for public key signature schemes which is yet another example of a dynamic non-interactive weakly verifiable puzzle.
Then, in Section \ref{section:bc} we focus on commitment protocols, which binding property can be seen as a non-dynamic interactive  weakly verifiable puzzle.
Finally, in Section \ref{section:att} we describe a problem of solving CAPTCHAs that can be considered to be a non-dynamic and non-interactive weakly verifiable puzzle.

\section{Message Authentication Codes}
\label{section:mac}
In this section we describe message authentication codes (MACs) and conclude that the game based definition of MAC
security can be seen being a dynamic non-interactive weakly verifiable puzzle.

Let us consider the setting in which two parties a \textit{sender} and a \textit{receiver} communicate over an \textit{insecure channel} where
messages of the sender may be intercepted, modified, and deleted by a third party called an \textit{adversary}.
The receiver needs a way to ensure that the received messages were indeed sent by the sender and were not modified by the adversary.
The solution is to use \textit{message authentication codes}.

Loosely speaking, MAC can be explained as follows.
Let the sender, receiver, and adversary be polynomial time algorithms and messages be represented as bitstrings.
Furthermore, we assume that the sender and the receiver share a secret key to which the adversary has no access.
The sender appends to every message a tag which is computed as a function of the key and the message.
The receiver, using the key, has a way to efficiently check whether an appended tag is valid for a received message.
The receiver accepts a message if the tag is valid, otherwise it rejects.
We require that it is hard for the adversary to find a tag and a message that was not sent before and
with non-negligible probability is accepted by the receiver.

Below we give the formal definition of a \textit{message authentication code} based on \cite{LectureNotesCrypo, Goldreich:2004:FCV:975541}.
%
\begin{definition}[Message Authentication Code]
  \label{def:mac}
  Let $\cM$ be a set of messages, $\cK$ a set of keys, and $\cT$ a set of tags.
  A \textit{message authentication code} (MAC) is defined by an efficiently computable function $f:\cM \times \cK \rightarrow \cT$.
  We say that a MAC is \textit{secure} if $f$ satisfies the following condition:

  Let $k$ be chosen uniformly at random from $\cK$ and $H$ be a polynomial size circuit with hard-coded $k$
  that takes as input a message $m \in \cM$ and outputs a tag $t \in \cT$ such that $f(m,k) = t$.
  We say that a MAC is secure if there is no probabilistic polynomial time algorithm
  with oracle access to $H$ that with non-negligible probability outputs $m \in \cM$
  as well as $t \in \cT$ such that $f(m, k) = t$ and $H$ has not been queried on $m$.
\end{definition}

We describe how the security definition of a MAC can be seen as a dynamic weakly verifiable puzzle
where at most one verification query is asked.
For fixed $f$ and $n \in \N$ the sender corresponds to the problem poser and the adversary to the solver.
Furthermore, the key $k_{\pi}$ depends on the randomness $\pi \in \{0,1\}^{n}$ used by the problem poser.
The set $\cQ$ is the set of messages $\cM$.

In the first phase, which is non-interactive, neither the problem poser nor the solver send any messages.
The problem poser outputs a hint circuit $\Gamma_H$ and a verification circuit $\Gamma_V$ where both circuits
have hard-coded $\pi$.
The circuit $\Gamma_H$ corresponds to the circuit $H$ from Definition \ref{def:mac} and takes as input
a message $m$ and outputs a tag $t$ such that $f(m, k_{\pi}) = t$.
The circuit $\Gamma_V$ takes as input $m \in \cM$ and $t \in \cT$ and outputs $1$ if and only if $f(m, k_{\pi}) = t$.
In the second phase, the solver takes no input ($x$ is empty string) and has oracle access to $\Gamma_H$, $\Gamma_V$ and
can make at most one verification query.

Thus, the task of the adversary to find a valid tag $t \in \cT$ for a message $m \in \cM$ such that the hint query for $m$
has not been asked before corresponds to making a successful verification query by the solver.
%
\section{Public Key Signature Schemes}
\label{section:pks}
In this section we describe \textit{public key signature schemes} (SIGs).
The game-based security definition of SIGs for a chosen message attack can be seen as a dynamic,
non-interactive weakly verifiable puzzle.

We now give a definition of a public key encryption scheme, and what it means for such a scheme to be secure.
These definitions are based on \cite{Goldreich:2004:FCV:975541}.

\begin{definition}[Public key signature scheme]
Let $\cQ$ be a set of messages. A \textit{public key signature scheme} is defined by a triple of polynomial time algorithms:
$G$ -- a probabilistic key-generation algorithm,
$V$ -- a verification algorithm,
$S$ -- a probabilistic signing algorithm,
such that the following conditions are satisfied:
\begin{enumerate}[-]
  \item The key-generation algorithm $G(1^n; \rho)$ outputs a pair of bitstrings $k_{priv} \in \{0,1\}^{*}$ and $k_{pub} \in \{0,1\}^{*}$ where $n \in \N$ is a security
    parameter and $\rho_G$ the randomness used by $G$. We call $k_{priv}$ a \textit{private key} and $k_{pub}$ a \textit{public key}.
  \item The signing algorithm $S$ takes as input $k_{priv}$, $q \in \cQ$, uses the randomness $\rho_S$ and outputs a signature $s \in \cS$.
  \item The verification algorithm $V$ takes as input $k_{pub}$, $q \in \cQ$, and $s \in \cS$ and outputs $b \in \{0,1\}$\footnote{For simplicity we assume that the verification
algorithm is deterministic.}.
  \item For every $k_{priv}$, $k_{pub}$ output by $G$ and every $q \in \cQ$ it holds
    \begin{align*}
      \Pr[V(k_{pub}, q, S(k_{priv}, q)) = 1] = 1,
    \end{align*}
    where the probability is over the randomness of $S$.
\end{enumerate}
\end{definition}
We say that $s \in \cS$ is a \textit{valid signature} for $q \in \cQ$ if and only if $V(k_{pub}, q, s) = 1$.
We define the security of public key signature scheme as follows.
%
\begin{definition}[Security of a public key signature scheme with respect to a chosen message attack]
  \label{def:sec_sig}
Let $H$ be a polynomial size probabilistic circuit that has hard-coded $k_{priv}$ and takes as input $q$ and outputs $S(k_{priv}, q)$.
An \textit{adversary} $A$ is a probabilistic polynomial time algorithm that takes as input $k_{pub}$ and has oracle access to $H$.
We say that $A$ \textit{succeeds} if it finds a valid signature $s \in \cS$ for a message $q \in \cQ$, and the oracle $H$ has not been queried for a signature of $q$.
The public key encryption scheme is \textit{secure} if there is no polynomial time adversary that succeeds with non-negligible probability.
\end{definition}
%
It seems that a game of breaking the security of a public key signature scheme is not weakly verifiable as the adversary can use $V$
to efficiently check the correctness of a solution. However, it is easy to see that this situation can be modeled as
a dynamic weakly verifiable puzzle where the solver has access to the verification circuit and can ask a polynomial number of verification queries.

More precisely, we now describe how breaking the security of a public key signature scheme can be
seen as a dynamic non-interactive weakly verifiable puzzle.

The problem poser corresponds to the key-generation algorithm $G$ and the solver to the adversary. The set $\cQ$ is a set of messages.
In the first phase, the problem poser obtains $k_{pub}$, $k_{priv}$ and sends $k_{pub}$ to the solver.
Then, the problem poser generates a hint circuit $\Gamma_H$ and a verification circuit $\Gamma_V$.
The hint circuit $\Gamma_H$ corresponds to the circuit $H$ from Definition \ref{def:sec_sig} and takes as input $q \in \cQ$ and outputs a valid signature for $q$.
The verification circuit $\Gamma_V$ takes as input $s \in \cS$ and $q \in \cQ$ and outputs $1$ if and only if $s \in \cS$ is a valid signature for $q \in \cQ$.
In the second phase, the solver takes as input a transcript of messages from the first phase which consists solely of $k_{pub}$.
Additionally, the solver has oracle access to $\Gamma_V$ and $\Gamma_H$ and can make a polynomial number of queries to $\Gamma_H$ and $\Gamma_V$.
The adversary calls to $V$ can be simulated by making a verification query to $\Gamma_V$.
It is clear that if the solver asks a successful verification query $(q,s)$,
then there exists an adversary that also finds a valid signature for $q$.

Thus, the task of finding a valid signature in a public key signature scheme can be seen as a weakly verifiable puzzle that
is dynamic but non-interactive as in the first phase only a single message is sent.

\section{Bit Commitment Protocols}
\label{section:bc}
In this section we describe \textit{bit commitment protocols}. The binding property of
a bit commitment protocol can be seen as an interactive non-dynamic weakly verifiable puzzle where $|\cQ| = 1$.

Loosely speaking, we consider the following \textit{bit commitment protocol} that involves two parties, a \textit{sender} and a \textit{receiver}.
We suppose that the sender and the receiver are polynomial time probabilistic algorithms.
The protocol consists of a \textit{commit phase} and a \textit{reveal phase}.
In the commit phase the sender and the receiver interact, as the result the sender commits to a value $b \in \{0,1\}$.
We require that after the commit phase it is hard for the receiver to guess $b$ correctly.
In the reveal phase the sender opens the commitment by sending to the receiver a pair $(b', y)$ where $y \in \{0,1\}^{*}$
should convince the receiver that the sender committed to the value $b' \in \{0,1\}$.
A desirable property of a bit commitment protocol is that in the reveal phase it is hard for
the sender to find two bitstrings $y_0$ and $y_1$ such that the receiver recognizes both $(0,y_0)$ and $(1, y_1)$ to be valid decommitments.

We base the following definition of a \textit{bit commitment protocol} on \cite{LectureNotesComThCrypto}.
\begin{definition}[Bit Commitment Protocol]
  \label{def:bit_commitment}
A \textit{bit commitment protocol} is defined by families of circuits $\{S_n\}_{n \in \N}$ and $\{R_n\}_{n \in \N}$
where $S_n = (S_1, S_2)$ is a two-phase probabilistic circuit, $R_n$ is a probabilistic circuit, and
$n~\in~\N$ is a security parameter. We call $S_n$ a \textit{sender} and $R_n$ a \textit{receiver}.
The circuit $S_1$ takes as input a pair $(b, \rho_S)$ where $b \in \{0,1\}$ is interpreted as a bit to which $S$ commits, and $\rho_S \in \{0,1\}^{*}$ is the randomness used by $S$.
The receiver $R_n$ uses the randomness $\rho_R \in \{0,1\}^{*}$.
The protocol consists of two phases. In the \textit{commit phase}, $S_1$ and $R_n$ engage in the protocol execution.
As the result, $S_n$ commits to $b$ and $R_n$ generates a circuit $V$.
The circuit $V$ takes as input $b' \in \{0,1\}$ and $y \in \{0,1\}^{*}$ and outputs a bit.
In the \textnormal{reveal phase}, the circuit $S_2$ takes as input a communication transcript from the commitment phase
$\langle S_1, R_n \rangle_{\mathit{trans}}$, the randomness $\rho_s$ and returns $(b', y)$.
We require a bit commitment protocol to have the following properties:
\begin{enumerate}[]
\item{\textnormal{\textbf{Correctness.}}} For a fixed $b \in \{0,1\}$, we have
  \begin{align*}
    \underset{\substack{\rho_S \in \{0,1\}^{*}, \rho_R \in \{0,1\}^{*} \\
        V := \langle S_1(b,\rho_S), R_n(\rho_R) \rangle_{R_n} \\
        (b',y) := S_2(\langle S_1(b,\rho_s), R_n(\rho_R) \rangle_{\mathit{trans}},\rho_S)}}{\Pr}\Big[V(b',y) = 1 \Big] \geq 1 - \epsilon(n),
  \end{align*}
where $\epsilon(n)$ is a negligible function of $n$.
\item{\textnormal{\textbf{Hiding.}}}
  For every $b \in \{0,1\}$ the probability over the randomness of $S_n$ and $R_n$ that any polynomial size circuit
  can guess $b$ correctly after the commit phase is at most $\frac{1}{2} + \epsilon(n)$ where $\epsilon(n)$ is a negligible function of $n$.
\item{\textnormal{\textbf{Binding.}}}
  For every probabilistic polynomial size two-phase circuit $S_n^*(\rho_S) := (S^*_1, S^*_2)(\rho_S)$ we have
  \begin{align*}
    \underset{\substack{
        \rho_S \in \{0,1\}^{*}, \rho_R \in \{0,1\}^{*} \\
        V := \langle S_1^*(b,\rho_S), R_n(\rho_R) \rangle_{R_n} \\ ((0, y_0), (1, y_1)) := S_2^*(\rho_S)}}{\Pr}[V(0,y_0) = 1 \land V(1,y_1) = 1] \leq \epsilon(n),
  \end{align*}
  where $\epsilon(n)$ is a negligible function in $n$.
\end{enumerate}
\end{definition}

Breaking the binding property of a bit commitment protocol can be seen as solving an interactive weakly verifiable puzzle.
The problem poser corresponds to the receiver and the solver to the circuit that tries to break the binding property of the bit commitment protocol.
% The number of hint queries is zero. The number of the verification queries is at most one. Thus, we the puzzle is non-dynamic.
% In the first phase, the solver uses its randomness to choose a bit to which it commits. Then, it engage in interaction with the problem poser.
When the interaction in the commit phase is completed the problem poser generates circuits $\Gamma_V$, $\Gamma_H$.
The circuit $\Gamma_V$ takes as input $y_0 \in \{0,1\}^{*}$, $y_1 \in \{0,1\}^{*}$ and outputs $1$
if and only if $V(0,y_0) = 1 \land V(1,y_1) = 1$. The hint circuit $\Gamma_H$ outputs $\bot$ for any query.
Thus, without loss of generality, we assume that the solver does not ask any verification queries.
Furthermore, $|\cQ| = 1$, therefore we do not write explicitly on which $q \in \cQ$ the solver asks a verification query.
In the second phase, the solver is given oracle access to $\Gamma_V$ and is allowed to ask at most one verification query.
% We emphasize that the solver has only oracle access to $\Gamma_V$. Therefore, it has no efficient way to check whether
% the solution is successful except asking a verification query.
Finally, we conclude that for the problem poser and the solver defined as above making a successful verification query
corresponds to breaking the binding property of a bit commitment protocol.

\section{Automated Turing Tests}
\label{section:att}
In this section we describe CAPTCHAs which can be seen as non-dynamic
and non-interactive weakly verifiable puzzles.

The goal of \textit{automated Turing tests} is to distinguish humans from computer programs.
An example of such a test is a \textit{CAPTCHA}, which is formally defined in \cite{von2003captcha}.
% Loosely speaking, a CAPTCHA is a test for which it is hard to write a computer program whose success
% probability is comparable to or higher than the one achieved by humans.
An example of a CAPTCHA is an image depicting a distorted text where the task is to guess the text used to generate the image.

Solving a CAPTCHA can be seen as a non-dynamic, non-interactive weakly verifiable puzzle.
In the first phase the problem poser sends the solver an image containing a distorted text.
Then, the problem poser generates circuits $\Gamma_V$ and $\Gamma_H$.
The circuit $\Gamma_V$ takes as input a bitstring $y$ and outputs $1$
if and only if $y$ correctly encodes the text depicted in the distorted image.
A hint circuit $\Gamma_H$ outputs $\bot$ in response to every query.
Therefore, without loss of generality, we assume that the solver makes no hint queries.

In the second phase, the solver takes as input the image, has oracle access to $\Gamma_V$
and can ask at most a single verification query. In general, for CAPTCHAs checking the correctness of a solution is
comparably hard to finding a correct solution.
Thus we conclude that a task of correctly guessing a text depicted in the image can be seen as a weakly verifiable non-interactive and non-dynamic puzzle.

It is unknown how good algorithms for solving a CAPTCHA can be. Thus, it is likely that a gap between the human
performance and the performance of the best computer programs may be small. We are interested in finding a way to amplify this gap.
In the papers \cite{holenstein2011general, dodis2009security} it was shown that this is possible by means of parallel repetition.

% is used where a solver is given $n$ independent weakly verifiable puzzles.
% The verifier for the parallel repetition accepts if the solver succeeds on at least some fraction of these puzzles.

% In Chapter \ref{ch:intro_weakly} we define the $k$-wise direct product of puzzles and state the similar theorem about hardness amplification
% for dynamic weakly verifiable puzzles as in \cite{DBLP:journals/corr/abs-1002-3534}.
% If this hardness amplification theorem was true, then it would be possible to construct a weakly verifiable puzzle
% for which is very hard for even the best computer programs but is still very easy for humans
% (under the assumption that humans have a noticeable advantage over computer programs).

%%% Local Variables:
%%% mode: latex
%%% TeX-master: "thesis"
%%% End:
