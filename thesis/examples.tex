\section{Examples}
\label{section:wvp_examples}
In this section we give examples of cryptographic constructions that are dynamic weakly verifiable puzzles.

\subsection{Message Authentication Codes}
\begin{todo}
  \textbf{TODO:} Runtime of problem poser and problem solver.
\end{todo}
We consider the setting in which two parties a \textit{sender} and a \textit{receiver} communicate over an insecure channel.
The messages of the sender may be read, modified, and replaced by a third party called an \textit{adversary}.
The receiver needs a way to ensure that received messages have been indeed sent by the sender and have not been modified by the adversary.
The solution is to use \textit{message authentication codes}.

Loosely speaking, the message authentication codes may be explained as follows.
Let sender, receiver, and adversary be polynomial time algorithms, and messages be represented as bitstrings.
Furthermore, we assume that the sender and the receive share a secrete key to which an adversary has no access.
The sender appends to every message a tag which is computed as a function of the key and the message.
The receiver, using the key, has a way to check whether an appended tag is valid for a received message.
The receiver accepts a message if the tag is valid, otherwise it rejects.
We require that it is hard for the adversary to find a tag and a message that is accepted by the receiver with non-negligible probability.
We give the following formal definition of \textit{Message Authentication Codes} based on \cite{LectureNotesCrypo} and \cite{Goldreich:2004:FCV:975541}.
\begin{definition}[Message Authentication Codes]
  Let $\cM \subseteq \{0,1\}^*$ be a set of messages, $\cK \subseteq \{0,1\}^{n}$ a set of keys and $\cT \subseteq \{0,1\}^*$ a set of tags where $n \in \N$.
  We define the \textnormal{message authentication code (MAC)} as am efficiently computable function $\cM \times \cK \rightarrow \cT$.
  Furthermore, we say that MAC is \textit{secure} if it satisfies the following condition:

  Let $k \xleftarrow{\$} \cK$ and $\Gamma_H: \cM \rightarrow \cT$ be a polynomial size circuit computing
  a tag for a message where a key is fixed to $k$. We say that MAC is secure if there is no probabilistic polynomial time algorithm with oracle access to $\Gamma_H$
  that with non-negligible probability outputs a message $m \in \cM$, as well as a corresponding tag $t \in \cT$ such that $f(m, k) = t$,
  and $\Gamma_H$ has not been queried for a tag of message $m$.

  % We say that a polynomial time algorithm $A$ with oracle access to $\Gamma_H$ \textit{breaks the security} of MAC if it find a message
  % $m$ and $t$ such that $\Gamma_V(m',t') = 1$ with probability .
\end{definition}
%
We show how MAC is captured by the dynamic interactive weakly verifiable puzzles.
For fixed $n$ the sender corresponds to a problem poser, the adversary to a problem solver, and
the $k$ key is a bitstring $\pi \in \{0,1\}^{n}$ taken as auxiliary input by a problem poser.
In the first phase, which is non interactive, the problem poser outputs a hint circuit
$\Gamma_H: \cM \rightarrow \cT$ that given a message computes a tag
and a verification circuit $\Gamma_V: \cM \times \cT \rightarrow \{0,1\}$ that on input $m \in \cM$ and $t \in \cT$
outputs one if and only if $f(m, k) = t$.

In the second phase the adversary takes no input ($x^*$ is empty string), as the first phase was non-interactive, and
is given oracle access to $\Gamma_H$ and $\Gamma_V$.
A task of finding by an adversary a valid tag $t' \in \cT$ for a message $m' \in \cM$ such that a hint for $m'$ has no been asked before
corresponds to asking a successful verification query by a problem poser to $\Gamma_V$.
% Conversely, if there is no polynomial time algorithm that succeeds in solving this dynamic weakly verifiable puzzle with non-negligible probability
% then MAC is secure.
%
\subsection{Public Key Signature Scheme}
% \begin{todo}
%   \textbf{TODO:} Add introduction that gives intuition about the Public Key Signature Schemes
% \end{todo}
First we give a definition of public key encryption scheme, and what it means for such a scheme to be secure.
These definitions are based on \cite{Goldreich:2004:FCV:975541}.

\begin{todo}
  \textbf{TODO:} Length of public and private key
\end{todo}

\begin{definition}[Public key signature scheme]
Let $\cQ$ be the set of messages. A \textnormal{public key signature scheme} is defined by a triple of probabilistic polynomial time algorithms:
$G$ -- the key generation algorithm,
$V$ -- the verification algorithm,
$S$ -- the signing algorithm,
such that the following conditions are satisfied:
\begin{enumerate}[-]
  \item $G(1^n)$ outputs a pair of bitstrings $k_{priv} \in \{0,1\}^{n}$ and $k_{pub} \in \{0,1\}^{n}$ where $n$ is a security parameter.
    We call $k_{priv}$ a private key and $k_{pub}$ a public key.
  \item The signing algorithm $S$ takes as input $k_{priv}$, $q \in \cQ$ and outputs a signature $s \in S$.
  \item The verification algorithm takes as input $k_{pub}$, $q \in \cQ$, and $s \in S$ and outputs a bit $b \in \{0,1\}$.
  \item For every $k_{priv}$, $k_{pub}$ output by $G$ and every $q \in \cQ$ it holds
    \begin{align*}
      \Pr[V(k_{pub}, q, S(k_{priv}, q))] = 1,
    \end{align*}
    where the probability is over the random coins of $V$ and $S$.
\end{enumerate}
\end{definition}
We say that $s \in S$ is a \textit{valid} signature for $q \in \cQ$ if and only if $V(k_{pub}, q, s) = 1$.
%
%TODO efficiency of the algorithms
%
\begin{definition}\textbf{(Security of public key signature scheme with respect to a chosen message attack)}
Let an \textit{adversary} be a probabilistic polynomial time algorithm that takes as input $k_{pub}$ and has oracle access to $S$.
We say that the adversary \textnormal{succeeds} if it finds a signature $s \in S$ for a message $q \in \cQ$ such that $V(k_{pub}, q, s) = 1$,
and the oracle $S$ has not been queried for a signature of $q$.
The public key encryption scheme is \textnormal{secure} if there is no polynomial time algorithm that succeeds with non negligible probability.
\end{definition}

We show now that the public key signature schemes defined as above can be represented as a dynamic interactive weakly verifiable puzzle.
In the first phase the problem poser uses algorithm $G(1^n)$ to obtain $k_{pub}$, $k_{priv}$ and sends to the adversary the public key $k_{pub}$.
The problem poser generates a hint circuit $\Gamma_H$ and a verification circuit $\Gamma_V$.
The hint circuit takes as input $q \in \cQ$ and outputs a signature for $q$. The verification circuit
takes as input $s \in S$ and $q \in Q$ and checks whether $s \in S$ is a valid signature for $q \in \cQ$.
In the second phase the problem solver takes as input a transcript of message from the first round which consists of a single message $k_{pub}$.
Additionally, it gain oracle access to $\Gamma_V$ and $\Gamma_H$.
It is clear that if the adversary asks a successful verification query $(q,s)$, then it also breaks the security of a public key signature scheme.

Public key signature schemes are types of puzzles that are dynamic but are not interactive as in the first phase only a single message is sent.
%
\subsection{Bit Commitments}
Let us consider the following \textit{bit commitment protocol} that involves two parties a \textit{sender} and a \textit{receiver}.
We suppose that the sender and the receiver are polynomial time probabilistic algorithms.
The protocol consists of a \textit{commit phase} and a \textit{reveal phase}.
In the commit phase the sender and the receiver interact, as the result the sender commits to a value $b \in \{0,1\}$.
In the reveal phase the sender opens the commitment by sending to the receiver $(y,b')$ where $y \in \{0,1\}^{*}$ and $b' \in \{0,1\}$.
We require that after the commit phase it is hard for the receiver to correctly guess $b$.
Additionally, in the \textit{reveal phase} it should be hard for the sender to persuade the receiver that it was committed to the value $\lnot b$.

We base the following definition of \textit{bit commitment protocol} on \cite{LectureNotesComThCrypto}.
\begin{definition}[Bit Commitment Protocol]
  \label{def:bit_commitment}
For a security parameter $n \in \N$ a \textnormal{bit commitment protocol} is defined by a pair $(S_n, R_n)$
where $S_n = (S_1, S_2)$ is a two phase probabilistic circuit, and $R_n$ is a probabilistic circuit.
The circuit $S_1$, used in the commit phase, takes as input a tuple $(b, \rho_S)$ where $b \in \{0,1\}$ is interpreted as a bit to which $S_n$
commits and $\rho_S \in \{0,1\}^{n}$ is the randomness used by the algorithm $S_n$.
The receiver $R_n$ takes only auxiliary input $\rho_R \in \{0,1\}^{n}$ that is the randomness used by $R_n$.
The protocol consists of two phases. In the commit phase circuits $S_1$, $R_n$ engage in the protocol execution.
As the result $S_1$ commits to $b$ and generates a circuit $\Gamma_V: \{0,1\} \times \{0,1\}^{*} \rightarrow \{0,1\}$.
In the reveal phase the circuit $S_2$ returns $(b', y)$. For fixed $b \in \{0,1\}$ and $n$ we require the bit commit protocol to have the following properties:
\begin{enumerate}[]
\item{\textnormal{\textbf{(Correctness)}}} For a fixed $b \in \{0,1\}$ we have
  \begin{align*}
    \underset{\substack{\Gamma_V := \langle S_1, R \rangle_{R} \\ (b,y) := S_2(k_{pub}, \rho_S) }}{\Pr}\Big[V(b,y) = 1 \Big] \geq 1 - \epsilon(n),
  \end{align*}
where $\epsilon(n)$ is a negligible function of $n$.
\item{\textnormal{\textbf{(Hiding)}}}
  \begin{todo}
    \textbf{TODO:} Describe it using equations, define somehow the guess of R? Maybe as a last message in the first phase of communication
  \end{todo}
  Probability over random coins of $S_n$ and $R_n$ that any polynomial size circuit
  can guess bit $b$ correctly after the commit phase is at most $\frac{1}{2} + \epsilon(n)$ where $\epsilon(n)$ is a negligible function of $n$.
\item{\textnormal{\textbf{(Binding)}}}
  For every polynomial size circuit $S_n$ we have
  \begin{align*}
    \underset{\substack{\Gamma_V := \langle S_1, R \rangle_{R} \\ (b,y) := S_2(k_{pub}, \rho_S)}}{\Pr}[\Gamma_V(0,y_0) = 1 \land V(1,y_1) = 1] \leq \epsilon(k),
  \end{align*}
  where $\epsilon(k)$ is a negligible function in $k$.
\end{enumerate}
\end{definition}

The bit commitment protocols can be generalized as dynamic interactive weakly verifiable puzzles for a
case where the number of hint queries amounts to zero and the number of the verification queries is at most one.
The sender corresponds to a problem solver, and the receiver is a problem poser.
Additionally, we require the problem solver to ask a verification query only on $q := \lnot b$ where $b$
is a bit to which the problem solver is committed after the first phase.
The first phase corresponds to the commit phase.
In the reveal phase the problem poser tries to find a bitstring $y$ such that $V(\lnot b, y) = 1$.

\subsection{Automated Turing Tests}
The goal of \textit{Automated Turing Tests} is to distinguish humans from computers which
is frequently used to prevent computer programs from accessing resources for humans.
An example is \textit{CAPTCHA} defined first in \cite{von2003captcha}.
Loossly speaking, CAPTCHA is a test that human can solve with probability close to 1, but it is hard to write a computer program
that has a success probability comparable to the one achieved by humans.
An example of CAPTACHA is an image depicting a distorted text. Most humans guess the text which is displayed on the image correctly, but it might be hard to write
a program for which it would also be easy. We note that the definition of hardness has not been particular well defined ,
and bases on opinions AI community opinions that distinguish between hard and easy AI problems \cite{von2003captcha}.

CAPTCHAs based on guessing the distorted text are weakly verifiable puzzles.
In the first round the problem poser and problem solver engage in interactive protocol, such that
after the execution of the protocol the problem poser has a way to verify the solution.
The problem poser in the second round takes as input a distorted image, and try to guess the text that was used to generated it.
The standard CAPTCHAs are non-dynamic, as the problem poser does not gain access to the hint oracle and
asks only a single verification query.

Our definition captures also the above type of problems, additionally it is also applicable in the broader context for a different
AI problems.

As it is not know how good the possible algorithm can be to recognize CAPTCHA it is likely that the gap between human
performance and a performance of computer programs may be small. Therefore, it is of interest to find a way to amplify this gap.
It turns out that it is indeed possiThe first ble which for not dynamic puzzles was proved in \cite{DBLP:journals/corr/abs-1002-3534}.
The proof presented in Chapter \ref{ch:main_result} applies also to the dynamic context.

\begin{todo}
  \textbf{TODO:} Give an optimization problem for gap amplification
\end{todo}



%%% Local Variables: 
%%% mode: latex
%%% TeX-master: "thesis"
%%% End: 
