A natural question is whether one can improve the result in Theorem \ref{th:sec_amp_for_dwvp} such that a solver
for the $k$-wise direct product of puzzles that satisfies
\begin{align}
    \label{weakerAssumptions}
    \underset{\substack{\pi^{(k)} \in \{0,1\}^{kn} \\ \rho \in \{0,1\}^{*}}}{\Pr}\left[\mathit{Success}^{P_{kn}^{(g)}, C}(\pi^{(k)}, \rho) = 1\right]
    \geq \underset{u \leftarrow \mu_\delta^k}{\Pr}[g(u) = 1] + \varepsilon
\end{align}
suffices to obtain a solver $D$ for a single puzzle that with high probability satisfies
  \begin{align}
    \underset{\substack{\pi \in \{0,1\}^{n} \\ \rho \in \{0,1\}^{*}}}
    {\Pr}\left[\Success^{P_{n}^{(1)},D}(\pi, \rho) = 1\right] \geq \delta + \frac{\epsilon}{6k}.
  \end{align}
In Theorem \ref{th_sec_amp_dwvp_assum} we loose the factor $\frac{1}{16(h+v)}$ because of the domain partitioning that let us avoid collisions between hint and verification queries.

We do not see how to find a more efficient partitioning such that the assumption \eqref{weakerAssumptions} is sufficient.
We also tried unsuccessfully other methods:
\begin{enumerate}[]
  \item showing that the influence of the hint queries can be somehow bounded such that the weaker assumption \eqref{weakerAssumptions} holds. \\
  \item amending the algorithm $\Gen$ such that only one call to the solver for the $k$-wise direct product of puzzles is necessary.
\end{enumerate}

It seems to be likely that we have to loose a factor $\frac{1}{v}$ as we make just a single verification query.
For example let $C_v$ be a solver for the $1$-wise direct product that asks no hint queries and at most $v$ verification queries.
Additionally, $C_v$ succeeds with probability $1$ and a verification query on which it succeeds is chosen uniformly at random from $\{0,1, \cdots, v\}$.
All queries except this one are unsuccessful.
Thus, in a situation where we allow to ask only a single verification query we do not see a way to succeed with probability higher than~$\frac{1}{v}$.

Therefore, we tried to show that the result obtain in Theorem \ref{th:sec_amp_for_dwvp} is optimal.
Full domain hash is a type of sigature scheme that can be seen as dynamic weakly verifiable puzzle.
Coron \cite{coron2000exact, coron2002optimal} proves that security of RSA implies security of FDH.
Loosely speaking, he shows that if there is no efficient algorithm that breaks security of RSA with probability higher
than $\epsilon_{\mathit{RSA}}$ then there is no algorithm that breaks security of FDH with probability higher than
$\epsilon_{\mathit{FDH}} \approx h \cdot \epsilon_{\mathit{RSA}}$ where $h$ is the number of hint queries that a solver makes when solving FDH.
Coron shows that this result is optimal. However, we do not see how the technique used in \cite{coron2002optimal} can be used in our setting.

We also tried to apply the \textit{black box separation technique} by Holenstein and Haitner \cite{haitner2009possibility}.
The intuition, behind our approach was to use a solver for the $k$-wise direct product that asks a large number of hint queries such that
the success probability of $C$ in every next run decreases drastically.
% In the technique of \cite{haitner2009possibility} a impossibility of a black box reduction is shown between the problem that has access
% to some possibility computationally unbounded oracle for a certain problem. It is shown that it is possible to solve a certain problem,
% but not with high probability not possible to solve a different problem (write it more formally).

However, in our setting it only makes sense to consider situations where the number of hint and verification queries is bounded by some polynomial.
Furthermore, the solver for the $k$-wise direct product cannot solve puzzle on each position independently (in this setting it is possible
to prove the hardness amplification with a bound as in \ref{weakerAssumptions}).
Finally, as we assume that the solver for the $k$-wise direct product of puzzles may not obtain direct access to hint and verification oracles.
Instead, these queries are answered by the solver that runs the solver $D$ for a single puzzle. We cannot exclude a situation where $D$ is dishonest.
This makes the solver fairly complicated so that we do not see a way to apply the technique of \cite{haitner2009possibility} in this setting.

Therefore, the question whether the result in Theorem \ref{th:sec_amp_for_dwvp} is optimal remains open.

% \begin{todo}
%   \textbf{TODO:} Why fixing all puzzles is not possible \\
%   \textbf{TODO:} Why we believe that it is not possible to do any better \\
%   \textbf{TODO:} Mention the number of the verification queries  \\
%   \textbf{TODO:} Try to explain somehow the number of hint queries \\
%   \textbf{TOOD:} What we tried to prove the lower bound \\
%   \textbf{TOOD:} Why it was not possible \\
%   \textbf{TODO:} Why the methods of Coron does not work \\
%   \textbf{TODO:} When does parallel repetition fails to amplify hardness \\
%   - no easy way to handle inefficient solver without visible effect on the number of hint queries \\
%   - complicated breaker + puzzles has to depend on each other, \\
%   - has to ask many hint and verification queries in each phase on the more less the same domain \\
%   - it interacts with the solver that is computationally bounded, hence limitation on the number of hint and verification queries \\
%   - the solver may be dishonest answer the queries with different (not true response)
% \end{todo}

% Local Variables:
% mode: latex
% TeX-master: "thesis"
% End:
