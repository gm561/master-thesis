\subsection{Optimality of the result}
\label{st:discussion}
A natural question is whether one can improve the result in Theorem \ref{th:sec_amp_for_dwvp} such that a solver
for the $k$-wise direct product of puzzles that satisfies
\begin{align}
    \label{weakerAssumptions}
    \underset{\substack{\pi^{(k)} \in \{0,1\}^{kn} \\ \rho \in \{0,1\}^{*}}}{\Pr}\left[\mathit{Success}^{P_{kn}^{(g)}, C}(\pi^{(k)}, \rho) = 1\right]
    \geq \underset{u \leftarrow \mu_\delta^k}{\Pr}[g(u) = 1] + \varepsilon
\end{align}
suffices to obtain a solver $D$ for a single puzzle that with high probability satisfies
  \begin{align}
    \underset{\substack{\pi \in \{0,1\}^{n} \\ \rho \in \{0,1\}^{*}}}
    {\Pr}\left[\Success^{P_{n}^{(1)},D}(\pi, \rho) = 1\right] \geq \delta + \frac{\epsilon}{6k}.
  \end{align}
In Theorem \ref{th_sec_amp_dwvp_assum} the coefficient $\frac{1}{16(h+v)}$ comes from the fact that we need to partition the query domain
to avoid collisions between hint and verification queries.
To improve this result we would need to deal with collisions using a different method or show that the influence of the hint queries that can prevent
verification queries from succeeding is not big such that the theorem holds even when a solver as in \ref{weakerAssumptions} is used.
Unfortunately, we do not see a way to partition the domain in a more efficient manner or to amend the algorithm $\Gen$.

Furthermore, it seems to be natural that the we have to loose a factor of $\frac{1}{v}$ as instead of making $v$ verification queries we make just a single query.
For a case of a solver $C_v$ for the $1$-wise direct product that always solves the puzzle but makes at least $v-1$ unsuccessful verification
queries and a single successful verification and asks no hint queries.
The solver output by $\Gen$ has to guess which of the $v$ queries is the correct one.
It seems that if $C_v$ chooses uniformly at random which of the verification queries it the successful one then the solve $C$ cannot succeed with
probability higher than $\frac{1}{v}$.

Coron \cite{coron2000exact, coron2002optimal} shows the reduction between RSA and full domain hash (FDH)
(which is a type of a signature scheme that can be seen as a dynamic interactive weakly verifiable puzzle).
He shows that the fact that corresponds to the number of hint queries in this reduction is necessary.
Unfortunately, we do not see how the proof technique used by Coron can be applied to our setting.

We also tried to apply the \textit{black box separation technique} by Holenstein and Heitner \cite{haitner2009possibility}.
The intuition, behind our approach was to use a solver for the $k$-wise direct product that asks large amount of hint queries such that
the success probability in next runs decreases drastically.
In the technique of \cite{haitner2009possibility} a impossibility of a black box reduction is shown between the problem that has access
to some possibility computationally unbounded oracle for a certain problem. It is shown that it is possible to solve a certain problem,
but not with high probability not possible to solve a different problem (write it more formally).

However, in our setting it only makes sense to consider situations where the number of hint and verification queries is bounded.
Thus, this solver cannot asks exponential number of queries. Furthermore, the solve for the $k$-wise direct product cannot solve puzzle on each position
independently (in this setting it is possible to prove the hardness amplification with a bound as in \ref{weakerAssumptions}).
These makes the solver fairly complicated so that we do not see a way how to use the technique of \cite{haitner2009possibility} in this setting.


% \begin{todo}
%   \textbf{TODO:} Why fixing all puzzles is not possible \\
%   \textbf{TODO:} Why we believe that it is not possible to do any better \\
%   \textbf{TODO:} Mention the number of the verification queries  \\
%   \textbf{TODO:} Try to explain somehow the number of hint queries \\
%   \textbf{TOOD:} What we tried to prove the lower bound \\
%   \textbf{TOOD:} Why it was not possible \\
%   \textbf{TODO:} Why the methods of Coron does not work \\
%   \textbf{TODO:} When does parallel repetition fails to amplify hardness \\
%   - no easy way to handle inefficient solver without visible effect on the number of hint queries \\
%   - complicated breaker + puzzles has to depend on each other, \\
%   - has to ask many hint and verification queries in each phase on the more less the same domain \\
%   - it interacts with the solver that is computationally bounded, hence limitation on the number of hint and verification queries \\
%   - the solver may be dishonest answer the queries with different (not true response)
% \end{todo}
%
%
% Local Variables:
% mode: latex
% TeX-master: "thesis"
% End:
