A natural question is whether one can improve the result from Theorem \ref{th:sec_amp_for_dwvp}
such that a solver $C$ for the $k$-wise direct product of puzzles that satisfies
\begin{align}
    \label{weakerAssumptions}
    \underset{\substack{\pi^{(k)} \in \{0,1\}^{kn} \\ \rho \in \{0,1\}^{*}}}{\Pr}\left[\mathit{Success}^{P_{kn}^{(g)}, C}(\pi^{(k)}, \rho) = 1\right]
    \geq \underset{u \leftarrow \mu_\delta^k}{\Pr}[g(u) = 1] + \varepsilon
\end{align}
can be used by a solver $D$ for a single puzzle such that with high probability it holds
  \begin{align}
    \underset{\substack{\pi \in \{0,1\}^{n} \\ \rho \in \{0,1\}^{*}}}
    {\Pr}\left[\Success^{P_{n}^{(1)},D}(\pi, \rho) = 1\right] \geq \delta + \frac{\epsilon}{6k}.
  \end{align}
In Theorem \ref{th:sec_amp_for_dwvp} we lose the factor $\frac{1}{16(h+v)}$ because of the domain partitioning that let us avoid collisions between hint and verification queries.

Unfortunately, we do not see how to find a more efficient partitioning such that the assumption \eqref{weakerAssumptions} is sufficient.
We also tried unsuccessfully different approaches:
\begin{enumerate}[-]
  \item Showing that the influence of hint queries can be somehow bounded such that the weaker assumption \eqref{weakerAssumptions} is sufficient.
  \item Amending the algorithm $\Gen$ such that only one call to the solver for the $k$-wise direct product of puzzles is necessary.
\end{enumerate}

Intuitively, it seems that we must loose a factor $\frac{1}{v}$ as we allow a solver $D$ to make just a single verification query.
For example, let us consider a solver $C$ for the $1$-wise direct product that asks no hint queries and at most $v$ verification queries,
among which exactly one is always successful.
Furthermore, this successful verification query is chosen uniformly at random independently from other queries.
Thus, in the situation where we allow to ask only a single verification query (as in Theorem \ref{th:sec_amp_for_dwvp})
we do not see how $D$ could succeed with probability higher than~$\frac{1}{v}$.

We tried to show that the result obtained in Theorem \ref{th:sec_amp_for_dwvp} is optimal.
Firstly, we attempted to apply the \textit{black-box separation technique} by Holenstein and Haitner \cite{haitner2009possibility}.
The intuition behind our approach was to use a solver $C$ for the $k$-wise direct product that asks a large number of hint queries such that
the success probability of $C$, when its execution is repeated, decreases drastically.
In our setting it makes sense to consider situations where the number of hint and verification queries is bounded by some polynomial.
Furthermore, the solver for the $k$-wise direct product cannot solve a puzzle on each position independently
(otherwise it is possible to show that the assumption \eqref{weakerAssumptions} is sufficient).
Finally, we have to take into account that the solver $C$ for the $k$-wise direct product of puzzles does not obtain direct access to hint and verification oracles.
Instead, the queries are answered by a solver circuit $D$ for a single puzzle that runs $C$. Thus, we cannot exclude a situation where the queries of $C$
are answered dishonestly. All this makes it fairly complicated to apply the black-box separation technique to our setting.

We also tried to use the technique of Coron \cite{coron2000exact, coron2002optimal}.
He proves that the security of RSA implies the security of the full domain hash (FDH).
The full domain hash is a signature scheme that can be seen as a dynamic weakly verifiable puzzle.
Loosely speaking, Coron shows that if there is no efficient algorithm that breaks the security of RSA with probability higher
than $\epsilon_{\mathit{RSA}}$, then there is no efficient algorithm that breaks the security of FDH with probability higher than
$\epsilon_{\mathit{FDH}} \approx h \cdot \epsilon_{\mathit{RSA}}$, where $h$ is the number of hint queries made by the solver for FDH.
Coron proves that this result is optimal. However, we do not see how his technique, used to show the optimality result, can be applied to our setting.

Therefore, the question whether our result is optimal remains open.

% \begin{todo}
%   \textbf{TODO:} Why fixing all puzzles is not possible \\
%   \textbf{TODO:} Why we believe that it is not possible to do any better \\
%   \textbf{TODO:} Mention the number of the verification queries  \\
%   \textbf{TODO:} Try to explain somehow the number of hint queries \\
%   \textbf{TOOD:} What we tried to prove the lower bound \\
%   \textbf{TOOD:} Why it was not possible \\
%   \textbf{TODO:} Why the methods of Coron does not work \\
%   \textbf{TODO:} When does parallel repetition fails to amplify hardness \\
%   - no easy way to handle inefficient solver without visible effect on the number of hint queries \\
%   - complicated breaker + puzzles has to depend on each other, \\
%   - has to ask many hint and verification queries in each phase on the more less the same domain \\
%   - it interacts with the solver that is computationally bounded, hence limitation on the number of hint and verification queries \\
%   - the solver may be dishonest answer the queries with different (not true response)
% \end{todo}

% Local Variables:
% mode: latex
% TeX-master: "thesis"
% End:
