\subsection{Optimality of the result}
\label{st:discussion}
A natural question to ask is whether it is possible to prove hardness amplification where a solver for the $k$-wise direct product of puzzles satisfies
\begin{align}
    \label{weakerAssumptions}
    \underset{\substack{\pi^{(k)} \in \{0,1\}^{kn} \\ \rho \in \{0,1\}^{*}}}{\Pr}\left[\mathit{Success}^{P_{kn}^{(g)}, C}(\pi^{(k)}, \rho) = 1\right]
    \geq \underset{u \leftarrow \mu_\delta^k}{\Pr}[g(u) = 1] + \varepsilon.
\end{align}
In Theorem \ref{th_sec_amp_dwvp_assum} the coefficient $\frac{1}{16(h+v)}$ comes from the fact that we need to partition the query domain using the hash function
such that we avoid collisions between hint and verification queries.
To obtain \ref{weakerAssumptions} we would need to find a way to avoid collisions using a different method.
Unfortunately, we did not see an easy way to achieve that.

Furthermore, it seems to be natural that the we have to loose a factor of $\frac{1}{v}$ as instead of asking $v$ verification queries we ask just a single query.
Thus for a trivial case where we have a solver $C_v$ for the $1$-wise direct product always solves the puzzle but asks $v-1$ unsuccessful verification
queries and a single successful verification and asks no hint queries.
The solver output by $\Gen$ has to guess which of the $v$ queries is the correct one.
It is clear, that if the solver $C_v$ chooses uniformly at random which of the verification queries it the successful one then the solve $C$ cannot succeed with
probability higher than $\frac{1}{v}$.

For the verification queries a similar results is obtained by Coron \cite{coron2000exact}.

Thus, we tries to show that the bound stated in Theorem \ref{th:sec_amp_for_dwvp} is optimal using
the black box separation techniques similar to the one \ref{haitner2009possibility}.
The intuition, behind our approach was to use a solver for the $k$-wise direct product that asks many hint queries
such that the success probability in next runs decreases quickly.
In the technique of \cite{haitner2009possibility} a impossibility of a black box reduction is shown between the problem that has access
to some possibility computationally unbounded oracle for certain problem. It is shown that it is possible to solve a certain problem, but
not with high probability not possible to solve a different problem (write it more formally).

However, in our setting it only makes sense to consider situations where the number of hint and verification queries is bounded.
Thus, this solver cannot asks exponential number of queries. Furthermore, the solve for the $k$-wise direct product cannot solve puzzle on each position
independently (in this setting it is possible to prove the hardness amplification with a bound as in \ref{weakerAssumptions}).
These makes the solver fairly complicated so that we do not see a way how to use the technique of \cite{haitner2009possibility} in this setting.


% \begin{todo}
%   \textbf{TODO:} Why fixing all puzzles is not possible \\
%   \textbf{TODO:} Why we believe that it is not possible to do any better \\
%   \textbf{TODO:} Mention the number of the verification queries  \\
%   \textbf{TODO:} Try to explain somehow the number of hint queries \\
%   \textbf{TOOD:} What we tried to prove the lower bound \\
%   \textbf{TOOD:} Why it was not possible \\
%   \textbf{TODO:} Why the methods of Coron does not work \\
%   \textbf{TODO:} When does parallel repetition fails to amplify hardness \\
%   - no easy way to handle inefficient solver without visible effect on the number of hint queries \\
%   - complicated breaker + puzzles has to depend on each other, \\
%   - has to ask many hint and verification queries in each phase on the more less the same domain \\
%   - it interacts with the solver that is computationally bounded, hence limitation on the number of hint and verification queries \\
%   - the solver may be dishonest answer the queries with different (not true response)
% \end{todo}
%
%
% Local Variables:
% mode: latex
% TeX-master: "thesis"
% End:
