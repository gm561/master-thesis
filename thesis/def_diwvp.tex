\section{The Definition}
\label{section:wvp}
Let us combine definitions of puzzles introduced in \cite{dodis2009security} and \cite{holenstein2011general}
and define a \textit{dynamic interactive weakly verifiable puzzle} as follows.
\begin{definition}[Dynamic interactive weakly verifiable puzzle]
  \label{def:dwvp}%
  A~\textit{dynamic interactive weakly verifiable puzzle (DIWVP)} is defined by a family of probabilistic circuits $\{P_n\}_{n \in \N}$.
  A circuit belonging to $\{P_n\}_{n \in \N}$ is called a \textit{problem poser}.
  A \textit{solver} $C := (C_1, C_2)$ is a probabilistic two-phase circuit.
  We write $P_n(\pi)$ to denote the execution of $P_n$ with the randomness fixed to $\pi \in \{0,1\}^n$ and $C(\rho) := (C_1,C_2)(\rho)$
  to denote the execution of both $C_1$ and $C_2$ with the randomness fixed to $\rho \in \{0,1\}^{*}$.

  In the first phase, the problem poser $P_n(\pi)$ and the solver $C_1(\rho)$ interact.
  As a result of the interaction $P_n(\pi)$ outputs a \textit{verification circuit} $\Gamma_{V}$ and a~probabilistic \textit{hint circuit} $\Gamma_{H}$.
  The circuit $C_1(\rho)$ produces no output. The circuit $\Gamma_{V}$ takes as input $q \in \cQ$ (for some set $\cQ$ of indices),
  $y \in \{0,1\}^*$ and outputs a bit. We say that an answer $(q,y)$ is a \textit{correct solution} if and only if $\Gamma_V(q,y) = 1$.
  The circuit $\Gamma_H$ on input $q \in \cQ$ outputs a hint such that $\Gamma_V(q,\Gamma_H(q))~=~1$.

  In the second phase, $C_2$ takes as input $x := \langle P_n(\pi), C_1(\rho) \rangle_{\trans}$
  and has oracle access to $\Gamma_V$ and $\Gamma_H$.
  The execution of $C_2$ with the input $x$ and the randomness fixed to $\rho$
  is denoted by $C_2(x; \rho)$. The queries of $C_2$ to $\Gamma_V$ and $\Gamma_H$ are called \textit{verification queries} and \textit{hint queries}, respectively.
  We say that the circuit $C_2$ \textit{succeeds} if and only if it makes a verification query $(q,y)$ such that $\Gamma_V(q,y)~=~1$,
  and it has not previously asked for a hint query on this $q$.
\end{definition}

If it is clear from the context, we omit the subscript $n$ and write $P$ instead of~$P_n$.

A verification query $(q,y)$ of $C$ for which a hint query on this $q$ has been asked before cannot be a verification query for which $C$ succeeds.
Therefore, without loss of generality throughout this chapter, we make the assumption that $C$ does not ask verification queries on $q$
for which a hint query has been asked before. Moreover, we assume that once $C$ asked a verification query that succeeds,
it does not ask any further hint or verification queries.

There is no loss of generality in assuming that the problem poser and the solver are defined by probabilistic circuits.
Definition \ref{def:dwvp} embraces also the case where the problem poser and the solver are probabilistic polynomial time algorithms.
We use the well known fact \cite{LectureNotesCT} that a probabilistic polynomial time algorithm can be transformed into an equivalent
family of probabilistic Boolean circuits of the polynomial size\footnote{Theorem 6.10 from \cite{LectureNotesCT} is stated for probabilistic, polynomial time,
oracle algorithms with a single bit of output, but it can be adopted to the case where an output is longer than a single bit.}.

We use the term \textit{weakly verifiable} to emphasize that there is no easy way
for the solver to check the correctness of a solution except for asking a verification query.
We call a weakly verifiable puzzle \textit{dynamic} if the number of hint queries is greater than zero.
Furthermore, we say that a weakly verifiable puzzle is \textit{interactive} if in the first
phase the number of messages exchanged between the problem poser and the solver is greater than one.
Finally, we say that a weakly verifiable puzzle is \textit{non-dynamic} if $|Q| = 1$
and \textit{non-interactive} if the number of messages sent in the first phase is at most one.

Definition \ref{def:dwvp} generalizes and combines the previous approaches that study
\textit{weakly verifiable puzzles} \cite{canetti2005hardness},
\textit{dynamic weakly verifiable puzzles} \cite{dodis2009security}, and \textit{interactive weakly verifiable puzzles} \cite{holenstein2011general}.

%
\section{The Hardness Amplification Theorem}
\label{section:hardness_amplification_diwvp}
In this section we define the $k$-wise direct product of puzzles and state the hardness amplification theorem for dynamic interactive weakly verifiable puzzles.

\begin{definition}[$k$-wise direct-product of DIWVPs]
  \label{def:k_wise_direct_product}
  Let $g: \{0,1\}^{k}\!\rightarrow\!\{0,1\}$ be a binary monotone function and $P_n^{(1)}$ a problem poser as in Definition~\ref{def:dwvp}.
  The $k$-wise direct product of $P_n^{(1)}$ is a DIWVP defined by a circuit $P_{kn}^{(g)}$.
  We write $P_{kn}^{(g)}(\pi^{(k)})$ to denote the execution of $P_{kn}^{(g)}$ with the randomness fixed to $\pi^{(k)} := (\pi_1, \dots, \pi_k)$
  where $\pi_i \in \{0,1\}^n$ for all $1 \leq i \leq n$. Let $(C_1, C_2)(\rho)$ be a probabilistic two-phase circuit called a \textit{solver}.
  In the first phase, the problem poser $P_{kn}^{(g)}(\pi^{(k)})$ sequentially interacts in $k$ rounds with the circuit $C_1(\rho)$.
  In the $i$-th round $C_1(\rho)$ interacts with $P_n^{(1)}(\pi_i)$, and as the result $P_{n}^{(1)}(\pi_i)$ generates circuits $\Gamma_V^i, \Gamma_H^i$.
  Finally, after $k$ rounds $P_{kn}^{(g)}(\pi^{(k)})$ outputs a verification circuit
\begin{align*}
  \Gamma_V^{(g)} (q, y_1, \dots, y_k) := g(\Gamma_V^{1}(q, y_1), \dotsc, \Gamma_V^{k}(q, y_k))
\end{align*}
and a hint circuit
\begin{align*}
  \Gamma_H^{(k)} (q) := (\Gamma_H^{1}(q), \dotsc, \Gamma_H^{k}(q)).
\end{align*}
\end{definition}

For the $k$-wise direct product of puzzles we require the solver to make a verification query on a single index $q \in \cQ$.
Otherwise, if verification queries of the form $((q_1,y_1), \dotsc, (q_k, y_k))$ were allowed,
solving the $k$-wise direct product of puzzles would be trivial.
One could fix some $\bar{q} \in \cQ$ and ask
$k$ hint queries such that the $i$-th hint query is of the form $(q_1, \dotsc,q_{i-1}, \bar{q}, q_{i+1}, \dotsc, q_k)$ where
for all $1 \leq j \leq k$ such that $i \neq j$ it holds $q_j \in \cQ \setminus \{\bar{q}\}$.
The solver obtains correct solutions for $\bar{q}$ to the puzzles on all $k$ positions and
can trivially make a successful verification query.

On the following code listing we define an experiment $\Success$ such that
it outputs $1$ if and only if the solver $C$ asks a successful verification query.
%
%\begin{codeblock}
\begin{restatable}{codeblock}{success}
  \textbf{Experiment} $\Success^{P, C}(\pi, \rho)$
  \medskip \hrule
  \textbf{Oracles:} A problem poser $P$, a solver $C = (C_1, C_2)$ for $P$.\\
  \textbf{Input:}  Bitstrings $\pi \in \{0,1\}^n$, $\rho \in \{0,1\}^*$.\\
  \textbf{Output:} A bit $b \in \{0,1\}$.
  \medskip\hrule
  \Run $\langle P(\pi), C_1(\rho) \rangle$ \\
  \IndI $(\Gamma_V, \Gamma_H) := \langle P(\pi), C_1(\rho) \rangle_{P}$ \\
  \IndI $x := \langle P(\pi), C_1(\rho) \rangle_{\mathit{trans}}$ \\ \\
  \Run $C_2^{\Gamma_V,\Gamma_H}(x; \rho)$ \\
  \IndI \If $C_2^{\Gamma_V, \Gamma_H}(x; \rho)$ asks a verification query $(q, y)$ s.t. $\Gamma_V(q, y) = 1$ \Then \\
  \IndII \Return $1$ \\
  \Return $0$
\end{restatable}
%\end{codeblock}
%
We define the \textit{success probability} of $C$ in solving a puzzle defined by $P$ as
\begin{align}
 \underset{\pi, \rho}{\Pr}[\Success^{P,C}(\pi, \rho) = 1].
\end{align}
Furthermore, for fixed $P$ we say that $C$ \textit{succeeds} for $\pi$, $\rho$ if $\Success^{P,C}(\pi, \rho) = 1$.

We now state our main theorem. Loosely speaking, we claim that it is possible to reduce a solver for the $k$-wise direct product of $P$
to a solver for a single puzzle defined by $P$. This implies that if there is no good solver for $P$, then also a good solver for
the $k$-wise direct product of $P$ does not exist.
%
\begin{restatable}[Hardness amplification for dynamic interactive weakly verifiable puzzles]{theorem}{hardnessAmpfDiwvp}
\label{th:sec_amp_for_dwvp}
Let $P_{n}^{(1)}$ be a fixed problem poser as in Definition \ref{def:dwvp}
and $P_{kn}^{(g)}$ a problem poser for the $k$-wise direct product of $P_{n}^{(1)}$ as in Definition~\ref{def:k_wise_direct_product}.
Additionally, let $C$ be a solver for $P_{kn}^{(g)}$ asking at most $h$ hint queries and $v$ verification queries.
There exists a probabilistic algorithm $\Gen$ with oracle access to $C$,
a binary monotone function $g:\{0,1\}^k \rightarrow \{0,1\}$, and a problem poser $P_{n}^{(1)}$.
The algorithm $\Gen$ takes as input parameters $\varepsilon$, $\delta$, $n$, $k$, $h$, $v$,
and outputs a solver circuit $D$ for $P_{n}^{(1)}$ such that: \\
If $C$ satisfies
  \begin{align}
    \label{th_sec_amp_dwvp_assum}
    \underset{\substack{\pi^{(k)} \in \{0,1\}^{kn} \\ \rho \in \{0,1\}^{*}}}{\Pr}\left[\mathit{Success}^{P_{kn}^{(g)}, C}(\pi^{(k)}, \rho) = 1\right]
    \geq 16(h+v)\Bigl(\underset{u \leftarrow \mu_\delta^k}{\Pr}[g(u) = 1] + \varepsilon \Bigr),
  \end{align}
then $D$ is a two-phase probabilistic circuit that almost surely over the randomness of Gen satisfies
  \begin{align}
    \underset{\substack{\pi \in \{0,1\}^{n} \\ \rho \in \{0,1\}^{*}}}
    {\Pr}\left[\Success^{P_{n}^{(1)},\widetilde{D}}(\pi, \rho) = 1\right] \geq \delta + \frac{\epsilon}{6k}.
  \end{align}
Additionally, $D$ requires oracle access to $g$, $P_{n}^{(1)}$, $C$, hint and verification circuits generated by the
problem poser $P_n^{(1)}$ after the first phase and asks at most $\frac{6k}{\epsilon}\log\left(\frac{6k}{\epsilon}\right) h$
hint queries and one verification query. Finally, $\Time(\Gen) = \poly(k, \frac{1}{\varepsilon}, n, v, h)$ with oracle calls.
\end{restatable}

The above theorem is very general in the sense that it does not pose any constrains
on the size of the circuits or the time complexity of the interactive protocol.
Additionally, we count each oracle call as a single step of the algorithm.

Let us consider, as an example, the case where the solver $C$ and the problem poser $P_{kn}^{(g)}$ are polynomial time probabilistic algorithms.
Furthermore, we assume that the solver $C$ is such that it satisfies \eqref{th_sec_amp_dwvp_assum}
and $\frac{1}{\epsilon}$ and $k$ are bounded by some polynomial $p(n)$.
Clearly the solver and the poser can be represented as polynomial size families of probabilistic circuits.
The running time of $\widetilde{Gen}$ is polynomial, therefore the size of $\widetilde{D}$ must also be polynomial.
Finally, the circuit $\widetilde{D}$ can be executed by a polynomial time algorithm.
Thereby, we obtain a typical polynomial time reduction as described in the literature \cite{Arora:2009:CCM:1540612, LectureNotesCrypo}.

The Theorem \ref{th:sec_amp_for_dwvp} holds with high probability over the randomness of $\widetilde{\Gen}$.
More precisely, the circuit output by $\widetilde{\Gen}$ satisfies the condition of the theorem with probability
at least $1 - p(k, n, \frac{1}{\epsilon}) \cdot 2^n$. Therefore, Theorem~\ref{th:sec_amp_for_dwvp} is meaningful if there exists a polynomial $p(n)$
that bounds~$k$~and~$\frac{1}{\epsilon}$.

We emphasize that the number of hint queries asked by $\widetilde{D}$ is greater than the number of hint queries asked by $C$ whereas the number of verification queries
is limited to at most one. For many cryptographic constructions making such an assumption about the number of hint and verification queries is reasonable.
In particular, we cannot assume that a solver for a single puzzle may ask more verification queries than a solver for the $k$-wise direct product of puzzles.

In Chapter \ref{ch:preliminaries} we defined a binary monotone function.
We note that the monotone restriction on $g$ in Theorem \ref{th:sec_amp_for_dwvp} is essential. For $g(b) := 1 - b$ a circuit that deliberately gives incorrect
answers satisfies $g$ with probability $1$ whereas a circuit that solves a puzzle successfully with probability
$\gamma > 0$ succeeds only with probability $1 - \gamma$.

We postpone the proof of Theorem \ref{th:sec_amp_for_dwvp} until Chapter~\ref{ch:main_result}.

%%% Local Variables:
%%% mode: latex
%%% TeX-master: "thesis"
%%% End:
