%% (Master) Thesis template
% Template version used: v1.4
%
% Largely adapted from Adrian Nievergelt's template for the ADPS
% (lecture notes) project.

%% We use the memoir class because it offers a many easy to use features.
\documentclass[11pt,a4paper,titlepage]{memoir}

%% Packages
%% ========

%% LaTeX Font encoding -- DO NOT CHANGE
\usepackage[OT1]{fontenc}

%% Babel provides support for languages.  'english' uses British
%% English hyphenation and text snippets like "Figure" and
%% "Theorem". Use the option 'ngerman' if your document is in German.
%% Use 'american' for American English.  Note that if you change this,
%% the next LaTeX run may show spurious errors.  Simply run it again.
%% If they persist, remove the .aux file and try again.
\usepackage[english]{babel}

%% Input encoding 'utf8'. In some cases you might need 'utf8x' for
%% extra symbols. Not all editors, especially on Windows, are UTF-8
%% capable, so you may want to use 'latin1' instead.
\usepackage[utf8]{inputenc}

%% This changes default fonts for both text and math mode to use Herman Zapfs
%% excellent Palatino font.  Do not change this.
\usepackage[sc]{mathpazo}

%% The AMS-LaTeX extensions for mathematical typesetting.  Do not
%% remove.
\usepackage{amsmath,amssymb,amsfonts,mathrsfs}

%% NTheorem is a reimplementation of the AMS Theorem package. This
%% will allow us to typeset theorems like examples, proofs and
%% similar.  Do not remove.
%% NOTE: Must be loaded AFTER amsmath, or the \qed placement will
%% break
\usepackage[amsmath,thmmarks]{ntheorem}

%% LaTeX' own graphics handling
\usepackage{graphicx}

%% We unfortunately need this for the Rules chapter.  Remove it
%% afterwards; or at least NEVER use its underlining features.
\usepackage{soul}

%% Some more packages that you may want to use.  Have a look at the
%% file, and consult the package docs for each.
%% See the TeXed file for more explanations

%% [OPT] Multi-rowed cells in tabulars
%\usepackage{multirow}

%% [REC] Intelligent cross reference package. This allows for nice
%% combined references that include the reference and a hint to where
%% to look for it.
\usepackage{varioref}

%% [OPT] Easily changeable quotes with \enquote{Text}
%\usepackage[german=swiss]{csquotes}

%% [REC] Format dates and time depending on locale
\usepackage{datetime}

%% [OPT] Provides a \cancel{} command to stroke through mathematics.
%\usepackage{cancel}

%% [NEED] This allows for additional typesetting tools in mathmode.
%% See its excellent documentation.
\usepackage{mathtools}

%% [ADV] Conditional commands
%\usepackage{ifthen}

%% [OPT] Manual large braces or other delimiters.
%\usepackage{bigdelim, bigstrut}

%% [REC] Alternate vector arrows. Use the command \vv{} to get scaled
%% vector arrows.
%\usepackage[h]{esvect}

%% [NEED] Some extensions to tabulars and array environments.
\usepackage{array}

%% [OPT] Postscript support via pstricks graphics package. Very
%% diverse applications.
%\usepackage{pstricks,pst-all}

%% [?] This seems to allow us to define some additional counters.
%\usepackage{etex}

%% [ADV] XY-Pic to typeset some matrix-style graphics
%\usepackage[all]{xy}

%% [OPT] This is needed to generate an index at the end of the
%% document.
%\usepackage{makeidx}

%% [OPT] Fancy package for source code listings.  The template text
%% needs it for some LaTeX snippets; remove/adapt the \lstset when you
%% remove the template content.
\usepackage{listings}
\lstset{language=TeX,basicstyle={\normalfont\ttfamily}}

%% [REC] Fancy character protrusion.  Must be loaded after all fonts.
\usepackage[activate]{pdfcprot}

%% [REC] Nicer tables.  Read the excellent documentation.
\usepackage{booktabs}

%%pseudocode and algorithms
\usepackage{algpseudocode}
\usepackage{algorithm}

%% define comments in single line
\algnewcommand{\LineComment}[1]{\State \(\triangleright\) #1}

%%placing in the right place
\usepackage{float}

\usepackage{caption}

\DeclareCaptionFormat{algor}{
  \hrulefill\par\offinterlineskip\vskip1pt
    \textbf{#1#2}#3\offinterlineskip\hrulefill}
\DeclareCaptionStyle{algori}{singlelinecheck=off,format=algor,labelsep=space}
\captionsetup[algorithm]{style=algori}

\algnewcommand\algorithmicinput{\textbf{Input:}}
\algnewcommand\Input{\item[\algorithmicinput]}

\algnewcommand\algorithmicauxinput{\textbf{Auxiliary input:}}
\algnewcommand\AuxInput{\item[\algorithmicauxinput]}

%indention in algorithms
\newcommand{\pushcode}[1][1]{\hskip\dimexpr#1\algorithmicindent\relax}

% hyphen in math mode
\def\hyph{\text{-}}



%% Our layout configuration.  DO NOT CHANGE.
%% Memoir layout setup

%% NOTE: You are strongly advised not to change any of them unless you
%% know what you are doing.  These settings strongly interact in the
%% final look of the document.

% Dependencies
\usepackage{ETHlogo}

% Turn extra space before chapter headings off.
\setlength{\beforechapskip}{0pt}

\nonzeroparskip
\parindent=0pt
\defaultlists

% Chapter style redefinition
\makeatletter

\if@twoside
  \pagestyle{Ruled}
  \copypagestyle{chapter}{Ruled}
\else
  \pagestyle{ruled}
  \copypagestyle{chapter}{ruled}
\fi
\makeoddhead{chapter}{}{}{}
\makeevenhead{chapter}{}{}{}
\makeheadrule{chapter}{\textwidth}{0pt}
\copypagestyle{abstract}{empty}

\makechapterstyle{bianchimod}{%
  \chapterstyle{default}
  \renewcommand*{\chapnamefont}{\normalfont\Large\sffamily}
  \renewcommand*{\chapnumfont}{\normalfont\Large\sffamily}
  \renewcommand*{\printchaptername}{%
    \chapnamefont\centering\@chapapp}
  \renewcommand*{\printchapternum}{\chapnumfont {\thechapter}}
  \renewcommand*{\chaptitlefont}{\normalfont\huge\sffamily}
  \renewcommand*{\printchaptertitle}[1]{%
    \hrule\vskip\onelineskip \centering \chaptitlefont\textbf{\vphantom{gyM}##1}\par}
  \renewcommand*{\afterchaptertitle}{\vskip\onelineskip \hrule\vskip
    \afterchapskip}
  \renewcommand*{\printchapternonum}{%
    \vphantom{\chapnumfont {9}}\afterchapternum}}

% Use the newly defined style
\chapterstyle{bianchimod}

\setsecheadstyle{\Large\bfseries\sffamily}
\setsubsecheadstyle{\large\bfseries\sffamily}
\setsubsubsecheadstyle{\bfseries\sffamily}
\setparaheadstyle{\normalsize\bfseries\sffamily}
\setsubparaheadstyle{\normalsize\itshape\sffamily}
\setsubparaindent{0pt}

% Set captions to a more separated style for clearness
\captionnamefont{\sffamily\bfseries\footnotesize}
\captiontitlefont{\sffamily\footnotesize}
\setlength{\intextsep}{16pt}
\setlength{\belowcaptionskip}{1pt}

% Set section and TOC numbering depth to subsection
\setsecnumdepth{subsection}
\settocdepth{subsection}

%% Titlepage adjustments
\pretitle{\vspace{0pt plus 0.7fill}\begin{center}\HUGE\sffamily\bfseries}
\posttitle{\end{center}\par}
\preauthor{\par\begin{center}\let\and\\\Large\sffamily}
\postauthor{\end{center}}
\predate{\par\begin{center}\Large\sffamily}
\postdate{\end{center}}

\def\@advisors{}
\newcommand{\advisors}[1]{\def\@advisors{#1}}
\def\@department{}
\newcommand{\department}[1]{\def\@department{#1}}
\def\@thesistype{}
\newcommand{\thesistype}[1]{\def\@thesistype{#1}}

\renewcommand{\maketitlehooka}{\noindent\ETHlogo[2in]}

\renewcommand{\maketitlehookb}{\vspace{1in}%
  \par\begin{center}\Large\sffamily\@thesistype\end{center}}

\renewcommand{\maketitlehookd}{%
  \vfill\par
  \begin{flushright}
    \sffamily
    \@advisors\par
    \@department, ETH Z\"urich
  \end{flushright}
}

\checkandfixthelayout

\setlength{\droptitle}{-48pt}

\makeatother

% This defines how theorems should look. Best leave as is.
\theoremstyle{plain}
\setlength\theorempostskipamount{0pt}

\usepackage[framemethod=tikz]{mdframed}
\newdimen\linenumbersep

\newcommand{\linenumber}[1]{%
  \linenumbersep 4pt%
  \advance\linenumbersep\mdflength{innerleftmargin}%
  \advance\linenumbersep\mdflength{innerlinewidth}%
  \advance\linenumbersep\mdflength{middlelinewidth}%
  \advance\linenumbersep\mdflength{outerlinewidth}%
  \advance\linenumbersep\mdflength{linewidth}%
  \makebox[0pt][r]{{\rmfamily\tiny#1}\hspace*{\linenumbersep}}}

\newenvironment{codeblock}%
   {\medskip\begin{mdframed}\setlength{\parindent}{0cm}}%
   {\end{mdframed}\medskip}
\newcommand{\Ind}{\mbox{}}
\newcommand{\IndI}{\mbox\qquad}
\newcommand{\IndII}{\mbox\qquad\qquad}
\newcommand{\IndIII}{\mbox\qquad\qquad\qquad}
\newcommand{\IndIIII}{\mbox\qquad\qquad\qquad\qquad}

\newenvironment{todo}%
   {\medskip\begin{mdframed}\setlength{\parindent}{0cm}}%
   {\end{mdframed}\medskip}


%%% Local Variables:
%%% mode: latex
%%% TeX-master: "thesis"
%%% End:


%% Theorem environments.  You will have to adapt this for a German
%% thesis.
%% Theorem-like environments

%% This can be changed according to language. You can comment out the ones you
%% don't need.

\numberwithin{equation}{chapter}

% declare theorem style
\declaretheoremstyle[
spaceabove=6pt, spacebelow=6pt,
headfont=\normalfont\bfseries,
notefont=\mdseries, notebraces={(}{)},
bodyfont=\normalfont\itshape,
postheadspace=1em,
%qed=\qedsymbol
]{thm_sty}

% declare definition style
\declaretheoremstyle[
spaceabove=6pt, spacebelow=6pt,
headfont=\normalfont\bfseries,
notefont=\mdseries, notebraces={(}{)},
bodyfont=\normalfont,
postheadspace=1em,
qed=\ensuremath{\lozenge}
]{def_sty}


%% English variants
\declaretheorem[style=thm_sty, name=Theorem, numberwithin=chapter]{theorem}
\declaretheorem[style=thm_sty, name=Observation, sibling=theorem]{observation}
\declaretheorem[style=thm_sty, name=Lemma, sibling=theorem]{lemma}
\declaretheorem[style=def_sty, name=Definition, sibling=theorem]{definition}
%\declaretheorem[style=def_sty, name=Proof]{proof}

%\newtheorem{theorem}{Theorem}[chapter]
% \newtheorem{example}[theorem]{Example}
% \newtheorem{remark}[theorem]{Remark}
% \newtheorem{corollary}[theorem]{Corollary}
% \newtheorem{lemma}[theorem]{Lemma}
% \newtheorem{proposition}[theorem]{Proposition}
% \newtheorem{observation}[theorem]{Observation}

%\theoremstyle{definition}
%\theorembodyfont{\normalfont}
%% end def with blacksquare symbol
%\theoremsymbol{\ensuremath{\lozenge}}
%\newtheorem{definition}[theorem]{Definition}

%%Proof environment with a small square as a "qed" symbol
%\theoremstyle{nonumberplain}
%\theorembodyfont{\normalfont}
%\theoremsymbol{\ensuremath{\square}}
%\theoremseparator{.}
%\newtheorem{proof}{Proof}



%% Helpful macros.
%% Custom commands
%% ===============

%% Special characters for number sets, e.g. real or complex numbers.
\newcommand{\C}{\mathbb{C}}
\newcommand{\D}{\mathbb{D}}
\newcommand{\K}{\mathbb{K}}
\newcommand{\N}{\mathbb{N}}
\newcommand{\M}{\mathbb{M}}
\newcommand{\Q}{\mathbb{Q}}
\newcommand{\R}{\mathbb{R}}
\newcommand{\T}{\mathbb{T}}
\newcommand{\X}{\mathbb{X}}
\newcommand{\Z}{\mathbb{Z}}


\newcommand{\cA}{\mathcal{A}}
\newcommand{\cB}{\mathcal{B}}
\newcommand{\cX}{\mathcal{X}}
\newcommand{\cH}{\mathcal{H}}
\newcommand{\cW}{\mathcal{W}}
\newcommand{\cG}{\mathcal{G}}
\newcommand{\cP}{\mathcal{P}}
\newcommand{\cR}{\mathcal{R}}
\newcommand{\cD}{\mathcal{D}}
\newcommand{\cF}{\mathcal{F}}
\newcommand{\cM}{\mathcal{M}}
\newcommand{\cK}{\mathcal{K}}
\newcommand{\cT}{\mathcal{T}}
\newcommand{\cS}{\mathcal{S}}
\newcommand{\cQ}{\mathcal{Q}}
\newcommand{\cV}{\mathcal{V}}
\newcommand{\cI}{\mathcal{I}}

%define our own code commands
%use capital latters as most of these commands is already defined
\renewcommand{\For}{\textbf{for }}
\renewcommand{\If}{\textbf{if }}
\renewcommand{\Else}{\textbf{else }}
\renewcommand{\Return}{\textbf{return }}
\newcommand{\Then}{\textbf{then }}
\newcommand{\Do}{\textbf{do: }}
\renewcommand{\And}{\textbf{and }}
\newcommand{\Or}{\textbf{or }}
\newcommand{\Run}{\textbf{run }}
\newcommand{\To}{\textbf{to }}
\renewcommand{\Repeat}{\textbf{Repeat }}

%% Fixed/scaling delimiter examples (see mathtools documentation)
\DeclarePairedDelimiter\abs{\lvert}{\rvert}
\DeclarePairedDelimiter\norm{\lVert}{\rVert}

%% Use the alternative epsilon per default and define the old one as \oldepsilon
\let\oldepsilon\epsilon
\renewcommand{\epsilon}{\ensuremath\varepsilon}

%% Also set the alternate phi as default.
\let\oldphi\phi
\renewcommand{\phi}{\ensuremath{\varphi}}

% New command that introduces a tab
\newcommand{\itab}[1]{\hspace{0em}\rlap{#1}}
\newcommand{\tab}[1]{\hspace{.2\textwidth}\rlap{#1}}

\DeclareMathOperator{\la0}{\leftarrow}
\DeclareMathOperator{\ra0}{\rightarrow}

\DeclareMathOperator{\hash}{\mathit{hash}}
\DeclareMathOperator{\CanonicalSuccess}{\mathit{CanonicalSuccess}}
\DeclareMathOperator{\Success}{\mathit{Success}}
\DeclareMathOperator{\poly}{\mathit{poly}}
\DeclareMathOperator{\p}{\mathit{p}}
\DeclareMathOperator{\Time}{\mathit{Time}}
\DeclareMathOperator{\Gen}{\text{Gen}}
\DeclareMathOperator{\FindHash}{\text{FindHash}}
\DeclareMathOperator{\Bad}{\mathit{Bad}}
\DeclareMathOperator{\Good}{\mathit{Good}}
\DeclareMathOperator{\trans}{\mathit{trans}}
\DeclareMathOperator{\Canonical}{\mathit{Canonical}}


%the DWVP for the permutation
\DeclareMathOperator{\PiDWVP}{\Pi_{DWVP}}
%the DWPV for the k-wise product of permutations
\DeclareMathOperator{\kPiDWVP}{\Pi_{DWVP}^{(k)}}

%% Make document internal hyperlinks wherever possible. (TOC, references)
%% This MUST be loaded after varioref, which is loaded in 'extrapackages'
%% above.  We just load it last to be safe.
\usepackage[linkcolor=black,colorlinks=true,citecolor=black,filecolor=black]{hyperref}


%% Document information
%% ====================

\title{Title of Thesis}
\author{Grzegorz Makosa}
\thesistype{Master Thesis}
\advisors{Advisors: Prof. Dr. Thomas Holenstein, Dr. Robin Künzler}
\department{Department of Computer Science}
\date{April 8, 2014}

\begin{document}

\frontmatter

% \begin{titlingpage}
%   \calccentering{\unitlength}
%   \begin{adjustwidth*}{\unitlength-24pt}{-\unitlength-24pt}
%     \maketitle
%   \end{adjustwidth*}
% \end{titlingpage}

%\begin{abstract}
% A good abstract explains in one line why the paper is important.
% It then goes on to give a summary of your major results, preferably couched in numbers with error limits.
% The final sentences explain the major implications of your work. A good abstract is concise, readable, and quantitative.
% Length should be ~ 1-2 paragraphs, approx. 400 words.
%
% Absrtracts generally do not have citations.
% Information in title should not be repeated.
% Be explicit.
% Use numbers where appropriate.
% Answers to these questions should be found in the abstract:
% What did you do?
% Why did you do it? What question were you trying to answer?
% How did you do it? State methods.
% What did you learn? State major results.
% Why does it matter? Point out at least one significant implication.

% 1) new hardness proof
% 2) it is possible to start with a weak crypto primitive and obtain strong one.
% 3) a natural puzzle fixing technique similar to the one used in the classical proofs of weak one-way function implying strong ones
% 4) uses an arbitrary monotone function to decide the result
% 5) compare with previous works
% 6)

We give a new proof of hardness amplification for Dynamic Weakly Verifiable Puzzles which is more general than the previous ones.

Our proof applies to the various cryptographic primitives like hash function,
one way functions, signature schemes and bit commitment protocols.

\end{abstract}


\cleartorecto
\tableofcontents
\mainmatter

\chapter{Introduction}
\section{Security amplification}
\section{Organization of the thesis}
In this section we set up notation and terminology used in the thesis.
%
\section{Notation and Definitions}
\textbf{(Algorithms, Bitstrings and Circuits)}
We define a Boolean circuit as a directed acyclic graph with input vertices and vertices implementing logical functions \textit{and}, \textit{or}, and \textit{not}.
We denote Boolean circuits using capital letters from the Greek and English alphabet.

We define a \textit{probabilistic circuit} as a Boolean circuit $C_{m,n} : \{0,1\}^{m} \times \{0,1\}^{n} \rightarrow \{0,1\}^{*}$,
For input $x \in \{0,1\}^{m}$ we write to denote $C_{m,n}(x;r)$ where $r \in \{0,1\}^{n}$ is called auxiliary input.
If a probabilistic circuit does not take input $x$, we slightly abuse notation and write $C_{n}(r)$.
Similarly we use $\{C_n\}$ to denote a family of probabilist circuits that takes only auxiliary input.
We make sure that it is clear from the context that probabilist circuits with only auxiliary input
are not confused with circuits that do not take auxiliary input.

For a (probabilist) circuit $C$ we write $\mathit{Size}(C)$ to denote the total number of vertices of $C$.

We define a \textit{two phase circuit} $C := (C_1, C_2)$ as a circuit where in the first phase a circuit $C_1$ is used and in the second phase a circuit $C_2$.
If $C_1$ and $C_2$ are probabilistic circuits we write $C(\delta) := (C_1, C_2)(\delta)$ to denote that in both phases $C_1$ and $C_2$ take
as auxiliary input the same bitstring $\delta$.

It is well known \cite{Arora:2009:CCM:1540612} that a probabilistic polynomial time algorithm can be represented as a circuit of polynomial size.
Moreover, it can be computed in polynomial time and logarithmic space.
Therefore, whenever we state a theorem about circuits we can also apply it to polynomial time algorithms.

We write $\mathit{poly}(\alpha_1, \dots, \alpha_n)$ to denote a polynomial on variables $\alpha_1, \dots, \alpha_n$.
For an algorithm $A$ we write $\mathit{Time}(A)$ to denote the number of steps it takes to execute $A$.
Similarly, as for probabilistic circuits we write the randomness used by a probabilistic algorithm explicitly as a bitstring provided as an auxiliary input.

\textbf{(Probabilities and distributions)}
For a finite set $\cR$ we write $r \xleftarrow{\$} \cR$ to denote that $r$ is chosen from $\cR$ uniformly at random.
For $\delta \in \R : 0 \leq \delta \leq 1$ we write $\mu_{\delta}$ to denote the Bernoulli distribution where outcome $1$ occurs with
probability $\delta$ and $0$ with probability $1-\delta$.
Moreover, we use $\mu_{\delta}^k$ to denote the probability distribution over $k$-tuples
where each element of a $k$-tuple is drawn independently according to $\mu_{\delta}$.
Finally, let $u \leftarrow \mu_{\delta}^k$ denote that a $k$-tuple $u$ is chosen according to $\mu_{\delta}^k$.

Let $(\Omega, \cF, \Pr)$ be a probability space and $n \in \N$. We say that an event $E_n \in \cF$
happens \textit{almost surely} or with \textit{high probability} if $\Pr[E_n] \geq 1 - 2^{-n} \mathit{poly}(n)$.

\begin{todo}
  \textbf{TODO:} Non-negligible function and probability
\end{todo}

\textbf{(Interactive protocols)} We are often interested in situations where two probabilistic algorithms interact with each other according to some protocol.
We limit ourselves to the cases where algorithms interact by means of messages representable by bitstrings.
A protocol execution between two probabilistic algorithms $A$ and $B$ is denoted by $\langle A, B \rangle$.
The output of $A$ in a protocol execution is denoted by $\langle A, B \rangle_A$ and of $B$ by $\langle A, B \rangle_B$.
A sequence of all messages sent by $A$ and $B$ in the protocol execution is called a communication transcript and
is denoted by $\langle A, B \rangle_{\mathit{trans}}$.

\textbf{(Oracle algorithms)}
\begin{todo}
  \textbf{TODO:} set up notation
\end{todo}

\begin{definition}[Polynomial time sampleable distribution]
We say that a distribution is polynomial time sampleable if it can be approximated by polynomial time algorithm
up to exponential factor.
\end{definition}

\begin{definition}[Pairwise independent family of efficient hash functions]
Let $\cD$ and $\cR$ be finite sets and $\cH$ be a family of functions mapping values from $\cD$ to values in $\cR$.
We say that $\cH$ is an \textnormal{efficient family of pairwise independent hash functions}
if $\cH$ has the following properties.

\textbf{(Pairwise independent)} For $\forall x \neq y \in \mathcal{D}$ and $\forall \alpha, \beta \in \cR$, it holds
\begin{displaymath}
\underset{\hash \la0 \cH}{\Pr}[hash(x) = \alpha \mid hash(y) = \beta] = \frac{1}{|\cR|}.
\end{displaymath}

\textbf{(Polynomial time sampleable)} For every $\mathit{hash} \in \cH$ the function $\mathit{hash}$ is sampleable in time $\mathit{poly}(\log|\cD|, \log|\cR|)$.

\textbf{(Efficiently computable)}
For every $hash \in \cH$ there exists an algorithm running in time $\mathit{poly}(\log|\cD|, \log|\cR|)$ which
on input $x \in \cD$ outputs $y \in \cR$ such that $y = hash(x)$.
\end{definition}

We note that the pairwise independence property is equivalent to
\begin{displaymath}
\underset{\hash \la0 \cH}{\Pr}[hash(x) = \alpha \land hash(y) = \beta] = \frac{1}{|\cR|^2}.
\end{displaymath}
It is well know \cite{Carter:1977:UCH:800105.803400} that there exists a family of hash functions meeting the above criteria.

%%% Local Variables:
%%% mode: latex
%%% TeX-master: "master"
%%% End:

\chapter{Hardness amplification of weakly verifiable puzzles}
\section{Weakly verifiable puzzles}
\section{Previous results}
\subsection{Dynamic puzzles}
\subsection{Interactive puzzles}
\section{Proof Interactive Dynamic puzzles}
\subsection{Our techniques}
The idea of the algorithm $Gen$ is to output a circuit $D$ that solves the input puzzle often.
We know that $C$ has good success probability for a $k$-wise product of $P^{(1)}$.
The algorithm $Gen$ tries to find a puzzle such that when $C$ runs with this puzzle fixed
on the first position, and disregards whether this puzzle is correctly solved
then the assumptions of Theorem \ref{th:sec_amp_for_dwvp} are true for a $k-1$-wise direct product.
If it is possible to find such a puzzle then $Gen$ could recurse and solve a smaller problem.
In the optimistic case we can reach $k=1$, which means that we found a good circuit for solving a single
puzzle by just fixing the initial puzzles of $C$.

Otherwise, when the first position is disregarded then the success probability of $C$ is not substantially better.
This is remarkable, as we know that $C$ performs good for $k$-wise product, it means that the first position is important,
in the sense that $C$ solves the puzzle on that position unusually often.
Therefore, it is reasonable to construct the circuit $D$ using $C$ by placing the input puzzle of $D$ on that position, and then
finding remaining $k-1$ puzzles. These $k-1$ remaining puzzles are generated by the circuit $D$, hence it is possible to check
whether they are correctly solved by the circuit $C$. We know that circuit $C$ has good success probability, and the puzzle on the first
position is important. Therefore, if we are able to find a $k-1$ puzzles such that the fact whether the $k$-wise direct product is correctly
solved depends on whether the puzzle on the first position is correctly solved then we can assume that $C$ is often correct on this first position.

There are some problems with this approach, first we have to ensure that we can make a decision when the algorithm $Gen$ should recurse and when not
correctly with high probability. Then, we have to show that it is possible to find a puzzles such that $C$ is often correct on the first position.
Finally, we also have to be sure that we do not ask a hint query, on the final verification query to the oracle.
To satisfy the last requirement we split $Q$.

%%% Local Variables:
%%% mode: latex
%%% TeX-master: "../thesis"
%%% End:

%
% Define high probability
%
\noindent
We write $\mu_{\delta}$ to denote the Bernoulli distribution, where outcome $1$ occurs with
probability $\delta$ and $0$ with probability $1-\delta$ where $0 \leq \delta \leq 1$.
Moreover, we use $\mu_{\delta}^k$ to denote a probability distribution over $k$-tuples,
where each bit of a $k$-tuple is drawn independently according to $\mu_{\delta}$.
Finally, let $u \leftarrow \mu_{\delta}^k$ denote that a $k$-tuple $u$ is chosen according to $\mu_{\delta}^k$.

The protocol execution between two probabilistic algorithms $A$ and $B$ is denoted by $\langle A, B \rangle$.
The output of $A$ in such a protocol execution is denoted by $\langle A, B \rangle_A$ and of $B$ by $\langle A, B \rangle_B$.
Finally, let $\langle A, B \rangle_{\text{trans}}$ denote the transcript of communication between $A$ and $B$.

We define a \textit{two phase algorithm} $A := (A_1, A_2)$ as an algorithm where in the first phase an algorithm $A_1$
is executed and in the second phase an algorithm $A_2$.

We say that an event happens \textit{almost surely} or with \textit{high probability} if
it occurs with probability at least $1 - 2^{-n} poly(n)$.

\begin{definition}[Dynamic weakly verifiable puzzle.]
  \label{def:dwvp}
  A dynamic weakly verifiable puzzle (DWVP) is defined by a probabilistic algorithm $P$
  called a problem poser.
  A problem solver $S := (S_1, S_2)$ for $P$ is a probabilistic two phase algorithm.
  We write $P_n(\pi)$ to denote the execution of $P$ with the randomness fixed to $\pi \in \{0,1\}^n$, and $(S_1,S_2)(\rho)$
  to denote the execution of both $S_1$ and $S_2$ with the randomness fixed to $\rho \in \{0,1\}^{*}$.

  In the first phase, the poser $P_n(\pi)$ and the solver $S_1(\rho)$ interact.
  As the result of the interaction $P_n(\pi)$ outputs a verification circuit $\Gamma_{V}$ and a hint circuit $\Gamma_{H}$.
  The algorithm $S_1(\rho)$ produces no output.
  The circuit $\Gamma_{V}$ takes as input $q \in Q$, an answer $y \in \{0,1\}^*$,
  and outputs a bit. We say that an answer $(q,y)$ is a correct solution if and only if $\Gamma_V(q,y) = 1$.
  The circuit $\Gamma_H$ on input $q \in Q$ outputs a hint such that $\Gamma_V(q,\Gamma_H(q)) = 1$.

  In the second phase, $S_2$ takes as input $x := \langle P_n(\pi), S_1(\rho) \rangle_{\text{trans}}$,
  and has oracle access to $\Gamma_V$ and $\Gamma_H$.
  The execution of $S_2$ with the input $x$ and the randomness fixed to $\rho$
  is denoted by $S_2(x, \rho)$. The queries of $S_2$ to $\Gamma_V$ and $\Gamma_H$ are called verification queries and hint queries respectively.
  The algorithm $S_2$ asks at most $h$ hint queries, $v$ verification queries, and succeeds if and only if
  it makes a verification query $(q,y)$ such that $\Gamma_V(q,y) = 1$, and it has not previously asked for a hint query on $q$.
\end{definition}
%
\begin{definition}[$k$-wise direct-product of DWVPs.]
  Let $g: \{0,1\}^{k} \rightarrow \{0,1\}$ be a monotone function and $P^{(1)}$ a problem poser as in Definition \ref{def:dwvp}.
  The $k$-wise direct product of $P^{(1)}$ is a DWVP defined by a probabilistic algorithm $P^{(g)}$.
  We write $P_{kn}^{(g)}(\pi^{(k)})$ to denote the execution of $P^{(g)}$ with the randomness fixed to $\pi^{(k)} := (\pi_1, \dots, \pi_k)$
  where for each $1 \leq i \leq n : \pi_i \in \{0,1\}^n.$
  Let $(S_1, S_2)(\rho)$ be a solver for $P^{(g)}$ as in Definition \ref{def:dwvp}.
  In the first phase, the algorithm $S_1(\rho)$ sequentially interacts in $k$ rounds with $P_{kn}^{(g)}(\pi^{(k)})$.
  In the $i$-th round $S_1(\rho)$ interacts with $P_n^{(1)}(\pi_i)$,
  and as the result $P_{n}^{(1)}(\pi_i)$ generates circuits $\Gamma_V^i, \Gamma_H^i$.
  Finally, after $k$ rounds $P_{kn}^{(g)}(\pi^{(k)})$ outputs a verification circuit
\begin{align*}
  \Gamma_V^{(g)} (q, y_1, \dots, y_k) := g(\Gamma_V^{1}(q, y_1), \dots, \Gamma_V^{k}(q, y_k))
\end{align*}
and a hint circuit
\begin{align*}
  \Gamma_H^{(k)} (q) := (\Gamma_H^{1}(q), \dots, \Gamma_H^{k}(q)).
\end{align*}
\end{definition}
%
If it is clear from a context, we omit the parameter $n$, and write $P(\pi)$ instead of $P_n(\pi)$ where $\pi \in \{0,1\}^{n}$.

A verification query $(q,y)$ of a solver $S$ for which a hint query on this $q$ has been asked before can not be a verification query that succeeds.
Therefore, without loss of generality, we make the assumption that $S$ does not ask verification queries on $q$
for which a hint query has been asked before. Furthermore, we assume that once $S$ asked a verification query that succeeds,
it does not ask any further hint or verification queries.

Let $C$ be a circuit that corresponds to a solver $S$ as in Definition \ref{def:dwvp}.
Similarly as for a two phase algorithm, we write $C(\rho) := (C_1, C_2)(\rho)$ to denote that $C$
in the first phase uses a circuit $C_1$ and in the second phase a circuit $C_2$.
Additionally, the randomness in both phases is fixed to $\rho \in \{0,1\}^{*}$.

%
\begin{codeblock}
  \textbf{Experiment $Success^{P, C}(\pi, \rho) $}
  \medskip
  \hrule
  \medskip
  \textbf{Oracle:} A problem poser $P$, a solver circuit $C = (C_1, C_2)$.\\
  \textbf{Input:}  Bitstrings $\pi \in \{0,1\}^n$, $\rho \in \{0,1\}^*$.\\
  \textbf{Output:} A bit $b \in \{0,1\}$.
  \medskip\hrule\medskip
  \Run $\langle P(\pi), C_1(\rho) \rangle$ \\
  \IndI $(\Gamma_V, \Gamma_H) := \langle P(\pi), C_1(\rho) \rangle_{P}$ \\
  \IndI $x := \langle P(\pi), C_1(\rho) \rangle_{\text{trans}}$ \\ \\
  \Run $C_2^{\Gamma_V,\Gamma_H}(x, \rho)$ \\
  \IndI \If $C_2^{\Gamma_V, \Gamma_H}(x, \rho)$ asks a verification query $(q, y)$ such that $\Gamma_V(q, y) = 1$ \Then \\
  \IndII \Return $1$ \\
  \Return $0$ \\
\end{codeblock}
%
We define the \textit{success probability} of $C$ in solving a puzzle defined by $P$ as
\begin{align}
 \underset{\pi, \rho}{\Pr}[Success^{P,C}(\pi, \rho) = 1].
\end{align}
Furthermore, we say that $C$ succeeds for $\pi$, $\rho$ if $Success^{P,C}(\pi, \rho) = 1$.
%
\begin{theorem}[Security amplification for a dynamic weakly verifiable puzzle.]
\label{th:sec_amp_for_dwvp}
Let $P^{(1)}$ be a fixed problem poser as in Definition \ref{def:dwvp}, and $P^{(g)}$ be a poser for the $k$-wise direct product of $P^{(1)}$.
There exists a probabilistic algorithm $Gen$ with oracle access to: a solver circuit $C$ for $P^{(g)}$,
a monotone function $g:\{0,1\}^k \rightarrow \{0,1\}$ and problem posers $P^{(1)}$, $P^{(g)}$.
Additionally, $Gen$ takes as input parameters $\varepsilon$, $\delta$, $n$, $k$
the number of verification queries $v$ and hint queries $h$ asked by $C$, and outputs a solver circuit $D$ for $P^{(1)}$ as in Definition \ref{def:dwvp}
such that the following holds: \\
If $C$ is such that
  \begin{align*}
    \underset{\substack{\pi^{(k)} \in \{0,1\}^{kn} \\ \rho \in \{0,1\}^{*}}}{\Pr}\left[Success^{P^{(g)}, C}(\pi^{(k)}, \rho) = 1\right]
    \geq 16(h+v)\left(\underset{u \leftarrow \mu_\delta^k}{\Pr}\left[g(u) = 1\right] + \varepsilon\right)
  \end{align*}
then $D$ satisfies almost surely
  \begin{align*}
    \underset{\substack{\pi \in \{0,1\}^{n} \\ \rho \in \{0,1\}^{*}}}
    {\Pr}\left[Success^{P^{(1)},D}(\pi, \rho) = 1\right] \geq (\delta + \frac{\varepsilon}{6k}).
  \end{align*}
Additionally, $D$ requires oracle access to $g$, $P^{(1)}$, $C$,
and asks at most $\frac{6k}{\epsilon}\log\left(\frac{6k}{\epsilon}\right) h$ hint queries and one verification query.
Finally, $\text{Size}(D) \leq \text{Size}(C) \cdot \frac{6k}{\varepsilon}$ and $\text{Time}(\text{Gen}) = \text{poly}(k, \frac{1}{\varepsilon}, n, v, h)$.
\end{theorem}
%
% The Theorem \ref{th:sec_amp_for_dwvp} implies that if there is no good solver for a puzzle defined by $P^{(1)}$, then a good solver for
% a $k$-wise direct product of $P^{(1)}$ does not exist.

% The idea of the algorithm $Gen$ is to output a circuit $D$ that solves the input puzzle often.
% We know that $C$ has good success probability for a $k$-wise product of $P^{(1)}$.
% The algorithm $Gen$ tries to find a puzzle such that when $C$ runs with this puzzle fixed
% on the first position, and disregards whether this puzzle is correctly solved
% then the assumptions of Theorem \ref{th:sec_amp_for_dwvp} are true for a $k-1$-wise direct product.
% If it is possible to find such a puzzle then $Gen$ could recurse and solve a smaller problem.
% In the optimistic case we can reach $k=1$, which means that we found a good circuit for solving a single
% puzzle by just fixing the initial puzzles of $C$.

% Otherwise, when the first position is disregarded then the success probability of $C$ is not substantially better.
% This is remarkable, as we know that $C$ performs good for $k$-wise product, it means that the first position is important,
% in the sense that $C$ solves the puzzle on that position unusually often.
% Therefore, it is reasonable to construct the circuit $D$ using $C$ by placing the input puzzle of $D$ on that position, and then
% finding remaining $k-1$ puzzles. These $k-1$ remaining puzzles are generated by the circuit $D$, hence it is possible to check
% whether they are correctly solved by the circuit $C$. We know that circuit $C$ has good success probability, and the puzzle on the first
% position is important. Therefore, if we are able to find a $k-1$ puzzles such that the fact whether the $k$-wise direct product is correctly
% solved depends on whether the puzzle on the first position is correctly solved then we can assume that $C$ is often correct on this first position.

% There are some problems with this approach, first we have to ensure that we can make a decision when the algorithm $Gen$ should recurse and when not
% correctly with high probability. Then, we have to show that it is possible to find a puzzles such that $C$ is often correct on the first position.
% Finally, we also have to be sure that we do not ask a hint query, on the final verification query to the oracle.
% To satisfy the last requirement we split $Q$.

%%% Local Variables:
%%% mode: latex
%%% TeX-master: "../master"
%%% End:


% Let $hash:Q\rightarrow\{0,1,\dots, 2(h+v)-1\}$, then a set $P_{hash} \subseteq Q$,
% defined with respect to $hash$, is the set of preimages of $0$ for $hash$.
Let $hash:Q\rightarrow\{0,1,\dots, 2(h+v)-1\}$,
the idea is to partition $Q$ such that the set of preimages of $0$ for $hash$ contains $q \in Q$ on which $C$ is not allowed to ask hint queries,
and the first successful verification query $(q,y)$ of $C$ is such that $hash(q) = 0$.
Therefore, if $C$ makes a verification query $(q,y)$ such that $hash(q) = 0$, then we know that no hint query is ever asked on this $q$.

In the experiment $CanonicalSuccess$ we denote the $i$-th query of $C$ by $q_i$ if it is a hint query, and by $(q_i, y_i)$ if it is a verification query.
A solver circuit $C$ succeeds in the experiment $CanonicalSuccess$ if it asks a successful verification query $(q_j, y_j)$ such that $hash(q_j) = 0$,
and no hint query $q_i$ is asked before $(q_j, y_j)$ such that $hash(q_i) = 0$.
%
\begin{codeblock}
  \textbf{Experiment $CanonicalSuccess^{P, C, hash}(\pi, \rho)$}
  \medskip

  \hrule

  \medskip
  \textbf{Oracle:} A problem poser $P$, a solver circuit $C = (C_1, C_2)$.\\
  \IndII A function $hash: Q \rightarrow \{0, \dots, 2(h+v) - 1\}$.\\
  \textbf{Input:}  Bitstrings $\pi \in \{0,1\}^n$ and $\rho \in \{0,1\}^*$. \\
  \textbf{Output:} A bit $b \in \{0,1\}$.

  \medskip\hrule\medskip
  Run $\langle P(\pi), C_1(\rho) \rangle$ \\
  \IndI $(\Gamma_V, \Gamma_H) := \langle P(\pi), C_1(\rho) \rangle_{P}$ \\
  \IndI $x := \langle P(\pi), C_1(\rho) \rangle_{\text{trans}}$ \\ \\
  Run $C_2^{\Gamma_V, \Gamma_H} (x, \rho)$ \\
  \IndI Let $(q_j,y_j)$ be the first verification query of $C_2$ such that $\Gamma_v(q_j, y_j) = 1$.\\
  \IndI If $C_2$ does not succeed let $(q_j, y_j)$ be an arbitrary verification query.\\
  \\
  \textbf{If} $(\forall i < j :  hash(q_i) = 0)$ \And $(hash(q_j) = 1)$ \And $(\Gamma_V(q_j, y_j) = 1)$ \then \\
  \IndI \textbf{return} 1\\
  \textbf{else}\\
  \IndI \textbf{return} 0
\end{codeblock}
%
We define the \textit{canonical success probability} of a solver $C$ for $P$ with respect to a function $hash$ as
\begin{align}
 \underset{\pi, \rho}{\Pr}[CanonicalSuccess^{P, C, hash}(\pi, \rho) = 1].
\end{align}
%
For fixed $hash$ and a problem poser $P$ a \textit{canonical success} of $C$ for $\pi, \rho$ is a situation where $CanonicalSuccess^{P, C, hash}(\pi, \rho) = 1$.

We show that if a solver circuit $C$ for $P^{(g)}$ often succeeds in the experiment $Success$, then it is
also often successful in the experiment $CanonicalSuccess$.

\begin{lemma}\textbf{(\boldmath{Success probability in solving a $k$-wise direct product of $P^{(1)}$ with respect to a function $hash$.)}}
\label{lemma:hash_function_probability}
For fixed $P^{(g)}$ let $C$ be a solver for $P^{(g)}$ with the success probability at least $\gamma$,
asking at most $h$ hint queries and $v$ verification queries.
There exists a probabilistic algorithm \textbf{FindHash} that takes as input:
parameters $\gamma$, $n$, $k$, the number of verification queries $v$ and hint queries $h$, and has
oracle access to $C$ and $P^{(g)}$. Furthermore, \textbf{FindHash} runs in time $O((h+v)^4/\gamma^4)$,
and with high probability outputs a function $hash \in \cH$
such that the canonical success probability of $C$ with respect to $hash$ is at least $\frac{\gamma}{8(h+v)}$.
\end{lemma}
%
\begin{proof}
We fix $P^{(g)}$ and a solver $C$ for $P^{(g)}$ in the whole proof of Lemma \ref{lemma:hash_function_probability}.
Let $\cH$ be a family of pairwise independent hash functions $Q \rightarrow \{0,1, \dots,2(h+v)-1\}$.
For all $m,n \in \{1, \dots, (h+v)\}$ and $k,l \in \{0,1,\dots,2(h+v)-1\}$ by the pairwise independence property of $\cH$, we have
\begin{align}
  \label{eq:hash_pr}
 \forall q_m,q_n \in Q, q_m \neq q_n : \underset{\textit{hash} \leftarrow \cH}{\Pr}[hash(q_m) = k \mid hash(q_n) = l] = \underset{\textit{hash} \leftarrow \cH}{\Pr}[hash(q_m) = k] = \frac{1}{2(h+v)}.
\end{align}
%
Let $\cP_{Success}$ be a set containing all $(\pi^{(k)},\rho)$ for which $Success^{P^{(g)}, C}(\pi^{(k)}, \rho) = 1$.
We choose uniformly at random $hash \leftarrow \cH$, and consider the experiment $CanonicalSuccess^{P^{(g)}, C, hash}(\pi^{(k)}, \rho)$.
We are interested in the probability of the event that for a fixed $(\pi, rho) \in \cP_{Success}$ the solver $C$ succeeds canonically.
Let $(q_j, y_j)$ denote the first query such that $\Gamma_V(q_j, y_j) = 1$.
We have
\begin{align*}
  &\underset{\textit{hash} \leftarrow \cH}{\Pr}[hash(q_j) = 0 \land (\forall i < j : hash(q_i) \neq 0)]\\
  &\IndII = \underset{\textit{hash} \leftarrow \cH}{\Pr}[\forall i < j : hash(q_i) \neq 0 \mid hash(q_j) = 0] \underset{\textit{hash} \leftarrow \cH}{\Pr}[hash(q_j) = 0] \\
  &\IndII \stackrel{(\ref{eq:hash_pr})}{=} \frac{1}{2(h+v)}\left(1 -\underset{\textit{hash} \leftarrow \cH}{\Pr}[\exists i < j : hash(q_i) = 0 \mid hash(q_j) = 0] \right) \\
  &\IndII \stackrel{(\ref{eq:hash_pr})}{=} \frac{1}{2(h+v)} \left( 1 -\underset{\textit{hash} \leftarrow \cH}{\Pr}[\exists i < j : hash(q_i) = 0] \right) \\
  &\IndII \stackrel{(\text{u.b})}{\geq} \frac{1}{2(h+v)} \left( 1 - \sum_{i < j} \underset{\textit{hash} \leftarrow \cH}{\Pr}[hash(q_i) = 0] \right) \\
  &\IndII \stackrel{(\ref{eq:hash_pr})}{\geq} \frac{1}{4(h+v)}.
\end{align*}

We denote the set of those $(\pi^{(k)},\rho)$ for which $CanonicalSuccess^{P^{(g)}, C, hash}(\pi^{(k)}, \rho) = 1$ by $\cP_{Canonical}$.
For $(\pi^{(k)}, \rho)$ for which $C$ succeeds canonically, we have $Success^{P^{(g)}, C}(\pi^{(k)}, \rho) = 1$.
Hence, $\cP_{Canonical} \subseteq \cP_{Success}$, and we conclude
\begin{align}
  \label{ineq:hash_high_prob}
\underset{\substack{\textit{hash} \leftarrow \cH \\ \pi^{(k)}, \rho}}{\Pr}\left[CanonicalSuccess^{P^{(g)}, C, hash}(\pi^{(k)}, \rho) = 1\right] &=
\underset{{(\pi^{(k)},\rho) \in \cP_{Success}}}{\mathbb{E}}\left[\underset{\substack{\textit{hash} \leftarrow \cH}}{\Pr}[X = 1]\right] \notag\\
&\geq \frac{\gamma}{4(h+v)}.
\end{align}
%
\begin{codeblock}
  \textbf{Algorithm: FindHash}$(\gamma, n, k, h, v)$
  \medskip
  \hrule
  \medskip
  \textbf{Oracle:} A problem poser $P^{(g)}$, a solver circuit $C$ for $P^{(g)}$.\\
  \textbf{Input:} Parameters $\gamma, n, k, h,v $\\
  \textbf{Output:} A function $hash:Q \rightarrow \{0,1, \dots, 2(h+v)-1 \}$.
  \medskip\hrule\medskip
  Let $\cH$ be a family of pairwise independent hash functions $Q \rightarrow \{0,1,\dots, 2(h+v)-1\}$\\
  \For $i = 1$ \To $32(h+v)^2/\gamma^2$ \Do \\
  \IndI $hash \xleftarrow{\$} \cH$ \\
  \IndI $count := 0$ \\
  \IndI \For $j := 1$ to $32(h+v)^2/\gamma^2$ \Do \\
  \IndII $\pi^{(k)} \xleftarrow{\$} \{0,1\}^{kn} $\\
  \IndII $\rho \xleftarrow{\$} \{0,1\}^*$ \\
  \IndII \If $CanonicalSuccess^{P^{(g)}, C, hash}(\pi^{(k)}, \rho) = 1$ \then \\
  \IndIII $count := count + 1$\\
  \IndI \If $\frac{\gamma^2}{32(h+v)^2} count \geq \frac{\gamma}{6(h+v)}$ \then \\
  \IndII \return $hash$\\
  \return $\bot$
\end{codeblock}
We show that \textbf{FindHash} chooses $hash$ such that the canonical success probability of $C$
with respect to $hash$ is at least $\frac{\gamma}{4(h+v)}$ almost surely.
Let $\cH_{Good}$ denote a family of functions $hash \in \cH$ for which
\begin{align}
  \label{eq:hash_good}
\underset{\pi^{(k)}, \rho}{\Pr}\left[CanonicalSuccess^{P^{(g)}, C, hash}(\pi^{(k)}, \rho) = 1\right] \geq \frac{\gamma}{8(h+v)},
\end{align}
and $\cH_{Bad}$ be the family of functions $hash \in \cH$ such that
\begin{align}
  \label{eq:hash_bad}
\underset{\pi^{(k)}, \rho}{\Pr}\left[CanonicalSuccess^{P^{(g)}, C^{(\cdot, \cdot)}, hash}(\pi^{(k)}, \rho) = 1\right] \leq \frac{\gamma}{16(h+v)}.
\end{align}
%
Let $N$ denote the number of iterations of the inner loop of \textbf{FindHash}.
For a fixed $hash$, we define binary random variables $X_1, \dots, X_{N}$ such that
\begin{align*}
  X_i =
  \begin{cases}
    1 & \text{if in the $i$-th iteration of the inner loop $count$ is increased}\\
    0 & \text{otherwise.}
  \end{cases}
\end{align*}
We show now that \textbf{FindHash} is unlikely to return $hash \in \cH_{Bad}$.
For $hash \in \cH_{Bad}$ by (\ref{eq:hash_bad}) we have $\mathbb{E}_{\pi^{(k)},\rho}[X_i] \leq \frac{\gamma}{16(h+v)}$.
Therefore, for any fixed $hash \in \cH_{Bad}$ using the Chernoff bound we get
\footnote{For $X = \sum_{i=1}^N X_i$ and $0 < \delta \leq 1$ we use the Chernoff bounds in the form
$\Pr[X \geq (1+\delta) \mathbb{E}[X]] \leq e^{- \mathbb{E}[X] \delta^2/3}$ and
$\Pr[X \leq (1-\delta) \mathbb{E}[X]] \leq e^{- \mathbb{E}[X] \delta^2/2}$.}
\begin{align*}
  \underset{\pi^{(k)},\rho}{\Pr} \left[\frac{1}{N} \sum_{i=1}^{N} X_i \geq \frac{\gamma}{12(h+v)} \right] \leq
  \underset{\pi^{(k)}, \rho}{\Pr}\left[\frac{1}{N} \sum_{i=1}^{N} X_i \geq (1 + \frac{1}{4}) \mathbb{E}[X_i]\right] \leq
  e^{-{\frac{\gamma}{16(h+v)}} N /48} \leq e^{-\frac{1}{24}\frac{(h+v)}{\gamma}}.
\end{align*}
%
The probability that $hash \in \cH_{Good}$, when picked, is not returned amounts
\begin{align*}
  \underset{\pi^{(k)}, \rho}{\Pr}\left[\frac{1}{N} \sum_{i=1}^{N} X_i \leq \frac{\gamma}{12(h+v)}\right] \leq
  \underset{\pi^{(k)}, \rho}{\Pr}\left[\frac{1}{N} \sum_{i=1}^{N} X_i \leq (1 - \frac{1}{3})\mathbb{E}[X_i]\right]
  \leq e^{-{\frac{\gamma}{8(h+v)}} N / 18} \leq e^{-\frac{2}{9} \frac{(h+v)}{\gamma}},
\end{align*}
where we once more used the Chernoff bound.
Now we show that the probability of picking a $hash \in \cH_{Good}$ is at least $\frac{\gamma}{8(h+v)}$.
We proof this statement by contradiction. We assume otherwise, namely that
$\underset{hash \leftarrow \cH}{\Pr}[hash \in \cH_{Good}] < \frac{\gamma}{8(g+v)}$.
We have
\begin{align*}
  &\underset{\substack{hash \leftarrow \cH \\ \pi, \rho}}{\Pr}[CanonicalSuccess^{P,C,hash}(\pi, \rho) = 1] \\
  &\IndI = \underset{\substack{hash \leftarrow \cH \\ \pi, \rho}}{\Pr}[CanonicalSuccess^{P,C,hash}(\pi, \rho) = 1 \mid hash \in \cH_{Good}]
  \underset{hash \leftarrow \cH}{\Pr}[hash \in \cH_{Good}] \\
  & \IndII + \underset{\substack{hash \leftarrow \cH \\ \pi, \rho}}{\Pr}[CanonicalSuccess^{P,C,hash}(\pi, \rho) = 1 \mid hash \notin \cH_{Good}]
  \underset{hash \leftarrow \cH}{\Pr}[hash \notin \cH_{Good}] \\
  & \IndI \leq \underset{hash \leftarrow \cH}{\Pr}[hash \in \cH_{Good}] +
  \underset{\substack{hash \leftarrow \cH \\ \pi, \rho}}{\Pr}[CanonicalSuccess^{P,C,hash}(\pi, \rho) = 1 \mid hash \notin \cH_{Good}] \\
  & \IndI < \frac{\gamma}{8(h+v)} + \frac{\gamma}{8(h+v)} = \frac{\gamma}{4(h+v)}.
\end{align*}
But this contradicts (\ref{ineq:hash_high_prob}).
Finally, we show that \textbf{FindHash} picks in one of its iteration $hash \in \cH_{Good}$ almost surely.
Let $K$ be the number of iterations of the outer loop of \textbf{FindHash}.
Let $Y_i$ be a random variable for the event
that in the $i$-th iteration of the outer loop $hash \in \cH_{Good}$ is picked.
Using $\underset{hash \leftarrow \cH}{\Pr}[hash \in \cH_{Good}] < \frac{\gamma}{8(g+v)}$ and  $K \leq \frac{32(h+v)^2}{\gamma^2}$ we conclude
\begin{align*}
  \underset{hash \leftarrow \cH}{\Pr}[ \bigcap_{1 \leq i \leq K} Y_i ] \leq \left(1 - \frac{\gamma}{8(h+v)}\right)^{K}
    \leq e^{-\frac{\gamma}{8(h+v)} K}
    \leq e^{-\frac{4(h+v)}{\gamma}}.
\end{align*}
\end{proof}
%%% Local Variables:
%%% mode: latex
%%% TeX-master: "../master"
%%% End:

%
\subsection{The hardness amplification proof for partitioned domains}
\label{st:amplification_proof}
\begin{todo}
  \textbf{TODO:} Add short introduction
\end{todo}

Let $C := (C_1, C_2)$ be a two-phase solver circuit as in Definition \ref{def:dwvp}.
We write $C_2^{(\cdot, \cdot)}$ to emphasize that $C_2$ does not obtain direct access to the hint and verification oracles.
Instead, whenever $C_2$ asks a hint or verification query, it is answered explicitly
as in the following code excerpt of the circuit $\widetilde{C}_2$.

\begin{codeblock}
  \textbf{Circuit} $\widetilde{C}_2^{\Gamma_H, C_2, \hash} (x, \rho)$
  \medskip \hrule
  \textbf{Oracle:} A hint circuit $\Gamma_H$, a circuit $C_2$, \\ \IndII a function $\hash : \cQ \rightarrow \{0,1,\dots, 2(h+v)-1\}$. \\
  \textbf{Input:} Bitstrings $x \in \{0,1\}^{*}$, $\rho \in \{0,1\}^{*}$. \\
  \textbf{Output:} A pair $(q, y)$ where $q \in \cQ$ and $y \in \{0,1\}^{*}$.
  \medskip\hrule
  \Run $C_2^{(\cdot, \cdot)}(x, \rho)$ \\
  \IndI \If $C_2^{(\cdot, \cdot)}(x, \rho)$ asks a hint query on $q$ \Then\\
  \IndII \If $\hash(q) = 0$ \Then\\
  \IndIII \Return $\bot$\\
  \IndII \textbf{else}\\
  \IndIII answer the query of $C_2^{(\cdot, \cdot)}(x, \rho)$ using $\Gamma_H(q)$\\
  \\
  \IndI \If $C_2^{(\cdot, \cdot)}(x, \rho)$ asks a verification query $(q, y)$ \Then \\
  \IndII \If $\hash(q) = 0 $ \textbf{then} \\
  \IndIII \Return $(q, y)$ \\
  \IndII \textbf{else} \\
  \IndIII answer the verification query of $C_2^{(\cdot, \cdot)}(x, \rho)$ with 0 \\
  \Return $\bot$
\end{codeblock}
%
Given $C := (C_1, C_2)$ we define a circuit $\widetilde{C} := (C_1, \widetilde{C}_2)$.
Every hint query $q$ asked by $\widetilde{C}$ is such that $hash(q) \neq 0$.
Furthermore, $\widetilde{C}$ asks no verification queries, instead it returns $(q,y)$ such that $hash(q) = 0$ or $\bot$.

For fixed $\pi$, $\rho$, and $hash$ we say that the circuit $\widetilde{C}$ \textit{succeeds} if
for $x := \langle P(\pi), C_1(\rho) \rangle_{\mathit{trans}}$,
$(\Gamma_V, \Gamma_H) := \langle P(\pi), C_1(\rho) \rangle_{P}$, we have
\begin{align*}
\Gamma_V(\widetilde{C}_2^{\Gamma_H, C_2, \hash}(x, \rho)) = 1.
\end{align*}
%
\begin{todo}
  \textbf{TODO:} Show the intuitive meaning of this lemma
\end{todo}

\begin{lemma}
  \label{lemma:ctilda_c}
  For fixed $P$, $C := (C_1, C_2)$, and $hash$ it holds
  \begin{align*}
    \underset{\pi, \rho}{\Pr}[\CanonicalSuccess^{P, C, \hash}(\pi, \rho) = 1]
    \leq
    \mkern13mu
    \underset{
      \mathclap{
      \substack{
        \pi, \rho \\
        x := \langle P(\pi), C_1(\rho) \rangle_{\mathit{trans}} \\
        (\Gamma_V, \Gamma_H) := \langle P(\pi), C_1(\rho) \rangle_{P}}}}
  {\Pr}[\Gamma_V(\widetilde{C}_2^{\Gamma_H, C_2, hash}(x, \rho)) = 1].
  \end{align*}
\end{lemma}
%
\begin{todo}
  \textbf{TODO:} Give an overview of this Lemma
\end{todo}
\begin{proof}[Proof of Lemma \ref{lemma:ctilda_c}]
If for some $\pi$, $\rho$, and $\hash$ the circuit $C := (C_1, C_2)$ succeeds canonically,
then for the same $\pi$, $\rho$, and $\hash$ the circuit $\widetilde{C} := (C_1, \widetilde{C}_2)$ also succeeds.
Using this observation, we conclude that
\begin{align*}
  &\underset{\pi, \rho}{\Pr}\left[\CanonicalSuccess^{P, C, \hash}(\pi, \rho) = 1\right] \\
  &\IndII \leq
  \mkern33mu
    \underset{
      \mathclap{
        \substack{\pi, \rho \\
        x := \langle P(\pi), C_1(\rho) \rangle_{\mathit{trans}} \\
        (\Gamma_V, \Gamma_H) := \langle P(\pi), C_1(\rho) \rangle_{P}
      }}}
    {\mathbb{E}}\mkern13mu\big[\Gamma_V(\widetilde{C}_2^{\Gamma_H, C_2, \hash}(x, \rho)) = 1\big] \\
  &\IndII =
  \mkern33mu
    \underset{
      \mathclap{
        \substack{\pi, \rho \\
        x := \langle P(\pi), C_1(\rho) \rangle_{\mathit{trans}} \\
        (\Gamma_V, \Gamma_H) := \langle P(\pi), C_1(\rho) \rangle_{P}
      }}}
    {\Pr}\mkern13mu\big[\Gamma_V(\widetilde{C}_2^{\Gamma_H, C_2, \hash}(x, \rho)) = 1\big]
\end{align*}
\end{proof}
%
%
\begin{todo}
  \textbf{TODO:} intuition behind the lemma \\
  \textbf{TODO:} bases on \cite{holenstein2011general}
\end{todo}
The following Lemma is analogous to Theorem 10 from \cite{holenstein2011general}.
\begin{lemma}[Hardness amplification for a dynamic interactive weakly verifiable puzzle with respect to $\hash$]
  \label{lemma:sec_amp_for_p_hash}
  Let $g: \{0,1\}^{k} \rightarrow \{0,1\}$ be a monotone function, $P_n^{(1)}$ a fixed
  problem poser and $\widetilde{C} := (C_1, \widetilde{C}_2)$ a probabilistic two-phase circuit
  with oracle access to a function $\hash: \cQ \rightarrow \{0,1,\dots, 2(h+v)-1\}$
  and a solver $C := (C_1, C_2)$ for $P_{kn}^{(g)}$ that asks at most $h$ hint queries and $v$ verification queries.
  There exists an algorithm Gen that takes as input parameters $\varepsilon$, $\delta$, $n$, $k$,
  has oracle access to $P_n^{(1)}$,  $\widetilde{C}$, $\hash$, $g$,
  and outputs a probabilistic two-phase circuit $D := (D_1, D_2)$ such that the following holds: \\
  If $\widetilde{C}$ is such that
  \begin{align*}
    \underset{\mathclap{\substack{
          \pi^{(k)} \in \{0,1\}^{kn}, \rho \in \{0,1\}^{*} \\
          x:= \langle P^{(g)}(\pi^{(k)}), {C}_1(\rho) \rangle_{\mathit{trans}} \\
          (\Gamma_H^{(k)}, \Gamma_V^{(g)}) := \langle P^{(g)}(\pi^{(k)}), C_1(\rho) \rangle_{P^{(g)}}}}}
    {\Pr}[\Gamma_V^{(g)}(\widetilde{C}_2^{\Gamma_H^{(k)}, C_2, \hash}(x,\rho)) = 1]
    \geq \underset{u \leftarrow \mu_\delta^k}{\Pr}[g(u) = 1] + \varepsilon,
  \end{align*}
  then $D$ satisfies almost surely over the randomness of Gen
  \begin{align*}
    \underset{
      \mathclap{
      \substack{
        \pi \in \{0,1\}^{n}, \rho \in \{0,1\}^{*} \\
        x := \langle P^{(1)}(\pi), D_1^{\widetilde{C}}(\rho) \rangle_{\mathit{trans}} \\
        (\Gamma_H, \Gamma_V) := \langle P^{(1)}(\pi), D_1^{\widetilde{C}}(\rho) \rangle_{P^{(1)}}}}}
    {\Pr}[\Gamma_V(D_2^{P^{(1)}, \widetilde{C}, \hash, g, \Gamma_H}(x, \rho)) = 1] \geq \delta + \frac{\varepsilon}{6k}.
  \end{align*}
  Furthermore, $D$
  asks at most $\frac{6k}{\epsilon}\log\left(\frac{6k}{\epsilon}\right) h$ hint queries and no verification queries.
  Finally, the running time of $\mathit{Gen}$ is polynomial in $k, \frac{1}{\varepsilon}, n$ with oracle calls.
\end{lemma}
We note that the circuit $D$ from Lemma \ref{lemma:sec_amp_for_p_hash} does not ask any verification queries,
instead it outputs a pair $(q, y)$ such that $\hash(q) = 0$ or $\bot$.

Before we give the proof of Lemma \ref{lemma:sec_amp_for_p_hash} we define additional algorithms.
First, in the following code listing the algorithm $\Gen$ from Lemma \ref{lemma:sec_amp_for_p_hash} is defined.
The procedures and circuits used by $\Gen$ are presented on the succeeding code listings.
\begin{codeblock}
  \textbf{Algorithm} $\Gen^{P^{(1)}, \widetilde{C}, g, \mathit{hash}}(\epsilon, \delta, n, k)$
  \medskip \hrule
  \textbf{Oracle:} A poser $P^{(1)}$, a solver $\widetilde{C}$ for $P^{(g)}$, functions $g: \{0,1\}^{k} \rightarrow \{0,1\}$, $hash:\cQ \rightarrow \{0,1, \dots, 2(h + v) - 1\}$. \\
  \textbf{Input:}  Parameters $\epsilon$, $\delta$, $n$, $k$.\\
  \textbf{Output:} A circuit $D$.
  \medskip\hrule
  \For $i:=1$ \To $\frac{6k}{\epsilon}n$ \Do \\
  \IndI $\pi^* \xleftarrow{\$} \{0,1\}^{n}$\\
  \IndI $\widetilde{S}_{\pi^*,0} := \text{EstimateSurplus}^{P^{(1)},  \widetilde{C}, g, hash}(\pi^*, 0, k, \epsilon, \delta,n)$\\
  \IndI $\widetilde{S}_{\pi^*,1} := \text{EstimateSurplus}^{P^{(1)},  \widetilde{C}, g, hash}(\pi^*, 1, k, \epsilon, \delta,n)$\\
  \IndI \If $ \exists b \in \{0,1\}: \widetilde{S}_{\pi^*,b} \geq (1 - \frac{3}{4k}) \epsilon$ \Then \\
  \IndII Let $C_1'$ have oracle access to $\widetilde{C}$, and have hard-coded $\pi^*$. \\
  \IndII Let $\widetilde{C}_2'$ have oracle access to $\widetilde{C}$, and have hard-coded $\pi^*$. \\
  \IndII $\widetilde{C}' := (C_1', \widetilde{C}_2')$ \\
  \IndII $g'(b_2, \dots, b_k) := g(b, b_2, \dots, b_k)$\\
  \IndII\Return $Gen^{P^{(1)}, \widetilde{C}', g', hash}(\epsilon, \delta, n, k-1)$ \\
  \textit{// all estimates are lower than $(1-\frac{3}{4k})\varepsilon$}\\
  \Return $D^{P^{(1)}, \widetilde{C}, hash, g}$
\end{codeblock}
We are interested in the probability that for $u \leftarrow \mu_{\delta}^k$ and a bit $b$ we have $g(b,u_2, \dotsc, u_k) = 1$.
The estimate of this probability is calculated by the algorithm EstimateFunctionProbability.
%
\begin{codeblock}
  \textbf{Algorithm} $\text{EstimateFunctionProbability}^{g}(b, k, \epsilon, \delta, n)$
  \medskip\hrule
  \textbf{Oracle:} A function $g : \{0,1\}^{k} \rightarrow \{0,1\}$.\\
  \textbf{Input:} A bit $b \in \{0,1\}$, parameters $k$, $\epsilon$, $\delta$, $n$. \\
  \textbf{Output:} An estimate $\widetilde{g}_b$ of $\Pr_{u \leftarrow \mu_{\delta}^{k}}[g(b,u_2, \dotsc, u_k) = 1]$.
  \medskip\hrule
  \For $i:=1$ \To $N := \frac{64k^2}{\epsilon^2} n$ \Do \\
  \IndI $u \leftarrow \mu_{\delta}^{k}$ \\
  \IndI $g_i := g(b, u_2, \dotsc, u_k)$ \\
  \Return $\frac{1}{N} \sum_{i=1}^{N} g_i$
\end{codeblock}
%
For fixed $\pi^{(k)}$, $\rho$, and $hash$ we say that the circuit $\widetilde{C} := (C_1, \widetilde{C}_2)$ \textit{succeeds on the $i$-th coordinate}
if for $x := \langle P^{(g)}(\pi^{(k)}), C_1(\rho) \rangle_{\mathit{trans}}$, $(\Gamma_V^{(g)}, \Gamma_H^{(k)}) := \langle P^{(g)}(\pi), C_1(\rho) \rangle_{P^{(g)}}$ and
$(q, y^{(k)}) := \widetilde{C}_2(x, \rho)$ we have
\begin{align*}
  \Gamma_V^i(q, y_i) = 1.
\end{align*}
%
\begin{lemma}
  \label{lemma:estimate_of_g}
  The algorithm $\text{EstimateFunctionProbability}^{g}(b, k, \epsilon, \delta, n)$ outputs an estimate $\widetilde{g}_b$
  such that $| \widetilde{g}_b - \Pr_{u \leftarrow \mu_{\delta}^{k}}\left[g(b,u_2, \dots, u_k) = 1\right] | \leq \frac{\epsilon}{8k}$ almost surely.
\end{lemma}
%
\begin{proof}
We fix notation as in the code excerpt of the algorithm EstimateFunctionProbability.
Let us define independent and identically distributed binary random variables $K_1, K_2, \dots, K_N$
such that for each $1 \leq i \leq N$ the random variable $K_i$ takes value $g_i$. We use the Chernoff bound to obtain
\begin{align*}
  &\Pr \Bigl[ \Bigl| \widetilde{g}_b - \Pr_{u \leftarrow \mu_{\delta}^{k}}\left[g(b,u_2, \dots, u_k) = 1\right] \Bigr| \geq \frac{\epsilon}{8k} \Bigr]\\
  &\IndII = \Pr \Bigl[\Bigl|\Bigl(\frac{1}{N} \sum_{i=1}^N K_i \Bigr) - \mathbb{E}_{u \leftarrow \mu_{\delta}^k}[g(b,u_2, \dots, u_k)]\Bigr|
    \geq \frac{\epsilon}{8k} \Bigr] \leq 2 \cdot e^{-n/3}.
\end{align*}
\end{proof}
%
The algorithm $\text{EvalutePuzzles}^{P^{(1)}, \widetilde{C}, \hash}(\pi^{(k)}, \rho, n, k)$
evaluates which of the $k$ puzzles of the $k$-wise direct product of $P^{(1)}$ are solved successfully by $\widetilde{C}(\rho) := (C_1,\widetilde{C}_2)(\rho)$.
To decide whether the $i$-th puzzle of the $k$-wise direct product is solved successfully we need to gain access to the verification circuit
for the puzzle generated in the $i$-th round of the interaction between $P^{(g)}$ and $\widetilde{C}$.
Therefore, the algorithm EvalutePuzzles runs $k$ times $P^{(1)}(\pi_i)$ to simulate the interaction with
$C_1(\rho)$ where in each round of interaction a fresh random bitstring $\pi_i \in \{0,1\}^{n}$ is used.

Let us introduce the additional notation.
We write $\langle P^{(1)}(\pi_i), C_1(\rho)\rangle^i$ to denote the $i$-th round of the sequential interaction.
Let $\langle P^{(1)}(\pi_i), C_1(\rho)\rangle^i_{P^{(1)}}$ be the output of $P^{(1)}(\pi_i)$ in the $i$-th round.
Finally, we write $\langle P^{(1)}(\pi_i), C_1(\rho)\rangle^i_{\mathit{trans}}$ to denote the transcript of communication in the $i$-th round.
We note that the $i$-th round of the interaction between $P^{(1)}$ and $C_1$ is well defined only if all previous rounds have been executed before.

For simplicity of the notation in the code excerpts of circuits $C_2$, $D_2$, and EvalutePuzzles we omit superscripts of some oracles.

Exemplary, for $\widetilde{C}_2^{\Gamma_H^{(k)}, C, \hash}$ we omit the superscript $C$ and instead write $\widetilde{C}_2^{\Gamma_H^{(k)}, \hash}$.
We make sure that it is clear from the context which oracles are used.

\begin{todo}
  \textbf{TODO:} Introduce this algorithm
\end{todo}

\begin{codeblock}
  \textbf{Algorithm} $\text{EvaluatePuzzles}^{P^{(1)}, \widetilde{C}, \hash}(\pi^{(k)}, \rho, n, k)$
  \medskip \hrule
  \textbf{Oracle:}  A problem poser $P^{(1)}$, a solver circuit $\widetilde{C} = (C_1, \widetilde{C}_2)$ for $P^{(g)}$,\\
  \IndII a function $hash : \cQ \rightarrow \{0,1,\dots, 2(h+v)-1\}$.\\
  \textbf{Input:} Bitstrings $\pi^{(k)} \in \{0,1\}^{kn}$, $\rho \in \{0,1\}^{*}$, parameters $n$, $k$.\\
  \textbf{Output}: A tuple $(c_1, \dots, c_k) \in \{0,1\}^{k}$.
  \medskip\hrule
  %
  \For $i:=1$ \To $k$ \Do \IndII \textit{//simulate $k$ rounds of interaction} \\
  \IndI $(\Gamma_V^{i}, \Gamma_H^{i}) := \langle P^{(1)}(\pi_i), C_1(\rho) \rangle_{P^{(1)}}^i$\\
  \IndI $x_i := \langle P^{(1)}(\pi_i), C_1(\rho) \rangle^i_{\mathit{trans}}$ \\
  $x := (x_1, \dots, x_k)$ \\
  $\Gamma_H^{(k)} := (\Gamma_H^1, \dotsc, \Gamma_H^k)$ \\
  $(q, y_1, \dots, y_k) := \widetilde{C}_2^{\Gamma_H^{(k)}, hash} (x, \rho)$ \\
  \If $(q, y_1, \dots, y_k) = \bot$ \Then \\
  \IndI \Return $(0, \dotsc, 0)$ \\
  $(c_1, \dotsc, c_k) := (\Gamma_V^{1}(q, y_1), \dotsc, \Gamma_V^{k}(q, y_k))$\\
  \Return $(c_1, \dotsc, c_k)$
\end{codeblock}
%
All puzzles used by EvalutePuzzles are generated internally. Thus, the algorithm can answer all queries of $\widetilde{C}_2$ itself.

We are interested in the success probability of $\widetilde{C}$ with the bitstring $\pi_1$ fixed to $\pi^*$ where
the fact whether $\widetilde{C}$ succeeds in solving the input puzzle defined by $P^{(1)}(\pi_1)$ placed on the first position is neglected,
and instead a bit $b$ is used. More formally, we define the surplus $S_{\pi^*, b}$ as
\begin{align}
  \label{eq:s_pi_b}
S_{\pi^*, b} = \underset{\pi^{(k)}, \rho}{\Pr}\left[g(b, c_2, \dots, c_k) = 1 \mid \pi_1 = \pi^*\right] - \underset{u \leftarrow \mu^{k}_{\delta}}{\Pr}\left[g(b, u_2, \dots, u_k) = 1\right],
\end{align}
where $(c_2, c_3, \dotsc, c_k)$ is obtained as in EvalutePuzzles.

The algorithm EstimateSurplus returns an estimate $\widetilde{S}_{\pi^*, b}$ for $S_{\pi^*, b}$.
%
\begin{codeblock}
  \textbf{Algorithm} $\text{EstimateSurplus}^{P^{(1)}, \widetilde{C}, g, \hash}(\pi^*, b, k, \epsilon, \delta, n)$
  \medskip\hrule
  \textbf{Oracle:} A problem poser $P^{(1)}$, a circuit $\widetilde{C}$ for $P^{(g)}$, functions \\
  \IndII $g: \{0,1\}^{k} \rightarrow \{0,1\}$ and  $\hash : \cQ \rightarrow \{0,1,\dots, 2(h+v)-1\}$.\\
  \textbf{Input:} A bistring $\pi^* \in \{0,1\}^{n}$, a bit $b \in \{0,1\}$, parameters $k$, $\epsilon$, $\delta$, $n$.\\
  \textbf{Output:} An estimate $\widetilde{S}_{\pi^*, b}$ for $S_{\pi^*, b}$.
  \medskip\hrule
  \For $i:=1$ \To $N := \frac{64k^2}{\epsilon^2}n$ \Do \\
  \IndI $(\pi_{2}, \dots, \pi_k) \xleftarrow{\$} \{0,1\}^{(k-1)n}$\\
  \IndI $\rho \xleftarrow{\$} \{0,1\}^{*}$\\
  \IndI $(c_1, \dots, c_k) := \text{EvalutePuzzles}^{P^{(1)}, \widetilde{C}, \hash}((\pi^*, \pi_2, \dots, \pi_k), \rho, n, k)$\\
  \IndI $\widetilde{s}_{\pi^*,b}^i := g(b, c_{2}, \dots, c_k)$\\
  $\widetilde{g}_b := \text{EstimateFunctionProbability}^{g}(b, k, \epsilon, \delta, n)$ \\
  \textbf{return} $\Bigl(\frac{1}{N} \sum_{i=1}^N \widetilde{s}_{\pi^*,b}^i \Bigr) - \widetilde{g}_b$
\end{codeblock}
%
\begin{lemma}
  \label{lemma:surplus_estimate}
The estimate $\widetilde{S}_{\pi^*,b}$ returned by EstimateSurplus differs from $S_{\pi^*, b}$ by at most $\frac{\epsilon}{4k}$ almost surely.
\end{lemma}

\begin{proof}
We use the union bound and similar argument as in Lemma \ref{lemma:estimate_of_g}
which yields that $\frac{1}{N} \sum_{i=1}^{N} \widetilde{s}_{\pi^*,b}^i$ differs from
$\mathbb{E}[g(b, c_2, \dots, c_k)]$ by at most $\frac{\epsilon}{8k}$ almost surely. Together, with Lemma $\ref{lemma:estimate_of_g}$ we conclude that the surplus estimate
returned by EstimateSurplus differs from $S_{\pi^*,b}$ by at most $\frac{\epsilon}{4k}$ with probability at least $1 - 2e^{-n}$.
\end{proof}
%
We define the following solver circuit $C' = (C_1', C_2')$ for the $(k-1)$--wise direct product of $P^{(1)}$.
\begin{todo}
  \textbf{TODO:} Give more intuition why we need this circuit and where it is used
\end{todo}
\begin{codeblock}
  \textbf{Circuit} $C_1'^{\widetilde{C}, P^{(1)}}(\rho)$
  \medskip \hrule
  \textbf{Oracle:} A solver circuit $\widetilde{C} = (C_1, \widetilde{C}_2)$ for $P^{(g)}$, a poser $P^{(1)}$. \\
  \textbf{Input:}  A bitstring $\rho \in \{0,1\}^{*}$. \\
  \textbf{Hard-coded:} A bitstring $\pi^* \in \{0,1\}^{n}$.
  \medskip\hrule
  Simulate $\langle P^{(1)}(\pi^*), C_1(\rho)\rangle^1$ \\
  Use $C_1(\rho)$ for the remaining $k-1$ rounds of interaction.
\end{codeblock}
%
\begin{codeblock}
  \textbf{Circuit} $\widetilde{C}_2'^{\Gamma_H^{(k-1)}, \widetilde{C}, \hash}(x^{(k-1)}, \rho)$
  \medskip \hrule
  \textbf{Oracle:} A hint oracle $\Gamma_H^{(k-1)} := (\Gamma_H^{2}, \dots, \Gamma_H^{k})$,\\
  \IndII a solver circuit $\widetilde{C} = (C_1, \widetilde{C}_2)$ for $P^{(g)}$, \\
  \IndII a function $\hash: \cQ \rightarrow \{0,1,\dots, 2(h+v)-1\}$. \\
  \textbf{Input:}  A transcript of $k-1$ rounds of interaction \\
  \IndII $x^{(k-1)} := (x_2, \dotsc, x_{k}) \in \{0,1\}^{*}$, a bitstring $\rho \in \{0,1\}^{*}$.\\
  \textbf{Hard-coded:} A bitstring $\pi^* \in \{0,1\}^{n}$. \\
  \textbf{Output:} A tuple $(q, y_2, \dots, y_k)$.
  \medskip\hrule
  Simulate $\langle P^{(1)}(\pi^*), C_1(\rho) \rangle^{1}$ \\
  \IndI $(\Gamma_H^*, \Gamma_V^*) := \langle P^{(1)}(\pi^*), C_1(\rho) \rangle^{1}_{P^{(1)}}$ \\
  \IndI $x^* := \langle P^{(1)}(\pi^*), C_1(\rho) \rangle^{1}_{\mathit{trans}}$ \\
  $\Gamma_H^{(k)} := (\Gamma_H^*, \Gamma_H^{2}, \dots, \Gamma_H^{k})$ \\
  $x^{(k)} := (x^*, x_2, \dots, x_{k})$ \\
  $(q, y_1, \dots, y_k) := \widetilde{C}_2^{\Gamma_H^{(k)}, \mathit{hash}}(x^{(k)}, \rho)$ \\
  \Return $(q, y_2, \dots, y_k)$
\end{codeblock}
%
We are ready to define the solver circuit $D = (D_1, D_2)$ for $P^{(1)}$ output by $\Gen$.
%
\begin{codeblock}
  \textbf{Circuit} $D_1^{\widetilde{C}}(r)$
  \medskip \hrule
  \textbf{Oracle:} A solver circuit $\widetilde{C} = (C_1, \widetilde{C}_2)$ for $P^{(g)}$.\\
  \textbf{Input:} A pair $r := (\rho, \sigma)$ where $ \rho \in \{0,1\}^{*}$ and $\sigma \in \{0,1\}^{*}$.
  \medskip\hrule
  Interact with the problem poser $\langle P^{(1)}, C_1(\rho) \rangle^1$. \\
  Let $x^* := \langle P^{(1)}, C_1(\rho) \rangle^1_{\mathit{trans}}$.
\end{codeblock}
%
\begin{codeblock}
  \textbf{Circuit} $D_2^{P^{(1)}, \widetilde{C}, \mathit{hash}, g,  \Gamma_H}(x^*, r)$
  \medskip \hrule
  \textbf{Oracle:} A poser $P^{(1)}$, a solver circuit $\widetilde{C} = (C_1, \widetilde{C}_2)$ for $P^{(g)}$, \\
  \IndII functions $hash : \cQ \rightarrow \{0,1, \dots, 2(h+v)-1\}$, $g:\{0,1\}^k \rightarrow \{0,1\}$, \\
  \IndII a hint circuit $\Gamma_H$ for $P^{(1)}$. \\
  \textbf{Input:} A communiation transcript $x^* \in \{0,1\}^{*}$, a bitstring $r := (\rho, \sigma)$ \\
  \IndII where $\rho \in \{0,1\}^{*}$ and $\sigma \in \{0,1\}^{*}$\\
  \textbf{Output}: A pair $(q, y^*)$.
  \medskip \hrule
  \For at most $\frac{6k}{\epsilon} \log(\frac{6k}{\epsilon})$ iterations \Do \\
  \IndI $(\pi_2, \dots, \pi_k) \leftarrow$ read next $(k-1)\cdot n$ bits from $\sigma$ \\
  \IndI Use $x^*$ to simulate the first round of interaction of $C_1(\rho)$ \\
  \IndI with the problem poser $P^{(1)}$.\\
  \IndI \For $i:=2$ \To $k$ \Do \\
  \IndII \Run $\langle P^{(1)}(\pi_i), C_1(\rho)\rangle^i$ \\
  \IndIII $(\Gamma_V^{i}, \Gamma_H^{i}) := \langle P^{(1)}(\pi_i), C_1(\rho) \rangle^i_{P^{(1)}}$ \\
  \IndIII $x_i := \langle P^{(1)}(\pi_i), C_1(\rho) \rangle^i_{\mathit{trans}}$ \\
  \IndI $\Gamma_H^{(k)}(q) := (\Gamma_H(q), \Gamma_H^{2}(q), \dots, \Gamma_H^{k}(q))$ \\
  \IndI $(q, y^*, y_2, \dots, y_k) := \widetilde{C}_2^{\Gamma_H^{(k)}, \hash}((x^*, x_2, \dotsc, x_k), \rho)$\\
  \IndI $(c_2, \dots, c_k) := (\Gamma_V^2(q, y_2), \dotsc, \Gamma_V^{k}(q, y_k))$ \\
  \IndI \If $g(1, c_{2}, \dots, c_k) = 1$ \And $g(0,c_{2}, \dots, c_k) = 0$ \Then \\
  \IndII \Return $(q, y^*)$ \\
  \Return $\bot$
%
\end{codeblock}
%
%
\begin{proof}[of Lemma \ref{lemma:sec_amp_for_p_hash}]
First, let us consider the case where $k=1$. The function $g: \{0,1\} \rightarrow \{0,1\}$ is either the identity or a constant function.
In the latter case, when $g$ is a constant function, Lemma \ref{lemma:sec_amp_for_p_hash} is vacuously true.
If $g$ is the identity function, then the circuit $D$ returned by Gen directly uses $\widetilde{C}$ to find a solution.
From the assumptions of Lemma \ref{lemma:sec_amp_for_p_hash} it follows that $\widetilde{C}$ succeeds with probability at least
$\delta + \epsilon$. Hence, $D$ trivially satisfies Lemma~\ref{lemma:sec_amp_for_p_hash}.

For the general case, we consider two possibilities.
Namely, either Gen in one of the iterations finds an estimate with high surplus such that $\widetilde{S}_{\pi, b} \geq (1-\frac{3}{4k})\epsilon$ and recurses,
or in all iterations it fails and outputs the circuit~$D$.

If it is possible to find an estimate with high surplus, then we introduce a new monotone function $g': \{0,1\}^{k-1} \rightarrow \{0,1\}$
such that $g'(b_2, \dots, b_k) := g(b, b_2, \dots, b_k)$ and a new circuit $\widetilde{C}' = (C_1', \widetilde{C}_2')$
with oracle access to $\widetilde{C} := (C_1, \widetilde{C}_2)$.
W apply Lemma \ref{lemma:surplus_estimate} and conclude that almost surely it holds
\begin{align*}
S_{\pi^*,b} \geq \widetilde{S}_{\pi^*, b} - \frac{\epsilon}{4k} \geq \Bigl(1 - \frac{1}{k}\Bigr)\epsilon.
\end{align*}
It follows that $\widetilde{C}'$ succeeds in solving the $(k\!-\!1)$--wise direct product of puzzles with probability at least
\begin{align*}
\Pr_{u \leftarrow \mu^{(k-1)}_{\delta}}[g'(u_1,\dots, u_{k-1} )] + \Bigl(1 - \frac{1}{k}\Bigr)\epsilon.
\end{align*}
We see that in this case $\widetilde{C}'$ satisfies the conditions of Lemma \ref{lemma:sec_amp_for_p_hash} for the $(k\!-\!1)$--wise direct product of puzzles.
Therefore, the recursive call to Gen with access to $g'$ and $\widetilde{C}$ returns $D = (D_1, D_2)$ that with high probability satisfies
\begin{align}
  \underset{
    \mathclap{
      \substack{
        \pi, \rho \\
        x := \langle P^{(1)}(\pi), D_1^{\widetilde{C}}(\rho) \rangle_{\mathit{trans}} \\
        (\Gamma_H, \Gamma_V) := \langle P^{(1)}(\pi), D_1^{\widetilde{C}}(\rho) \rangle_{P^{(1)}}}}}
  {\Pr}\Big[\Gamma_V\big(D_2^{P^{(1)}, \widetilde{C}, \hash, g, \Gamma_H}(x, \rho)\big) = 1\Big]
  &\geq \delta + \Bigl(1 - \frac{1}{k}\Bigr)\frac{\epsilon}{6(k-1)} \notag\\
  &= \delta + \frac{\epsilon}{6k}.
\end{align}
%
Let us bring our attention to the case where none of the estimates is greater than $(1-\frac{3}{4k})\epsilon$.
If all surpluses $S_{\pi,0}$ and $S_{\pi,1}$ were lower than $(1-\frac{1}{k})\epsilon$, then it would mean that $\widetilde{C}$
does not succeed on the remaining $k-1$ puzzles with much higher probability than an algorithm that solves each puzzle
independently with success probability $\delta$. However, from the assumptions of Lemma~\ref{lemma:sec_amp_for_p_hash}
we know that on all $k$ puzzles the success probability of $\widetilde{C}$ is higher.
Hence, we suspect that the first puzzle is correctly solved unusually often.
It remains to show that the fact that $\Gen$ fails to find a surplus estimate that is large implies that
with high probability there are only few surpluses greater than $(1-\frac{1}{k})\epsilon$ and their influence
is can be neglected. Additionally, we have to show that the circuit $D$ finds outputs an answer almost surely.

We fix notation as in the code listing of the circuit $D_2$.
Let us consider a single iteration of the outer loop of $D_2$ where values $\pi_2, \dotsc, \pi_k$ are fixed.
We write $\pi_1$ to denote randomness used by the problem poser to generate the input puzzle.
Furthermore, we define $c_1 := \Gamma_V(q,y_1)$ where $\Gamma_V$ is the verification circuit generated
by $P^{(1)}(\pi_1)$ in the first phase when interacting with $D_1(r)$.
We write $c := (c_1, c_2, \dotsc, c_k)$, and for $b \in \{0,1\}$ we define a set
\begin{align*}
\cG_{b}~:=~\big\{(b_1, b_2, \dots, b_k) : g(b, b_2, \dots, b_k) = 1 \big\}.
\end{align*}
Using this notation we express
\begin{align}
  \label{eqs:set_g}
  \underset{u \leftarrow \mu_{\delta}^k}{\Pr}[u \in \cG_b] = \underset{u \leftarrow \mu_{\delta}^k}{\Pr}[g(b, u_2, \dots, u_k) = 1]\notag\\
 \underset{\pi^{(k)}, \rho}{\Pr}[c \in \cG_b] = \underset{\pi^{(k)}, \rho}{\Pr}[g(b, c_2, \dots, c_k) = 1].
\end{align}
Let us fix randomness $\pi_1$ used by the problem poser to generate the input puzzle to $\pi^*$.
We use \eqref{eq:s_pi_b} and \eqref{eqs:set_g} to obtain
\begin{multline}
\label{eq:diff_s01}
\underset{u \leftarrow \mu_{\delta}^k}{\Pr}[u \in \cG_1] - \underset{u \leftarrow \mu_{\delta}^k}{\Pr}[u \in \cG_0] \\
 = \underset{\pi^{(k)}, \rho}{\Pr}[c \in \cG_1 \mid \pi_1 = \pi^*] - \underset{\pi^{(k)}, \rho}{\Pr}[c \in \cG_0 \mid \pi_1 = \pi^*] - (S_{\pi^*, 1} - S_{\pi^*,0})
\end{multline}
By monotonicity of $g$ it holds $\cG_0 \subseteq \cG_1$, and we write \eqref{eq:diff_s01} as
\begin{align}
  \label{eq:diff_s01_next}
  \underset{u \leftarrow \mu_{\delta}^k}{\Pr}[u \in \cG_1 \setminus \cG_0] = \underset{\pi^{(k)}, \rho}{\Pr}[c \in \cG_1 \setminus \cG_0 \mid \pi_1 = \pi^*] - (S_{\pi^*,1} - S_{\pi^*,0}).
\end{align}
Let us multiply both sides of \eqref{eq:diff_s01_next} by
\begin{align*}
\underset{
  \mathclap{
    \substack{r \\ x^* := \langle P^{(1)}(\pi^*), D_1(r) \rangle_{\mathit{trans}}
    \\ (\Gamma_V, \Gamma_H) := \langle P^{(1)}(\pi^*), D_1(r) \rangle_{P^{(1)}} }}}
{\Pr}\mkern13mu [\Gamma_V(D_2(x^*, r)) = 1]
 \mkern11mu / \underset{u \leftarrow \mu_{\delta}^k}{\Pr}[ u \in \cG_1 \setminus\cG_0],
\end{align*}
%
which yields
\begin{align}
\label{eq:pr_d_succ_0}
&\IndII\underset{
  \mathclap{
    \substack{r \\ x^* := \langle P^{(1)}(\pi^*), D_1(r) \rangle_{\mathit{trans}} \\ (\Gamma_V, \Gamma_H) := \langle P^{(1)}(\pi^*), D_1(r) \rangle_{P^{(1)}} }}}
{\Pr}\mkern13mu[\Gamma_V(D_2(x^*, r)) = 1] \notag\\
%
&\IndIII = \mkern13mu
  \underset{
    \mathclap{
      \substack{r \\ x^* := \langle P^{(1)}(\pi^*), D_1 (r) \rangle_{\mathit{trans}} \\ (\Gamma_V, \Gamma_H) := \langle P^{(1)}(\pi^*), D_1 (r) \rangle_{P^{(1)}} }}}
  {\Pr}\mkern13mu[\Gamma_V(D_2(x^*, r)) = 1]
  \underset{\pi^{(k)},\rho}{\Pr}[c \in \cG_1 \setminus \cG_0 \mid \pi_1 = \pi^*]
\frac{1}{\underset{u \leftarrow \mu_{\delta}^k}{\Pr}[ u \in \cG_1 \setminus \cG_0]}\notag\\
%
&\IndIIII - \mkern13mu
\underset{
  \mathclap{
  \substack{r \\ x^* := \langle P^{(1)}(\pi^*), D_1(r) \rangle_{\mathit{trans}} \\ (\Gamma_V, \Gamma_H) := \langle P^{(1)}(\pi^*), D_1(r) \rangle_{P^{(1)}} }}}
{\Pr}\mkern13mu[\Gamma_V(D_2(x^*, r)) = 1](S_{\pi^*,1} - S_{\pi^*,0})
\frac{1}{\underset{u \leftarrow \mu_{\delta}^k}{\Pr}[ u \in \cG_1 \setminus\cG_0]}.
\end{align}
Let us study the first summand of \eqref{eq:pr_d_succ_0}. First, we have
\begin{align}
  \label{eq:pr_gamma_v_0}
  \IndII &\underset{
    \mathclap{
      \substack{r \\
        x^* := \langle P^{(1)}(\pi^*), D_1 (r) \rangle_{\mathit{trans}} \\
        (\Gamma_V, \Gamma_H) := \langle P^{(1)}(\pi^*), D_1(r) \rangle_{P^{(1)}} }}}
  {\Pr}\mkern13mu[\Gamma_V(D_2(x^*, r)) = 1] \notag\\
  &\IndI = \underset{
    \mathclap{
      \substack{r \\
        x^* := \langle P^{(1)}(\pi^*), D_1 (r) \rangle_{\mathit{trans}} \\
        (\Gamma_V, \Gamma_H) := \langle P^{(1)}(\pi^*), D_1(r) \rangle_{P^{(1)}} }}}
  {\Pr}[\Gamma_V(D_2(x^*, r)) = 1 | D_2(x^*,r) \neq \bot]
  \underset{\mathclap{\substack{r \\ x^* = \langle P^{(1)}(\pi^*), D_1(r) \rangle_{\mathit{trans}}}}}{\Pr}[D_2(x^*,r) \neq \bot] \notag\\
  &\IndI \stackrel{(*)}{=}
  \underset{\pi^{(k)}, \rho}{\Pr}[c_1 = 1 \mid c \in \cG_1 \setminus \cG_0, \pi_1 = \pi^*]
  \underset{\mathclap{\substack{r \\ x^* = \langle P^{(1)}(\pi^*), D_1(r) \rangle_{\mathit{trans}}}}} {\Pr}[D_2(x^*,r) \neq \bot]
\end{align}
where in $(*)$ we use the observation that $D_2(x^*, r) \neq \bot$ occurs if and only if $D_2(x^*, r)$ finds $\pi^{(k)}$ such that $c \in \cG_1 \setminus \cG_0$.
Furthermore, conditioned on $c \in \cG_1 \setminus \cG_0$ we have that $\Gamma_V(D_2(x^*,r)) = 1$ happens if and only if $c_1 = 1$.
Inserting \eqref{eq:pr_gamma_v_0} to the numerator of the first summand of (\ref{eq:pr_d_succ_0}) yields
\begin{align}
  \label{ineq:start_for_case}
\IndI &\underset{
  \mathclap{
  \substack{r \\
    x^* := \langle P^{(1)}(\pi^*), D_1 (r) \rangle_{\mathit{trans}} \\
    (\Gamma_V, \Gamma_H) := \langle P^{(1)}(\pi^*), D_1(r) \rangle_{P^{(1)}} }}}
{\Pr}\mkern13mu[\Gamma_V(D_2(x^*, r)) = 1]
\underset{\pi^{(k)},\rho}{\Pr}[c \in \cG_1 \setminus \cG_0 \mid \pi_1 = \pi^*] \notag\\
  &\IndI = \underset{\mathclap{\substack{r
      \\ x^* = \langle P^{(1)}(\pi^*), D_1(r) \rangle_{\mathit{trans}}}}}
  {\Pr}\mkern13mu[D_2(x^*,r) \neq \bot]
  \underset{\pi^{(k)}, \rho}{\Pr}[c_1 = 1 \mid c \in \cG_1 \setminus \cG_0, \pi_1 = \pi^*]
  \underset{\pi^{(k)}, \rho}{\Pr}[c \in \cG_1 \setminus \cG_0 \mid \pi_1 = \pi^*].
\end{align}
We consider the following two cases. First, if $\Pr_{\pi^{(k)}, \rho}[ c \in \cG_1 \setminus \cG_0 \mid \pi_1 = \pi^*] \leq \frac{\epsilon}{6k}$ then
\begin{align}
  \label{ineq:case_0}
  \underset{\pi^{(k)}, \rho}{\Pr}[c_1 = 1 \mid c \in \cG_1 \setminus \cG_0, \pi_1 = \pi^*] \underset{\pi^{(k)}, \rho}{\Pr}[c \in \cG_1 \setminus \cG_0 \mid \pi_1 = \pi^*] \leq \frac{\epsilon}{6k}.
\end{align}
In case when $\Pr_{\pi^{(k)}, \rho}[c \in \cG_1 \setminus \cG_0 \mid \pi_1 = \pi^*] > \frac{\epsilon}{6k}$ the circuit $D_2$ outputs $\bot$
if and only if it fails in all $\frac{6k}{\epsilon} \log(\frac{6k}{\epsilon})$ iterations to find $\pi^{(k)}$ such that $c \in \cG_1 \setminus \cG_0$
which happens with probability
\begin{align}
  \label{ineq:case_1}
\underset{
  \mathclap{
    \substack{
      r \\
      x^* := \langle P^{(1)}(\pi^*), D_1(r) \rangle_{\mathit{trans}}}}}
{\Pr}[D_2(x^*,r) = \bot]
\leq \Big(1 - \frac{\epsilon}{6k}\Big)^{\frac{6k}{\epsilon}\log(\frac{6k}{\epsilon})} \leq \frac{\epsilon}{6k}.
\end{align}
We conclude that in both aforementioned cases using \eqref{ineq:start_for_case}, \eqref{ineq:case_0} and \eqref{ineq:case_1} the following holds
\begin{align}
  \label{ineq:first_part}
  &\underset{
    \mathclap{
    \substack{r \\
      x^* := \langle P^{(1)}(\pi^*), D_1(r) \rangle_{\mathit{trans}}}}}
  {\Pr}\mkern13mu[D_2(x^*,r) \neq \bot]
  \underset{\pi^{(k)}, \rho}{\Pr}[c_1 = 1 \mid c \in \cG_1 \setminus \cG_0, \pi_1 = \pi^*]
  \underset{\pi^{(k)}, \rho}{\Pr}[c \in \cG_1 \setminus \cG_0 \mid \pi_1 = \pi^*] \notag\\
  &\IndII \stackrel{\hphantom{(\ref{eq:s_pi_b})}}{\geq}
  \underset{\pi^{(k)}, \rho}{\Pr}[c_1 = 1 \mid c \in \cG_1 \setminus \cG_0, \pi_1 = \pi^*]\underset{\pi^{(k)}, \rho}
  {\Pr}[c \in \cG_1 \setminus \cG_0 \mid \pi_1 = \pi^*] - \frac{\epsilon}{6k} \notag\\
  &\IndII \stackrel{\hphantom{(\ref{eq:s_pi_b})}}{=}
  \underset{\pi^{(k)}, \rho}{\Pr}[c_1 = 1 \land c \in \cG_1 \setminus \cG_0 \mid \pi_1 = \pi^*] - \frac{\epsilon}{6k} \notag\\
  &\IndII \stackrel{\hphantom{(\ref{eq:s_pi_b})}}{=}
  \underset{\pi^{(k)}, \rho}{\Pr}[g(c) = 1 \mid \pi_1 = \pi^*] -  \underset{\pi^{(k)}, \rho}{\Pr}[c \in \cG_0 \mid \pi_1 = \pi^*] - \frac{\epsilon}{6k} \notag\\
  &\IndII \stackrel{(\ref{eq:s_pi_b})}{=}
   \underset{\pi^{(k)}, \rho}{\Pr}[g(c) = 1 \mid \pi_1 = \pi^*] -  \underset{u \leftarrow \mu_{\delta}^{(k)}}{\Pr}[u \in \cG_0]  - S_{\pi^*,0} - \frac{\epsilon}{6k}.
\end{align}
We insert \eqref{ineq:first_part} into \eqref{eq:pr_d_succ_0} and calculate the expected value over $\pi^*$ which yields
\begin{align}
  \label{ineq:ex_pr_gamma}
\underset{
  \mathclap{
    \substack{\pi, r \\ x := \langle P^{(1)}(\pi), D_1(r) \rangle_{\mathit{trans}} \\ (\Gamma_V, \Gamma_H) := \langle P^{(1)}(\pi), D_1(r) \rangle_{P^{(1)}} }}}
{\Pr}[\Gamma_V(D_2(x, r)) = 1]
&\geq \mathbb{E}_{\pi^*}\left[\frac{\Pr_{\pi^{(k)}, \rho}[g(c) = 1 | \pi_1 = \pi^*]
  - \Pr_{u \leftarrow \mu_{\delta}^{(k)}}[u \in \cG_0] - \frac{\epsilon}{6k}}{\Pr_{u \leftarrow \mu_{\delta}^{(k)}}[u \in \cG_1 \setminus \cG_0]}\right] \notag\\
&- \mathbb{E}_{\pi^*}\Bigl[\Bigl(
\underset{\mathclap{
  \substack{r \\ x^* := \langle P^{(1)}(\pi^*), D_1(r) \rangle_{\mathit{trans}} \\ (\Gamma_V, \Gamma_H) := \langle P^{(1)}(\pi^*), D_1(r) \rangle_{P^{(1)}} }}}
{\Pr}[\Gamma_V(D_2(x^*, r)) = 1](S_{\pi^*,1} - S_{\pi^*,0})
 + S_{\pi^*,0}\Bigr)
\frac{1}{\underset{u \leftarrow \mu_{\delta}^k}{\Pr}[ u \in \cG_1 \setminus\cG_0]}\Bigr].
\end{align}
We now show that if Gen does not recurse, then the majority of estimates is low almost surely.
Let us assume that
\begin{align}
\underset{\pi}{\Pr}\left[\left(S_{\pi,0} \leq (1 - \frac{1}{2k})\epsilon\right) \land \left( S_{\pi,1} \leq (1-\frac{1}{2k})\epsilon\right)\right] < 1 - \frac{\epsilon}{6k},
\end{align}
then Gen recurses almost surely, because the probability that
Gen does not find $\widetilde{S}_{\pi, b} \geq (1-\frac{3}{4k})\epsilon$ in all of the $\frac{6k}{\epsilon}n$ iterations is at most
\begin{align*}
  \Bigl(1 - \frac{\epsilon}{6k}\Bigr)^{\frac{6k}{\epsilon}n} \leq e^{-n}
\end{align*}
almost surely, where we used Lemma \ref{lemma:surplus_estimate}.
Therefore, under the assumption that Gen does not recurse with high probability it holds
\begin{align}
\underset{\pi, \rho}{\Pr}\left[\left(S_{\pi,0} \leq (1 - \frac{1}{2k})\epsilon\right) \land \left( S_{\pi,1} \leq (1-\frac{1}{2k})\epsilon\right)\right] \geq 1 - \frac{\epsilon}{6k}.
\end{align}
Let us define a set
\begin{align}
  \cW = \left\{ \pi :  \left(S_{\pi,0} \leq (1 - \frac{1}{2k})\epsilon\right) \land \left( S_{\pi,1} \leq (1-\frac{1}{2k})\epsilon \right) \right\}.
\end{align}
Additionally, let $\overline{\cW}$ denote the complement of $\cW$.
We bound the numerator of the second summand in (\ref{ineq:ex_pr_gamma})
\begin{align}
  \label{ineq:second_eq}
&\mathbb{E}_{\pi^*}\Big[ S_{\pi^*,0}
\mkern23mu
+
\mkern23mu
\underset{
  \mathclap{
  \substack{r \\ x^* := \langle P^{(1)}(\pi^*), D_1(r) \rangle_{\mathit{trans}}
    \\ (\Gamma_V, \Gamma_H) := \langle P^{(1)}(\pi^*), D_1 (r) \rangle_{P^{(1)}} }}}
{\Pr}\mkern13mu[\Gamma_V(D_2(x^*, r)) = 1]
(S_{\pi^*,1} - S_{\pi^*,0})\Big] \notag\\
%
&\IndII = \mathbb{E}_{\pi^*}\Bigl[ S_{\pi^*,0}
\mkern23mu + \mkern23mu
\underset{
  \mathclap{
  \substack{r \\ x^* := \langle P^{(1)}(\pi^*), D_1(r) \rangle_{\mathit{trans}}
    \\ (\Gamma_V, \Gamma_H) := \langle P^{(1)}(\pi^*), D_1 (r) \rangle_{P^{(1)}} }}}
{\Pr}\mkern13mu[\Gamma_V(D_2(x^*, r) = 1]
  (S_{\pi^*,1} - S_{\pi^*,0}) \bigm| \pi^* \in \overline{\cW}\Bigr] \Pr_{\pi^*}[\pi^* \in \overline{\cW}]\notag\\
&\IndIII +  \mathbb{E}_{\pi^*}\Bigl[ S_{\pi^*,0} \mkern13mu + \mkern13mu
\underset{
  \mathclap{
  \substack{r \\ x^* := \langle P^{(1)}(\pi^*), D_1(r) \rangle_{\mathit{trans}}
    \\ (\Gamma_V, \Gamma_H) := \langle P^{(1)}(\pi^*), D_1 (r) \rangle_{P^{(1)}} }}}
{\Pr}\mkern13mu[\Gamma_V(D_2(x^*, r)) = 1]
(S_{\pi^*,1} - S_{\pi^*,0})  \bigm| \pi^* \in \cW\Bigr] \Pr_{\pi^*}[\pi^* \in \cW] \notag\\
&\IndII \leq \frac{\epsilon}{6k} + \mathbb{E}_{\pi^*}\Bigl[ S_{\pi^*,0} \mkern23mu + \mkern23mu
\underset{
  \mathclap{
  \substack{r \\ x := \langle P^{(1)}(\pi^*), D_1(r) \rangle_{\mathit{trans}}
    \\ (\Gamma_V, \Gamma_H) := \langle P^{(1)}(\pi^*), D_1 (r) \rangle_{P^{(1)}} }}}
{\Pr}\mkern13mu\big[\Gamma_V(D_2^{\widetilde{C}}(x^*, r)) = 1\big]
\big(\bigl(1 - \frac{1}{2k}\bigr)\epsilon - S_{\pi^*,0}\big)  \bigm| \pi^* \in \cW \Bigr] \notag\\
% \Pr_{\pi^*}[\pi^* \in \cW] \notag\\
&\IndII \leq \frac{\epsilon}{6k} + (1 - \frac{1}{2k})\epsilon = (1 - \frac{1}{3k}) \epsilon.
\end{align}
Finally, we insert \eqref{ineq:second_eq} into \eqref{ineq:ex_pr_gamma} which yields
\begin{align*}
  \IndI
\underset{
  \mathclap{
  \substack{\pi, \rho \\ x := \langle P^{(1)}(\pi), D_1(\rho) \rangle_{\text{trans}}
    \\ (\Gamma_V, \Gamma_H) := \langle P^{(1)}(\pi), D_1 (\rho) \rangle_{P^{(1)}} }}}
{\Pr}\big[\Gamma_V(D_2(x, \rho)) = 1\big]
&\geq \underset{\pi^*}{\mathbb{E}}\left[\frac{{\Pr}_{\pi^{(k)}, \rho}[g(c) = 1 \mid \pi_1 = \pi^*] -
{\Pr}_{u \leftarrow \mu_{\delta}^{k}}[u \in G_0] - (1 - \frac{1}{6k})\epsilon} {\Pr_{u \leftarrow \mu_{\delta}^{k}}[u \in \cG_1 \setminus \cG_0]}\right] \notag.
 \end{align*}
 From the assumptions of Lemma \ref{lemma:sec_amp_for_p_hash} it follows that
 \begin{align}
   \label{eq:lemma_assum}
   \Pr_{\pi^{(k)}, \rho} [g(c) = 1] \geq \Pr_{u \leftarrow \mu_{\delta}^{(k)}}[g(u) = 1] + \epsilon.
 \end{align}
We observe that
\begin{align}
  \label{eq:gu_rel}
\underset{u \leftarrow \mu_{\delta}^k}{\Pr}[g(u) = 1]
&= \Pr[u \in \cG_0 \lor ( u \in \cG_1 \setminus \cG_0 \land u_1 = 1)] \notag\\
&= \Pr[u \in \cG_0] + \delta \Pr[u \in \cG_1 \setminus \cG_0].
\end{align}
 Using \eqref{eq:gu_rel} and \eqref{eq:lemma_assum} we obtain
 \begin{align}
   \label{eq:proof_final}
   \IndI
\underset{
  \mathclap{
  \substack{\pi, \rho \\ x := \langle P^{(1)}(\pi), D_1(\rho) \rangle_{\text{trans}}
    \\ (\Gamma_V, \Gamma_H) := \langle P^{(1)}(\pi), D_1 (\rho) \rangle_{P^{(1)}} }}}
{\Pr}\mkern13mu[\Gamma_V(D_2(x, \rho)) = 1]
 &\stackrel{\hphantom{\eqref{eq:gu_rel}}}{\geq} \frac{ {\Pr}_{u \leftarrow \mu_{\delta}^{k}}[g(u) = 1] + \epsilon -
 \Pr_{u \leftarrow \mu_{\delta}^{k}}[u \in \cG_0] - (1 - \frac{1}{6k})\epsilon} {\Pr_{u \leftarrow \mu_{\delta}^{k}}[u \in \cG_1 \setminus \cG_0]} \notag\\
 &\stackrel{\eqref{eq:gu_rel}}{\geq} \frac{\epsilon + \delta\Pr_{u \leftarrow \mu_{\delta}^{k}}[u \in \cG_1 \setminus \cG_0] - (1 - \frac{1}{6k})\epsilon}
{\Pr_{u \leftarrow \mu_{\delta}^{k}}[u \in \cG_1 \setminus \cG_0]} \geq \delta + \frac{\epsilon}{6k}.
\end{align}
Clearly, the running time of $\Gen$ is bounded by some polynomial $p(k, \frac{1}{\epsilon}, n)$ with oracle calls.
Furthermore, the algorithm $\Gen$ outputs a circuit $D$ such that it satisfies with probability at least $1 - \p(k, \frac{1}{\epsilon}, n) \cdot 2^n$
the statement of Lemma \ref{lemma:sec_amp_for_p_hash}. This concludes the proof of Lemma~\ref{lemma:sec_amp_for_p_hash}.
\end{proof}
%
%%% Local Variables:
%%% mode: latex
%%% TeX-master: "../master"
%%% End:

%
\begin{proof}[Theorem \ref{th:sec_amp_for_dwvp}]
We define the following circuits.
%
\begin{codeblock}
  \textbf{Circuit} $\widetilde{D}_2^{D, P^{(1)}, \hash, g, \Gamma_V, \Gamma_H}(x, \rho)$
  \medskip
  \hrule
  \medskip
  \textbf{Oracle:} A circuit $D :=(D_1, D_2)$ from Lemma \ref{lemma:sec_amp_for_p_hash}, a problem poser $P^{(1)}$, \\
  \IndII functions $\hash: Q \rightarrow \{0,1, \dots, 2(h+v) - 1\}$, $g: \{0,1\}^{k} \rightarrow \{0,1\}$ \\
  \IndII a verification oracle $\Gamma_V$, a hint oracle $\Gamma_H$.\\
  \textbf{Input:}  Bitstrings $x \in \{0,1\}^{*}$, $\rho \in \{0,1\}^{*}$.
  \medskip\hrule\medskip
  %
  $(q, y) := D_2^{P^{(1)}, \widetilde{C}, \hash, g, \Gamma_H}(x, \rho)$ \\
  Make a verification query to $\Gamma_V$ using $(q,y)$
\end{codeblock}
%
\begin{codeblock}
  \textbf{Algorithm} $\widetilde{\text{Gen}}^{P^{(1)}, g, C}(n, \epsilon, \delta, k, h, v)$
  \medskip \hrule \medskip
  \textbf{Oracle:} A problem poser $P^{(1)}$, a function $g: \{0,1\}^{k} \rightarrow \{0,1\}$, \\
  \IndII a solver circuit $C$ for $P^{(g)}$.  \\
  \textbf{Input:} Parameters $n$, $\epsilon$, $\delta$, $k$, $h$, $v$.
  \medskip\hrule\medskip
  %
  $hash := \text{FindHash}((h+v)\epsilon, n, h, v)$ \\
  Let $\widetilde{C} := (C_1, \widetilde{C}_2)$ be as in Lemma \ref{lemma:ctilda_c} with oracle access to $C$, $\hash$. \\
  $D := Gen^{P^{(1)},  \widetilde{C},  g, \hash}(\epsilon, \delta, n, k)$ \\
  \Return $\widetilde{D} := (D_1, \widetilde{D}_2)$
\end{codeblock}
%
We show that Theorem \ref{th:sec_amp_for_dwvp} follows from Lemma \ref{lemma:hash_function_probability} and Lemma \ref{lemma:sec_amp_for_p_hash}.
We fix $P^{(1)}$, $g$, $P^{(g)}$. Given a solver circuit $C = (C_1, C_2)$, asking $h$ hint queries and $v$ verification queries, such that
\begin{align*}
    \underset{\pi^{(k)}, \rho}{\Pr}\left[\Success^{P^{(g)}, C}(\pi^{(k)}, \rho) = 1\right] \geq 16(h+v)\left(\underset{u \leftarrow \mu_\delta^k}{\Pr}\left[g(u) = 1\right] + \varepsilon\right)
\end{align*}
we satisfy conditions of Lemma \ref{lemma:hash_function_probability}. Therefore, $\widetilde{\text{Gen}}$ can use the algorithm FindHash to find $\hash$ such that
\begin{align*}
    \underset{\pi^{(k)}, \rho}{\Pr}\left[\CanonicalSuccess^{P^{(g)}, C, \hash}(\pi^{(k)}, \rho) = 1\right] \geq \underset{u \leftarrow \mu_\delta^k}{\Pr}\left[g(u) = 1\right] + \varepsilon
\end{align*}
almost surely.
By Lemma \ref{lemma:ctilda_c} we know that it is possible to build $\widetilde{C} = (C_1, \widetilde{C}_2)$ such that
\begin{align*}
    \underset{
      \mathclap {
      \substack{\pi^{(k)}, \rho \\
        x := \langle P^{(g)}(\pi^{(k)}), C_1(\rho) \rangle_{\mathit{trans}} \\
        (\Gamma_V^{(g)}, \Gamma_H^{(k)}) := \langle P^{(g)}(\pi^{(k)}), C_1(\rho) \rangle_{P^{(g)}}
      }}}
    {\Pr}\mkern13mu[\Gamma_V^{(g)}(\widetilde{C}_2^{\Gamma_H^{(k)}, C_2, \hash}(x, \rho)) = 1]
    \geq
\underset{u \leftarrow \mu_\delta^k}{\Pr}\left[g(u) = 1\right] + \varepsilon.
\end{align*}
Now, we use Gen to obtain a circuit $D = (D_1, D_2)$, which by Lemma \ref{lemma:sec_amp_for_p_hash} satisfies
\begin{align}
  \label{eq:succ_prob_d}
    \underset{
      \mathclap{
      \substack{\pi, \rho \\ x := \langle P^{(1)}(\pi), D_1^{\widetilde{C}}(\rho) \rangle_{\mathit{trans}} \\
        (\Gamma_H, \Gamma_V) := \langle P^{(1)}(\pi), D_1^{\widetilde{C}}(\rho) \rangle_{P^{(1)}}}}}
    {\Pr}\mkern13mu[\Gamma_V(D_2^{P^{(1)}, \widetilde{C}, hash, g, \Gamma_H}(x, \rho)) = 1] \geq (\delta + \frac{\varepsilon}{6k})
\end{align}
almost surely.
Finally, $\widetilde{\text{Gen}}$ outputs $\widetilde{D} = (D_1, \widetilde{D}_2)$ with oracle access to $D$, $P^{(1)}$, $hash$, $g$ such that with high probability it holds
\begin{align*}
    \underset{\pi, \rho}{\Pr}\left[\Success^{P^{(1)},\widetilde{D}}(\pi, \rho) = 1\right] \geq (\delta + \frac{\varepsilon}{6k}).
\end{align*}
The running time of FindHash is $\mathit{poly}(h,v,\frac{1}{\epsilon},n)$ with oracle calls and of Gen $\mathit{poly(k, \frac{1}{\epsilon}, n)}$ with oracle calls.
Thus, the overall running time of $\widetilde{\mathit{Gen}}$ is  $\mathit{poly(k,\frac{1}{\epsilon},h,v,n,t)}$ with oracle calls.
Furthermore, the circuit $\widetilde{D}$ asks at most $\frac{6k}{\epsilon} \log(\frac{6k}{\epsilon})h$ hint queries and one verification query.
Finally, we have $\mathit{Size}(\widetilde{D}) \leq \mathit{Size}(C) \cdot \frac{6k}{\epsilon}$.
This finishes the proof of Theorem \ref{th:sec_amp_for_dwvp}.
\end{proof}

%%% Local Variables:
%%% mode: latex
%%% TeX-master: "../master"
%%% End:


%% Your real content!
% \section{Weakly Verifiable Puzzles}
The notion of a \textit{weakly verifiable puzzles} was introduced by Cannetti et al. \cite{canetti2005hardness}.
Intuitively, it is a problem generated by a party called a~poser
and solved by a solver. One does not require the solver to have an efficient way to verify~a solution.
On the other hand, the poser has access to some secret information which makes the task of verifying a solution easy.

An example of a weakly verifiable puzzle is a CAPTCHA (an automated Turing test) defined by Ahn et al. \cite{von2003captcha}.
It is easily solvable by humans but is at least mildly harder to solve for computer programs.
The poser generates an instance of a CAPTCHA together with the unique solution.
Therefore, it can trivially verify solutions.
The solution space is often relatively small,
thus it must be hard for the solver being a computer program to verify a~solution.

\section{Hardness Amplification}
An important task in cryptography is to turn a certain problem that is only mildly hard into one that is substantially harder.
An example is the well known hardness amplification lemma for one-way function by Yao \cite{yao1982theory}.
This result implies that it is possible to build a strong one-way function from a function that is only weakly one-way.

A similar hardness amplification statement can be studied for weakly verifiable puzzles.
Given a puzzle that is solved with substantial probability, we want to construct a puzzle that is considerable harder to solve.

There are two approaches to amplify hardness for weakly verifiable puzzles.
Both base on combining several weakly-hard puzzles into a construction that is strongly-hard.
First, one can use \textit{sequential repetition} where puzzles are solved in rounds that start one after another.
Ahn et al. \cite{von2003captcha} observed that sequential repetition amplifies hardness for weakly verifiable puzzle.
However, this approach may be inefficient as it increases the number of communication rounds.
Often a better solution from the practical reasons is to amplify hardness by \textit{parallel repetition}
where several independent puzzles are sent in one round.
We note that Bellare et al. \cite{bellare1997does} show that parallel repetition may fail to amplify hardness in certain settings.

To prove hardness amplification one has to show that the following implication holds
\begin{align*}
  A \implies B,
\end{align*}
where $A$ is a statement that a problem $P$ is hard, and $B$ denotes that a problem $Q$ is hard.
It turns out that it is often easier to consider the following logically equivalent implication
\begin{align*}
  \lnot B \implies \lnot A.
\end{align*}
This approach is used in this thesis. More precisely, we assume existence of an algorithm that successfully
solves the parallel repetition of weakly verifiable puzzles with substantial probability.
Under this assumption we construct an algorithm that success probability in solving a single puzzle is substantial.

\section{Previous works}
The proof of Yao for amplifying hardness of one--way functions relies to a great extent on the fact that it is possible
for an adversary to easily verify correctness of a solution. Therefore, to show hardness amplification
for weakly verifiable puzzles a different approach has to be developed.

Cannetti et al. \cite{canetti2005hardness} define weakly verifiable puzzles and give the hardness amplification proof for these puzzles.

Hardness amplification for weakly verifiable puzzles in a situation where the solver can make several mistakes but still
successful solve a parallel repetition of puzzles is studied by Impagliazzo et al. \cite{impagliazzo2007chernoff}.
They introduce a \textit{threshold function} that determines the minimal fraction of puzzles that must be successfully solved.

Holenstein and Schoenebeck \cite{holenstein2011general} give a more general proof for hardness amplification
of weakly verifiable puzzles. They use an arbitrary \textit{binary monotone function} in the place of a threshold function.
Additionally, they introduce \textit{interactive weakly verifiable puzzles} where a puzzle is defined by
an interactive protocol between the poser and the solver. These puzzles generalize, for example, the binding property of commitment protocols.

Dodis et al. \cite{dodis2009security} extend the proof of hardness amplification of Imaglizazzo et al.
Not only, they consider threshold functions but also introduce \textit{dynamic interactive weakly verifiable puzzles} where a puzzle is defined together with a set of indices $\cQ$.
The solver must give a correct answer on any $q \in \cQ$. Furthermore, it can verify correctness of a limited number of solutions and
obtain several hints understood as correct solutions on some $q \in \cQ$. The solver must solve a puzzle on $q$ for which it did not ask a hint before.
A dynamic weakly verifiable puzzle can be seen as, for example, a game of breaking security of MAC.
In this setting $\cQ$ is a set of messages. The task of the solver is to provide a valid tag for $q \in \cQ$.
The solver can obtain tags of messages of his choice, but must provide a correct tag for a message for which it did not obtain a hint before.

\section{Contribution of the Thesis}
In this thesis we introduce the notion of \textit{dynamic interactive weakly verifiable puzzles} that generalize dynamic and interactive puzzles.
We give a proof of hardness amplification for this type of puzzles by combining techniques
used by Dodis et al. \cite{dodis2009security} and Holenstein et al. \cite{holenstein2011general}.
As a result we show that it is possible to amplify hardness of dynamic interactive weakly verifiable puzzles where
to create an instance of a puzzle the solver and the poser engage in an interactive protocol.
Furthermore, the solver can verify a limited number of solutions and obtain correct solutions on some $q \in \cQ$.
Finally, the proof applies also to a setting where a monotone binary function is used to decide which puzzles of the parallel repetition of puzzles
have to be successfully solved.
%
\section{Organization of the Thesis}
In Chapter \ref{ch:preliminaries} we set up the notation and terminology used in the thesis.
We also define a \textit{monotone binary function} and a \textit{pairwise independent family of hash functions}.

Next, in Chapter \ref{ch:diwvp_main_thm} we introduce \textit{dynamic interactive weakly verifiable
puzzles} and state the hardness amplification theorem for this type of puzzles.
Proving this theorem is the main focus of this thesis.

Chapter \ref{ch:examples_wvcp} is devoted to the different types of cryptographic primitives that
can be seen as weakly verifiable.
We briefly describe MACs, signature schemes, commitment protocols, and CAPTCHAs and explain why
these constructions are generalized by dynamic interactive weakly verifiable puzzles.

Then, in Chapter \ref{ch:previous_results} we give an outline of the earlier studies of weakly
verifiable puzzles and compare them with the techniques used in this thesis.

Finally, in Chapter \ref{ch:main_result} we give the proof of hardness amplification
for dynamic interactive weakly verifiable puzzles and discuss this result.

% Local Variables:
% mode: latex
% TeX-master: "thesis"
% End:
% \chapter{Writing scientific texts in English}

This chapter was originally a separate document written by Reto
Spöhel.  It is reprinted here so that the template can serve as a
quick guide to thesis writing, and to provide some more example
material to give you a feeling for good typesetting.

% We're going to need an extra theorem-like environment for this
% chapter
\theoremstyle{plain}
\theoremsymbol{}
\newtheorem{Rule}[theorem]{Rule}

\section{Basic writing rules}

The following rules need little further explanation; they are best
understood by looking at the example in the booklet by Knuth et al.,
§2--§3.

\begin{Rule}
  Write texts, not chains of formulas.
\end{Rule}

More specifically, write full sentences that are logically
interconnected by phrases like `Therefore', `However', `On the other
hand', etc.\ where appropriate.

\begin{Rule}
  Displayed formulas should be embedded in your text and punctuated
  with it.
\end{Rule}

In other words, your writing should not be divided into `text parts'
and `formula parts'; instead the formulas should be tied together by
your prose such that there is a natural flow to your writing.

\section{Being nice to the reader}

Try to write your text in such a way that a reader enjoys reading
it. That's of course a lofty goal, but nevertheless one you should
aspire to!

\begin{Rule}
  Be nice to the reader.
\end{Rule}

Give some intuition or easy example for definitions and theorems which
might be hard to digest. Remind the reader of notations you introduced
many pages ago -- chances are he has forgotten them. Illustrate your
writing with diagrams and pictures where this helps the reader. Etc.

\begin{Rule}
  Organize your writing.
\end{Rule}

Think carefully about how you subdivide your thesis into chapters,
sections, and possibly subsections.  Give overviews at the beginning
of your thesis and of each chapter, so the reader knows what to
expect. In proofs, outline the main ideas before going into technical
details. Give the reader the opportunity to `catch up with you' by
summing up your findings periodically.

\emph{Useful phrases:} `So far we have shown that \ldots', `It remains
to show that \ldots', `Recall that we want to prove inequality (7), as
this will allow us to deduce that \ldots', `Thus we can conclude that
\ldots. Next, we would like to find out whether \ldots', etc.

\begin{Rule}
  Don't say the same thing twice without telling the reader that you
  are saying it twice.
\end{Rule}

Repetition of key ideas is important and helpful. However, if you
present the same idea, definition or observation twice (in the same or
different words) without telling the reader, he will be looking for
something new where there is nothing new.

\emph{Useful phrases:} `Recall that [we have seen in Chapter 5 that]
\ldots', `As argued before / in the proof of Lemma 3, \ldots', `As
mentioned in the introduction, \ldots', `In other words, \ldots', etc.

\begin{Rule}
  Don't make statements that you will justify later without telling
  the reader that you will justify them later.
\end{Rule}

This rule also applies when the justification is coming right in the
next sentence!  The reasoning should be clear: if you violate it, the
reader will lose valuable time trying to figure out on his own what
you were going to explain to him anyway.

\emph{Useful phrases:} `Next we argue that \ldots', `As we shall see,
\ldots', `We will see in the next section that \ldots, etc.


\section{A few important grammar rules}

\begin{Rule}
  \label{rule:no-comma-before-that}
  There is (almost) \emph{never} a comma before `that'.
\end{Rule}

It's really that simple. Examples:
\begin{quote}
  We assume that \ldots\\
  \emph{Wir nehmen an, dass \ldots}

  It follows that \ldots\\
  \emph{Daraus folgt, dass \ldots}

  `thrice' is a word that is seldom used.\\
  \emph{`thrice' ist ein Wort, das selten verwendet wird.}
\end{quote}
Exceptions to this rule are rare and usually pretty obvious. For
example, you may end up with a comma before `that' because `i.e.' is
spelled out as `that is':
\begin{quote}
  For \(p(n)=\log n/n\) we have \ldots{} However, if we choose \(p\) a
  little bit higher, that is \(p(n)=(1+\varepsilon)\log n/n\) for some
  \(\varepsilon>0\), we obtain that\ldots
\end{quote}
Or you may get a comma before `that' because there is some additional
information inserted in the middle of your sentence:
\begin{quote}
  Thus we found a number, namely \(n_0\), that satisfies equation (13).
\end{quote}
If the additional information is left out, the sentence has no comma:
\begin{quote}
  Thus we found a number that satisfies equation (13).
\end{quote}
(For `that' as a relative pronoun, see also
Rules~\ref{rule:non-defining-has-comma}
and~\ref{rule:defining-without-comma} below.)

\begin{Rule}
  There is usually no comma before `if'.
\end{Rule}

Example:
\begin{quote}
  A graph is not \(3\)-colorable if it contains a \(4\)-clique.\\
  \emph{Ein Graph ist nicht \(3\)-färbbar, wenn er eine \(4\)-Clique
    enthält.}
\end{quote}
However, if the `if' clause comes first, it is usually separated from
the main clause by a comma:
\begin{quote}
  If a graph contains a \(4\)-clique, it is not \(3\)-colorable .\\
  \emph{Wenn ein Graph eine \(4\)-Clique enthält, ist er nicht
    \(3\)-färbbar.}
\end{quote}

There are more exceptions to these rules than to
Rule~\ref{rule:no-comma-before-that}, which is why we are not
discussing them here. Just keep in mind: don't put a comma before `if'
without good reason.

\begin{Rule}
  \label{rule:non-defining-has-comma}
  Non-defining relative clauses have commas.
\end{Rule}
\begin{Rule}
  \label{rule:defining-without-comma}
  Defining relative clauses have no commas.
\end{Rule}

In English, it is very important to distinguish between two types of
relative clauses: defining and non-defining ones. This is a
distinction you absolutely need to understand to write scientific
texts, because mistakes in this area actually distort the meaning of
your text!

It's probably easier to explain first what a \emph{non-defining}
relative clause is. A non-defining relative clauses simply gives
additional information \emph{that could also be left out} (or given in
a separate sentence). For example, the sentence
\begin{quote}
  The \textsc{WeirdSort} algorithm, which was found by the famous
  mathematician John Doe, is theoretically best possible but difficult
  to implement in practice.
\end{quote}
would be fully understandable if the relative clause were left out
completely. It could also be rephrased as two separate sentences:
\begin{quote}
  The \textsc{WeirdSort} algorithm is theoretically best possible but
  difficult to implement in practice. [By the way,] \textsc{WeirdSort}
  was found by the famous mathematician John Doe.
\end{quote}
This is what a non-defining relative clause is. \emph{Non-defining
  relative clauses are always written with commas.} As a corollary we
obtain that you cannot use `that' in non-defining relative clauses
(see Rule~\ref{rule:no-comma-before-that}!). It would be wrong to
write
\begin{quote}
  \st{The \textsc{WeirdSort} algorithm, that was found by the famous
    mathematician John Doe, is theoretically best possible but
    difficult to implement in practice.}
\end{quote}
A special case that warrants its own example is when `which' is
referring to the entire preceding sentence:
\begin{quote}
  Thus inequality (7) is true, which implies that the Riemann
  hypothesis holds.
\end{quote}
As before, this is a non-defining relative sentence (it could be left
out) and therefore needs a comma.

So let's discuss \emph{defining} relative clauses next. A defining
relative clause tells the reader \emph{which specific item the main
  clause is talking about}. Leaving it out either changes the meaning
of the sentence or renders it incomprehensible altogether.  Consider
the following example:

\begin{quote}
  The \textsc{WeirdSort} algorithm is difficult to implement in
  practice. In contrast, the algorithm that we suggest is very simple.
\end{quote}

Here the relative clause `that we suggest' cannot be left out -- the
remaining sentence would make no sense since the reader would not know
which algorithm it is talking about. This is what a defining relative
clause is. \textit{Defining relative clauses are never written with
  commas.} Usually, you can use both `that' and `which' in defining
relative clauses, although in many cases `that' sounds better.

As a final example, consider the following sentence:
\begin{quote}
  For the elements in \(\mathcal{B}\) which satisfy property (A), we
  know that equation (37) holds.
\end{quote}
This sentence does not make a statement about all elements in
\(\mathcal{B}\), only about those satisfying property (A). The relative
clause is \emph{defining}. (Thus we could also use `that' in place of
`which'.)

In contrast, if we add a comma the sentence reads
\begin{quote}
  For the elements in \(\mathcal{B}\), which satisfy property (A), we
  know that equation (37) holds.
\end{quote}

Now the relative clause is \emph{non-defining} -- it just mentions in
passing that all elements in \(\mathcal{B}\) satisfy property (A). The
main clause states that equation (37) holds for \emph{all} elements in
\(\mathcal{B}\). See the difference?


\section[Things you (usually) don't say in English]%
{Things you (usually) don't say in English -- and what to say
  instead}
\label{sec:list}

Table~\ref{tab:things-you-dont-say} lists some common mistakes and
alternatives.  The entries should not be taken as gospel -- they don't
necessarily mean that a given word or formulation is wrong under all
circumstances (obviously, this depends a lot on the context). However,
in nine out of ten instances the suggested alternative is the better
word to use.

\begin{table}
  \centering
  \caption{Things you (usually) don't say}
  \label{tab:things-you-dont-say}
  \begin{tabular}{lll}
    \toprule
    \st{It holds (that) \dots} & We have \dots & \emph{Es gilt \dots}\\
    \multicolumn{3}{l}{\quad\footnotesize(`Equation (5) holds.' is fine, though.)}\\
    \st{$x$ fulfills property $\mathcal{P}$.}& \(x\) satisfies property \(\mathcal{P}\). & \emph{\(x\) erfüllt Eigenschaft \(\mathcal{P}\).} \\
    \st{in average} & on average & \emph{im Durchschnitt}\\
    \st{estimation} & estimate   & \emph{Abschätzung}\\
    \st{composed number} & composite number & \emph{zusammengesetzte Zahl}\\
    \st{with the help of} & using & \emph{mit Hilfe von}\\
    \st{surely} & clearly & \emph{sicher, bestimmt}\\
    \st{monotonously increasing} & monotonically incr. & \emph{monoton steigend}\\
    \multicolumn{3}{l}{\quad\footnotesize(Actually, in most cases `increasing' is just fine.)}\\
    \bottomrule
  \end{tabular}
\end{table}

%%% Local Variables:
%%% mode: latex
%%% TeX-master: "thesis"
%%% End:

% \chapter{Typography}


\section{Punctuation}

\begin{Rule}
  Use opening (`) and closing (') quotation marks correctly.
\end{Rule}

In \LaTeX, the closing quotation mark is typed like a normal
apostrophe, while the opening quotation mark is typed using the French
\emph{accent grave} on your keyboard (the \emph{accent grave} is the
one going down, as in \emph{frère}).

Note that any punctuation that \emph{semantically} follows quoted
speech goes inside the quotes in American English, but outside in
Britain.  Also, Americans use double quotes first.  Oppose
\begin{quote}
  ``Using `lasers,' we punch a hole in \ldots\ the Ozone Layer,''
  Dr.\ Evil said.
\end{quote}
to
\begin{quote}
  `Using ``lasers'', we punch a hole in \ldots\ the Ozone Layer',
  Dr.\ Evil said.
\end{quote}

\begin{Rule}
  Use hyphens (-), en-dashes (--) and em-dashes (---) correctly.
\end{Rule}

A hyphen is only used in words like `well-known', `$3$-colorable'
etc., or to separate words that continue in the next line (which is
known as hyphenation).  It is entered as a single ASCII hyphen
character (\texttt{-}).

To denote ranges of numbers, chapters, etc., use an en-dash (entered
as two ASCII hyphens \texttt{--}) with no spaces on either side.  For
example, using Equations (1)--(3), we see\ldots

As the equivalent of the German \emph{Gedankenstrich}, use an en-dash
with spaces on both sides -- in the title of Section \ref{sec:list},
it would be wrong to use a hyphen instead of the dash. (Some English
authors use the even longer emdash (---) instead, which is typed as
three subsequent hyphens in \LaTeX. This emdash is used without spaces
around it---like so.)


\section{Spacing}

\begin{Rule}
  \label{rule:no-manual-spacing}
  Do not add spacing manually.
\end{Rule}

You should never use the commands \lstinline-\\- (except within
tabulars and arrays), \lstinline[showspaces=true]-\ - (except to
prevent a sentence-ending space after Dr.\ and such),
\lstinline-\vspace-, \lstinline-\hspace-, etc.  The choices programmed
into \LaTeX{} and this style should cover almost all cases.  Doing it
manually quickly leads to inconsistent spacing, which looks terrible.
Note that this list of commands is by no means conclusive.

\begin{Rule}
  Judiciously insert spacing in maths where it helps.
\end{Rule}

This directly contradicts Rule~\ref{rule:no-manual-spacing}, but in
some cases \TeX{} fails to correctly decide how much spacing is
required.  For example, consider
\begin{displaymath}
  f(a,b) = f(a+b, a-b).
\end{displaymath}
In such cases, inserting a thin math space \lstinline-\,- greatly
increases readability:
\begin{displaymath}
  f(a,b) = f(a+b,\, a-b).
\end{displaymath}

Along similar lines, there are variations of some symbols with
different spacing.  For example, Lagrange's Theorem states that
\(\abs{G}=[G:H]\abs{H}\), but the proof uses a bijection \(f\colon
aH\to bH\).  (Note how the first colon is symmetrically spaced, but
the second is not.)

\begin{Rule}
  Learn when to use \lstinline[showspaces=true]-\ - and
  \lstinline-\@-.
\end{Rule}

Unless you use `french spacing', the space at the end of a sentence is
slightly larger than the normal interword space.

The rule used by \TeX{} is that any space following a period,
exclamation mark or question mark is sentence-ending, except for
periods preceded by an upper-case letter.  Inserting \lstinline-\-
before a space turns it into an interword space, and inserting
\lstinline-\@- before a period makes it sentence-ending.  This means
you should write
\begin{lstlisting}
Prof.\ Dr.\ A. Steger is a member of CADMO\@.
If you want to write a thesis with her, you
should use this template.
\end{lstlisting}
which turns into
\begin{quote}
  Prof.\ Dr.\ A. Steger is a member of CADMO\@.  If you want to write
  a thesis with her, you should use this template.
\end{quote}
The effect becomes more dramatic in lines that are stretched slightly
during justification:
\begin{quote}
  \parbox{\linewidth}{\hbox to \linewidth{%
      Prof.\ Dr.\ A. Steger is a member of CADMO\@.  If you}}
\end{quote}

\begin{Rule}
  Place a non-breaking space (\lstinline-~-) right before references.
\end{Rule}

This is actually a slight simplification of the real rule, which
should invoke common sense.  Place non-breaking spaces where a line
break would look `funny' because it occurs right in the middle of a
construction, especially between a reference type (Chapter) and its
number.


\section{Choice of `fonts'}

Professional typography distinguishes many font attributes, such as
family, size, shape, and weight.  The choice for sectional divisions
and layout elements has been made, but you will still occasionally
want to switch to something else to get the reader's attention.  The
most important rule is very simple.

\begin{Rule}
  When emphasising a short bit of text, use \lstinline-\emph-.
\end{Rule}

In particular, \emph{never} use bold text (\lstinline-\textbf-).
Italics (or Roman type if used within italics) avoids distracting the
eye with the huge blobs of ink in the middle of the text that bold
text so quickly introduces.

Occasionally you will need more notation, for example, a consistent
typeface used to identify algorithms.

\begin{Rule}
  Vary one attribute at a time.
\end{Rule}

For example, for \textsc{WeirdSort} we only changed the shape to small
caps.  Changing two attributes, say, to bold small caps would be
excessive (\LaTeX{} does not even have this particular variation).
The same holds for mathematical notation: the reader can easily
distinguish \(g_n\), \(G(x)\), \(\mathcal{G}\) and \(\mathsf{G}\).

\begin{Rule}
  Never underline or uppercase.
\end{Rule}

No exceptions to this one, unless you are writing your thesis on a
typewriter.  Manually.  Uphill both ways.  In a blizzard.


\section{Displayed equations}

\begin{Rule}
  Insert paragraph breaks \emph{after} displays only where they
  belong.  Never insert paragraph breaks \emph{before} displays.
\end{Rule}

\LaTeX{} translates sequences of more than one linebreak (i.e., what
looks like an empty line in the source code) into a paragraph break in
almost all contexts.  This also happens before and after displays,
where extra spacing is inserted to give a visual indication of the
structure.  Adding a blank line in these places may look nice in the
sources, but compare the resulting display

\begin{displaymath}
  a = b
\end{displaymath}

to the following:
\begin{displaymath}
  a = b
\end{displaymath}
The first display is surrounded by blank lines, but the second is not.
It is bad style to start a paragraph with a display (you should always
tell the reader what the display means first), so the rule follows.

\begin{Rule}
  Never use \lstinline-eqnarray-.
\end{Rule}

It is at the root of most ill-spaced multiline displays.  The
\package{amsmath} package provides better alternatives, such as the
\lstinline-align- family
\begin{align*}
  f(x) &= \sin x, \\
  g(x) &= \cos x,
\end{align*}
and \lstinline-multline- which copes with excessively long equations:
\begin{multline*}
  \def\P{\mathrm P}
  \P\bigl[X_{t_0} \in (z_0, z_0+dz_0],\ldots, X_{t_n}\in(z_n,z_n+dz_n]\bigr]
  \\= \nu(dz_0) K_{t_1}(z_0,dz_1) K_{t_2-t_1}(z_1,dz_2)\cdots
  K_{t_n-t_{n-1}}(z_{n-1},dz_n).
\end{multline*}


\section{Floats}

By default this style provides floating environments for tables and
figures.  The general structure should be as follows:
\begin{lstlisting}
\begin{figure}
  \centering
  % content goes here
  \caption{A short caption}
  \label{some-short-label}
\end{figure}
\end{lstlisting}
Note that the label must follow the caption, otherwise the label will
refer to the surrounding section instead.  Also note that figures
should be captioned at the bottom, and tables at the top.

The whole point of floats is that they, well, \emph{float} to a place
where they fit without interrupting the text body.  This is a frequent
source of confusion and changes; please leave it as is.

\begin{Rule}
  Do not restrict float movement to only `here'
  \textnormal{(\lstinline-h-)}.
\end{Rule}

If you are still tempted, you should avoid the float altogether and
just show the figure or table inline, similar to a displayed equation.

%%% Local Variables:
%%% mode: latex
%%% TeX-master: "thesis"
%%% End:

% \chapter{Example Chapter}

Dummy text.

\section{Example Section}

Dummy text.

\subsection{Example Subsection}

Dummy text.

\subsubsection{Example Subsubsection}

Dummy text.

\paragraph{Example Paragraph}

Dummy text.

\subparagraph{Example Subparagraph}

Dummy text.


\appendix

\chapter{Appendix}
\section{Concentration Bounds}
\begin{lemma}[Chernoff Bounds]
For independent Bernoulli distributed random variables $X_1, \dotsc, X_n$ with $X := \sum_{i=1}^n X_i$
and $\Pr[X_i = 1] = p_i$ for all $ 1 \leq  i \leq n$ the following inequalities hold
\begin{gather}
\label{ineq:ch0}
\Pr[X \geq (1+\delta) \mathbb{E}[X]] \leq e^{- \mathbb{E}[X] \delta^2/3} \\
\label{ineq:ch1}
\Pr[X \leq (1-\delta) \mathbb{E}[X]] \leq e^{- \mathbb{E}[X] \delta^2/2},
% \label{ineq:ch2}
% \Pr[|X - \mathbb{E}[X]| \geq \delta \mathbb{E}[X]] \leq 2 e^{- \mathbb{E}[X] \delta^2 / 3},
\end{gather}
where $0 \leq \delta \leq 1$.

For independent and identically distributed Bernoulli random variables $X_1, \dotsc, X_n$ with $X := \sum_{i=1}^n X_i$
where $\Pr[X_i = 1] = p$ for some $p \in (0,1)$ and for all $ 1 \leq  i \leq n$ and
for any $\epsilon > 0$ we have
\begin{gather}
\label{ineq:ch3}
\Pr[X \geq (p + \varepsilon)n] < e^{-\frac{\epsilon^2n}{2}} \\
\label{ineq:ch2}
\Pr[| X - \mathbb{E}[X]| \geq \epsilon\mathbb{E}[X]] \leq 2e^{-\frac{\epsilon^2 \mathbb{E}[X]}{2}}.
\end{gather}
\end{lemma}
\vspace*{\fill}
\pagebreak

\section{The Proof of Lemma \ref{lemma:sec_amp_for_p_hash} Under the Simplifying Assumptions}
\label{st:proofSimAssm}
We prove Lemma \ref{lemma:sec_amp_for_p_hash} in the case where $\widetilde{Gen}$ does not find the randomness for which the surplus is large.
For the sake of simplicity we make the following assumptions
\begin{gather}
  \label{pr:always_d}
\underset{
  \mathclap{
  \substack{\pi, \rho \\ x := \langle P^{(1)}(\pi), D_1(\rho) \rangle_{\trans}}}}
{\Pr}[\widetilde{D}_2(x,\rho) \neq \bot] = 1 \\
  \label{pr:low_surpluses}
\forall \pi \in \{0,1\}^{n} : S_{\pi, b} \leq \Big(1 - \frac{1}{k}\Big)\epsilon.
\end{gather}
In \eqref{pr:always_d} we assume that $\widetilde{D}$ always outputs an answer, and in \eqref{pr:low_surpluses} that all surpluses are low.
In the complete proof of Lemma \ref{lemma:sec_amp_for_p_hash} these assumptions fail only slightly such that it is possible
to obtain the desired result. However, the calculations are fairly lengthy.
The following simplified proof is intended to give the intuition behind the full proof.
We have
\begin{align*}
\underset{
  \mathclap{
  \substack{\pi, \rho \\ x := \langle P^{(1)}(\pi), D_1(\rho) \rangle_{\trans}
    \\ (\Gamma_V, \Gamma_H) := \langle P^{(1)}(\pi), D_1 (\rho) \rangle_{P^{(1)}} }}}
{\Pr}[\Gamma_V&(\widetilde{D}_2(x,\rho)) = 1]
  \overset{\eqref{pr:always_d}}{=}
\underset{
  \mathclap{
  \substack{\pi, \rho \\ x := \langle P^{(1)}(\pi), D_1(\rho) \rangle_{\trans}
    \\ (\Gamma_V, \Gamma_H) := \langle P^{(1)}(\pi), D_1 (\rho) \rangle_{P^{(1)}} }}}
{\Pr}[\Gamma_V(\widetilde{D}_2(x,\rho)) = 1 \mid \widetilde{D}_2(x,\rho) \neq \bot] \\
  &\overset{\hphantom{a}(*)\hphantom{a}}{=} \underset{\pi^{(k)}}{\Pr}[c_1 = 1 \mid c \in \cG_1 \setminus \cG_0] \\
  &\overset{\eqref{eq:diff_s01_next}}{=} \underset{\pi^*}{\mathbb{E}}
  \left[\frac{\Pr_{\pi^{(k)}}[c_1 = 1 \mid c \in \cG_1 \setminus \cG_0] \big(\Pr_{\pi^{(k)}}[c \in \cG_1 \setminus \cG_0] - (S_{\pi^*,1} - S_{\pi^*,0})\big)}
  {\underset{u \leftarrow \mu_{\delta}^{k}}{\Pr}[u \in \cG_1 \setminus \cG_0]}\right] \\
  % &\overset{\hphantom{a(*)}\hphantom{a}}{\geq} \underset{\pi^*}{\mathbb{E}}
  % \left[\frac{\Pr[c_1 = 1 \land c \in \cG_1 \setminus \cG_0]  - (S_{\pi^*,1} - S_{\pi^*,0})}
  % {\underset{u \leftarrow \mu_{\delta}^{k}}{\Pr}[u \in \cG_1 \setminus \cG_0]}\right] \\
  &\overset{\eqref{pr:low_surpluses}}{\geq} \underset{\pi^*}{\mathbb{E}}
  \left[\frac{\Pr_{\pi^{(k)}}[g(c)=1] - \Pr_{\pi^{(k)}}[c \in \cG_0]  - (1 - \frac{1}{k})\epsilon + S_{\pi^*,0}}
  {\underset{u \leftarrow \mu_{\delta}^{k}}{\Pr}[u \in \cG_1 \setminus \cG_0]}\right] \\
  &\overset{\eqref{eq:s_pi_b}}{=} \underset{\pi^*}{\mathbb{E}}
  \left[\frac{\Pr_{\pi^{(k)}}[g(c)=1] - \Pr_{u \leftarrow \mu_{\delta}^{k}}[u \in \cG_0] - S_{\pi^*,0}  - (1 - \frac{1}{k})\epsilon + S_{\pi^*,0}}
  {\underset{u \leftarrow \mu_{\delta}^{k}}{\Pr}[u \in \cG_1 \setminus \cG_0]}\right]\\
  &\overset{\substack{\eqref{eq:gu_rel}\\\eqref{eq:lemma_assum}}}{\geq} \underset{\pi^*}{\mathbb{E}}
  \left[\frac{\delta \Pr_{u \leftarrow \mu_{\delta}^{k}}[u \in \cG_1 \setminus \cG_0] + \epsilon  - (1 - \frac{1}{k})\epsilon}
  {\underset{u \leftarrow \mu_{\delta}^{k}}{\Pr}[u \in \cG_1 \setminus \cG_0]}\right] \geq \delta + \frac{\epsilon}{k},
  %
  % &\overset{\hphantom{\eqref{eq:diff_s01_next}}}{=} \frac{\Pr_{\pi^{(k)}}[c_1 = 1 \land c \in \cG_1 \setminus \cG_0]}{\Pr_{\pi^{(k)}}[c \in \cG_1 \setminus \cG_0]} \\
  % &\overset{\eqref{eq:diff_s01_next}}{=} \underset{\pi^*}{\mathbb{E}}\left[\frac{\Pr_{\pi^{(k-1)}}[c_1^* = 1 \land c \in \cG_1 \setminus \cG_0]
  %   \big(\Pr_{\pi^{(k-1)}}[c \in \cG_0 \setminus \cG_1] - (S_{\pi^*,1} - S_{\pi^*,0})\big)}
  % {\underset{\pi^{(k-1)}}{\Pr}[c \in \cG_0 \setminus \cG_1] \underset{u \leftarrow \mu_{\delta}^{k}}{\Pr}[u \in \cG_1 \setminus \cG_0]}\right]  \\
  % &\overset{\hphantom{\eqref{eq:diff_s01_next}}}{\geq} \underset{\pi^*}{\mathbb{E}}\left[\frac{\Pr_{\pi^{(k-1)}}[c_1^* = 1 \land c \in \cG_1 \setminus \cG_0]
  %   \big(\Pr_{\pi^{(k-1)}}[c \in \cG_0 \setminus \cG_1] - (1 - \frac{1}{k})\epsilon\big)}
  % {\underset{\pi^{(k-1)}}{\Pr}[c \in \cG_0 \setminus \cG_1] \underset{u \leftarrow \mu_{\delta}^{k}}{\Pr}[u \in \cG_1 \setminus \cG_0]}\right]  \\
  % &\overset{\hphantom{\eqref{eq:diff_s01_next}}}{\geq} \frac{\Pr_{\pi^{(k)}}[c_1 = 1 \land c \in \cG_1 \setminus \cG_0] -
  %   (1 - \frac{1}{k})\epsilon}{\underset{u \leftarrow \mu_{\delta}^{k}}{\Pr}[u \in \cG_1 \setminus \cG_0]} \\
  % &\overset{\eqref{eq:gu_rel}}{\geq} \frac{\delta \underset{}{\Pr}[u \in \cG_1 \setminus \cG_0] + \epsilon - (1-\frac{1}{k})\epsilon}{\underset{}{\Pr}[u \in \cG_1 \setminus \cG_0]} \\
  % &\overset{\hphantom{\eqref{eq:diff_s01_next}}}{\geq} \delta +  \frac{\epsilon}{k},
\end{align*}
where in $(*)$ we use the facts that $\widetilde{D}_2(x,\rho) \neq \bot$ if and only if $\widetilde{D}_2$ finds $c \in \cG_1 \setminus \cG_0$
and conditioned on $\widetilde{D}_2(x,\rho) \neq \bot$  we have that $\Gamma_V(\widetilde{D}_2(x,r)) = 1$ if and only if $c_1 = 1$.

\backmatter

% \bibliographystyle{plain}
% \bibliography{refs}

\end{document}
