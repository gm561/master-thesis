%% (Master) Thesis template
% Template version used: v1.4
%
% Largely adapted from Adrian Nievergelt's template for the ADPS
% (lecture notes) project.

%% We use the memoir class because it offers a many easy to use features.
\documentclass[11pt,a4paper,titlepage]{memoir}

%% Packages
%% ========

%% LaTeX Font encoding -- DO NOT CHANGE
\usepackage[OT1]{fontenc}

%% Babel provides support for languages.  'english' uses British
%% English hyphenation and text snippets like "Figure" and
%% "Theorem". Use the option 'ngerman' if your document is in German.
%% Use 'american' for American English.  Note that if you change this,
%% the next LaTeX run may show spurious errors.  Simply run it again.
%% If they persist, remove the .aux file and try again.
\usepackage[english]{babel}

%% Input encoding 'utf8'. In some cases you might need 'utf8x' for
%% extra symbols. Not all editors, especially on Windows, are UTF-8
%% capable, so you may want to use 'latin1' instead.
\usepackage[utf8]{inputenc}

%% This changes default fonts for both text and math mode to use Herman Zapfs
%% excellent Palatino font.  Do not change this.
%\usepackage[sc]{mathpazo}

%% The AMS-LaTeX extensions for mathematical typesetting.  Do not
%% remove.
\usepackage{amsmath,amssymb,amsfonts,mathrsfs}

%% NTheorem is a reimplementation of the AMS Theorem package. This
%% will allow us to typeset theorems like examples, proofs and
%% similar.  Do not remove.
%% NOTE: Must be loaded AFTER amsmath, or the \qed placement will
%% break
\usepackage[amsmath,thmmarks]{ntheorem}

%% LaTeX' own graphics handling
\usepackage{graphicx}

%% We unfortunately need this for the Rules chapter.  Remove it
%% afterwards; or at least NEVER use its underlining features.
\usepackage{soul}

%% Some more packages that you may want to use.  Have a look at the
%% file, and consult the package docs for each.
\input{extrapackages}

%% Our layout configuration.  DO NOT CHANGE.
\input{layoutsetup}

%% Theorem environments.  You will have to adapt this for a German
%% thesis.
%% Theorem-like environments

%% This can be changed according to language. You can comment out the ones you
%% don't need.

\numberwithin{equation}{chapter}

%% German theorems
%\newtheorem{satz}{Satz}[chapter]
%\newtheorem{beispiel}[satz]{Beispiel}
%\newtheorem{bemerkung}[satz]{Bemerkung}
%\newtheorem{korrolar}[satz]{Korrolar}
%\newtheorem{definition}[satz]{Definition}
%\newtheorem{lemma}[satz]{Lemma}
%\newtheorem{proposition}[satz]{Proposition}

%% English variants
\newtheorem{theorem}{Theorem}[chapter]
\newtheorem{example}[theorem]{Example}
\newtheorem{remark}[theorem]{Remark}
\newtheorem{corollary}[theorem]{Corollary}
\newtheorem{lemma}[theorem]{Lemma}
\newtheorem{proposition}[theorem]{Proposition}
\newtheorem{observation}[theorem]{Observation}

\theoremstyle{definition}
\theorembodyfont{\normalfont}
%% end def with blacksquare symbol
\theoremsymbol{\ensuremath{\lozenge}}
\newtheorem{definition}[theorem]{Definition}

%% Proof environment with a small square as a "qed" symbol
\theoremstyle{nonumberplain}
\theorembodyfont{\normalfont}
\theoremsymbol{\ensuremath{\square}}
\theoremseparator{.}
\newtheorem{proof}{Proof}

%\newtheorem{beweis}{Beweis}

\declaretheorem[name=Theorem, numberwithin=chapter]{thm}


%% Helpful macros.
%% Custom commands
%% ===============

%% Special characters for number sets, e.g. real or complex numbers.
\newcommand{\C}{\mathbb{C}}
\newcommand{\K}{\mathbb{K}}
\newcommand{\N}{\mathbb{N}}
\newcommand{\Q}{\mathbb{Q}}
\newcommand{\R}{\mathbb{R}}
\newcommand{\Z}{\mathbb{Z}}
\newcommand{\X}{\mathbb{X}}

\newcommand{\cX}{\mathcal{X}}
\newcommand{\cH}{\mathcal{H}}
\newcommand{\cW}{\mathcal{W}}
\newcommand{\cG}{\mathcal{G}}
\newcommand{\cB}{\mathcal{B}}
\newcommand{\cP}{\mathcal{P}}
\newcommand{\cR}{\mathcal{R}}
\newcommand{\cD}{\mathcal{D}}

%define our own code commands
%use capital latters as most of these commands is already defined
\renewcommand{\For}{\textbf{for }}
\renewcommand{\If}{\textbf{if }}
\renewcommand{\Else}{\textbf{else }}
\renewcommand{\Return}{\textbf{return }}
\newcommand{\Then}{\textbf{then }}
\newcommand{\Do}{\textbf{do: }}
\renewcommand{\And}{\textbf{and }}
\newcommand{\Or}{\textbf{or }}
\newcommand{\Run}{\textbf{run }}
\newcommand{\To}{\textbf{to }}

%% Fixed/scaling delimiter examples (see mathtools documentation)
\DeclarePairedDelimiter\abs{\lvert}{\rvert}
\DeclarePairedDelimiter\norm{\lVert}{\rVert}

%% Use the alternative epsilon per default and define the old one as \oldepsilon
\let\oldepsilon\epsilon
\renewcommand{\epsilon}{\ensuremath\varepsilon}

%% Also set the alternate phi as default.
\let\oldphi\phi
\renewcommand{\phi}{\ensuremath{\varphi}}

% New command that introduces a tab
\newcommand{\itab}[1]{\hspace{0em}\rlap{#1}}
\newcommand{\tab}[1]{\hspace{.2\textwidth}\rlap{#1}}

\DeclareMathOperator{\la0}{\leftarrow}
\DeclareMathOperator{\ra0}{\rightarrow}

\DeclareMathOperator{\hash}{\mathit{hash}}
\DeclareMathOperator{\CanonicalSuccess}{\mathit{CanonicalSuccess}}
\DeclareMathOperator{\Success}{\mathit{Success}}

%the DWVP for the permutation
\DeclareMathOperator{\PiDWVP}{\Pi_{DWVP}}
%the DWPV for the k-wise product of permutations
\DeclareMathOperator{\kPiDWVP}{\Pi_{DWVP}^{(k)}}

%% Make document internal hyperlinks wherever possible. (TOC, references)
%% This MUST be loaded after varioref, which is loaded in 'extrapackages'
%% above.  We just load it last to be safe.
\usepackage[linkcolor=black,colorlinks=true,citecolor=black,filecolor=black]{hyperref}


%% Document information
%% ====================

\title{On amplification of weakly verifiable dynamic cryptographic primitives}
\author{Grzegorz Makosa}
\thesistype{Master Thesis}
\advisors{Advisors: Prof. Dr. Thomas Holenstein, Dr. Robin Künzler}
\department{Department of Computer Science}
\date{April 8, 2014}

\begin{document}

\frontmatter

% \begin{titlingpage}
%   \calccentering{\unitlength}
%   \begin{adjustwidth*}{\unitlength-24pt}{-\unitlength-24pt}
%     \maketitle
%   \end{adjustwidth*}
% \end{titlingpage}
%\begin{abstract}
% A good abstract explains in one line why the paper is important.
% It then goes on to give a summary of your major results, preferably couched in numbers with error limits.
% The final sentences explain the major implications of your work. A good abstract is concise, readable, and quantitative.
% Length should be ~ 1-2 paragraphs, approx. 400 words.
%
% Abstracts generally do not have citations.
% Information in title should not be repeated.
% Be explicit.
% Use numbers where appropriate.
% Answers to these questions should be found in the abstract:
% What did you do?
% Why did you do it? What question were you trying to answer?
% How did you do it? State methods.
% What did you learn? State major results.
% Why does it matter? Point out at least one significant implication.
%
%
% 1) new hardness proof
% 2) it is possible to start with a weak crypto primitive and obtain strong one.
% 3) a natural puzzle fixing technique similar to the one used in the classical proofs of weak one-way function implying strong ones
% 4) uses an arbitrary monotone function to decide the result
% 5) compare with previous works
% 6)
%
We give a proof of hardness amplification for a new type of Weakly Verifiable Puzzles that generalize the previous research.

An important question in cryptography is whether it is possible to build cryptographic construction that is strongly hard to solve
from a one that is only weakly hard. The well known result of this type is about building a strong one-way function using functions that
are only weakly hard.

The problem of hardness amplification has been studied for various variants of Weakly Verifiable Puzzles
where a solution cannot be efficiently verified by a party that solves a puzzle.
It has been shown \cite{Dodis:2009:SAI:1530441.1530450} that it possible to amplify hardness for Dynamic Weakly Verifiable Puzzles,
which covers cryptographic primitives like message authentication codes and signature schemes.
In \cite{DBLP:journals/corr/abs-1002-3534} a similar result has been proven for Interactive Weakly Verifiable Puzzles that
embraces constructions like bit commitment protocols.

In this thesis we introduce the notion of \textit{Dynamic Interactive Weakly Verifiable Puzzles}
that generalize previous definitions of Dynamic and Interactive Weakly Verifiable Puzzles.
Furthermore, we prove that if there is no probabilistic algorithm that solves a single puzzle with probability more than $\delta - \epsilon$,
then there is no algorithm that solves $k$ independent instances of puzzles with probability higher than an algorithm that solves each of the $k$ puzzles
independently with probability $\delta$ and where an arbitrary binary monotone function is used to decide which of the $k$ puzzle has to be solved.
\end{abstract}


\cleartorecto
\tableofcontents
\mainmatter

\chapter{Introduction}
\section{Security amplification theorems}
Introduction to security amplification theorems and hardness implication statements.
Example of classical results. Problems captured by weakly verifiable puzzles.
Contribution of this thesis.

\section{Organization of the thesis}
Overview of the content of the succeeding chapters.

\chapter{Preliminaries}
In this chapter we set up notation and terminology used in the Thesis.
%
\section{Notation and Definitions}
\textbf{(Algorithms, Bitstrings and Circuits)}
We define a \textit{Boolean circuit} as a directed acyclic graph with input vertices and vertices implementing logical functions \textit{and}, \textit{or}, and \textit{not}.
We denote Boolean circuits using capital letters from the Greek or English alphabet.
We define a \textit{probabilistic circuit} as a Boolean circuit $C_{m,n} : \{0,1\}^{m} \times \{0,1\}^{n} \rightarrow \{0,1\}^{*}$.
Additionally, we write $C_{m,n}(x;r)$ which should be understood as a probabilistic circuit taking as input  $x \in \{0,1\}^{m}$
and auxiliary input $r \in \{0,1\}^{n}$.
If a probabilistic circuit does not take any input, we abuse notation and write $C_{n}(r)$.
Similarly, we use $\{C_n\}_{n \in \N}$ to denote a family of probabilist circuits that takes only auxiliary input.
We make sure that it is clear from the context that probabilistic circuits with only auxiliary input
are not confused with circuits that do not take auxiliary input.
For a (probabilistic) circuit $C$ we write $\mathit{Size}(C)$ to denote the total number of vertices of $C$.
A \textit{(probabilistic) polynomial size circuit} is a (probabilistic) circuit of size polynomial in the number of input vertices (including auxiliary input).
We define a \textit{two phase circuit} $C := (C_1, C_2)$ as a circuit where in the first phase a circuit $C_1$ is used and in the second phase a circuit $C_2$.
If $C_1$ and $C_2$ are probabilistic circuits we write $C(\delta) := (C_1, C_2)(\delta)$ to denote that in both phases $C_1$ and $C_2$ take
as auxiliary input the same bitstring $\delta$.

\begin{todo}
  \textbf{TODO:} Does it hold for search problems and for algorithms with not a single bit of output.
\end{todo}
It is well known \cite{Arora:2009:CCM:1540612} that a probabilistic polynomial time algorithm can be represented as a circuit of polynomial size.
Moreover, it can be computed in polynomial time and logarithmic space.
Therefore, whenever we state a theorem about circuits it can be also generalized for the polynomial time algorithms.

We write $\mathit{poly}(\alpha_1, \dots, \alpha_n)$ to denote a polynomial on variables $\alpha_1, \dots, \alpha_n$.
For an algorithm $A$ we write $\mathit{Time}(A)$ to denote the number of steps it takes to execute $A$.
We say that $A$ runs in \textit{polynomial time} if the number of steps required to evaluate $A$ is bounded by $poly(|x|)$, where $|x|$ denotes
the length of the input that $A$ takes.
Similarly, as for probabilistic circuits we write the randomness used by a probabilistic algorithm explicitly as a bitstring provided as an auxiliary input.

\textbf{(Probabilities and distributions)}
For a finite set $\cR$ we write $r \xleftarrow{\$} \cR$ to denote that $r$ is chosen from $\cR$ uniformly at random.
For $\delta \in \R : 0 \leq \delta \leq 1$ we write $\mu_{\delta}$ to denote the Bernoulli distribution where outcome $1$ occurs with
probability $\delta$ and $0$ with probability $1-\delta$.
Moreover, we use $\mu_{\delta}^k$ to denote the probability distribution over $k$-tuples
where each element of a $k$-tuple is drawn independently according to $\mu_{\delta}$.
Finally, let $u \leftarrow \mu_{\delta}^k$ denote that a $k$-tuple $u$ is chosen according to $\mu_{\delta}^k$.

Let $(\Omega_n, \cF_n, \Pr)$ be a probability space and $n \in \N$.
Let $E_n \in \cF_n$ denote an event that probability depends on $n$.
We say that $E_n$ happens \textit{almost surely} or with \textit{high probability} if $\Pr[E_n] \geq 1 - 2^{-n} \mathit{poly}(n)$.

\textbf{(Functions)} We call a function $f: \N \rightarrow \R$ \textit{negligible} if for every polynomial $\poly(n)$
there exists $n_0 \in \N$ such that for all $n \in \N : n > n_0$ the following holds
\begin{align*}
f(n) < \frac{1}{\poly(n)}.
\end{align*}
On the other hand, we say that a function $f: \N \rightarrow \R$ is \textit{non-negligible} if
there exists a polynomial $\poly(n)$ such that for some $n_0 \in \N$ and for all $n \in \N : n > n_0$ we have
\begin{align*}
  f(n) \geq \frac{1}{\poly(n)}.
\end{align*}
We say that a function $f$ is \textit{efficiently computable} if there exists a polynomial time algorithm computing $f$.

\textbf{(Interactive protocols)}
We are often interested in situations where two probabilistic circuits interact with each other according to some protocol.
We limit ourselves to the cases where circuits interact by means of messages representable by bitstrings.
Let $\{A_n\}_{n \in \N}$ and $\{B_n\}_{n \in \N}$ be families of circuits such that $A_n : \{0,1\}^{*} \rightarrow \{0,1\}^{*}$ and $B_n : \{0,1\}^{*} \rightarrow \{0,1\}^{*}$.
An \textit{interactive protocol} is defined by a $\{A_n\}_{n \in \N}$ and $\{B_n\}_{n \in \N}$ where
for random bitstrings $\rho_A \in \{0,1\}^{n}$, $\rho_B \in \{0,1\}^{n}$ in the first round $m_0 := A_n(\rho_A)$ and in the second round $m_1 := B_n(\rho_B, m_1)$.
In general in the $k$-th round we have $m_k := A_n(\rho_A, m_1, m_2, \cdots, m_{k-1})$ and in the $k+1$-th round $B_n(\rho_B, m_1, m_2, \cdots, m_{k-1}, m_{k})$.
A protocol execution between two probabilistic circuits $A$ and $B$ is denoted by $\langle A, B \rangle$.
The output of $A$ in a protocol execution is denoted by $\langle A, B \rangle_A$ and of $B$ by $\langle A, B \rangle_B$.
A sequence of all messages sent by $A$ and $B$ in the protocol execution is called a communication transcript and
is denoted by $\langle A, B \rangle_{\mathit{trans}}$.

\textbf{(Oracle algorithms)}
We use notions of \textit{oracle circuits} following the standard definition included in the literature \cite{Goldreich:2004:FCV:975541}.
If a circuit $A$ gain oracle access to a circuit $B$, we write $A^{B}$. If additionally $B$ gain oracle access to a circuit $C$
we write $A^{B^{}C}$. However, to simplify notation we often write $A^{B}$ instead of $A^{B^C}$.
We make sure that it is clear from the context which oracle is accessed by $B$.

\begin{definition}[Polynomial time sampleable distribution]
We say that a distribution is \textnormal{polynomial time sampleable} if it can be approximated by an algorithm running in time $\mathit{poly}(\log|\cD|, \log|cR|)$
up to an exponential factor.
\end{definition}

\begin{definition}[Pairwise independent family of efficient hash functions]
Let $\cD$ and $\cR$ be finite sets and $\cH$ be a family of functions mapping values from $\cD$ to values in $\cR$.
We say that $\cH$ is a \textnormal{family of pairwise independent efficient hash functions}
if $\cH$ has the following properties.

\textbf{(Pairwise independent)} For $\forall x \neq y \in \mathcal{D}$ and $\forall \alpha, \beta \in \cR$, it holds
\begin{displaymath}
\underset{\hash \la0 \cH}{\Pr}[hash(x) = \alpha \mid hash(y) = \beta] = \frac{1}{|\cR|}.
\end{displaymath}

\textbf{(Polynomial time sampleable)} For every $\mathit{hash} \in \cH$ the function $\mathit{hash}$ is sampleable in time $\mathit{poly}(\log|\cD|, \log|\cR|)$.

\textbf{(Efficiently computable)}
For every $hash \in \cH$ there exists an algorithm running in time $\mathit{poly}(\log|\cD|, \log|\cR|)$ which
on input $x \in \cD$ outputs $y \in \cR$ such that $y = hash(x)$.
\end{definition}

We note that the pairwise independence property is equivalent to
\begin{displaymath}
\underset{\hash \la0 \cH}{\Pr}[hash(x) = \alpha \land hash(y) = \beta] = \frac{1}{|\cR|^2}.
\end{displaymath}
It is well know \cite{Carter:1977:UCH:800105.803400} that there exists families of functions meeting the above criteria.

%%% Local Variables:
%%% mode: latex
%%% TeX-master: "master"
%%% End:


\chapter{Weakly verifiable cryptographic primitives}
In this chapter we introduce the notion of weakly verifiable puzzles. In section \ref{def:dwvp} we provide a formal definitions that
is followed by a series of cryptographic primitives that are captured by this notion.
Finally, in Section \ref{st:previous_results} we give an overview of the earlier research in this area
that is primarily covered in \cite{canetti2004hardness}, \cite{Dodis:2009:SAI:1530441.1530450}, and \cite{DBLP:journals/corr/abs-1002-3534}.

\section{Weakly verifiable puzzles}
\begin{definition}[Dynamic weakly verifiable puzzle.]
  \label{def:dwvp}
  A dynamic weakly verifiable puzzle (DWVP) is defined by a family of probabilistic circuits $\{P_n\}$.
  A circuit belonging to $\{P_n\}$ is called a problem poser.
  A solver $C := (C_1, C_2)$ for $P_n$ is a probabilistic two phase circuit.
  We write $P_n(\pi)$ to denote the execution of $P_n$ with the randomness fixed to $\pi \in \{0,1\}^n$, and $(C_1,C_2)(\rho)$
  to denote the execution of both $C_1$ and $C_2$ with the randomness fixed to $\rho \in \{0,1\}^{*}$.

  In the first phase, the problem poser $P_n(\pi)$ and the solver $C_1(\rho)$ interact.
  As the result of the interaction $P_n(\pi)$ outputs a verification circuit $\Gamma_{V}$ and a hint circuit $\Gamma_{H}$.
  The circuit $C_1(\rho)$ produces no output.
  The circuit $\Gamma_{V}$ takes as input $q \in Q$, an answer $y \in \{0,1\}^*$,
  and outputs a bit. We say that an answer $(q,y)$ is a correct solution if and only if $\Gamma_V(q,y) = 1$.
  The circuit $\Gamma_H$ on input $q \in Q$ outputs a hint such that $\Gamma_V(q,\Gamma_H(q)) = 1$.

  In the second phase, $C_2$ takes as input $x := \langle P_n(\pi), C_1(\rho) \rangle_{\mathit{trans}}$,
  and has oracle access to $\Gamma_V$ and $\Gamma_H$.
  The execution of $C_2$ with the input $x$ and the randomness fixed to $\rho$
  is denoted by $C_2(x, \rho)$. The queries of $C_2$ to $\Gamma_V$ and $\Gamma_H$ are called verification queries and hint queries respectively.
  The circuit $C_2$ succeeds if and only if it makes a verification query $(q,y)$ such that $\Gamma_V(q,y) = 1$,
  and it has not previously asked for a hint query on $q$.
\end{definition}

\section{Examples}
\subsection{Message authentication codes}
\subsection{Public key encryption}
\subsection{Bit commitments}
\subsection{CAPTACHs}

\section{Previews results}
\label{st:previous_results}
\subsection{Results of R.Canetti, S.Halevi, and M.Steiner}
\subsection{Results of Y.Dodis, R.Impagliazzo, R.Jaiswal, V.Kabanets}
\subsection{Results of T.Holenstein and G.Scheonebeck}
\section{Limitations of security amplification}

\input{interactive_proof/interactive_proof}

\appendix
\chapter{Appendix}
\section{Basic inequalities}
\begin{lemma}[Chernoff Bounds]
For independent, identically distributed Bernoulli random variables $X_1, \dots, X_n$ with $X := \sum_{i=1}^n X_i$
with $\Pr[X_i = 1] = p_i$ and $\Pr[X_i = 0] = 1 - p_i$ for all $ 1 \leq  i \leq n$.
we have the following inequalities for $0 \leq \delta \leq 1$ and $\mathbb{E}[X] = \sum_{i=1}^{n} p_i$:
\begin{gather}
\label{ineq:ch0}
\Pr[X \geq (1+\delta) \mathbb{E}[X]] \leq e^{- \mathbb{E}[X] \delta^2/3} \\
\label{ineq:ch1}
\Pr[X \leq (1-\delta) \mathbb{E}[X]] \leq e^{- \mathbb{E}[X] \delta^2/2} \\
\label{ineq:ch2}
\Pr[|X - \mathbb{E}[X]| \geq \delta \mathbb{E}[X]] \leq 2 e^{- \mathbb{E}[X] \delta^2 / 3}.
\end{gather}
\end{lemma}

\backmatter

\bibliographystyle{alpha}
\bibliography{refs}

\end{document}
