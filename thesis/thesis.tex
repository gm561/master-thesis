%% (Master) Thesis template
% Template version used: v1.4
%
% Largely adapted from Adrian Nievergelt's template for the ADPS
% (lecture notes) project.

%% We use the memoir class because it offers a many easy to use features.
\documentclass[11pt,a4paper,titlepage]{memoir}

%% Packages
%% ========

%% LaTeX Font encoding -- DO NOT CHANGE
\usepackage[OT1]{fontenc}

%% Babel provides support for languages.  'english' uses British
%% English hyphenation and text snippets like "Figure" and
%% "Theorem". Use the option 'ngerman' if your document is in German.
%% Use 'american' for American English.  Note that if you change this,
%% the next LaTeX run may show spurious errors.  Simply run it again.
%% If they persist, remove the .aux file and try again.
\usepackage[english]{babel}

%% Input encoding 'utf8'. In some cases you might need 'utf8x' for
%% extra symbols. Not all editors, especially on Windows, are UTF-8
%% capable, so you may want to use 'latin1' instead.
\usepackage[utf8]{inputenc}

%% This changes default fonts for both text and math mode to use Herman Zapfs
%% excellent Palatino font.  Do not change this.
%\usepackage[sc]{mathpazo}

%% The AMS-LaTeX extensions for mathematical typesetting.  Do not
%% remove.
\usepackage{amsmath,amssymb,amsfonts,mathrsfs}

%% NTheorem is a reimplementation of the AMS Theorem package. This
%% will allow us to typeset theorems like examples, proofs and
%% similar.  Do not remove.
%% NOTE: Must be loaded AFTER amsmath, or the \qed placement will
%% break
\usepackage[amsmath,thmmarks]{ntheorem}

%% LaTeX' own graphics handling
\usepackage{graphicx}

%% We unfortunately need this for the Rules chapter.  Remove it
%% afterwards; or at least NEVER use its underlining features.
\usepackage{soul}

%% Some more packages that you may want to use.  Have a look at the
%% file, and consult the package docs for each.
\input{extrapackages}

%% Our layout configuration.  DO NOT CHANGE.
\input{layoutsetup}

%% Theorem environments.  You will have to adapt this for a German
%% thesis.
%% Theorem-like environments

%% This can be changed according to language. You can comment out the ones you
%% don't need.

\numberwithin{equation}{chapter}

%% German theorems
%\newtheorem{satz}{Satz}[chapter]
%\newtheorem{beispiel}[satz]{Beispiel}
%\newtheorem{bemerkung}[satz]{Bemerkung}
%\newtheorem{korrolar}[satz]{Korrolar}
%\newtheorem{definition}[satz]{Definition}
%\newtheorem{lemma}[satz]{Lemma}
%\newtheorem{proposition}[satz]{Proposition}

%% English variants
\newtheorem{theorem}{Theorem}[chapter]
\newtheorem{example}[theorem]{Example}
\newtheorem{remark}[theorem]{Remark}
\newtheorem{corollary}[theorem]{Corollary}
\newtheorem{lemma}[theorem]{Lemma}
\newtheorem{proposition}[theorem]{Proposition}
\newtheorem{observation}[theorem]{Observation}

\theoremstyle{definition}
\theorembodyfont{\normalfont}
%% end def with blacksquare symbol
\theoremsymbol{\ensuremath{\lozenge}}
\newtheorem{definition}[theorem]{Definition}

%% Proof environment with a small square as a "qed" symbol
\theoremstyle{nonumberplain}
\theorembodyfont{\normalfont}
\theoremsymbol{\ensuremath{\square}}
\theoremseparator{.}
\newtheorem{proof}{Proof}

%\newtheorem{beweis}{Beweis}

\declaretheorem[name=Theorem, numberwithin=chapter]{thm}


%% Helpful macros.
%% Custom commands
%% ===============

%% Special characters for number sets, e.g. real or complex numbers.
\newcommand{\C}{\mathbb{C}}
\newcommand{\K}{\mathbb{K}}
\newcommand{\N}{\mathbb{N}}
\newcommand{\Q}{\mathbb{Q}}
\newcommand{\R}{\mathbb{R}}
\newcommand{\Z}{\mathbb{Z}}
\newcommand{\X}{\mathbb{X}}

\newcommand{\cX}{\mathcal{X}}
\newcommand{\cH}{\mathcal{H}}
\newcommand{\cW}{\mathcal{W}}
\newcommand{\cG}{\mathcal{G}}
\newcommand{\cB}{\mathcal{B}}
\newcommand{\cP}{\mathcal{P}}
\newcommand{\cR}{\mathcal{R}}
\newcommand{\cD}{\mathcal{D}}

%define our own code commands
%use capital latters as most of these commands is already defined
\renewcommand{\For}{\textbf{for }}
\renewcommand{\If}{\textbf{if }}
\renewcommand{\Else}{\textbf{else }}
\renewcommand{\Return}{\textbf{return }}
\newcommand{\Then}{\textbf{then }}
\newcommand{\Do}{\textbf{do: }}
\renewcommand{\And}{\textbf{and }}
\newcommand{\Or}{\textbf{or }}
\newcommand{\Run}{\textbf{run }}
\newcommand{\To}{\textbf{to }}

%% Fixed/scaling delimiter examples (see mathtools documentation)
\DeclarePairedDelimiter\abs{\lvert}{\rvert}
\DeclarePairedDelimiter\norm{\lVert}{\rVert}

%% Use the alternative epsilon per default and define the old one as \oldepsilon
\let\oldepsilon\epsilon
\renewcommand{\epsilon}{\ensuremath\varepsilon}

%% Also set the alternate phi as default.
\let\oldphi\phi
\renewcommand{\phi}{\ensuremath{\varphi}}

% New command that introduces a tab
\newcommand{\itab}[1]{\hspace{0em}\rlap{#1}}
\newcommand{\tab}[1]{\hspace{.2\textwidth}\rlap{#1}}

\DeclareMathOperator{\la0}{\leftarrow}
\DeclareMathOperator{\ra0}{\rightarrow}

\DeclareMathOperator{\hash}{\mathit{hash}}
\DeclareMathOperator{\CanonicalSuccess}{\mathit{CanonicalSuccess}}
\DeclareMathOperator{\Success}{\mathit{Success}}

%the DWVP for the permutation
\DeclareMathOperator{\PiDWVP}{\Pi_{DWVP}}
%the DWPV for the k-wise product of permutations
\DeclareMathOperator{\kPiDWVP}{\Pi_{DWVP}^{(k)}}

%% Make document internal hyperlinks wherever possible. (TOC, references)
%% This MUST be loaded after varioref, which is loaded in 'extrapackages'
%% above.  We just load it last to be safe.
\usepackage[linkcolor=black,colorlinks=true,citecolor=black,filecolor=black]{hyperref}

\allowdisplaybreaks[1]

%% Document information
%% ====================

\title{On amplification of weakly verifiable dynamic cryptographic primitives}
\author{Grzegorz Makosa}
\thesistype{Master Thesis}
\advisors{Advisors: Prof. Dr. Thomas Holenstein, Dr. Robin Künzler}
\department{Department of Computer Science}
\date{April 8, 2014}

\begin{document}

\frontmatter

% \begin{titlingpage}
%   \calccentering{\unitlength}
%   \begin{adjustwidth*}{\unitlength-24pt}{-\unitlength-24pt}
%     \maketitle
%   \end{adjustwidth*}
% \end{titlingpage}
%\begin{abstract}
% A good abstract explains in one line why the paper is important.
% It then goes on to give a summary of your major results, preferably couched in numbers with error limits.
% The final sentences explain the major implications of your work. A good abstract is concise, readable, and quantitative.
% Length should be ~ 1-2 paragraphs, approx. 400 words.
%
% Abstracts generally do not have citations.
% Information in title should not be repeated.
% Be explicit.
% Use numbers where appropriate.
% Answers to these questions should be found in the abstract:
% What did you do?
% Why did you do it? What question were you trying to answer?
% How did you do it? State methods.
% What did you learn? State major results.
% Why does it matter? Point out at least one significant implication.
%
%
% 1) new hardness proof
% 2) it is possible to start with a weak crypto primitive and obtain strong one.
% 3) a natural puzzle fixing technique similar to the one used in the classical proofs of weak one-way function implying strong ones
% 4) uses an arbitrary monotone function to decide the result
% 5) compare with previous works
% 6)
%
We give a proof of hardness amplification for a new type of Weakly Verifiable Puzzles that generalize the previous research.

An important question in cryptography is whether it is possible to build cryptographic construction that is strongly hard to solve
from a one that is only weakly hard. The well known result of this type is about building a strong one-way function using functions that
are only weakly hard.

The problem of hardness amplification has been studied for various variants of Weakly Verifiable Puzzles
where a solution cannot be efficiently verified by a party that solves a puzzle.
It has been shown \cite{Dodis:2009:SAI:1530441.1530450} that it possible to amplify hardness for Dynamic Weakly Verifiable Puzzles,
which covers cryptographic primitives like message authentication codes and signature schemes.
In \cite{DBLP:journals/corr/abs-1002-3534} a similar result has been proven for Interactive Weakly Verifiable Puzzles that
embraces constructions like bit commitment protocols.

In this thesis we introduce the notion of \textit{Dynamic Interactive Weakly Verifiable Puzzles}
that generalize previous definitions of Dynamic and Interactive Weakly Verifiable Puzzles.
Furthermore, we prove that if there is no probabilistic algorithm that solves a single puzzle with probability more than $\delta - \epsilon$,
then there is no algorithm that solves $k$ independent instances of puzzles with probability higher than an algorithm that solves each of the $k$ puzzles
independently with probability $\delta$ and where an arbitrary binary monotone function is used to decide which of the $k$ puzzle has to be solved.
\end{abstract}


\cleartorecto
\tableofcontents
\mainmatter

\chapter{Introduction}
\section{Security Amplification Theorems}
The cryptography is a science where not much is known about the computational lower bounds of certain problems
as many of them would lead to proving (or disproving) that $P \neq NP$.
However, certain problems seems to be more likely to be hard than others.
Thus, we often suppose that a certain problem $P$ is hard and under this assumption
study hardness of a problem $Q$.

An arising question is whether it is possible to turn a problem $P$ that is only mildly hard to solve into a problem $Q$
that solving is much harder. The statements of this types are called hardness amplification.
One of the most well known results is by Yao who proved that it is indeed possible to build a strongly secure one way function
from a one way function that is only weakly secure. The proof of Yao requires inverting correctly certain number of weak one-way function
which yields a function that is one-way.

We are interested in a question whether it is possible to achieve a similar result holds also for other basic cryptographic constructions
like signature schemes, message authentication codes, bit commitment protocols, or artificial intelligence problems, just to name a few.

In this Thesis we study \textit{weakly verifiable puzzles} that generalize many commonly encountered cryptographic primitives.
The characteristic property of weakly verifiable puzzles is that we do not insist that verifying the correctness of a solution by a problem
solver is efficient. An example is CAPTCHA where a problem of checking whether a solution is correct is comparable hard to finding a correct solution.
We give a single proof of hardness amplification for all these primitives.

In general there are two methods to amplify hardness.
The first one is sequentially repetition where a protocol is repeated such that one round starts after the previous is finished.
It has been observed that sequential repetition amplifies hardness of weakly verifiable puzzles by \cite{von2003captcha}.
However, this approach is costly as it increase number of rounds creating additional communication burden.

Therefore, the approach that is more useful from practical point of view is the hardness amplification by means of the
parallel repetition where all puzzles are send at once. We state that it has been shown that parallel repetition
does not always amplify hardness \cite{bellare1997does}.

Commonly when proving the reduction statements of the form
\begin{align*}
  P \implies Q
\end{align*}
where $P$ is a statement that certain problem is hard, and $Q$ is a certain problem which hardness is implied by $P$.
We use logically equivalent statement
\begin{align*}
  \lnot Q \implies \lnot P.
\end{align*}
This means that we under the assumption that a problem $Q$ is easy, we try to prove that problem $P$ is easy.
This approach is also taken by us in this Thesis where in order to show hardness amplification for parallel repetition,
we assume that there exists an algorithm that solves direct product of puzzle with substantial probability.
Under this assumption we show that there exists a solver for a single puzzle with access to the solver for direct product of puzzles that
probability of solving a single puzzle is also substantial.

\section{Weakly verifiable puzzles}
Security of cryptographic primitives is often defined by
a game in which an adversary has access to certain resources, but must not solve a certain problem.
The game base security for many cryptographic constructions can be modeled as a puzzle.
If it is not possible for an adversary to efficiently check the correctness of the solution then
we call such a puzzle weakly verifiable. We are interested whether how to amplify hardness of this game
and whether solving parallel repetition is considerable harder for the adversary.

The classical proof of Yao found in \cite{Goldreich:2000:FCB:519078} relies in the great extent on the fact that
it is easy to verify correctness of a solution.
Thus, in order to show the hardness amplification for weakly verifiable puzzles a different approach has to be found.

Weakly verifiable puzzles has been introduced and studied by Ran Cannetti, Sahi Halevi, and Michael Steiner in \cite{canetti2004hardness}.
They proved that simultaneous solving a number of weakly verifiable puzzles is much harder than solving a single puzzle.

There are situations where we do not require that all puzzles are solved but just a fraction of them.
A prominent example are hard AI problems like CAPTCHAs where a human and computer programs should be distinguished.
A human has an advantage over a computer programs in solving certain problems, but still it is possible to make mistakes.

The first proof where a threshold function is used and which show that it is hard to solve even a fraction of weakly
verifiable puzzles correctly comes from by Russell Impagliazzo, Ragesh Jaiswal, and Valentine Kabanets \cite{impagliazzo2007chernoff}.
A similar broaden proof by Yevgeniy Dodis, Russell Impagliazzo, Ragesh Jaiswal, and Valentine Kabanets \cite{Dodis:2009:SAI:1530441.1530450}
takes into account situations where a solver for weakly verifiable puzzle can ask limited number of queries
that helps to efficiently verify a solution, and can obtain limited number of hints that should help to solve a puzzle.
It captures the case of cryptographic primitives like message authentication codes and signature schemes.
However, both these proof are technically involved and require non trivial intuition.

Thomas Holenstein and Grant Schoenebeck \cite{DBLP:journals/corr/abs-1002-3534} gave a simple and more general proof of hardness amplification for of weakly verifiable puzzles
where only a fraction of puzzles have to be solved correctly.
Furthermore, the puzzles considered by them generalized a situations where a two parties a solve and a puzzle poser intact in an interactive protocol before
a solver is given a puzzle to solve. Such an approach generalize to bit commitment protocols.
What is more, the proof given by Thomas Holenstein and Grant Schoenebeck follows a similar intuitive approach as a proof given in \cite{canetti2004hardness}.

\section{Contribution of the Thesis}
Our main contribution is applying the proof technique given in \cite{DBLP:journals/corr/abs-1002-3534}
in the context of dynamic puzzles that are considered in \cite{Dodis:2009:SAI:1530441.1530450}.
As a result we prove that it is possible to amplify hardness of weakly verifiable puzzles where a solver
can ask hint and verification queries, an instance of a puzzle is created in an interactive protocol, and
which puzzles have to solved is determined by a monotone binary function which generalize a situation where only
a fraction of puzzles have to be solved.

\section{Organization of the Thesis}
In Chapter \ref{ch:preliminaries} we lay down notation and terminology used in the Thesis.
We introduce constructions like pairwise independent hash functions used in later chapters.

Then, in Chapter \ref{ch:intro_weakly} we define a dynamic interactive weakly verifiable puzzle and
given an overview of cryptographic primitives that game based security definition can be modeled as
a puzzle. Furthermore, we give an outline of earlier studies of weakly verifiable puzzles, and
compare it to the puzzles studied in this Thesis.

Finally, in Chapter \ref{ch:main_result} we formulate and prove the main theorem of this Thesis that
shows that it is possible to amplify hardness of dynamic interactive weakly verifiable puzzles.

\chapter{Preliminaries}
\label{ch:preliminaries}
In this chapter we set up notation and terminology used in the Thesis.
%
\section{Notation and Definitions}
\textbf{(Algorithms, Bitstrings and Circuits)}
We define a \textit{Boolean circuit} as a directed acyclic graph with input vertices and vertices implementing logical functions \textit{and}, \textit{or}, and \textit{not}.
We denote Boolean circuits using capital letters from the Greek or English alphabet.
We define a \textit{probabilistic circuit} as a Boolean circuit $C_{m,n} : \{0,1\}^{m} \times \{0,1\}^{n} \rightarrow \{0,1\}^{*}$.
Additionally, we write $C_{m,n}(x;r)$ which should be understood as a probabilistic circuit taking as input  $x \in \{0,1\}^{m}$
and auxiliary input $r \in \{0,1\}^{n}$.
If a probabilistic circuit does not take any input, we abuse notation and write $C_{n}(r)$.
Similarly, we use $\{C_n\}_{n \in \N}$ to denote a family of probabilist circuits that takes only auxiliary input.
We make sure that it is clear from the context that probabilistic circuits with only auxiliary input
are not confused with circuits that do not take auxiliary input.
For a (probabilistic) circuit $C$ we write $\mathit{Size}(C)$ to denote the total number of vertices of $C$.
A \textit{(probabilistic) polynomial size circuit} is a (probabilistic) circuit of size polynomial in the number of input vertices (including auxiliary input).
We define a \textit{two phase circuit} $C := (C_1, C_2)$ as a circuit where in the first phase a circuit $C_1$ is used and in the second phase a circuit $C_2$.
If $C_1$ and $C_2$ are probabilistic circuits we write $C(\delta) := (C_1, C_2)(\delta)$ to denote that in both phases $C_1$ and $C_2$ take
as auxiliary input the same bitstring $\delta$.

\begin{todo}
  \textbf{TODO:} Does it hold for search problems and for algorithms with not a single bit of output.
\end{todo}
It is well known \cite{Arora:2009:CCM:1540612} that a probabilistic polynomial time algorithm can be represented as a circuit of polynomial size.
Moreover, it can be computed in polynomial time and logarithmic space.
Therefore, whenever we state a theorem about circuits it can be also generalized for the polynomial time algorithms.

We write $\mathit{poly}(\alpha_1, \dots, \alpha_n)$ to denote a polynomial on variables $\alpha_1, \dots, \alpha_n$.
For an algorithm $A$ we write $\mathit{Time}(A)$ to denote the number of steps it takes to execute $A$.
We say that $A$ runs in \textit{polynomial time} if the number of steps required to evaluate $A$ is bounded by $poly(|x|)$, where $|x|$ denotes
the length of the input that $A$ takes.
Similarly, as for probabilistic circuits we write the randomness used by a probabilistic algorithm explicitly as a bitstring provided as an auxiliary input.

\textbf{(Probabilities and distributions)}
For a finite set $\cR$ we write $r \xleftarrow{\$} \cR$ to denote that $r$ is chosen from $\cR$ uniformly at random.
For $\delta \in \R : 0 \leq \delta \leq 1$ we write $\mu_{\delta}$ to denote the Bernoulli distribution where outcome $1$ occurs with
probability $\delta$ and $0$ with probability $1-\delta$.
Moreover, we use $\mu_{\delta}^k$ to denote the probability distribution over $k$-tuples
where each element of a $k$-tuple is drawn independently according to $\mu_{\delta}$.
Finally, let $u \leftarrow \mu_{\delta}^k$ denote that a $k$-tuple $u$ is chosen according to $\mu_{\delta}^k$.

Let $(\Omega_n, \cF_n, \Pr)$ be a probability space and $n \in \N$.
Let $E_n \in \cF_n$ denote an event that probability depends on $n$.
We say that $E_n$ happens \textit{almost surely} or with \textit{high probability} if $\Pr[E_n] \geq 1 - 2^{-n} \mathit{poly}(n)$.

\textbf{(Functions)} We call a function $f: \N \rightarrow \R$ \textit{negligible} if for every polynomial $\poly(n)$
there exists $n_0 \in \N$ such that for all $n \in \N : n > n_0$ the following holds
\begin{align*}
f(n) < \frac{1}{\poly(n)}.
\end{align*}
On the other hand, we say that a function $f: \N \rightarrow \R$ is \textit{non-negligible} if
there exists a polynomial $\poly(n)$ such that for some $n_0 \in \N$ and for all $n \in \N : n > n_0$ we have
\begin{align*}
  f(n) \geq \frac{1}{\poly(n)}.
\end{align*}
We say that a function $f$ is \textit{efficiently computable} if there exists a polynomial time algorithm computing $f$.

\textbf{(Interactive protocols)}
We are often interested in situations where two probabilistic circuits interact with each other according to some protocol.
We limit ourselves to the cases where circuits interact by means of messages representable by bitstrings.
Let $\{A_n\}_{n \in \N}$ and $\{B_n\}_{n \in \N}$ be families of circuits such that $A_n : \{0,1\}^{*} \rightarrow \{0,1\}^{*}$ and $B_n : \{0,1\}^{*} \rightarrow \{0,1\}^{*}$.
An \textit{interactive protocol} is defined by a $\{A_n\}_{n \in \N}$ and $\{B_n\}_{n \in \N}$ where
for random bitstrings $\rho_A \in \{0,1\}^{n}$, $\rho_B \in \{0,1\}^{n}$ in the first round $m_0 := A_n(\rho_A)$ and in the second round $m_1 := B_n(\rho_B, m_1)$.
In general in the $k$-th round we have $m_k := A_n(\rho_A, m_1, m_2, \cdots, m_{k-1})$ and in the $k+1$-th round $B_n(\rho_B, m_1, m_2, \cdots, m_{k-1}, m_{k})$.
A protocol execution between two probabilistic circuits $A$ and $B$ is denoted by $\langle A, B \rangle$.
The output of $A$ in a protocol execution is denoted by $\langle A, B \rangle_A$ and of $B$ by $\langle A, B \rangle_B$.
A sequence of all messages sent by $A$ and $B$ in the protocol execution is called a communication transcript and
is denoted by $\langle A, B \rangle_{\mathit{trans}}$.

\textbf{(Oracle algorithms)}
We use notions of \textit{oracle circuits} following the standard definition included in the literature \cite{Goldreich:2004:FCV:975541}.
If a circuit $A$ gain oracle access to a circuit $B$, we write $A^{B}$. If additionally $B$ gain oracle access to a circuit $C$
we write $A^{B^{}C}$. However, to simplify notation we often write $A^{B}$ instead of $A^{B^C}$.
We make sure that it is clear from the context which oracle is accessed by $B$.

\begin{definition}[Polynomial time sampleable distribution]
We say that a distribution is \textnormal{polynomial time sampleable} if it can be approximated by an algorithm running in time $\mathit{poly}(\log|\cD|, \log|cR|)$
up to an exponential factor.
\end{definition}

\begin{definition}[Pairwise independent family of efficient hash functions]
Let $\cD$ and $\cR$ be finite sets and $\cH$ be a family of functions mapping values from $\cD$ to values in $\cR$.
We say that $\cH$ is a \textnormal{family of pairwise independent efficient hash functions}
if $\cH$ has the following properties.

\textbf{(Pairwise independent)} For $\forall x \neq y \in \mathcal{D}$ and $\forall \alpha, \beta \in \cR$, it holds
\begin{displaymath}
\underset{\hash \la0 \cH}{\Pr}[hash(x) = \alpha \mid hash(y) = \beta] = \frac{1}{|\cR|}.
\end{displaymath}

\textbf{(Polynomial time sampleable)} For every $\mathit{hash} \in \cH$ the function $\mathit{hash}$ is sampleable in time $\mathit{poly}(\log|\cD|, \log|\cR|)$.

\textbf{(Efficiently computable)}
For every $hash \in \cH$ there exists an algorithm running in time $\mathit{poly}(\log|\cD|, \log|\cR|)$ which
on input $x \in \cD$ outputs $y \in \cR$ such that $y = hash(x)$.
\end{definition}

We note that the pairwise independence property is equivalent to
\begin{displaymath}
\underset{\hash \la0 \cH}{\Pr}[hash(x) = \alpha \land hash(y) = \beta] = \frac{1}{|\cR|^2}.
\end{displaymath}
It is well know \cite{Carter:1977:UCH:800105.803400} that there exists families of functions meeting the above criteria.

%%% Local Variables:
%%% mode: latex
%%% TeX-master: "master"
%%% End:


\chapter{Weakly Verifiable Cryptographic Primitives}
\label{ch:intro_weakly}
This chapter gives an overview of weakly verifiable cryptographic primitives.
We start by formulating a definition of a \textit{dynamic interactive weakly verifiable puzzle} in Section \ref{section:wvp}.
To provide the Reader more intuition in Section \ref{section:wvp_examples} we describe a~series of well known cryptographic primitives
that are weakly verifiable. Section \ref{st:previous_results} is devoted to the previous research concerning different types of weakly verifiable puzzles.
%
\section{The Definition}
\label{section:wvp}
Let us combine definitions of puzzles introduced in \cite{dodis2009security} and \cite{holenstein2011general}
and define a \textit{dynamic interactive weakly verifiable puzzle} as follows.
\begin{definition}[Dynamic interactive weakly verifiable puzzle]
  \label{def:dwvp}%
  A~\textit{dynamic interactive weakly verifiable puzzle (DIWVP)} is defined by a family of probabilistic circuits $\{P_n\}_{n \in \N}$.
  A circuit belonging to $\{P_n\}_{n \in \N}$ is called a \textit{problem poser}.
  A \textit{solver} $C := (C_1, C_2)$ is a probabilistic two-phase circuit.
  We write $P_n(\pi)$ to denote the execution of $P_n$ with the randomness fixed to $\pi \in \{0,1\}^n$ and $C(\rho) := (C_1,C_2)(\rho)$
  to denote the execution of both $C_1$ and $C_2$ with the randomness fixed to $\rho \in \{0,1\}^{*}$.

  In the first phase, the problem poser $P_n(\pi)$ and the solver $C_1(\rho)$ interact.
  As a result of the interaction, $P_n(\pi)$ outputs a \textit{verification circuit} $\Gamma_{V}$ and a~probabilistic \textit{hint circuit} $\Gamma_{H}$.
  The circuit $C_1(\rho)$ produces no output. The circuit $\Gamma_{V}$ takes as input $q \in \cQ$ (for some set $\cQ$ of indices),
  $y \in \{0,1\}^*$ and outputs a bit. We say that an answer $(q,y)$ is a \textit{correct solution} if and only if $\Gamma_V(q,y) = 1$.
  The circuit $\Gamma_H$ on input $q \in \cQ$ outputs a hint such that $\Gamma_V(q,\Gamma_H(q))~=~1$.

  In the second phase, $C_2$ takes as input $x := \langle P_n(\pi), C_1(\rho) \rangle_{\trans}$
  and has oracle access to $\Gamma_V$ and $\Gamma_H$.
  The execution of $C_2$ with the input $x$ and the randomness fixed to $\rho$
  is denoted by $C_2(x; \rho)$. The queries of $C_2$ to $\Gamma_V$ and $\Gamma_H$ are called \textit{verification queries} and \textit{hint queries}, respectively.
  We say that the circuit $C_2$ \textit{succeeds} if and only if it makes a verification query $(q,y)$ such that $\Gamma_V(q,y)~=~1$,
  and it has not previously asked for a hint query on this $q$.
\end{definition}

If it is clear from the context, we omit the subscript $n$ and write $P$ instead of~$P_n$.

A verification query $(q,y)$ of $C$ for which a hint query on this $q$ has been asked before cannot be a verification query for which $C$ succeeds.
Therefore, without loss of generality throughout this chapter, we make the assumption that $C$ does not ask verification queries on $q$
for which a hint query has been asked before. Moreover, we assume that once $C$ asked a verification query that succeeds,
it does not ask any further hint or verification queries.

There is no loss of generality in assuming that the problem poser and the solver are defined by probabilistic circuits.
Definition \ref{def:dwvp} also embraces the case where the problem poser and the solver are probabilistic polynomial time algorithms.
We use the well known fact \cite{LectureNotesCT} that a probabilistic polynomial time algorithm can be transformed into an equivalent
family of probabilistic Boolean circuits of the polynomial size\footnote{Theorem 6.10 from \cite{LectureNotesCT} is stated for probabilistic, polynomial time,
oracle algorithms with a single bit of output, but it can be adopted to the case where an output is longer than a single bit.}.

We use the term \textit{weakly verifiable} to emphasize that there is no easy way
for the solver to check the correctness of a solution except for asking a verification query.
We call a weakly verifiable puzzle \textit{dynamic} if the number of hint queries is greater than zero.
Furthermore, we say that a weakly verifiable puzzle is \textit{interactive} if in the first
phase the number of messages exchanged between the problem poser and the solver is greater than one.
Finally, we say that a weakly verifiable puzzle is \textit{non-dynamic} if $|\cQ| = 1$
and \textit{non-interactive} if the number of messages sent in the first phase is at most one.

Definition \ref{def:dwvp} generalizes and combines the previous approaches that study
\textit{weakly verifiable puzzles} \cite{canetti2005hardness},
\textit{dynamic weakly verifiable puzzles} \cite{dodis2009security}, and \textit{interactive weakly verifiable puzzles} \cite{holenstein2011general}.

%
\section{The Hardness Amplification Theorem}
\label{section:hardness_amplification_diwvp}
In this section we define the $k$-wise direct product of puzzles and state the hardness amplification theorem for dynamic interactive weakly verifiable puzzles.

\begin{definition}[$k$-wise direct-product of DIWVPs]
  \label{def:k_wise_direct_product}
  Let $g: \{0,1\}^{k}\!\rightarrow\!\{0,1\}$ be a binary monotone function and $P_n^{(1)}$ a problem poser as in Definition~\ref{def:dwvp}.
  The \textit{$k$-wise direct product of $P_n^{(1)}$} is a dynamic interactive weakly verifiable puzzle defined by a circuit $P_{kn}^{(g)}$.
  We write $P_{kn}^{(g)}(\pi^{(k)})$ to denote the execution of $P_{kn}^{(g)}$ with the randomness fixed to $\pi^{(k)} := (\pi_1, \dots, \pi_k)$
  where $\pi_i \in \{0,1\}^n$ for all $1 \leq i \leq n$. Let $(C_1, C_2)(\rho)$ be a probabilistic two-phase circuit called a \textit{solver}.
  In the first phase, the problem poser $P_{kn}^{(g)}(\pi^{(k)})$ sequentially interacts in $k$ rounds with $C_1(\rho)$.
  In the $i$-th round $C_1(\rho)$ interacts with $P_n^{(1)}(\pi_i)$, and as a result $P_{n}^{(1)}(\pi_i)$ generates circuits $\Gamma_V^i, \Gamma_H^i$.
  Finally, after $k$ rounds $P_{kn}^{(g)}(\pi^{(k)})$ outputs a verification circuit
\begin{align*}
  \Gamma_V^{(g)} (q, y_1, \dots, y_k) := g(\Gamma_V^{1}(q, y_1), \dotsc, \Gamma_V^{k}(q, y_k))
\end{align*}
and a hint circuit
\begin{align*}
  \Gamma_H^{(k)} (q) := (\Gamma_H^{1}(q), \dotsc, \Gamma_H^{k}(q)).
\end{align*}
\end{definition}

For the $k$-wise direct product of puzzles we require the solver to make a verification query on a single index $q \in \cQ$.
Otherwise, if verification queries of the form $((q_1,y_1), \dotsc, (q_k, y_k))$ were allowed,
solving the $k$-wise direct product of puzzles would be trivial.
One could fix some $\bar{q} \in \cQ$ and ask
$k$ hint queries such that the $i$-th hint query is of the form $(q_1, \dotsc,q_{i-1}, \bar{q}, q_{i+1}, \dotsc, q_k)$ where
for all $1 \leq j \leq k$ such that $i \neq j$ it holds $q_j \in \cQ \setminus \{\bar{q}\}$.
The solver obtains correct solutions for $\bar{q}$ to the puzzles on all $k$ positions and
can trivially make a successful verification query.

In the following code listing, we define an experiment $\Success$ such that
it outputs $1$ if and only if $C$ asks a successful verification query.
%
%\begin{codeblock}
\begin{restatable}{codeblock}{success}
  \textbf{Experiment} $\Success^{P, C}(\pi, \rho)$
  \medskip \hrule
  \textbf{Oracles:} A problem poser $P$, a solver $C := (C_1, C_2)$ for $P$.\\
  \textbf{Input:}  Bitstrings $\pi \in \{0,1\}^n$, $\rho \in \{0,1\}^*$.\\
  \textbf{Output:} A bit $b \in \{0,1\}$.
  \medskip\hrule
  \Run $\langle P(\pi), C_1(\rho) \rangle$ \\
  \IndI $(\Gamma_V, \Gamma_H) := \langle P(\pi), C_1(\rho) \rangle_{P}$ \\
  \IndI $x := \langle P(\pi), C_1(\rho) \rangle_{\mathit{trans}}$ \\ \\
  \Run $C_2^{\Gamma_V,\Gamma_H}(x; \rho)$ \\
  \IndI \If $C_2^{\Gamma_V, \Gamma_H}(x; \rho)$ asks a verification query $(q, y)$ s.t. $\Gamma_V(q, y) = 1$ \Then \\
  \IndII \Return $1$ \\
  \Return $0$
\end{restatable}
%\end{codeblock}
%
We define the \textit{success probability} of $C$ in solving a puzzle defined by $P$ as
\begin{align}
 \underset{\pi, \rho}{\Pr}[\Success^{P,C}(\pi, \rho) = 1].
\end{align}
Furthermore, for fixed $P$ we say that $C$ \textit{succeeds} for $\pi$, $\rho$ if $\Success^{P,C}(\pi, \rho) = 1$.

We now state our main theorem. Loosely speaking, we claim that it is possible to reduce a solver for the $k$-wise direct product of $P$
to a solver for a single puzzle defined by $P$. This implies that if there is no good solver for $P$, then also a good solver for
the $k$-wise direct product of $P$ does not exist.
%
\begin{restatable}[Hardness amplification for dynamic interactive weakly verifiable puzzles]{theorem}{hardnessAmpfDiwvp}
\label{th:sec_amp_for_dwvp}
Let $P_{n}^{(1)}$ be a fixed problem poser as in Definition \ref{def:dwvp}
and $P_{kn}^{(g)}$ a problem poser for the $k$-wise direct product of $P_{n}^{(1)}$ as in Definition~\ref{def:k_wise_direct_product}.
Additionally, let $C$ be a solver for $P_{kn}^{(g)}$ asking at most $h$ hint queries and $v$ verification queries.
There exists a probabilistic algorithm $\Gen$ with oracle access to $C$,
a binary monotone function $g:\{0,1\}^k \rightarrow \{0,1\}$, and a problem poser $P_{n}^{(1)}$.
The algorithm $\Gen$ takes as input parameters $n$, $\varepsilon$, $\delta$, $k$, $h$, $v$,
and outputs a two-phase probabilistic solver circuit $D$ for $P_{n}^{(1)}$ such that: \\
If $C$ satisfies
  \begin{align}
    \label{th_sec_amp_dwvp_assum}
    \underset{\substack{\pi^{(k)} \in \{0,1\}^{kn} \\ \rho \in \{0,1\}^{*}}}{\Pr}\left[\mathit{Success}^{P_{kn}^{(g)}, C}(\pi^{(k)}, \rho) = 1\right]
    \geq 16(h+v)\Bigl(\underset{u \leftarrow \mu_\delta^k}{\Pr}[g(u) = 1] + \varepsilon \Bigr),
  \end{align}
then $D$ almost surely over the randomness of $\Gen$ satisfies
  \begin{align}
    \underset{\substack{\pi \in \{0,1\}^{n} \\ \rho \in \{0,1\}^{*}}}
    {\Pr}\left[\Success^{P_{n}^{(1)},D}(\pi, \rho) = 1\right] \geq \delta + \frac{\epsilon}{6k}.
  \end{align}
Additionally, $D$ requires oracle access to $g$, $P_{n}^{(1)}$, $C$, hint and verification circuits generated by
the problem poser after the first phase and asks at most $\frac{6k}{\epsilon}\log\left(\frac{6k}{\epsilon}\right) h$
hint queries and one verification query. Finally, $\Time(\Gen) = \poly(k, \frac{1}{\varepsilon}, n, v, h)$ with oracle calls.
\end{restatable}

The above theorem is very general in the sense that it does not impose any constraints
on the size of the circuits or the time complexity of the interactive protocol.
Additionally, we count each oracle call as a single step.

Let us consider, as an example, the case where $C$ and $P_{kn}^{(g)}$ are polynomial time probabilistic algorithms.
Furthermore, we assume that $C$ is such that it satisfies \eqref{th_sec_amp_dwvp_assum}
and $\frac{1}{\epsilon}$ and $k$ are bounded by some polynomial $p(n)$.
Clearly, there are polynomial size families of probabilistic circuits that correspond to $P_{kn}^{(g)}$ and $C$.
%Clearly the solver and the problem poser can be represented as polynomial size families of probabilistic circuits.
The running time of $Gen$ is polynomial, therefore the size of $D$ must also be polynomial.
Finally, the circuit $D$ can be executed by a polynomial time algorithm.
Thereby, we obtain a polynomial time reduction similar as described in the literature \cite{Arora:2009:CCM:1540612, LectureNotesCrypo}.

Theorem \ref{th:sec_amp_for_dwvp} holds with high probability over the randomness of $\Gen$.
More precisely, the circuit output by $\Gen$ satisfies the condition of the theorem with probability
at least $1 - p(k, n, \frac{1}{\epsilon}) \cdot 2^{-n}$. Therefore, Theorem~\ref{th:sec_amp_for_dwvp} is meaningful if there exists $p(n)$
that bounds~$k$~and~$\frac{1}{\epsilon}$.

We emphasize that the number of hint queries asked by $D$ is greater than the number of hint queries asked by $C$ whereas the number of verification queries
is limited to at most one. For many cryptographic constructions, making such an assumption about the number of hint and verification queries is reasonable.
In particular, we cannot assume that a solver for a single puzzle may ask more verification queries than a solver for a $k$-wise direct product of puzzles.

In Chapter \ref{ch:preliminaries} we defined a binary monotone function.
The monotone restriction on $g$ in Theorem \ref{th:sec_amp_for_dwvp} is essential. For $g(b) := 1 - b$ a solver circuit that deliberately gives incorrect
answers satisfies $g$ with probability $1$ whereas a circuit that solves a puzzle successfully with probability
$\gamma > 0$ succeeds only with probability $1 - \gamma$.

We prove Theorem \ref{th:sec_amp_for_dwvp} in Chapter~\ref{ch:main_result}.

%%% Local Variables:
%%% mode: latex
%%% TeX-master: "thesis"
%%% End:

\section{Examples}
\label{section:wvp_examples}
In this section we give examples of cryptographic constructions that motivates studies of different types of weakly verifiable puzzles.

\subsection{Message Authentication Codes}
\begin{todo}
  \textbf{TODO:} Define as game
\end{todo}
Let us consider a setting in which two parties a \textit{sender} and a \textit{receiver} communicate over an insecure channel.
Messages of the sender may be intercepted, modified, and replaced by a third party called an \textit{adversary}.
The receiver needs a way to ensure that received messages have been indeed sent by the sender and have not been modified by the adversary.
The solution is to use \textit{message authentication codes}.

Loosely speaking, the message authentication codes may be explained as follows.
Let sender, receiver, and adversary be polynomial time algorithms, and messages be represented as bitstrings.
Furthermore, we assume that the sender and the receive share a secrete key to which an adversary has no access.
The sender appends to every message a tag which is computed as a function of the key and the message.
The receiver, using the key, has a way to check whether an appended tag is valid for a received message.
The receiver accepts a message if the tag is valid, otherwise it rejects.
We require that it is hard for the adversary to find a tag and a message, not sent before, that is accepted by the receiver with non-negligible probability.
We give the following formal definition of \textit{Message Authentication Code} based on \cite{LectureNotesCrypo} and \cite{Goldreich:2004:FCV:975541}.
\begin{definition}[Message Authentication Code]
  \label{def:mac}
  Let $\cM$ be a set of messages, $\cK$ a set of keys and $\cT$ a set of tags.
  We define the \textnormal{message authentication code (MAC)} as an efficiently computable function $\cM \times \cK \rightarrow \cT$.
  Furthermore, we say that MAC is \textit{secure} if it satisfies the following condition:

  Let $k \xleftarrow{\$} \cK$ be fixed and $H$ be a polynomial size circuit that takes as input a message $m \in \cM$ and outputs a tag $t \in \cT$
  such that $f(m,k) = t$. We say that MAC is secure if there is no probabilistic polynomial time algorithm
  with oracle access to $H$ that with non-negligible probability outputs a message $m \in \cM$
  as well as a corresponding tag $t \in \cT$ such that $f(m, k) = t$, and $\Gamma_H$ has not been queried for a tag of message $m$.
\end{definition}
%
We show how MAC is captured by notion of dynamic weakly verifiable puzzles where at most one verification query is asked.
For fixed $f$ and $n \in \N$ the sender corresponds to the problem poser, the adversary to the problem solver,
and the key is a bitstring $\pi \in \{0,1\}^{n}$ taken as auxiliary input by the problem poser.
In the first phase, which is non interactive, the problem poser outputs a hint circuit
$\Gamma_H$ and a verification circuit $\Gamma_V$ where both circuits have hard-coded $\pi$.
The circuit $\Gamma_H$ takes as input a message and outputs a tag for this message and corresponds to the circuit $H$ from Definition \ref{def:mac}.
The circuit $\Gamma_V$ that as input $m \in \cM$ and $t \in \cT$ and outputs a bit $1$ if and only if $f(m, \pi) = t$.
In the second phase the problem solver takes no input ($x^*$ is empty string) and is given oracle access to $\Gamma_H$ and $\Gamma_V$.
We consider a case where at most on verification query is asked.
Thus, a task of finding by an adversary a valid tag $t \in \cT$ for a message $m \in \cM$ such that a hint for $m$ has not been asked before
corresponds to asking a successful verification query by a problem poser to $\Gamma_V$.
%
\subsection{Public Key Signature Scheme}
% \begin{todo}
%   \textbf{TODO:} Add introduction that gives intuition about the Public Key Signature Schemes
% \end{todo}
First, we give a definition of public key encryption scheme, and what it means for such a scheme to be secure.
These definitions are based on \cite{Goldreich:2004:FCV:975541}.

\begin{definition}[Public key signature scheme]
Let $\cQ$ be the set of messages. A \textnormal{public key signature scheme} is defined by a triple of probabilistic polynomial time algorithms:
$G$ -- the key generation algorithm,
$V$ -- the verification algorithm,
$S$ -- the signing algorithm,
such that the following conditions are satisfied:
\begin{enumerate}[-]
  \item $G(1^n)$ outputs a pair of bitstrings $k_{priv} \in \{0,1\}^{n}$ and $k_{pub} \in \{0,1\}^{n}$ where $n \in \N$ is a security parameter.
    We call $k_{priv}$ a private key and $k_{pub}$ a public key.
  \item The signing algorithm $S$ takes as input $k_{priv} \in \{0,1\}^{n}$, $q \in \cQ$ and outputs a signature $s \in S$.
  \item The verification algorithm $V$ takes as input $k_{pub} \in \{0,1\}^{n}$, $q \in \cQ$, and $s \in S$ and outputs a bit $b \in \{0,1\}$.
  \item For every $k_{priv}$, $k_{pub}$ output by $G$ and every $q \in \cQ$ it holds
    \begin{align*}
      \Pr[V(k_{pub}, q, S(k_{priv}, q))] = 1,
    \end{align*}
    where the probability is over the random coins of $V$ and $S$.
\end{enumerate}
\end{definition}
We say that $s \in S$ is a \textit{valid} signature for $q \in \cQ$ if and only if $V(k_{pub}, q, s) = 1$.
%
%TODO efficiency of the algorithms
%
\begin{definition}\textbf{(Security of public key signature scheme with respect to a chosen message attack)}
Let an \textnormal{adversary} $A$ be a probabilistic polynomial time algorithm that takes as input $k_{pub}$ and has oracle access to $S$.
We say that $A$ \textnormal{succeeds} if it finds a signature $s \in S$ for a message $q \in \cQ$ such that $V(k_{pub}, q, s) = 1$
and the oracle $S$ has not been queried for a signature of $q$.
The public key encryption scheme is \textnormal{secure} if there is no polynomial time adversary that succeeds with non-negligible probability.
\end{definition}
%
We will show now that a public key signature scheme defined as above can be represented as a dynamic weakly verifiable puzzle.
Let a problem poser correspond to entity that generates $k_{pub}$ and $k_{priv}$ and a problem solver to an adversary.
In the first phase, the problem poser uses algorithm $G(1^n)$ to obtain $k_{pub}$, $k_{priv}$ and sends to the adversary the public key $k_{pub}$.
Then the problem poser generates a hint circuit $\Gamma_H$ and a verification circuit $\Gamma_V$.
The hint circuit $\Gamma_H$ takes as input $q \in \cQ$ and outputs a signature for $q$. The verification circuit
$\Gamma_V$ takes as input $s \in S$ and $q \in \cQ$ and checks whether $s \in S$ is a valid signature for $q \in \cQ$.
In the second phase, the problem solver takes as input a transcript of message from the first round which consists solely of a single message $k_{pub}$.
Additionally, it is given oracle access to $\Gamma_V$ and $\Gamma_H$.
It is clear that if the adversary asks a successful verification query $(q,s)$,
then it also forges a signature.

Thus, a game of forging a signature of a public key signature schemes is a weakly verifiable puzzle that
is dynamic but not interactive as in the first phase only a single message is sent.
%
\subsection{Bit Commitments}
Let us consider the following \textit{bit commitment protocol} that involves two parties a \textit{sender} and a \textit{receiver}.
We suppose that the sender and the receiver are polynomial time probabilistic algorithms.
The protocol consists of a \textit{commit phase} and a \textit{reveal phase}.
In the commit phase the sender and the receiver interact, as the result the sender commits to a value $b \in \{0,1\}$.
In the reveal phase the sender opens the commitment by sending to the receiver a pair $(y,b')$ where $y \in \{0,1\}^{*}$ and $b' \in \{0,1\}$.
We require that after the commit phase it is hard for the receiver to correctly guess $b$. Additionally, in the \textit{reveal phase}
it should be hard for the sender to persuade the receiver that it was committed to the value $\lnot b$.

We base the following definition of \textit{bit commitment protocol} on \cite{LectureNotesComThCrypto}.
\begin{definition}[Bit Commitment Protocol]
  \label{def:bit_commitment}
For a security parameter $n~\in~\N$ a \textnormal{bit commitment protocol} is defined by a pair $(S_n, R_n)$
where $S_n = (S_1, S_2)$ is a two phase probabilistic circuit, and $R_n$ is a probabilistic circuit.
We call $S_n$ a sender and $R_n$ a receiver. The circuit $S_1$, used in the commit phase,
takes as input a pair $(b, \rho_S)$ where $b \in \{0,1\}$ is interpreted as a bit to which $S_n$
commits, and $\rho_S \in \{0,1\}^{n}$ is the randomness used by the algorithm $S_n$.
The receiver $R_n$ takes only auxiliary input $\rho_R \in \{0,1\}^{*}$ that is the randomness used by $R_n$.
The protocol consists of two phases. In the commit phase, circuits $S_1$ and $R_n$ engage in the protocol execution.
As the result $S_1$ commits to $b$ and $R_n$ generates a verification circuit $\Gamma_V$.
The circuit $\Gamma_V$ takes as input a bit $b' \in \{0,1\}$ and a bitstring $y \in \{0,1\}^{*}$ and outputs a bit.
In the reveal phase the circuit $S_2$ takes as input a transcript communication transcript from the first phase
$\langle S_1, R_n \rangle_{\mathit{trans}}$, the bitstring $\rho_s$ and returns $(b', y)$.
We require a bit commitment protocol to have the following properties:
\begin{enumerate}[]
\item{\textnormal{\textbf{(Correctness)}}} For a fixed $b \in \{0,1\}$ we have
  \begin{align*}
    \underset{\substack{\rho_S \in \{0,1\}^{*}, \rho_R \in \{0,1\}^{n} \\
        \Gamma_V := \langle S_1(b,\rho_S), R_n(\rho_r) \rangle_{R_n} \\
        (b',y) := S_2(\langle S_1(b,\rho_s), R_n(\rho_R) \rangle,\rho_S)}}{\Pr}\Big[\Gamma_V(b',y) = 1 \Big] \geq 1 - \epsilon(n),
  \end{align*}
where $\epsilon(n)$ is a negligible function of $n$.
\item{\textnormal{\textbf{(Hiding)}}}
  \begin{todo}
    \textbf{TODO:} Describe it using equations, define somehow the guess of R? Maybe as a last message in the first phase of communication
  \end{todo}
  Probability over random coins of $S_n$ and $R_n$ that any polynomial size circuit
  can guess bit $b$ correctly after the commit phase is at most $\frac{1}{2} + \epsilon(n)$ where $\epsilon(n)$ is a negligible function of $n$.
\item{\textnormal{\textbf{(Binding)}}}
  For every polynomial size circuit $S_n$ we have
  \begin{align*}
    \underset{\substack{\Gamma_V := \langle S_1, R \rangle_{R} \\ (b',y) := S_2(\rho_S)}}{\Pr}[\Gamma_V(0,y_0) = 1 \land \Gamma_V(1,y_1) = 1] \leq \epsilon(k),
  \end{align*}
  where $\epsilon(k)$ is a negligible function in $k$.
\end{enumerate}
\end{definition}

\begin{todo}
  \textbf{TODO:} This is not clear ... access to the oracle etc.\\
  \textbf{TODO:} Why it is not possible to verify -- i.e. the sender does not even
  know the function that is used by the receiver to validate decommitment.
\end{todo}

The bit commitment protocols can be generalized as an interactive weakly verifiable puzzle as follows.
The number of hint queries amounts to zero and the number of the verification queries is at most one.
The sender corresponds to a problem solver, and the receiver is a problem poser.
Additionally, we require the problem solver to ask a verification query $(b',y)$ only on $b' := \lnot b$ where $b$
is a bit to which the problem solver is committed after the first phase.
The first phase corresponds to the commit phase.
The second phase is the reveal phase where the problem poser tries to find a bitstring $y$ such that $\Gamma_V(\lnot b, y) = 1$.

\subsection{Automated Turing Tests}
The goal of \textit{Automated Turing Tests} is to distinguish humans from computers which
is frequently used to prevent computer programs from accessing resources for humans.
An example is \textit{CAPTCHA} defined first in \cite{von2003captcha}.
Loossly speaking, CAPTCHA is a test that human can solve with probability close to 1, but it is hard to write a computer program
that has a success probability comparable to the one achieved by humans.
An example of CAPTACHA is an image depicting a distorted text. Most humans guess the text which is displayed on the image correctly, but it might be hard to write
a program for which it would also be easy. We note that the definition of hardness has not been particular well defined ,
and bases on opinions AI community opinions that distinguish between hard and easy AI problems \cite{von2003captcha}.

CAPTCHAs based on guessing the distorted text are weakly verifiable puzzles.
In the first round the problem poser and problem solver engage in interactive protocol, such that
after the execution of the protocol the problem poser has a way to verify the solution.
The problem poser in the second round takes as input a distorted image, and try to guess the text that was used to generated it.
The standard CAPTCHAs are non-dynamic, as the problem poser does not gain access to the hint oracle and
asks only a single verification query.

Our definition captures also the above type of problems, additionally it is also applicable in the broader context for a different
AI problems.

As it is not know how good the possible algorithm can be to recognize CAPTCHA it is likely that the gap between human
performance and a performance of computer programs may be small. Therefore, it is of interest to find a way to amplify this gap.
It turns out that it is indeed possiThe first ble which for not dynamic puzzles was proved in \cite{DBLP:journals/corr/abs-1002-3534}.
The proof presented in Chapter \ref{ch:main_result} applies also to the dynamic context.

\begin{todo}
  \textbf{TODO:} Give an optimization problem for gap amplification
\end{todo}

% why is it hard to automatically check the solution for CAPTCHAs
%todo define information theoretic security
% \subsection{Information theoretically secure constructions}
% The definition presented in \ref{def:dwvp} applies also to information theoretic secure constructions.

%%% Local Variables:
%%% mode: latex
%%% TeX-master: "thesis"
%%% End:

\label{st:previous_results}
In the last chapter we gave an overview of different types of cryptographic primitives that motivated studies of weakly verifiable puzzles.
The focus of this chapter is on giving the outline of the previous research concerning different types of weakly verifiable puzzles
We give a short overview of the techniques used in the series of papers \cite{canetti2004hardness, Dodis:2009:SAI:1530441.1530450, DBLP:journals/corr/abs-1002-3534}
and aim to provide some intuition and insight into the problem of the hardness amplification for dynamic interactive weakly verifiable puzzles.
First, we introduce the notion of the \textit{weakly verifiable puzzles} that are neither interactive nor dynamic and are studied in
\cite{canetti2004hardness}. Then, in Section \ref{section:dijk} we bring our focus on the dynamic non-interactive puzzles studies in \cite{Dodis:2009:SAI:1530441.1530450}.
Finally, in Section \ref{section:iwvp} we give an overview of results of Holenstein and Schoenebeck \cite{DBLP:journals/corr/abs-1002-3534} where
non-dynamic and interactive weakly verifiable puzzles have been studied.
%
\section{Weakly Verifiable Puzzles}
\label{subsec:chs}
The notion of \textit{weakly verifiable puzzles} has been coined by Canetti, Halevi, and Steiner in the paper
\textit{Hardness amplification of weakly verifiable puzzles} \cite{canetti2004hardness}.
In comparison to Definition \ref{def:dwvp} the puzzles considered in \cite{canetti2004hardness} are non-dynamic and non-interactive.
Moreover, the number of verification queries is limited to one. This constitutes a special case of Definition \ref{def:dwvp}.
In this section we provide the definition of weakly verifiable puzzles (WVP) that closely follows the one contained in \cite{canetti2004hardness}
and state the theorem of hardness amplification of weakly verifiable puzzles in a similar vein as in \cite{canetti2004hardness}.
Finally, we give an intuition behind the proof of this theorem. It is noteworthy that the main proof of this Thesis, contained in Chapter \ref{ch:main_result},
uses many ideas of the work of Canetti, Halevi, and Steiner \cite{canetti2004hardness}.
%
\subsection{The definition}
The following definition corresponds to the definition of weakly verifiable puzzles from \cite{canetti2004hardness}
but uses the notation and terminology used in this Thesis.

\begin{definition}[Weakly Verifiable Puzzles, informal \cite{canetti2004hardness}]
  \label{def:wvp}
A \textit{weakly verifiable puzzle} is defined by a pair of polynomial time algorithms:
a probabilistic puzzle--generation algorithm $G$ and a deterministic verification algorithm $V$.
We write $G(1^k; \rho)$ to denote that $G$ takes as input a bitstring $1^k$, where $k$ is a security parameter,
and the randomness $\rho \in \{0,1\}^{*}$.
The algorithm $G$ outputs a bitstring $p \in \{0,1\}^{*}$ and a check information $c \in \{0,1\}^{*}$.
The \textnormal{verifier} $V$ is a deterministic algorithm that takes as input $p$, $c$, an answer $a \in \{0,1\}^{*}$
and outputs $b \in \{0,1\}$.

A \textnormal{solver} $S$ for $G$ is a polynomial time probabilistic algorithm that
takes as input $p$ and outputs $a$. We denote the randomness used by $S$ as $\pi \in \{0,1\}^{*}$
and define the \textnormal{success probability} of $S$ in solving a puzzle defined by $P$ as
\begin{align*}
  \underset{\substack{\rho \in \{0,1\}^{*}, \pi \in \{0,1\}^{*} \\ (p,c):=G(1^k; \rho) \\ a := S(p, \pi)}}{\Pr}\Big[ V(p,c,a) = 1\Big].
\end{align*}
We write $P := (G,V)$ to denote a weakly verifiable puzzle $P$ defined by algorithms $G$ and $V$.
\end{definition}
We compare the above definition with Definition \ref{def:dwvp}.
First, we note that in Definition \ref{def:dwvp} we use the language of probabilistic circuits.
This is more general than probabilistic polynomial time algorithms.
Next, we see that in Definition \ref{def:wvp} the algorithm $G$ is parameterized by
a bitstring $1^k$ meaning that the length of a random bitstring taken by $G$ is bounded by $poly(k)$.
For a fixed $k$, without loss of generality, we can model the algorithm $G(1^k; \rho)$ as a polynomial size probabilistic circuit
that does not take as input $1^k$, but just a bitstring $\rho$ of length $\mathit{poly}(k)$.
The security parameters from Definition \ref{def:wvp} and Definition \ref{def:dwvp} are not equivalent,
as in the later definition the security parameter limits the length of the random bistring.
Moreover, in Definition \ref{def:wvp} a verification algorithm takes as input $p$, $c$, $a$.
Again, without loss of generality, we can assume that bitstrings $p$ and $c$ are hard-coded
in the circuit $\Gamma_V$ from Definition \ref{def:dwvp}. Hence, the algorithm $V$ corresponds to $\Gamma_V$.
The puzzles considered in Definition \ref{def:wvp} are non--dynamic. Thus, there is no corresponding
element for a hint circuit $\Gamma_H$ from Definition~\ref{def:dwvp}.
Finally, the puzzles described in Definition~\ref{def:wvp} are non--interactive.

\subsection{The hardness amplification}
We will formulate now the definition of the $n$-fold repetition of weakly verifiable puzzles along the lines of \cite{canetti2004hardness}.
Later in this Thesis, for other types of weakly verifiable puzzles, we use the notion of a $k$-wise direct product of puzzles instead\footnote{We
note that the terminology used to name the parallel repetition of weakly verifiable puzzles is not consistent.
In \cite{canetti2004hardness} the notion of the \textit{$n$-fold direct product of puzzles} is used whereas in \cite{Dodis:2009:SAI:1530441.1530450}
a similar construction that captures dynamic puzzles is named the \textit{$k$-wise repetition of puzzles}.
In this Thesis we use the latter name.}.
%
\begin{definition}\textbf{($n$-fold repetition of weakly verifiable puzzles, \cite{canetti2004hardness}.)}
  \label{def:n-fold-rep}
  Let $n~\in~\N$ and a weakly verifiable puzzle $P = (G,V)$ be fixed.
  We define the $n$-fold repetition of $P$ as a weakly verifiable puzzle where the puzzle--generation algorithm
  $G^{(n)}$ takes as input $1^k$, as an auxiliary input a bitstring $\rho \in \{0,1\}^{*}$
  and outputs tuples $p^{(n)} := (p_1, \dotsc, p_n) \in \{0,1\}^{*}$ and $c^{(n)} := (c_1, \dotsc, c_k) \in \{0,1\}^{*}$
  where for each $1 \leq i \leq n$ a pair $(p_i, c_i)$ is an independent instance of a weakly verifiable puzzle defined by $G$ and $V$ with security parameter $k$.
  Finally, the verification algorithm $V^{(n)}$ takes as input $p^{(n)}$, $c^{(n)}$, an answer $a^{(n)}$, and outputs $b \in \{0,1\}$
  such that $b = 1$ if and only if for all $1 \leq i \leq n$ we have $V(p_i, c_i, a_i) = 1$.
 \end{definition}
%
Let us give some notation and terminology. We write $P^{(n)} := (G^{(n)}, V^{(n)})$ to denote the $n$--fold repetition of $P$.
For $P^{(n) } := (G^{(n)},V^{(n)})$ when writing a \textit{puzzle on the $i$-th coordinate} we refer to the $i$-th puzzle of the $n$--fold repetition of WVP
(this puzzle corresponds to the one generated by $G^{(n)}_i$,  $V^{(xn)}_i$).

The $n$-fold repetition of weakly verifiable puzzles is solved successfully if and only if all $n$ puzzles are solved successfully.
In contrast, in Chapter \ref{ch:main_result} we are interested in a more general situation where a monotone function $g: \{0,1\}^{n} \rightarrow \{0,1\}$ is used to decide
which coordinates of the $n$-fold repetition of puzzles have to be solved correctly. A precise definition is given in Section \ref{st:main_theorem}.
Clearly, we can assume that $g$ is such that all coordinates have to be solved successfully which matches the case considered in this section.

The main theorem proved in \cite{canetti2004hardness} states that it is possible to turn a good solver for $P^{(n)}$ to a good solver for $P$.
%
\begin{theorem}[Hardness amplification of weakly verifiable puzzles, \cite{canetti2004hardness}]
  \label{thm:wvp_chs}
Let $n:\N \rightarrow \N$ and $\delta: \N \rightarrow (0,1)$ be efficiently computable functions and $q \in \N$ a \textit{slackness parameter}.
Moreover, let $P = (G,V)$ be a weakly verifiable puzzle. We denote the running time of the
puzzle--generation algorithm $G$ for $P$ by $T_G$ and of the verification algorithm $V$ for $P$ by $T_V$.
If $S^{(n)}$ is a solver for the $n$--fold repetition of $P$ that success probability is at least $\delta^{n}$
and the running time is $T$, then there exists a solver $S$ for $G$ with oracle access to $S^{(n)}$ that success
probability is at least $\delta(1-\frac{1}{q})$ and the running time is $O\Big(\frac{nq^3}{\delta^{2n-1}}(T + nT_G + nT_V)\Big)$.
\end{theorem}
%
The following algorithm is used in \cite{canetti2004hardness} to prove Theorem \ref{thm:wvp_chs}.
It transforms $S^{(n)}$ for $P^{(n)}$ with success probability at least $\delta^{n}$ to a solver for a single puzzle $P$ with probability
at least $\delta(1  - \frac{1}{q})$. The slackness parameter $q$ is introduced as it is not possible to achieve
the perfect hardness amplification. We note that in the analysis of the running time of $\mathit{CHS\text{--}solver}$
we explicitly take into account the time needed for the oracle calls to $S^{(n)}, V, G$.

Let us denote by $p \in \{0,1\}^{*}$ an output of $G$, which is taken as input by $\mathit{CHS\text{--}solver}$.
To make notation shorter in the following code excerpts we do not write randomness used by $G$ explicitly.

\begin{codeblock}
  \textbf{Algorithm:} $\mathit{CHS\text{--}solver}^{S^{(n)},V,G}(p, n, k, q, \delta)$
  \medskip\hrule
  \textbf{Oracle:} A solver $S^{(n)}$ for $P^{(n)}$, a verification algorithm $V$ for $P$, a puzzle--generation algorithm $G$ for $P$.\\
  \textbf{Input:}  A bistring $p \in \{0,1\}^{*}$, parameters $n, k, q, \delta$.
  \medskip\hrule
  $\mathit{prefix} := \emptyset$\\
  \For $i = 1$ \To $n\!-\!1$ \Do \\
  \IndI $p^* := \mathit{ExtendPrefix}^{S^{(n)}, V, G}(\mathit{prefix}, i, n, k, q, \delta)$\\
  \IndI \If $p^* = \bot$ \Then \Return $\mathit{OnlinePhase}^{S^{(n)}, V, G}(\mathit{prefix}, p, i, n, k, q, \delta)$ \\
  \IndI \Else $\mathit{prefix} := \mathit{prefix} \circ p^*$\\
  $ a^{(n)} := S^{(n)}(\mathit{prefix} \circ p)$ \\
  \Return $a_n$
\end{codeblock}
%
\begin{codeblock}
  \textbf{Algorithm:} $\mathit{OnlinePhase^{S^{(n)}, V, G}(\mathit{prefix}, p, i, n, k, q, \delta)}$
  \medskip \hrule
  \textbf{Oracle:} A solver algorithm $S^{(n)}$ for $P^{(n)}$, a puzzle--generation algorithm $G$ for $P$, a~verification algorithm $V$ for~$P$.\\
  \textbf{Input:} A $(v-1)$--tuple of bitstrings $\mathit{prefix}$, a bitstring $p \in \{0,1\}^{*}$, \\ parameters $v, n, k, q, \delta$.
  \medskip\hrule
  \Repeat $\Big\lceil\frac{6q \ln (6q)}{\delta^{n-i+1}}\Big\rceil$ times \\
  \IndI $((p_{i+1}, \dotsc, p_{n}),(c_{i+1}, \dots, c_n)) := G^{(n-i-1)}(1^k)$\\
  \IndI $a^{(n)} := S^{(n)}(\mathit{prefix}, p, p_{i+1}, \dotsc, p_n)$\\
  \IndI \If $\forall_{i+1 \leq j \leq n} V(p_j, c_j, a_j) = 1$ \Then \Return $a_i$\\
  \Return $\bot$
\end{codeblock}
%
\begin{codeblock}
  \textbf{Algorithm:} $\mathit{ExtendPrefix^{S^{(n)}, V, G}(prefix, i, n, k, q, \delta)}$
  \medskip \hrule
  \textbf{Oracle:} A solver algorithm $S^{(n)}$ for $P^{(n)}$, a puzzle--generation algorithm $G$ for $P$, a~verification algorithm $V$ for~$P$.\\
  \textbf{Input:} A $(i-1)$--tuple of puzzles $\mathit{prefix}$, parameters $i, n, k, q, \delta$.
  \medskip\hrule
  \Repeat $\Big\lceil \frac{6q}{\delta^{n-v+1}} \ln (\frac{18qn}{\delta}) \Big\rceil$ times \\
  \IndI $(p^*, c^*) := G(1^k) $\\
  \IndI $\bar{\nu}_i := \mathit{EstimateResSuccProb}^{G,V}(\mathit{prefix} \circ p^*, i, n, k, q, \delta)$\\
  \IndI \If $\bar{\nu}_i \geq \delta^{n-i}$ \Then \Return $p^*$ \\
  \Return $\bot$
\end{codeblock}
%
\begin{codeblock}
  \textbf{Algorithm:} $\mathit{EstimateResSuccProb}^{S^{(n)},V, G}(\mathit{prefix}, i, n, k, q, \delta)$
  \medskip \hrule
  \textbf{Oracle:} A solver algorithm for $P^{(n)}$, a verification algorithm $V$ for $P$, a~generation algorithm $G$ for~$P$\\
  \textbf{Input:} A $i$--tuple of puzzles $\mathit{prefix}$, parameters $i, n, k, q, \delta$.
  \medskip\hrule
  $successes := 0$ \\
  \Repeat $M := \Big\lceil \frac{84q^2}{\delta^{n-i}} \ln \Big(\frac{18qn \cdot N_i}{\delta} \Big) \Big\rceil$ times \\
  \IndI $((p_{i+1}, \dotsc, p_n), (c_{i+1}, \dotsc, c_n)) := G^{(n-i)}(1^k)$\\
  \IndI $a^{(n)} := A(\mathit{prefix}, p_{i+1}, \dotsc, p_{n})$\\
  \IndI \If $\forall_{i + 1\leq j \leq n} : V(p_j, c_j, a_j) = 1$ \Then $\mathit{successes := successes + 1}$ \\
  \Return $successes / M$
\end{codeblock}
%
%TODO: does this intuition works when $A$ is not deterministic i.e. cells then depends on the randomness of $A$.
A detailed proof of Theorem \ref{thm:wvp_chs} is presented in \cite{canetti2004hardness}.
We limit ourselves to providing the intuition why the CHS--solver transforms a good solver
for the $n$--wise direct product of $P$ to a good solver for $P$.

Let us consider the $n$-fold repetition of $P$, and for simplicity a deterministic solver $S^{(n)}$ for $P^{(n)} := (G^{(n)}, V^{(n)})$.
Furthermore, we write $p^{(n)}, c^{(n)}$ to denote the output of $G^{(n)}$.
We define a matrix $M$ as follows. The columns of $M$ are labeled with all possible bitstrings $p_1$
whereas the rows are labeled with all possible tuples $(p_2, \dotsc, p_n)$ output by algorithm $G^{(n)}$
executed with different randomness.
A cell of $M$ contains a binary $n$-tuple such that the $i$-th bit equals $1$ if and only if $V_i(p_i, c_i, a_i) = 1$ where
 $a^{(n)} := S^{(n)}(p^{(n)})$ and $p^{(n)}$ is a tuple of bitstring inferred by a column and a row of the cell.
We make the following observation.
%
\begin{observation}
\label{obs:wvp_matrix}
For a deterministic polynomial time algorithm $S^{(n)}$ that successfully solves the $n$--fold repetition of $P$ with probability at least $\delta^{n}$
the matrix $M$ defined as above has either a column with $\delta^{(n-1)}$ fractions of cells that are all one vectors, or
a conditional probability that a cell is of the form $1^n$ given that the last $(n-1)$ bits of the cell are equal 1 is at least $\delta$.
\end{observation}
%
We show, at least intuitively using Observation \ref{obs:wvp_matrix}, how the algorithm $\mathit{CHS\text{--}solver}$ can be used to solve a puzzle defined by $P$
with substantial probability given oracle access to $S^{(n)}$ for $P^{(n)}$.
The algorithm starts with the first position and tries to fix a bitstring $p^*$ on this position such that the success probability of $S^{(n)}$ on the remaining $(n-1)$
position is at least $\delta^{(n-1)}$. If it is possible to find $p^*$ such that this condition is satisfied, then we fix $p^*$
on this position and repeat the whole procedure again in the consecutive iteration for the next position.
If $\mathit{CHS\text{--}solver}$ fails to find a bitstring $p^*$, then we assume that there is no column of $M$ that contains a $\delta^{(n-1)}$ fraction
of cells that are of the form $1^n$. We use Observation~\ref{obs:wvp_matrix} to conclude that the conditional probability of
solving the first puzzle given that all puzzles on the remaining position are solved successfully is at least~$\delta$.
We place the input puzzle $p$ on this position and note that all remaining puzzles are generated by $\mathit{CHS\text{--}solver}$.
Thus, it is possible to efficiently verify whether these puzzles are successful solved by $S^{(n)}$.

Obviously, the algorithm $\mathit{CHS\text{--}solver}$ can still fail. First, it may happen that it does not find a column
with a high fraction of puzzles that are solved successfully, although such a column exists.
Secondly, we cannot exclude a situation where no such column exists, but the algorithm fails to find a cell such that last $(n\!-\!1)$ bits are 1.
Finally, it is also possible that an estimate returned by $\mathit{EstimateResSuccProb}$ is incorrect.

It is possible to show that all these events happen with small probability.
Therefore, at least intuitively we see that the algorithm $\mathit{CHS\text{--}solver}$
solves a single WVP puzzle successfully with probability at least $\delta(1\!-\!\frac{1}{q})$ almost surely.

In Chapter \ref{ch:main_result} we study a more general class of puzzles that are not only weakly verifiable but also dynamic and interactive.
Furthermore, we allow a more general situation where a solver successfully solves the $n$-fold repetition of puzzles\footnote{Actually,
in Chapter \ref{ch:main_result} we define the $k$-wise repetition of puzzles following the terminology used in \cite{Dodis:2009:SAI:1530441.1530450}
which is equivalent to the $n$-fold repetition of puzzles.}
although it succeeds only on some coordinates of the $n$-fold repetition of $P$.
It turns out that it is possible to use a similar technique of fixing puzzles on consecutive positions of the $n$-fold repetition of
puzzles to prove the hardness amplification in this more general setting.
%
\section{Dynamic Weakly Verifiable Puzzles}
\label{section:dijk}
Some of the cryptographic constructions presented in Section~\ref{section:wvp_examples}
are not only weakly verifiable but also dynamic (MAC and SIG). This type of puzzles are defined and studied in \cite{Dodis:2009:SAI:1530441.1530450}.
We give a short overview of this work, state the definition of a \textit{dynamic weakly verifiable puzzle} that closely follows
the one included in \cite{Dodis:2009:SAI:1530441.1530450}. Finally, we provide intuition for the proof of the hardness amplification of DWVP
included in \cite{Dodis:2009:SAI:1530441.1530450}.

\subsection{The definition}
\begin{definition}[Dynamic Weakly Verifiable Puzzle.]
  \label{def:dwvp_dodis}
  A \textnormal{dynamic weakly verifiable puzzle} (DWVP) is defined by a distribution $\cD$ on pairs $(x, \alpha)$
  where $\alpha \in \{0,1\}^{*}$ is an advice used to generate and evaluate responses and $x \in \{0,1\}^{*}$ is
  a bitstring taken as input by the solver.
  Furthermore, we consider a set $\cQ$ and a probabilistic polynomial time computable relation $R$ such that
  $R(\alpha, q, r) = 1$ if and only if $r$ is a correct answer to $q \in \cQ$ on the set of puzzle determined by $\alpha$.
  Finally, let $H(\alpha, q)$ be a probabilistic polynomial time computable \textnormal{hint} relation.

  A solver $S$ takes as input $x$ and can ask hint queries on $q \in \cQ$ which are answered using $H(\alpha, q)$ and verification
  queries of the form $(q,r)$ answered by means of $R(\alpha, q, r)$.
  We say that $S$ succeeds if and only if it makes a verification query on $(q,r)$ such that
  $R(\alpha,q,r) = 1$ and it
  has not previously asked for a hint query on this $q$. We write $P := (\cD, R, H)$ to denote a DWVP with a distribution
  $\cD$ of pairs $(x, \alpha)$, and $R$, $H$ being a verification and hint relations respectively.
\end{definition}
%
We show now how the above definition is generalized by Definition \ref{def:dwvp}.
First, instead of considering a distribution on pairs $(x,\alpha)$ in Definition \ref{def:dwvp}
we use a probabilistic problem poser that outputs circuits $\Gamma_H$ and $\Gamma_V$ that corresponds to hint
and verification relations respectively.
Furthermore, the problem poser may interact in the first phase with the problem solver.
In particular, the problem poser can send a bitstring $x$ as a message in the first phase.
Thus, Definition \ref{def:dwvp} captures a more general case where the distribution of
puzzles is defined by both the problem poser and the problem solver.

We define the \textit{$n$-wise direct product of DWVPs} which is conceptually similar to the $n$-fold repetition of WVPs.
%
\begin{definition}[$n$-wise direct product of DWVPs]
For a dynamic weakly verifiable puzzle $P~:=~(\cD, R, H)$ we define the $n$-wise direct product of $P$
as a DWVP with a distribution $\cD^{(n)}$ on tuples $(x_1, \alpha_1), \dotsc, (x_n, \alpha_n)$.
Furthermore, the hint relation is defined by $H^{(n)}(q, \alpha_1, \dotsc, \alpha_n) := (H(\alpha_1, q), \dotsc, H(\alpha_n, q))$ and
the verification relation $R^{(n)}(\alpha_1, \dotsc, \alpha_n, r_1, \dotsc, r_n, q)$ evaluates to $1$ if and only if
for $1\!\leq\!i\!\leq\!n$ at least $n - (1 - \gamma)\delta n$ is such that $R(\alpha_i, q, r_i)~=~1$ where $0~\leq~\gamma,\delta~\leq~1$.
%
We write $P^{(n)} := (D^{(n)}, H^{(n)}, R^{(n)})$ to denote the $n$-wise direct product of $P := (D,H,R)$.
%
\end{definition}

In contrast to the $n$-fold repetition of puzzles defined in the previous section, here we
require the solver to succeed only on a fraction of puzzles.

Dynamic weakly verifiable puzzles generalize a games of breaking security of message authentication codes and public signature schemes.
In case of MAC the adversary takes $x$ which is an empty string. For the public signature schemes $x$ is the public key.

We write $(\cH_{\mathit{hint}}, \cV_{\mathit{verif}}) \leftarrow S(x; \delta)$
to denote the execution of a solver $S$ with input $x \in \{0,1\}^{*}$ and using randomness $\delta \in \{0,1\}^{*}$.
Furthermore, $\cH_{hint}$ is the set of all hint queries asked by $S$ and $\cV_{verif}$ is
a set of all pairs $(q,a)$ of verification queries asked in the execution of $S$.

With no loss of generality we make the assumption that the solver does not make
hint queries on the successful verification queries.
We define the \textit{success probability} of a solver $S$ for $P := (G,V)$ as
\begin{align*}
  \underset{\substack{\delta \in \{0,1\}^{*} \\(x,\alpha) \leftarrow \cD \\ (\cH_{hint}, \cV_{verif}) \leftarrow S(x,\delta))}}
  {\Pr}\big[\exists (q,a) \in \cV_{verif} : q \notin \cH_{hint} \land V(q,a) := 1 \big]
\end{align*}

\subsection{The hardness amplification theorem}
\begin{theorem}\textbf{(Hardness amplification for dynamic weakly verifiable puzzles).}
\label{lemma:dwvp}
Let $S^{(n)}$ be a probabilistic algorithm for $P^{(n)}$ that succeeds with
probability at least $\epsilon$, where $\epsilon \geq (800/\gamma\delta) \cdot (h+v) \cdot e^{-\gamma^2\delta n/40}$, and $h$ and $v$
denote the number of hint and verification queries asked by $S^{(n)}$ respectively.
Then there exists a probabilistic algorithm $S$ that succeeds in solving $P$ with probability at least
$1-\delta$ making $O(h(h+v)/\epsilon) \cdot \log(1/\gamma\delta)$ hint queries and at most one verification query.
Furthermore, the running time is $\mathit{poly}(h,v,\frac{1}{\epsilon}, t, \omega, \log(1/\gamma\delta))$ where
$\omega$ is time needed to ask a single hint query.
\end{theorem}

It is worth seeing why the approach presented in the previous section that works well for the $n$-fold repetition of WVP
cannot be applied for the $k$-wise direct product of DWVP (moving aside for a moment the issue of solving only a fraction of puzzle successfully).
For DWVP the algorithm $\mathit{CHS\text{--}solver}$ breaks in the $\mathit{OnlinePhase}$ where
the solver $S^{(n)}$ can be called multiple times.
It is possible that in one of these runs $S^{(n)}$ asks a hint query on $q$
for which in one of the later runs a verification query $(q,r)$ is asked
for which algorithm would return an answer for the input puzzle
(in other words the condition $\forall_{v+1 \leq i \leq n} V(p_i, c_i, a_i) = 1$ is satisfied).
However, the fact that a hint query on this $q$ has been asked makes it impossible to ask a successful verification query on this $q$.
Thus, we can not dismiss a situation where the success probability of $S^{(n)}$ decreases with the number of iterations.

The solution proposed in \cite{Dodis:2009:SAI:1530441.1530450} is to partition the set $\cQ$ into a set of \textit{attacking queries} $\cQ_{\mathit{attack}}$
and a set of \textit{advice queries} $\cQ_{\mathit{adv}}$. The idea is to allow a solver for the $n$-wise direct product to ask hint
queries only on $q \in \cQ_{\mathit{adv}}$, and to halt the execution whenever a hint query is asked on $q \in \cQ_{\mathit{attack}}$.

It is possible, for a solver $S$ that asks at most $h$ hint queries and $v$ verification queries,
to find a function $\cQ \rightarrow \{0,1,\dotsc 2(h+v)-1\}$ such that the success probability of $S$ with respect to
$\cQ_{\mathit{attack}}$ and $\cQ_{\mathit{adv}}$ is multiplied by $\frac{1}{8(h+v)}$.
If $h$ and $v$ are not too big, then the success probability of $S$ can be still substantial.
More formally, for a function $\hash:\cQ \rightarrow \{0,1,\dotsc, 2(h+v)\!-\!1 \}$
we define $\cQ_{\mathit{attack}} := \{q \in \cQ : \hash(q) = 0 \}$ and $\cQ_{adv} := \{q \in \cQ: \hash(q) \neq 0\}$

In \cite{Dodis:2009:SAI:1530441.1530450} the following lemma is proved.
\begin{lemma}
  \label{lemma:hash_function_previous}
Let $S$ be a solver for DWVP which success probability is at least $\delta$, the running time is at most $t$,
and the number of hint and verification queries is at most $h$ and $v$ respectively.
There exists a probabilistic algorithm that runs in time $poly(h,v,\frac{1}{\delta},t)$
that outputs a function $\hash : \cQ \rightarrow \{0,1, \dotsc, 2(h+v)-1\}$
that partitions $\cQ$ to $\cQ_{attack}$ and $\cQ_{adv}$ such that
with probability at least $\frac{\delta}{8(h+v)}$ the first successful verification query $(q',a)$ asked by $S$ is such that $q' \in \cQ_{attack}$
and all previous hint and verification queries has been asked on $q \in \cQ_{adv}$.
\end{lemma}
A function $\hash$ can be found by using a natural sampling technique.
We follow exactly the same approach of the partitioning the domain in the main proof of this Thesis in Section \ref{st:domain_partition}.

Let $H_{\alpha}(q)$ denote a polynomial time probabilistic algorithm that takes as input $q$,
has hard-coded $\alpha$ and outputs $H(\alpha, q)$.
Similarly, we use $R_{\alpha}(q,r)$ to denote a polynomial time probabilistic algorithm that computes relations
$R(\alpha, q, r)$ and has hard-coded bitstring $\alpha$.
The following algorithm is used in \cite{Dodis:2009:SAI:1530441.1530450} in the proof of Lemma \ref{lemma:dwvp}.
It gains oracle access to $R_{\alpha}$ and $H_{\alpha}$ as well as a function $\hash$ from Lemma \ref{lemma:hash_function_previous}.
%
\begin{codeblock}
  \textbf{Algorithm:} $\mathit{DWVP\text{--}solver}^{S^{(n)}, \hash, H_{\alpha}^{(n)}, R_{\alpha}^{(n)}}(x)$
  \medskip
  \hrule
  \textbf{Oracle:}  A solver $S^{(n)}$ for $P^{(n)}$, a function $hash : Q \rightarrow \{0,1, \dotsc, 2(h+v)-1\}$.\\
  \textbf{Input:} A bistring $x \in \{0,1\}^{*}$.
  \medskip\hrule
  \Repeat at most $O(\frac{h+v}{\epsilon} \cdot \log(\frac{1}{\gamma\delta}))$ times \\
  \IndI Let $i \xleftarrow{\$} \{1, \dotsc, n\}$ be a position for $x$.\\
  \IndI Generate $(x_1, \alpha_1), \dotsc, (x_{i-1}, \alpha_{i-1}), (x_{i+1}, \alpha_{i+1}), \dotsc, (x_n, \alpha_n)$ \\
  \IndI using $(n-1)$ calls to $P$ each time with fresh randomness.\\
  \IndI \Run $S^{(n)}(x_1, \dotsc, x_{i-1}, x, x_{i+1}, \dotsc, x_n)$\\
  \IndII \If $S^{(n)}$ asks a hint query on $q$ \Then \\
  \IndIII \If $hash(q) \neq 0$ \Then abort current run of $S^{(n)}$\\
  \IndIII Ask a verification query $r := H(q)$\\
  \IndIII Let $(r_1, \dotsc, r_{i-1}, r_{i+1}, \dotsc, r_{n})$ be hints for query $q$ for puzzle\\
  \IndIII sets $(x_1, \dotsc, x_{i-1}, x_{i+1}, x_n)$\\
  \IndIII Answer the hint query of $S^{(n)}$ using $(r_1, \dots, r_{i-1}, r, r_{i+1}, r_n)$\\
  \IndII \If $S^{(n)}$ asks a verification query $(q, r_1, \dots, r_n)$ \Then \\
  \IndIII \If $hash(q) = 0$ \Then answer the query with $0$\\
  \IndIII Let $m := |j: V(q,r_j) = 1, j \neq i|$\\
  \IndIII \If $m \geq n - n(1-\gamma)\delta$ \Then \\
  \IndIIII make a verification query $(q, r_i)$ and halt.\\
  \IndIII \Else with probability $\rho^{m - n(1-\gamma)\delta}$ ask a verification query \\
  \IndIIII $(q, r_i)$ and halt. \\
  \IndIII Halt the current run of $S^{(n)}$ and go to the next iteration.\\
  \Return $\bot$
\end{codeblock}

% \begin{todo}
%   \textbf{TODO:} explain that hints can be auto generated.
% \end{todo}
%The above algorithm substantially differs from the one used in \cite{canetti2004hardness}.
In the above algorithm we execute multiple times a solver $S^{(n)}$ for the $k$-wise direct product of DWVPs.
In each iteration the position for $x \in \{0,1\}^{*}$ is chosen uniformly at random.
The remaining $(n-1)$ puzzles are generated by the algorithm, thus it is possible to answer
all hint and verification queries for these puzzles.
We use a function $\hash$ to partition the query domain and assume that $\hash$ is such
that the success probability of $S^{(n)}$ with respect to $\hash$ is at least $\frac{\delta}{8(h+v)}$.
We check on which $q$ the solver $S^{(n)}$ asks hint and verification queries.
If a hint query is asked on $q$ such that $hash(q) = 0$ then the execution of $S^{(n)}$
is aborted and we go to the next iteration. This way we make sure that the algorithm
never asks a hint query that could prevent a verification query from succeeding.

%TODO unfold this a bit more
If a verification query is asked on $q$ such that $hash(q) \neq 0$ we answer such a verification
query with $0$.

Finally, in case when $S^{(n)}$ asks a verification query using index $q$ such that $hash(q) = 0$, then
we use a soft decision system to decide whether to ask a verification query.
The idea is that if there are many puzzles among the ones generated by the algorithm that are solved successfully,
then it is likely that also the input puzzle is solved successfully.
We discount $\gamma\delta n$ to take into account that not all puzzles have to be solved successfully.
The detail calculations provided in \cite{Dodis:2009:SAI:1530441.1530450} show that this approach
yields a demanded result. We do not give a more detail description of the proof of \cite{Dodis:2009:SAI:1530441.1530450} as
in Chapter \ref{ch:main_result}, except the domain partitioning, we use a different technique.

In case of weakly verifiable primitives like CAPTCHAs we assume that most people have at least slightly higher probability of solving
these kind of puzzles than the best computer programs. Still, it may happen that humans do not solve all puzzles.
This is a motivation to introduce a threshold function such that on average solutions of humans are treated as solved successfully
but the ones of computer programs on average are classified as not successful.
This motivates a study of the situations where only a fraction of puzzles is solved successfully.

In Chapter \ref{ch:main_result} we consider a weakly verifiable puzzles that are not only dynamic but also interactive.
We use a very similar technique to partition domain $Q$ into advice and hint queries as presented in \cite{Dodis:2009:SAI:1530441.1530450}.
Instead of the requirement to succeed only on a fraction of puzzles we consider an arbitrary, monotone function $g : \{0,1\}^{n} \rightarrow \{0,1\}$
that determines on which coordinates the solver has to succeed in order to successfully solve the $n$-wise direct product of puzzles.

To show the hardness amplification for the $n$-wise direct product with the domain partitioned we use
the approach similar to the one presented in Section~\ref{subsec:chs}. Namely, we try to find a good position for the input puzzle instead of
choosing the position uniformly on random as in \cite{Dodis:2009:SAI:1530441.1530450}.

\section{Interactive Weakly Verifiable Puzzles}
\label{section:iwvp}
% \begin{todo}
%   \textbf{TODO:} Justify why you do not give a more detail description of the
%   algorithm as in the previous sections . \\
%   \textbf{TODO:} Explain in detail what is in your work and what is here.
%   What are my contributions.
% \end{todo}
%
The hardness amplification of interactive weakly verifiable puzzles has been studied by T.Holenstein and G.Schoenebeck in \cite{DBLP:journals/corr/abs-1002-3534}.
We will give now an overview of this work and compare it with our approach.

\subsection{The definition}
The following definition of an \textit{interactive weakly verifiable puzzle} closely follow the one from \cite{DBLP:journals/corr/abs-1002-3534}.
\begin{definition}
An \textit{interactive weakly verifiable puzzle} is defined by a protocol given by two probabilistic algorithms $P$ and $S$.
The algorithm $P$ is called the problem poser and produces as output a verification circuit $\Gamma$.
The algorithm $S$ called the problem solver produces no output.
Furthermore, the \textnormal{success probability} of the algorithm $S$ in solving an interactive weakly verifiable puzzle defined by $(P,S)$ is:
\begin{align*}
  \underset{\substack{\rho, \pi \\ \Gamma^{(g)} := \langle P(\rho), S(\pi) \rangle_{P}}}{\Pr}\Big[\Gamma^{(g)}(\langle P(\rho),S(\pi) \rangle_{S}) = 1 \Big].
\end{align*}
\end{definition}
We are intrested in the hardness amplification of interactive weakly verifiable puzzles. Thus, similarly as in the previous sections
we define the $k$-wise direct product of puzzles.
\begin{definition}\textbf{($\boldsymbol{k}$-wise direct product of interactive weakly verifiable puzzles)}
Let $g: \{0,1\}^{k} \rightarrow \{0,1\}$ be a monotone function and $(P,S)$ be a fixed interactive weakly verifiable puzzle.
The $k$-wise direct product of $(P,S)$ defined by $(P^{(g)}, S^{(g)})$ is an interactive weakly verifiable puzzles in which the sender and the receiver
sequentially interact in $k$ rounds where in each round $(P,S)$ is used to generate an instance of interactive weakly verifiable puzzle.
As the result circuits $\Gamma^{(1)}, \dotsc, \Gamma^{(k)}$ for $P$ are generated.
Finally, $P^{(g)}$ outputs the circuit $\Gamma^{(g)}(y_1, \dotsc, y_k) := g(\Gamma^{(1)}(y_1), \dotsc, \Gamma^{(k)}(y_k))$.
\end{definition}

Similarly as in Definition \ref{def:dwvp} the puzzles considered in \cite{DBLP:journals/corr/abs-1002-3534} are interactive.
Furthermore, a monotone binary function is used to determine whether the $k$-wise repetition
has been successfully solved. Unlike, puzzles in Definition \ref{def:dwvp}, the puzzles studied by T.Holenstein and G.Schoenebeck
are non-dynamic. Thus, only a verification circuit $\Gamma$ is generated and no hint circuit is ever used.

\subsection{The hardness amplification theorem}
The following hardness amplification theorem is proved in \cite{DBLP:journals/corr/abs-1002-3534}.
\begin{theorem}
There exists an algorithm $\mathit{Gen}(C,g,\epsilon, \delta, n)$ which takes as input a solver circuit $C$ for the $k$-wise
direct product of $P$, a monotone function $g: \{0,1\}^{*} \rightarrow \{0,1\}$, and parameters $\epsilon,\delta,n$.
The algorithm $\mathit{Gen}$ outputs a solver circuit $D$ for $P$ such that the following holds.
If $C$ is such that
\begin{align*}
\Pr\Big[\Gamma^{(g)}(\langle P^{(g)}, C \rangle_C) = 1\Big] \geq \Pr_{u \leftarrow \mu_{\delta}^{(k)}} \Big[ g(u) = 1 \Big] + \epsilon,
\end{align*}
then, $D$ satisfies almost surely,
\begin{align*}
  \Pr\Big[ \Gamma(\langle P, D\rangle_{D}) = 1\Big] \geq \delta + \frac{\epsilon}{6k}.
\end{align*}
Additionally, $\mathit{Gen}$ and $D$ only require oracle access to $g$ and $C$.
Furthermore, $\mathit{Size}(D) \leq \mathit{Size}(C) \cdot \frac{6k}{\epsilon} \log(\frac{6k}{\epsilon})$,
and $\Time (\mathit{Gen}) = \poly(k, \frac{1}{\epsilon}, n)$ with oracle calls.
\end{theorem}

First, we notice that the above definition does not impose any restrictions on the time complexity of the poser and the solver.
We consider a general approach where $\mathit{Gen}$ is used to define a polynomial time reduction between a solver for the $k$-wise
direct product of puzzles to a solver for a single puzzle.
Furthermore, in the previous sections we considered solvers for the $k$-wise direct product that were compared with the algorithms that either
solves all puzzles (\cite{canetti2004hardness}) or allowed a fraction of puzzles to be solved incorrectly (\cite{Dodis:2009:SAI:1530441.1530450}).
In the above definition a more general case is considered where we use a binary monotone function $g$.
More precisely we are interested functions that are binary monotonously non-decreasing.

The proof technique used by T.Holenstein and G.Schoenebeck is similar to the one presented in Section \ref{subsec:chs}.
In chapter \ref{ch:main_result} we use very similar approach and fix puzzles on consecutive coordinates of the $n$-wise direct product.

% \begin{todo}
%   \textbf{TODO:} Compare to the work of CHS i.e. what we use there to compare the puzzles
% \end{todo}

% In order to estimate how much better a solver circuit $C$ for the $n$-wise direct product performs when
% a puzzle on the first position is fixed a notion of a surplus $S_{\pi^*, b}$ is introduced:
% \begin{align*}
% S_{\pi^*, b} := \Pr_{\pi^{(k)}} [ c \in \cG_b | \pi_1 = \pi^* ] - \Pr_{u \leftarrow \mu_{\delta}^{k}} [u \in \cG_b],
% \end{align*}
% which intuitively tells us how much better a solver $C$ performs when a first puzzle is always solved correctly (case when $b = 1$)
% or is always solved incorrectly (when $b = 0$).
% Now we observe the following fact. If there exists a puzzle which is fixed on the first position for which the surplus is bigger than
% $(1 - \frac{1}{k})\epsilon$ then we can fixed a this first puzzle and inductively solve the problem for the $(k-1)$-direct product of puzzles.
% \begin{todo}
%   \textbf{TODO:} what do it mean
% \end{todo}
% On the other hand, if we there is no such puzzle to fix on the first position it means that when we fix the first bit of $g$ then
% the performance between the solver $C$ and an algorithm that solves puzzle on each position independently with probability $\delta$
% is similar. However, we know that when the first bit of a function $g$ is not fixed then the solver $C$ is better.
% Thus, we draw a conclusion that the puzzle on the first position has to be solved unusually often.

% In a case it would be possible to fix all $(k-1)$ puzzles except the last one, the proof become trivial as we know that a function
% $g$ with the first $k-1$ bits fixed is either the identity or a constant function.
% %TODO write why it is true in this case

% Thus, it is enough to show that if it is not possible to find an estimate that is low then
% if we place an input puzzle on this position and we can find remaining $k-1$ puzzle such that
% $c \in \cG_1 \setminus \cG_0$ then this puzzle is solved with substantial probability.
% The whole proof is given in \cite{DBLP:journals/corr/abs-1002-3534}, and requires some probability manipulations.

% %cite tell more what is used in our proof which technique do we use.
% Our proof of the hardness amplification for dynamic interactive weakly verifiable puzzles closely follows the one given in \cite{DBLP:journals/corr/abs-1002-3534}.
% \begin{todo}
%   \textbf{TODO:} Why do we consider fixing 0/1 on the first position
%   \textbf{TODO:} How the technique is generalized to approach of CHS \\
%   \textbf{TODO:} Explain why we compare to such a probability i.e. why we consider with $\mu$ \\
%   \textbf{TODO:} Why the requirement for $g$ being a monotone function is interesting \\
%   \textbf{TODO:} Give the intuition behind the proof. \\
%   \textbf{TODO:} Give a proof under the simplified assumptions? \\
%   1) The algorithm always output an answer \\
%   2) For every pi the surpluses $S_{\pi^*, 0}$ and $S_{\pi^*, 1}$ re less than $(1-\frac{1}{k})\epsilon$.\\
% \end{todo}
%%% Local Variables:
%%% mode: latex
%%% TeX-master: "thesis"
%%% End:

%
\section{Limitations of Security Amplification}
%
% the main theorem
\chapter{Hardness amplification for weakly verifiable puzzles}
\label{ch:main_result}
In the previous chapter we gave an overview of the former studies of different types of weakly verifiable puzzles.
We also defined a~more general notion of a~dynamic interactive weakly verifiable puzzle.
The focus of this chapter is on a constructive proof of hardness amplification for dynamic interactive weakly verifiable puzzles.
In Section \ref{st:main_theorem} we formulate the theorem which is then proved in the succeeding sections.
We begin with constructing an algorithm that finds an efficiently computable function that is used
to partition the domain of hint and verification queries. Next, we give a proof of hardness amplification
under the assumption that the domain is well partitioned. Finally, in Section \ref{st:put_together}
we complete the proof by combining the previous steps.
%
% section{Main theorem}
\input{interactive_proof/interactive_proof}
%
% section{Domain partitioning}

\begin{lemma}
\label{lemma:hash_function_probability}
\textbf{Success probability with respect to hash function.} \\
For a fixed $P^{(g)}$ let $C$ succeed in solving the $k$-wise direct product of DWVP produced by $P^{(g)}$
with probability $\gamma$ making $h$ hint and $v$ verification queries.
There exists a probabilistic algorithm, with oracle access to $C$, that runs in time $O((h+v)^4/\gamma^4)$
and with high probability outputs a function $hash: Q \rightarrow \{0, \dots, 2(h+v)-1\}$ such that success probability of
$C$ in random experiment $E$ with respect to the set $P_{hash}$ is at least $\frac{\gamma}{8(h+v)}$.
\end{lemma}
%
% Proof of existence of hash function with required properties
%
\begin{proof}
Let $\cH$ be a family of pairwise independent hash functions $Q \rightarrow \{0,1, \dots,2(h+v)-1\}$.
By a pairwise independence property of $\cH$ we know that for all $i \neq j \in \{1, \dots, (h+v)\}$ and $k,l \in \{0,1,\dots,2(h+v)-1\}$
we have the following
\begin{align}
  \label{eq:hash_pr}
 \forall q_i,q_j \in Q : \underset{\textit{hash} \leftarrow \cH}{\Pr}[hash(q_i) = k \mid hash(q_j) = l] = \underset{\textit{hash} \leftarrow \cH}{\Pr}[hash(q_i) = k] = \frac{1}{2(h+v)}.
\end{align}
For a fixed $P^{(g)}$ and $(\pi_1, \dots, \pi_k)$ in the random experiment $A$ we define a binary random variable $X$ for the event that $hash(q_j) = 0$, and for
every query $q_i$ asked before $q_j$ $hash(q_i) \neq 0$.
By definition of conditional probability
\begin{align*}
  \underset{\textit{hash} \leftarrow \cH}{\Pr}[X=1] &= \underset{\textit{hash} \leftarrow \cH}{\Pr}[hash(q_j) = 0 \land \forall i < j : hash(q_i) \neq 0] \\
  &=\underset{\textit{hash} \leftarrow \cH}{\Pr}[\forall i < j : hash(q_i) \neq 0 \mid hash(q_j) = 0] \underset{\textit{hash} \leftarrow \cH}{\Pr}[hash(q_j) = 0].
\end{align*}
Now we use (\ref{eq:hash_pr}) and obtain
\begin{align*}
\underset{\textit{hash} \leftarrow \cH}{\Pr}[X=1] =
\frac{1}{2(h+v)}\left(1 -\underset{\textit{hash} \leftarrow \cH}{\Pr}[\exists i < j : hash(q_i) = 0 \mid hash(q_j) = 0] \right)
\end{align*}
Using pairwise independence property we conclude
\begin{align*}
\underset{\textit{hash} \leftarrow \cH}{\Pr}[X=1] = \frac{1}{2(h+v)} \left( 1 -\underset{\textit{hash} \leftarrow \cH}{\Pr}[\exists i < j : hash(q_i) = 0] \right).
\end{align*}
Finally, we use union bound and the fact $j \leq (h+v)$ to get
\begin{align*}
\underset{\textit{hash} \leftarrow \cH}{\Pr}[X=1] \geq
\frac{1}{2(h+v)} \left( 1 - \sum_{i < j} \underset{\textit{hash} \leftarrow \cH}{\Pr}[hash(q_i) = 0] \right) \geq \frac{1}{4(h+v)}
\end{align*}
Let $G$ denote the set of all $(\pi_1, \dots, \pi_k)$ for which $C$ succeeds in the random experiment $A$.
Then
\begin{align*}
\underset{\substack{\textit{hash} \leftarrow \cH \\ (\pi_1, \dots, \pi_k)}}{\Pr}[X=1] &=
\sum_{(\pi_1, \dots, \pi_k) \in G} \underset{\textit{hash} \leftarrow \cH}{\Pr}[X=1 \mid (\pi_1, \dots, \pi_k)] \cdot \underset{(\widetilde{\pi}_1, \dots, \widetilde{\pi}_k)}{\Pr}[(\widetilde{\pi}_1, \dots, \widetilde{\pi}_k) = (\pi_1, \dots, \pi_k)]\\
&\geq \frac{1}{4(h+v)} \sum_{(\pi_1, \dots, \pi_k) \in G} \underset{(\widetilde{\pi}_1, \dots, \widetilde{\pi}_k)}{\Pr}[(\widetilde{\pi}_1, \dots, \widetilde{\pi}_k) = (\pi_1, \dots, \pi_k)] = \frac{\gamma}{4(h+v)}
\end{align*}

\begin{codeblock}
  \textbf{Algorithm: FindHash}

  \medskip

  \hrule

  \medskip

  %TODO define the circuit $C$ globally do not forget about limit on number of hint and verification queries
  \textbf{Oracle:} A solver circuit for $k$-wise direct product of DWVP $C^{(\cdot, \cdot)}$ with oracle access to hint and verification oracle.\\
  %TODO better describe this hash functions
  \textbf{Input:} $\cH$ a family of pairwise independent hash functions $Q \rightarrow \{0,1,\dots, 2(h+v)-1\}$
  \medskip\hrule\medskip
  For $i = 1$ to $16(h+v)^2/\gamma^2$ \\
  \IndI $hash \xleftarrow{\$} \cH$ \\
  \IndI $count := 0$ \\
  \IndI \For $j := 1$ to $16(h+v)^2/\gamma^2$ \\
  \IndII $(\pi_1, \dots, \pi_k) \xleftarrow{\$} \{0,1\}^{kl} $\\
  \IndII Run $A^{P^{(g)},C^{(\cdot,\cdot)}}(\pi_1, \dots, \pi_k)$\\
  \IndIII Let $(\widetilde{q},y^{(k)})$ be the first successful verification query. \\
  \IndIII Let $G$ be a set of all $q$ used in hint or verification queries asked before $(\widetilde{q},y^{(k)})$.\\
  \IndII \If $\Gamma_V^{(g)}(\widetilde{q},y^{(k)}) = 1 \land G \subseteq P_{hash}$\\
  \IndIII $count := count + 1$\\
  \IndI \If $count \geq 4(h+v)/\gamma$ \\
  \IndII \return $hash$\\
  \return $\bot$
\end{codeblock}
We show that the algorithm \textbf{FindHash} chooses a hash function such
that almost surly the success probability of $C$ in random experiment $E$
with respect to set $P_{hash}$ is at least $\frac{\gamma}{4(h+v)}$.
Let $\cH_{Good}$ denote the family of hash functions for which $\underset{(\pi_1, \dots, \pi_k)}{\Pr}[X] \geq \frac{\gamma}{4(h+v)}$
and $X_1, \dots, X_i$ be binary random variables such that for a fixed hash function
\begin{align*}
  X_i =
  \begin{cases}
    1 & \text{if in $i$th iteration variable $count$ is increased}\\
    0 & \text{otherwise .}
  \end{cases}
\end{align*}
We first show that it is unlikely that the algorithm \textbf{FindHash} returns $hash \notin \cH_{Good}$.
For $hash \notin \cH_{Good}$ we have $\mathbb{E}_{(\pi_1, \dots, \pi_k)}[X_i] < \frac{\gamma}{4(h+v)}$.
We use Chernoff inequality and obtain
%FIXME write down the definition of $X_i$ correctly
%FIXME write the number over which you sum correctly
\begin{align*}
  \underset{(\pi_1, \dots, \pi_k)}{\Pr} \left[\frac{1}{N} \sum_{i=1}^{N} X_i \geq (1 + \delta) \frac{\gamma}{4(h+v)} \right] \leq
  \underset{(\pi_1, \dots, \pi_k)}{\Pr}\left[\frac{1}{N} \sum_{i=1}^{N} X_i \geq (1 + \delta) \mathbb{E}[X_i]\right] \leq
  e^{-{\frac{\gamma}{4(h+v)}} N \delta^2 /3}
\end{align*}
%
The probability that $hash \in \cH_{Good}$ is not returned by the algorithm is
\begin{align*}
  \underset{(\pi_1, \dots, \pi_k)}{\Pr}[\frac{1}{N} \sum_{i=1}^{N} X_i \leq (1 - \delta) \frac{\gamma}{4(h+v)}] \leq
  \underset{(\pi_1, \dots, \pi_k)}{\Pr}[\frac{1}{N} \sum_{i=1}^{N} X_i \leq (1 - \delta) \mathbb{E}[X_i]] \leq e^{-{\frac{\gamma}{4(h+v)}} N \delta^2 /3}
\end{align*}
%
Finally, we show that almost surely \textbf{FindHash} picks in one of its iteration a hash function that is in $\cH_{Good}$.
From the fact that the random variable $X$ is binary distributed we have
\begin{align*}
  \underset{\substack{\textit{hash} \leftarrow \cH \\ (\pi_1, \dots, \pi_k)}}{\mathbb{E}}[X] \geq \frac{\gamma}{4(h+v)}
\end{align*}
Let $Y_i$ be a binary random variable
\begin{align*}
  Y_i =
  \begin{cases}
    1 & \text{in $i$th iteration $hash \in \cH_{Good}$ is picked} \\
    0 & \text{otherwise .}
  \end{cases}
\end{align*}
We make use of the fact that if a function from $\cH_{Good}$ is picked, then it is returned almost surely. Therefore,
$\mathbb{E}[Y_i] \geq \frac{\gamma}{4(h+v)}$ and we can use Chernoff bound to obtain
\begin{align*}
  \underset{hash \leftarrow \cH}{\Pr}\left[\frac{1}{K} \sum_{i=1}^{K} Y_i = 0\right] &\leq
  \underset{hash \leftarrow \cH}{\Pr}\left[\frac{1}{K} \sum_{i=1}^{K} Y_i \leq (1-\delta) \frac{\gamma}{4(h+v)}\right] \\
  &\leq \underset{hash \leftarrow \cH}{\Pr}\left[\frac{1}{K} \sum_{i=1}^{K} Y_i \leq (1-\delta) \mathbb{E}[Y_i] \right] \leq e^{-\delta^2K \mathbb{E}[Y_i]/2 }
\end{align*}
We see that the bound stated in the lemma \ref{lemma:hash_function_probability} is achieved for valid for $\delta = \frac{1}{2}$ and $K = N = 16(h+v)^2/\gamma^2$
\end{proof}
%%% Local Variables: 
%%% mode: latex
%%% TeX-master: "../master"
%%% End: 

%
% section{Amplification proof for partitioned domain}
\begin{codeblock}
  \textbf{Experiment $E^{P^{(g)}, C^{(.)(.)}, hash}(\pi_1, \dots, \pi_k)$} \\
  Solving $k$-wise direct product of DWVP with respect to the set $P_{hash}$
  \medskip

  \hrule

  \medskip
  \textbf{Oracle:} Problem poser for k-wise direct product $P^{(g)}$ \\
  \IndI A solver circuit for $k$-wise direct product $C^{(\cdot, \cdot)}$ \\
  \IndI A function $hash: Q \leftarrow \{0, \dots, 2(h+v) - 1\}$\\
  \textbf{Input:} Random bitstring $(\pi_1, \dots, \pi_k) \in \{0,1\}^{kl}$\\

  \medskip\hrule\medskip

  $\pi^{(k)} := \left(\pi_1, \dots, \pi_k \right)$\\
  $(x^{k}, \Gamma_V^{(g)}, \Gamma_H^{(k)}) := P^{(g)}(\pi^{k})$\\
  Run $C^{\Gamma_V^{(g)}, \Gamma_H^{(k)}} (x^{(k)})$ \\
  \IndI Let $(q_j,y_j^{(k)})$ be the first successful verification query if $C^{\Gamma_V^{(g)}, \Gamma_H^{(k)}}$ succeeds or \\
  \IndI an arbitrary verification query when it fails.\\
  \textbf{If} $(\forall i < j :  q_i \notin P_{hash} )$ and $q_j \in P_{hash}$ and $\Gamma_V^{(g)}(q_j, y_j^{(k)}) = 1$ \\
  \IndI \textbf{return} 1\\
  \textbf{else}\\
  \IndI \textbf{return} 0\\
\end{codeblock}
%
A canonical success is a situation when a solver $C$ for fixed $hash$ and $P^{(1)}$ succeeds in a random experiment $E$.
%
\begin{codeblock}
  \textbf{Random experiment $F^{P^{(1)}, D, hash}(\pi)$} \\
  Solving a single DWVP with respect to the set $P_{hash}$
  \medskip

  \hrule

  \medskip

  \textbf{Oracle:}
  A dynamic weakly verifiable puzzle $P^{(1)}$ \\
  \IndI A solver circuit for a single DWVP $D$ \\
  \IndI A function $hash: Q \rightarrow \{0,1,\dots, 2(h+v)-1\}$ \\
  \textbf{Input:} Random bitstring $\pi \in \{0,1\}^l$
  %TODO length of the bitstring maybe it is fixed as it is input to P^{(1)} ? like l see the end of the paper by Imaginazzo.
  \medskip\hrule\medskip

  $(x, \Gamma_v, \Gamma_H) := P^{(1)}(\pi)$ \\
  Run $D^{\Gamma_V, \Gamma_H}(x)$ \\
  \IndI Let $(\widetilde{q_j},\widetilde{r_j})$ be the first successful verification query if $D^{\Gamma_V, \Gamma_H}(x)$ succeeds or \\
  \IndI an arbitrary verification query when it fails. \\
  \If $(\forall i < j :  q_i \notin  P_{hash} )$ and $q_j \in P_{hash}$ and $\Gamma_V(q_j) = 1$ \then \\
  \IndI \return 1 \\
  \textbf{else}\\
  \IndI \return 0\\

\end{codeblock}
%
%
\begin{lemma}
  \label{lemma:sec_amp_for_p_hash}
  \textbf{Security amplification of a dynamic weakly verifiable puzzle with respect to set $P_{hash}$.} \\
  For a fixed dynamic weakly verifiable puzzle $P^{(1)}$ there exists an algorithm\\
  $Gen(C, g, \varepsilon, \delta, n, v, h, hash)$, which takes as input a circuit $C$, a monotone function $g$,
  a function $hash : Q \rightarrow \{0, \dots, 2(h+v)-1\}$, parameters $\varepsilon, \delta, n$,
  number of verification $v$, and hint $h$ queries asked by $C$, and outputs a circuit $D$
  such that following holds: \\
  If $C$ is such that \\
  \begin{align*}
    \underset{(\pi_1, \dots, \pi_k)}{\Pr}[E^{P^{(g)}, C, Hash}(\pi_1, \dots, \pi_k)=1] \geq \underset{\mu \leftarrow \mu_\delta^k}{\Pr}[g(\mu) = 1] + \varepsilon
  \end{align*}
  then $D$ satisfies almost surely
  \begin{align*}
    \underset{\pi}{\Pr}[F^{P^{(1)},D, Hash}(\pi) = 1] \geq (\delta + \frac{\varepsilon}{6k})
  \end{align*}
  and $Size(D) \leq Size(C)\frac{6k}{\varepsilon}$ and $Time(Gen) = poly(k, \frac{1}{\varepsilon}, n, v, h)$.
\end{lemma}

\begin{todo}
  \textbf{TODO:} The circuit should return the solutions to puzzles. Then we just need to call circuit $ \Gamma_v $ to eval. it.
  But there should be an assumption that the circuit always returns a tuple in $P_{hash}$ and does not ask hint or verification queries
  on this tuple.
\end{todo}

\begin{codeblock}
  \textbf{Circuit $\widetilde{C}^{\Gamma_V^{(g)}, \Gamma_H^{(g)}, hash, C} (x_1, \dots, x_k)$} \\
  Circuit $\widetilde{C}$ has good canonical success probability.
  \medskip

  \hrule

  \medskip

  \textbf{Oracle:} $\Gamma_V^{(g)}, \Gamma_H^{(g)}, hash, C$ \\
  \textbf{Input:} k-wise direct product of puzzles $(x_1, \dots, x_k)$

  \medskip\hrule\medskip
  Run $C^{(\cdot,\cdot)}(x_1, \dots, x_k)$ \\
  \IndI \If $C$ asks a hint query $q$ \then\\
  \IndII \If $q \in P_{hash}$ \then\\
  \IndIII \return $\bot$\\
  \IndII \textbf{else}\\
  \IndIII answer the hint query with $\Gamma_H^{(k)}(q)$\\
  \\
  \IndI \textbf{If} $C$ asks a verification query $(q, y_1, \dots, y_k)$ \textbf{then} \\
  \IndII \textbf{If} $q \in P_{hash}$ \textbf{then} \\
  \IndIII \text{ask the verification query} $(q, y_1, \dots, y_k)$ \\
  \IndIII \textbf{stop the execution} \\
  \IndII \textbf{else} \\
  \IndIII answer verification query with 0 \\
  \textbf{return} $\bot$
\end{codeblock}
The key difference between circuits $C$ and $\widetilde{C}$
is that if $\widetilde{C}$ asks a verification query $(q, y_1, \dots, y_k)$ then $q \in P_{hash}$.
This means that if $\widetilde{C}$ succeeds then it also succeeds canonically.

\begin{lemma}
  For fixed $P^{(g)}$ it is true that
  \begin{align*}
  \underset{(\pi_1, \dots, \pi_k)}{\Pr}[E^{P^{(g)}, C, Hash}(\pi_1, \dots, \pi_k) = 1] \leq \underset{(\pi_1, \dots, \pi_k)}{\Pr}[\Gamma_V^{(g)} (\widetilde{C}^{\Gamma_V^{(g)}, \Gamma_H^{(g)}, Hash}(\pi_1, \dots, \pi_k)) = 1].
  \end{align*}
\end{lemma}

\begin{proof}
We fix the randomness $(\pi_1, \dots, \pi_k)$ used in the random experiment $E$.
Let $x^{(k)} = (x_1, \dots, x_k)$ be a set of puzzles generated in the random experiment $E$ for the randomness $(\pi_1, \dots, \pi_k)$.
If $C$ succeeds canonically for the set of puzzles $x^{(k)}$, then also circuit $\widetilde{C}$ that runs $C$ on the same set of puzzles succeeds.
Using the definition of conditional expectation, we conclude that
\begin{align*}
  \underset{}{\Pr}[E^{P^{(g)}, C, hash}(\pi^{(k)}) = 1] &=
  \sum_{\pi^{(k)} \in \{0,1\}^{kl}}\underset{}{\Pr}[E^{P^{(g)}, C, hash}(\widetilde{\pi}^{(k)}) = 1 | \pi^{(k)} = \widetilde{\pi}^{(k)}] \underset{}{\Pr}[\pi^{(k)} = \widetilde{\pi}^{(k)}] \\
  &\leq
  \sum_{\pi^{(k)} \in \{0,1\}^{kl}}\underset{}{\Pr}[E^{P^{(g)}, \widetilde{C}, hash}(\widetilde{\pi}^{(k)}) = 1 | \pi^{(k)} = \widetilde{\pi}^{(k)}] \underset{}{\Pr}[\pi^{(k)} = \widetilde{\pi}^{(k)}] \\
  &= \underset{}{\Pr}[E^{P^{(g)}, \widetilde{C}, hash}(\pi^{(k)}) = 1]
\end{align*}

\end{proof}

\begin{codeblock}
  \textbf{Algorithm $Gen(\widetilde{C},g,\epsilon,\delta,n)$}
  \medskip

  \hrule

  \medskip

  \textbf{Oracle:} $\widetilde{C}, g$ \\
  \textbf{Input:}  $\epsilon, \delta, n$\\
  \textbf{Output:} A circuit $D$

  \medskip\hrule\medskip
  \If \text{the number of puzzles to solve equals one} \then \\
  \IndI \return $\widetilde{C}$ \\ \\
  \textbf{For} $i:=1$ to $\frac{6k}{\epsilon}\log(n)$ \\
  \IndI $\pi^* \leftarrow \{0,1\}^{l}$\\
  \IndI $\widetilde{S}_{\pi^*,0} := EvaluateSurplus(\pi^*, 0)$\\
  \IndI $\widetilde{S}_{\pi^*,1} := EvaluateSurplus(\pi^*, 1)$\\
  \IndI \textbf{If} $\widetilde{S}_{\pi^*,0} \geq (1 - \frac{3}{4k}) \epsilon$ or $\widetilde{S}_{\pi^*,1} \geq (1 - \frac{3}{4k}) \epsilon$ \\
  \IndII $\widetilde{C}' := \widetilde{C}$ with the first input fixed on $\pi^*$\\
  \IndII\textbf{return} $Gen(\widetilde{C}', g, \epsilon, \delta, n)$ \\
  // all estimates are lower than $(1-\frac{3}{4k})\varepsilon$\\
  \return $D^{\widetilde{C}}$ \\
  \\
  \textbf{EvaluateSurplus}($\pi^*, b$) \\
  \IndI \textbf{For} $i:=1$ to $N_k$ \\
  \IndII $(\pi_2, \dots, \pi_k) \xleftarrow{\$} \{0,1\}^{(k-1)l}$\\
  \IndII $(c_1, \dots, c_k) := EvalutePuzzles(\pi^*, \pi_2, \dots, \pi_k)$\\
  \IndII $\widetilde{S}_{\pi^*,b}^i := g(b, c_2, \dots, c_k) - \underset{(u_2, \dots, u_k)}{\Pr}[g(b, u_2, \dots, u_k) = 1] $\\
  \IndI \textbf{return} $\frac{1}{N_k} \sum_{i=1}^{N_k} \widetilde{S}_{\pi^*,b}^i$\\
  \\
  \textbf{EvalutePuzzles}($\pi^{(k)}$)\\
  \IndI $(x^{(k)}, \Gamma_V^{(g)}, \Gamma_H^{(k)}) := P^{(g)}(\pi^{(k)})$ \\
  \IndI \textbf{For} $i:=1$ to $k$\\
  \IndII $(x_i, \Gamma_V^{i}, \Gamma_H^{i}) := P^{(1)}(\pi_i)$\\
  \IndI $(q,y^{k}) := \widetilde{C}^{\Gamma_V^{(g)}, \Gamma_H^{(k)}}(x_1, x_2, \dots, x_k)$\\
  \IndI \textbf{For} $i:=1$ to $k$\\
  \IndII $c_i := \Gamma_v^{i}(q, y_i)$\\
  \IndI \textbf{return} $(c_1, \dots, c_k)$\\
\end{codeblock}
%
%
%
\begin{todo}
  \textbf{TODO:} Circuit $\widetilde{C}$ gets as input puzzle find a nice way to genereate the puzzles as it is used in many places in the code.
   Also make EvalutePuzzles more general maybe it should take $\widetilde{C}$ as input?
\end{todo}
\begin{codeblock}
  \textbf{Circuit $D^{\widetilde{C}}$}
  \medskip

  \hrule

  \medskip

  \textbf{Oracle:}  $\widetilde{C}, P^{(1)}$\\
  \textbf{Input:}  puzzle $x^*$, a random bitstring $r \in \{0,1\}^{*}$

  \medskip\hrule\medskip
  \textbf{For} $i:=1$ to $\frac{6k}{\epsilon} \log(\frac{6k}{\epsilon})$\\
  \IndI $\pi^{(k)} \leftarrow \{0,1\}^{kl}$ //read $k \cdot l$ bits from $r$  \\
  \IndI $(c_1, \dots, c_k) := EvaluatePuzzles(\pi^{(k)})$\\
  \IndI \textbf{If} $g(1,c_2, \dots, c_k) = 1$ and $g(0,c_2, \dots, c_k) = 0$\\
  \IndII $(q, y_1, \dots, y_k) := \widetilde{C}(x^*, x_2, \dots, x_k)$\\
  \IndII \textbf{return} $y_1$\\
  \textbf{return} $\bot$ \\

\end{codeblock}
%
% The algorithm $Gen$ recursively builds the circuit that have high success probability in solving a dynamic weakly verifiable puzzle.
% When algorithm recurses it fixes a puzzle on the first position on the input of circuit $\widetilde{C}$ which yields a new circuit $\widetilde{C}'$.
% It happens only in the situation when for some fixed $\pi^*$ circuit $\widetilde{C}$ performs good on the remaining $k-1$ puzzles.
%
For $k=1$ function $g(b)$ is either identity or a constant function.
If $g$ is identity then the success probability of $\widetilde{C}$ is as least $\delta + \epsilon$
and $\widetilde{C}$ can be directly used to solve a puzzle. If the function $g$ is constant the statement is vacuously true.

Let $(q, y_1, \dots, y_k)$ denote the output of $\widetilde{C}$.
Additionally, let us denote by $c_i = \Gamma_V(q, y_i)$ whether $(q,y_i)$ is a correct solution for a single puzzle.
We define surplus as the following quantity:
\begin{align}
  \label{eq:s_pi_b}
S_{\pi^*, b} = \underset{\pi^{(k)}}{\Pr}[g(b, c_2, \dots, c_k) = 1] - \underset{\mu^{(k)}}{\Pr}[g(b, u_2, \dots, u_k) = 1]
\end{align}
The surplus $S_{\pi^*, b}$ tells us how good the algorithm $\widetilde{C}$ performs when the first puzzle is fixed, and value of $c_1$ is neglected.
The procedure \textbf{EvaluateSurplus} returns the estimate for $\widetilde{S}_{\pi^*, b}$.
All puzzles used during obtaining the estimate are generated by $\textbf{EvaluatePuzzles}$.
Therefore, it is possible to provide answers for all hint and verification queries.
The returned estimate $\widetilde{S}_{\pi^*,b}$ that differs from $S_{\pi*, b}$
by at most $\frac{\epsilon}{4k}$ almost surely.
Therefore, if $\widetilde{S}_{\pi^*,b} \geq (1-\frac{3}{4k})\epsilon$ then
with high probability $S_{\pi*,b} \geq (1-\frac{1}{k})\epsilon$.
In this case we use a new monotone binary function $g'(b_2, \dots, b_k) := g(b, b_2, \dots, b_k)$, and fix the first puzzle of $\widetilde{C}$ for the one generated
by using the randomness $\pi^*$. The new circuit satisfies the conditions of Lemma \ref{lemma:sec_amp_for_p_hash} which means that we can
use algorithm $Gen$ for the new circuit $\widetilde{C}$ and monotone function $g'$.

If all estimates are less than $(1-\frac{1}{4k})\epsilon$, then intuitively $\widetilde{C}$
does not perform much better on the remaining $k-1$ puzzles than an algorithm that solves each puzzle independent with probability $\delta$.
However, from the assumption we know that on all $k$ puzzles $\widetilde{C}$ has high success probability.
It means that in this case the first puzzle has to be correctly solved with substantial probability.

\begin{todo}
  \textbf{TODO:} Explain the intuition why it may happen that we still can fail in the case of circuit $\widetilde{D}$.
\end{todo}

We have to show that the success probability when $Gen$ does not recurse is substantial.
We fix a randomness $\pi^*$ and thus also a puzzle $x^*$. For this fixed puzzle using (\ref{eq:s_pi_b}) we get
\begin{align}
\label{eq:diff_s01}
  &\underset{\mu_{\delta}^k}{\Pr}[g(1, \mu_2, \dots, \mu_k)=1] - \underset{\mu_{\delta}^k}{\Pr}[g(0, \mu_2, \dots, \mu_k)=1] = \notag\\
&\IndI  \underset{\pi^{(k)}}{\Pr}[g(1, c_2, \dots, c_k) =1 \mid \pi_1 = \pi^*] - \underset{\pi^k}{\Pr}[g(0, c_2, \dots, c_k) = 1 \mid \pi_1 = \pi^*] - (S_{\pi^*,1} - S_{\pi^*,0})
\end{align}
\begin{todo}
  \textbf{TODO:} Better explain why we can write $\Pr(g() = 1 \land g() = 0)$ as the equivalence for the difference.
\end{todo}
From the monotonicity of $g$ we know that for any set of tuples $(b_1, \dots, b_k)$
and sets $G_0 = \{ (b_1, b_2, \dots, b_k): g(0, b_2, \dots, b_k) = 1\}$, $ G_1 = \{(b_1, b_2, \dots, b_k) : g(1, b_2, \dots, b_k) = 1 \}$
we have $G_0 \subseteq G_1$. Hence, we can write (\ref{eq:diff_s01}):
\begin{align}
  \label{eq:diff_s01_next}
  &\underset{\mu_{\delta}^k}{\Pr}[g(1, \mu_2, \dots, \mu_k) = 1 \land g(0, \mu_2, \dots, \mu_k) = 0] = \notag\\
&\IndI  \underset{\pi^{(k)}}{\Pr}[g(1, c_2, \dots, c_k) = 1 \land g(0, c_2, \dots, c_k) = 0 \mid \pi_1 = \pi^*] - (S_{\pi^*,1} - S_{\pi^*,0}).
\end{align}
Let $G_{\mu^{(k)}}$ denote the event $g(1, \mu_2, \dots, \mu_k) = 1 \land g(0, \mu_2, \dots, \mu_k) = 0$, and correspondingly
$G_{\pi^{(k)}} := g(1, \pi_2, \dots, \pi_k) = 1 \land g(0, \pi_2, \dots, \pi_k) = 0$.
Then multiplying and dividing $\underset{}{\Pr}[\Gamma_v^{(g)}(D(x^*, \pi^{(k)})) = 1 \mid \pi_1 = \pi^*]$ by (\ref{eq:diff_s01_next}) we get
\begin{align}
\label{eq:pr_d_succ_0}
  \underset{r}{\Pr}[\Gamma_V^{(g)}(D(x^*, r)) = 1 \mid \pi_1 = \pi^*] &=
  \frac{\underset{r}{\Pr}[\Gamma_V^{(g)}(D(x^*, r)) = 1 \mid \pi_1 = \pi^*] \underset{\pi^{(k)}}{\Pr}[G_{\pi} \mid \pi_1 = \pi^*]} {\underset{\mu_{\delta}^{k}}{\Pr}[G_{\mu}]} \notag\\
  & \IndI - \frac{\underset{r}{\Pr}[\Gamma_V^{(g)}(D(x^*, r)) = 1 \mid \pi_1 = \pi^*](S_{\pi^*,1} - S_{\pi^*,0})}{\underset{\mu_{\delta}^{k}}{\Pr}[G_{\mu}]}
\end{align}
%
% \begin{todo}
%   \textbf{TODO:} Define $c_1, \dots, c_k$ correctly from the paper it is not known whether it is in the iteration or the final one
% \end{todo}
If output of circuit $D(x^*,r) \neq \bot$ then we denote $c_i := \Gamma_V^{i}(q, y_i)$.
We can write the first summand of (\ref{eq:pr_d_succ_0}) as
\begin{align}
  &\underset{r}{\Pr}[\Gamma_V^{(g)}(D(x^*,r)) = 1 \mid \pi_1 = \pi^*] \underset{\pi^{(k)}}{\Pr}[G_{\pi} \mid \pi_1 = \pi^*] = \notag\\
  &\IndI \underset{r}{\Pr}[D(x^*,r) \neq \bot \mid \pi_1 = \pi^*]
  \underset{\pi^{(k)}}{\Pr}[c_1 = 1 \mid G_{\pi}, \pi_1 = \pi^*]
  \underset{\pi^{(k)}}{\Pr}[G_{\pi} \mid \pi_1 = \pi^*]
\end{align}
where we make use of the fact that the event $G_{\pi}$ implies $D(x^*, r) \neq \bot$.
We consider two cases.
If $\underset{\pi^{k}}{\Pr}[g(1, c_2, \dots, c_k) = 1 \land g(0, c_2, \dots,c_k ) = 0 \mid \pi_1 = \pi^*] \leq \frac{\epsilon}{6k}$ then also
\begin{align}
  \underset{\pi^{(k)}}{\Pr}[c_1 = 1 \mid G_{\pi}, \pi_1 = \pi^*] \underset{\pi^{(k)}}{\Pr}[G_{\pi} \mid \pi_1 = \pi^*] \leq \frac{\epsilon}{6k}
\end{align}
and in the case when $\underset{\pi^{k}}{\Pr}[g(1, c_2, \dots, c_k) = 1 \land g(0, c_2, \dots,c_k ) = 0] > \frac{\epsilon}{6k}$ then circuit $D$ outputs $\bot$
only if it fails in all $\frac{6k}{\epsilon} \log(\frac{6k}{\epsilon})$ iterations to find $\pi^{(k)}$ such that $g(1, c_2, \dots, c_k) = 1 \land g(0, c_2, \dots, c_k) = 0$
which happens with probability
\begin{align}
\underset{r}{\Pr}[D(x^*,r) = \bot \mid \pi_1 = \pi^*] \leq (1 - \frac{\epsilon}{6k})^{\frac{6k}{\epsilon}\log(\frac{\epsilon}{6k})} \leq \frac{\epsilon}{6k}.
\end{align}
We conclude that in both cases we have
\begin{align}
  &\underset{r}{\Pr}[D(x^*,r) \neq \bot \mid \pi_1 = \pi^*]
  \underset{\pi^{(k)}}{\Pr}[c_1 = 1 \mid G_{\pi}, \pi_1 = \pi^*]
  \underset{\pi^{(k)}}{\Pr}[G_{\pi} \mid \pi_1 = \pi^*] \notag\\
  &\IndII \geq \underset{\pi^{(k)}}{\Pr}[c_1 = 1 \mid G_{\pi}, \pi_1 = \pi^*]\underset{\pi^{(k)}}{\Pr}[G_{\pi} \mid \pi_1 = \pi^*] - \frac{\epsilon}{6k}
\end{align}
Using definition of conditional probability we get
\begin{align*}
  &\underset{r}{\Pr}[D(x^*,r) \neq \bot \mid \pi_1 = \pi^*]
  \underset{\pi^{(k)}}{\Pr}[c_1 = 1 \mid G_{\pi}, \pi_1 = \pi^*]
  \underset{\pi^{(k)}}{\Pr}[G_{\pi} \mid \pi_1 = \pi^*] \notag\\
  &\IndII = \underset{\pi^{(k)}}{\Pr}[c_1 = 1 \land g(1, c_2,\dots, c_k) = 1 \land g(0, c_2, \dots, c_k) = 0 \mid \pi_1 = \pi^*] - \frac{\epsilon}{6k} \\
  &\IndII = \underset{\pi^{(k)}}{\Pr}[g(c_1, c_2,\dots, c_k) = 1 \land g(0, c_2, \dots, c_k) = 0 \mid \pi_1 = \pi^*] - \frac{\epsilon}{6k} \\
  &\IndII = \underset{\pi^{(k)}}{\Pr}[g(c_1, c_2,\dots, c_k) = 1 \mid \pi_1 = \pi^*] -  \underset{\pi^{(k)}}{\Pr}[g(0, c_2, \dots, c_k) = 0 \mid \pi_1 = \pi^*] - \frac{\epsilon}{6k} \\
\end{align*}
and finally by (\ref{eq:s_pi_b})
\begin{align}
  &\underset{r}{\Pr}[D(x^*,r) \neq \bot \mid \pi_1 = \pi^*]
  \underset{\pi^{(k)}}{\Pr}[c_1 = 1 \mid G_{\pi}, \pi_1 = \pi^*]
  \underset{\pi^{(k)}}{\Pr}[G_{\pi} \mid \pi_1 = \pi^*] \notag\\
  &\IndII = \underset{\pi^{(k)}}{\Pr}[g(c_1, c_2,\dots, c_k) = 1 \mid \pi_1 = \pi^*] -  \underset{\mu_{\delta}^{(k)}}{\Pr}[g(0, \mu_2, \dots, \mu_k) = 0 \mid \pi_1 = \pi^*]  - S_{\pi^*,0} - \frac{\epsilon}{6k}.
\end{align}
We insert this result into equation (\ref{eq:pr_d_succ_0}) to get
\begin{align}
\label{eq:pr_d_succ_1}
  &\underset{r,\pi}{\Pr}[D(x,r) = 1] = \mathbb{E_{\pi}}[\underset{r}{\Pr}[D(x,r) = 1 \mid \pi_1 = \pi^*]] \notag\\
&\IndI = \mathbb{E}_{\pi}\left[\frac{{\Pr}_{\pi^{(k)}}[g(c) = 1 \mid \pi_1 = \pi^*] -
{\Pr}_{\mu_{\delta}^{(k)}}[g(0, \mu_2, \dots, \mu_k) = 0 \mid \pi_1 = \pi^*] - \frac{\epsilon}{6k}} {{\Pr}_{\mu_{\delta}^{k}}[G_{\mu}]}\right] \notag\\
&\IndII - \mathbb{E}_{\pi}\left[\frac{
  S_{\pi^*,0} + \Pr_r [\Gamma_V^{(g)}(D(x^*,r)) = 1 \mid \pi_1 = \pi^*](S_{\pi^*,1} - S_{\pi^*,0})}
{{\Pr}_{\mu_{\delta}^{k}}[G_{\mu}]}\right]
\end{align}
For the second summand we want to show first that almost all estimates all low if we do not recurse.
Let assume that
\begin{align}
\underset{\pi}{\Pr}\left[\left(S_{\pi,0} \leq (1 - \frac{1}{2k})\epsilon\right) \land \left( S_{\pi,1} \leq (1-\frac{1}{2k})\epsilon\right)\right] < 1 - \frac{\epsilon}{6k},
\end{align}
then the algorithm would recurse almost surely.
Therefore, under the assumption that we do not recurse, we have almost surely
\begin{align}
\underset{\pi}{\Pr}\left[\left(S_{\pi,0} \leq (1 - \frac{1}{2k})\epsilon\right) \land \left( S_{\pi,1} \leq (1-\frac{1}{2k})\epsilon\right)\right] \geq 1 - \frac{\epsilon}{6k}.
\end{align}
Let us define a set
\begin{align}
  \X = \left\{ \pi :  \left(S_{\pi,0} \leq (1 - \frac{1}{2k})\epsilon\right) \land \left( S_{\pi,1} \leq (1-\frac{1}{2k})\epsilon \right) \right\}
\end{align}
and the complement of this set $\X^c$.
We bound the second summand in (\ref{eq:pr_d_succ_1})
\begin{align}
&\mathbb{E}_{\pi}\left[ S_{\pi^*,0} + \Pr_r [\Gamma_V^{(g)}(D(x^*,r)) = 1 \mid \pi_1 = \pi^*](S_{\pi^*,1} - S_{\pi^*,0}) \right] \notag\\
&\IndII = \mathbb{E}_{\pi \in \X^c}\left[ S_{\pi^*,0} + \Pr_r [\Gamma_V^{(g)}(D(x^*,r)) = 1 \mid \pi = \pi^*](S_{\pi^*,1} - S_{\pi^*,0}) \right] \notag\\
&\IndIII +  \mathbb{E}_{\pi \in \X}\left[ S_{\pi^*,0} + \Pr_r [\Gamma_V^{(g)}(D(x^*,r)) = 1 \mid \pi = \pi^*](S_{\pi^*,1} - S_{\pi^*,0}) \right] \\
&\IndII \leq \frac{\epsilon}{6k} + \mathbb{E}_{\pi \in \X^c}\left[ S_{\pi^*,0} + \Pr_r [\Gamma_V^{(g)}(D(x^*,r)) = 1 \mid \pi = \pi^*]((1 - \frac{1}{2k})\epsilon - S_{\pi^*,0}) \right] \\
&\IndII \leq \frac{\epsilon}{6k} + 1 - \frac{\epsilon}{2k} = 1 - \frac{\epsilon}{3k}
\end{align}
Finally, we insert this result into equation (\ref{eq:pr_d_succ_1}) and make use of the fact
\begin{align*}
\underset{}{\Pr}[g(u) = 1] &= \underset{}{\Pr}[(g(0, \mu_2, \dots, \mu_k) = 1) \lor ( g(1,\mu_2, \dots, \mu_k) = 1 \land g(0, \mu_2, \dots, \mu_k) = 0 \land \mu_1 = 1)] \notag\\
&= \Pr[g(0,\mu_2, \dots,\mu_k) = 1] + \Pr[g(1,\mu_2,\dots,\mu_k) = 1 \land g(0, \mu_2, \dots, \mu_k) = 0] \Pr[\mu_1 = 1]
\end{align*}
which yields
\begin{align*}
  \underset{r,\pi}{\Pr}[D(x,r) = 1]
&\geq \mathbb{E}_{\pi}\left[\frac{{\Pr}_{\pi^{(k)}}[g(c) = 1 \mid \pi_1 = \pi^*] -
{\Pr}_{\mu_{\delta}^{(k)}}[g(0, \mu_2, \dots, \mu_k) = 0] - (1 - \frac{1}{6k})\epsilon} {{\Pr}_{\mu_{\delta}^{k}}[G_{\mu}]}\right] \notag
 \end{align*}
 Using the assumptions of Lemma \ref{lemma:sec_amp_for_p_hash}, we get
 \begin{align*}
   \underset{r,\pi}{\Pr}[D(x,r) = 1]
 &\geq \frac{ {\Pr}_{\mu_{\delta}^{(k)}}[g(\mu) = 1] + \epsilon +
 \Pr_{\mu_{\delta}^{(k)}}[g(0, \mu_2, \dots, \mu_k) = 0] - (1 - \frac{1}{6k})\epsilon}
 {\Pr_{\mu_{\delta}^{k}}[G_{\mu}]} \notag\\
 &\geq \frac{\epsilon +
\delta\Pr_{\mu_{\delta}^{(k)}}[G_{\mu}] - (1 - \frac{1}{6k})\epsilon}
{\Pr_{\mu_{\delta}^{k}}[G_{\mu}]} \geq \delta + \frac{\epsilon}{6k}
\end{align*}

%%% Local Variables:
%%% mode: latex
%%% TeX-master: "../master"
%%% End:

%
% subsection{Putting it together}
%
\subsection{The proof of hardness amplification theorem}
\label{st:mainTheoremProof}
In this section we give the proof of Theorem \ref{th:sec_amp_for_dwvp}.
This proof uses tow lemma, that we state proofs are given in the succeeding sections.

\begin{todo}
  \textbf{TODO:} intuition behind the lemma
\end{todo}
The following lemma is analogous to Lemma 4 from \cite{dodis2009security}.
\begin{lemma}[Canonical success probability with respect to a function $\hash$, \cite{dodis2009security}]
\label{lemma:hash_function_probability}
For a fixed problem poser $P_n$ let $C$ be a solver for $P_n$ with the success probability at least $\gamma$,
asking at most $h$ hint queries and $v$ verification queries.
Moreover, let $\cH$ be a family of pairwise independent efficient hash functions $\cQ \rightarrow \{0,1, \dots,2(h+v)-1\}$.
There exists a probabilistic algorithm FindHash that takes as input parameters $\gamma$, $n$, $h$, $v$, and has oracle access to $C$ and $P_n$.
Furthermore, FindHash runs in time polynomial in $(h,v,\frac{1}{\gamma},n)$ and with high probability outputs a function $\hash \in \cH$
such that the canonical success probability of $C$ with respect to $\hash$ is at least $\frac{\gamma}{16(h+v)}$.
\end{lemma}

\begin{todo}
  \textbf{TODO:} intuition behind the lemma
\end{todo}
The following lemma is analogous to Theorem 10 from \cite{holenstein2011general}.
\begin{lemma}[Hardness amplification with respect to $\hash$, \cite{holenstein2011general}]
  \label{lemma:sec_amp_for_p_hash}
  Let $g: \{0,1\}^{k} \rightarrow \{0,1\}$ be a monotone function, $P_n^{(1)}$ a fixed
  problem poser and $\widetilde{C} := (C_1, \widetilde{C}_2)$ a probabilistic two-phase circuit
  with oracle access to a function $\hash: \cQ \rightarrow \{0,1,\dots, 2(h+v)-1\}$
  and a solver $C := (C_1, C_2)$ for $P_{kn}^{(g)}$ that asks at most $h$ hint queries and $v$ verification queries.
  There exists an algorithm Gen that takes as input parameters $\varepsilon$, $\delta$, $n$, $k$,
  has oracle access to $P_n^{(1)}$,  $\widetilde{C}$, $\hash$, $g$,
  and outputs a probabilistic two-phase circuit $D := (D_1, D_2)$ such that the following holds: \\
  If $\widetilde{C}$ is such that
  \begin{align*}
    \underset{\mathclap{\substack{
          \pi^{(k)} \in \{0,1\}^{kn}, \rho \in \{0,1\}^{*} \\
          x:= \langle P^{(g)}(\pi^{(k)}), {C}_1(\rho) \rangle_{\mathit{trans}} \\
          (\Gamma_H^{(k)}, \Gamma_V^{(g)}) := \langle P^{(g)}(\pi^{(k)}), C_1(\rho) \rangle_{P^{(g)}}}}}
    {\Pr}[\Gamma_V^{(g)}(\widetilde{C}_2^{\Gamma_H^{(k)}, C_2, \hash}(x,\rho)) = 1]
    \geq \underset{u \leftarrow \mu_\delta^k}{\Pr}[g(u) = 1] + \varepsilon,
  \end{align*}
  then $D$ satisfies almost surely over the randomness of Gen
  \begin{align*}
    \underset{
      \mathclap{
      \substack{
        \pi \in \{0,1\}^{n}, \rho \in \{0,1\}^{*} \\
        x := \langle P^{(1)}(\pi), D_1^{\widetilde{C}}(\rho) \rangle_{\mathit{trans}} \\
        (\Gamma_H, \Gamma_V) := \langle P^{(1)}(\pi), D_1^{\widetilde{C}}(\rho) \rangle_{P^{(1)}}}}}
    {\Pr}[\Gamma_V(D_2^{P^{(1)}, \widetilde{C}, \hash, g, \Gamma_H}(x, \rho)) = 1] \geq \delta + \frac{\varepsilon}{6k}.
  \end{align*}
  Furthermore, $D$
  asks at most $\frac{6k}{\epsilon}\log\left(\frac{6k}{\epsilon}\right) h$ hint queries and no verification queries.
  Finally, the running time of $\mathit{Gen}$ is polynomial in $k, \frac{1}{\varepsilon}, n$ with oracle calls.
\end{lemma}
The circuit $D$ from Lemma \ref{lemma:sec_amp_for_p_hash} does not ask any verification queries,
instead it outputs a pair $(q, y)$ such that $\hash(q) = 0$ or $\bot$.

\begin{proof}[Proof of Theorem \ref{th:sec_amp_for_dwvp}]
Let $Gen$ be the algorithm from Theorem \ref{th:sec_amp_for_dwvp} that outputs
the solver circuit $D$ for $P$. First, in the following code listing we define $Gen$.

In the following code listing we define the algorithm Gen from Theorem \ref{th:sec_amp_for_dwvp}.
The functions FindHash is as in Lemma \ref{lemma:f}
%
\begin{codeblock}
  \textbf{Algorithm} $\widetilde{\text{Gen}}^{P^{(1)}, g, C}(n, \epsilon, \delta, k, h, v)$
  \medskip\hrule
  \textbf{Oracle:} A problem poser $P^{(1)}$, a function $g: \{0,1\}^{k} \rightarrow \{0,1\}$, \\
  \IndII a solver circuit $C$ for $P^{(g)}$.  \\
  \textbf{Input:} Parameters $n$, $\epsilon$, $\delta$, $k$, $h$, $v$.\\
  \textbf{Return:} A circuit $\widetilde{D} = (D_1, \widetilde{D}_2)$.
  \medskip\hrule
  %
  $hash := \text{FindHash}((h+v)\epsilon, n, h, v)$ \\
  Let $\widetilde{C} := (C_1, \widetilde{C}_2)$ be as in Lemma \ref{lemma:ctilda_c} with oracle access to $C$, $\hash$. \\
  $D := Gen^{P^{(1)},  \widetilde{C},  g, \hash}(\epsilon, \delta, n, k)$ \\
  \Return $\widetilde{D} := (D_1, \widetilde{D}_2)$
\end{codeblock}
%
\begin{codeblock}
  \textbf{Circuit} $\widetilde{D}_2^{D, P^{(1)}, \hash, g, \Gamma_V, \Gamma_H}(x, \rho)$
  \medskip \hrule
  \textbf{Oracle:} A circuit $D :=(D_1, \widetilde{D}_2)$ from Lemma \ref{lemma:sec_amp_for_p_hash}, a problem poser $P^{(1)}$, \\
  \IndII functions $\hash: Q \rightarrow \{0,1, \dots, 2(h+v) - 1\}$, $g: \{0,1\}^{k} \rightarrow \{0,1\}$ \\
  \IndII a verification oracle $\Gamma_V$, a hint oracle $\Gamma_H$.\\
  \textbf{Input:}  Bitstrings $x \in \{0,1\}^{*}$, $\rho \in \{0,1\}^{*}$.
  \medskip\hrule
  $(q, y) := D_2^{P^{(1)}, \widetilde{C}, \hash, g, \Gamma_H}(x, \rho)$ \\
  Ask verification query $(q,y)$ to $\Gamma_V$.
\end{codeblock}
%
We show that Theorem \ref{th:sec_amp_for_dwvp} follows from Lemma \ref{lemma:hash_function_probability} and Lemma \ref{lemma:sec_amp_for_p_hash}.
We fix $P^{(1)}$, $g$, $P^{(g)}$ in the whole proof and consider a solver circuit $C = (C_1, C_2)$, asking at most $h$ hint queries and $v$ verification queries such that
\begin{align*}
    \underset{\pi^{(k)}, \rho}{\Pr}\left[\Success^{P^{(g)}, C}(\pi^{(k)}, \rho) = 1\right] \geq 16(h+v)\Big(\underset{u \leftarrow \mu_\delta^k}{\Pr}\left[g(u) = 1\right] + \varepsilon\Big).
\end{align*}
First, we note that $C$ meets the requirements of Lemma \ref{lemma:hash_function_probability}.
Furthermore, trivially success probability of $C$ is at least $(h+v)\epsilon$.
Thus, the algorithm $\widetilde{\text{Gen}}$ can call FindHash to obtain $\hash:\cQ \rightarrow \{0,1,\dotsc, 2(h+v)-1\}$
such that with high probability it holds
\begin{align*}
    \underset{\pi^{(k)}, \rho}{\Pr}\left[\CanonicalSuccess^{P^{(g)}, C, \hash}(\pi^{(k)}, \rho) = 1\right] \geq \underset{u \leftarrow \mu_\delta^k}{\Pr}\left[g(u) = 1\right] + \varepsilon.
\end{align*}
Additionally, the running time of $\FindHash$ is $\mathit{poly}(h,v,\frac{1}{\epsilon},n)$ with oracle calls.
Applying Lemma \ref{lemma:ctilda_c} for $C=(C_1, C_2)$ we obtain a circuit $\widetilde{C} = (C_1, \widetilde{C}_2)$ that satisfies
\begin{align*}
    \underset{
      \mathclap {
      \substack{\pi^{(k)}, \rho \\
        x := \langle P^{(g)}(\pi^{(k)}), C_1(\rho) \rangle_{\mathit{trans}} \\
        (\Gamma_V^{(g)}, \Gamma_H^{(k)}) := \langle P^{(g)}(\pi^{(k)}), C_1(\rho) \rangle_{P^{(g)}}
      }}}
    {\Pr}\mkern13mu[\Gamma_V^{(g)}(\widetilde{C}_2^{\Gamma_H^{(k)}, C_2, \hash}(x, \rho)) = 1]
    \geq
\underset{u \leftarrow \mu_\delta^k}{\Pr}\left[g(u) = 1\right] + \varepsilon.
\end{align*}
Now, we use the algorithm Gen as in Lemma \ref{lemma:sec_amp_for_p_hash} that yields a circuit $D = (D_1, D_2)$ which with high probability satisfies
\begin{align}
  \label{eq:succ_prob_d}
    \underset{
      \mathclap{
      \substack{\pi, \rho \\ x := \langle P^{(1)}(\pi), D_1^{\widetilde{C}}(\rho) \rangle_{\mathit{trans}} \\
        (\Gamma_H, \Gamma_V) := \langle P^{(1)}(\pi), D_1^{\widetilde{C}}(\rho) \rangle_{P^{(1)}}}}}
    {\Pr}\mkern13mu[\Gamma_V(D_2^{P^{(1)}, \widetilde{C}, hash, g, \Gamma_H}(x, \rho)) = 1] \geq (\delta + \frac{\varepsilon}{6k}).
\end{align}
Finally, $\widetilde{\text{Gen}}$ outputs $\widetilde{D} = (D_1, \widetilde{D}_2)$ with oracle access to $D$, $P^{(1)}$, $hash$, $g$ such that with high probability it holds
\begin{align*}
    \underset{\pi, \rho}{\Pr}\left[\Success^{P^{(1)},\widetilde{D}}(\pi, \rho) = 1\right] \geq \delta + \frac{\varepsilon}{6k}.
\end{align*}
The running time of $\Gen$ is polynomial in $k, \frac{1}{\epsilon}, n)$ with oracle calls.
Thus, the overall running time of $\widetilde{\mathit{Gen}}$ is  polynomial in $k,\frac{1}{\epsilon},h,v,n$ with oracle access.
Furthermore, the circuit $\widetilde{D}$ asks at most $\frac{6k}{\epsilon} \log(\frac{6k}{\epsilon})h$ hint queries and one verification query.
Finally, we have $\mathit{Size}(\widetilde{D}) \leq \mathit{Size}(C) \cdot \frac{6k}{\epsilon}$.
This finishes the proof of Theorem~\ref{th:sec_amp_for_dwvp}.
\end{proof}

%%% Local Variables:
%%% mode: latex
%%% TeX-master: "../master"
%%% End:

%
\section{Discussion}
\subsection{Optimality of the result}
%
%
\appendix
\chapter{Appendix}
\section{Basic inequalities}
\begin{lemma}[Chernoff Bounds]
For independent, identically distributed Bernoulli random variables $X_1, \dots, X_n$ with $X := \sum_{i=1}^n X_i$
with $\Pr[X_i = 1] = p_i$ and $\Pr[X_i = 0] = 1 - p_i$ for all $ 1 \leq  i \leq n$.
we have the following inequalities for $0 \leq \delta \leq 1$ and $\mathbb{E}[X] = \sum_{i=1}^{n} p_i$:
\begin{gather}
\label{ineq:ch0}
\Pr[X \geq (1+\delta) \mathbb{E}[X]] \leq e^{- \mathbb{E}[X] \delta^2/3} \\
\label{ineq:ch1}
\Pr[X \leq (1-\delta) \mathbb{E}[X]] \leq e^{- \mathbb{E}[X] \delta^2/2} \\
\label{ineq:ch2}
\Pr[|X - \mathbb{E}[X]| \geq \delta \mathbb{E}[X]] \leq 2 e^{- \mathbb{E}[X] \delta^2 / 3}.
\end{gather}
\end{lemma}

\backmatter

\bibliographystyle{alpha}
\bibliography{refs}

\end{document}
