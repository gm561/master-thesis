%% (Master) Thesis template
% Template version used: v1.4
%
% Largely adapted from Adrian Nievergelt's template for the ADPS
% (lecture notes) project.

%% We use the memoir class because it offers a many easy to use features.
\documentclass[11pt,a4paper,titlepage]{memoir}

%% Packages
%% ========

%% LaTeX Font encoding -- DO NOT CHANGE
\usepackage[OT1]{fontenc}

%% Babel provides support for languages.  'english' uses British
%% English hyphenation and text snippets like "Figure" and
%% "Theorem". Use the option 'ngerman' if your document is in German.
%% Use 'american' for American English.  Note that if you change this,
%% the next LaTeX run may show spurious errors.  Simply run it again.
%% If they persist, remove the .aux file and try again.
\usepackage[english]{babel}

%% Input encoding 'utf8'. In some cases you might need 'utf8x' for
%% extra symbols. Not all editors, especially on Windows, are UTF-8
%% capable, so you may want to use 'latin1' instead.
\usepackage[utf8]{inputenc}

%% This changes default fonts for both text and math mode to use Herman Zapfs
%% excellent Palatino font.  Do not change this.
%\usepackage[sc]{mathpazo}

%% The AMS-LaTeX extensions for mathematical typesetting.  Do not
%% remove.
\usepackage{amsmath,amssymb,amsfonts,mathrsfs}

%% NTheorem is a reimplementation of the AMS Theorem package. This
%% will allow us to typeset theorems like examples, proofs and
%% similar.  Do not remove.
%% NOTE: Must be loaded AFTER amsmath, or the \qed placement will
%% break
\usepackage[amsmath,thmmarks]{ntheorem}

%% LaTeX' own graphics handling
\usepackage{graphicx}

%% We unfortunately need this for the Rules chapter.  Remove it
%% afterwards; or at least NEVER use its underlining features.
\usepackage{soul}

%% Some more packages that you may want to use.  Have a look at the
%% file, and consult the package docs for each.
%% See the TeXed file for more explanations

%% [OPT] Multi-rowed cells in tabulars
%\usepackage{multirow}

%% [REC] Intelligent cross reference package. This allows for nice
%% combined references that include the reference and a hint to where
%% to look for it.
\usepackage{varioref}

%% [OPT] Easily changeable quotes with \enquote{Text}
%\usepackage[german=swiss]{csquotes}

%% [REC] Format dates and time depending on locale
\usepackage{datetime}

%% [OPT] Provides a \cancel{} command to stroke through mathematics.
%\usepackage{cancel}

%% [NEED] This allows for additional typesetting tools in mathmode.
%% See its excellent documentation.
\usepackage{mathtools}

%% [ADV] Conditional commands
%\usepackage{ifthen}

%% [OPT] Manual large braces or other delimiters.
%\usepackage{bigdelim, bigstrut}

%% [REC] Alternate vector arrows. Use the command \vv{} to get scaled
%% vector arrows.
%\usepackage[h]{esvect}

%% [NEED] Some extensions to tabulars and array environments.
\usepackage{array}

%% [OPT] Postscript support via pstricks graphics package. Very
%% diverse applications.
%\usepackage{pstricks,pst-all}

%% [?] This seems to allow us to define some additional counters.
%\usepackage{etex}

%% [ADV] XY-Pic to typeset some matrix-style graphics
%\usepackage[all]{xy}

%% [OPT] This is needed to generate an index at the end of the
%% document.
%\usepackage{makeidx}

%% [OPT] Fancy package for source code listings.  The template text
%% needs it for some LaTeX snippets; remove/adapt the \lstset when you
%% remove the template content.
\usepackage{listings}
\lstset{language=TeX,basicstyle={\normalfont\ttfamily}}

%% [REC] Fancy character protrusion.  Must be loaded after all fonts.
\usepackage[activate]{pdfcprot}

%% [REC] Nicer tables.  Read the excellent documentation.
\usepackage{booktabs}

%%pseudocode and algorithms
\usepackage{algpseudocode}
\usepackage{algorithm}

%% define comments in single line
\algnewcommand{\LineComment}[1]{\State \(\triangleright\) #1}

%%placing in the right place
\usepackage{float}

\usepackage{caption}

\DeclareCaptionFormat{algor}{
  \hrulefill\par\offinterlineskip\vskip1pt
    \textbf{#1#2}#3\offinterlineskip\hrulefill}
\DeclareCaptionStyle{algori}{singlelinecheck=off,format=algor,labelsep=space}
\captionsetup[algorithm]{style=algori}

\algnewcommand\algorithmicinput{\textbf{Input:}}
\algnewcommand\Input{\item[\algorithmicinput]}

\algnewcommand\algorithmicauxinput{\textbf{Auxiliary input:}}
\algnewcommand\AuxInput{\item[\algorithmicauxinput]}

%indention in algorithms
\newcommand{\pushcode}[1][1]{\hskip\dimexpr#1\algorithmicindent\relax}

% hyphen in math mode
\def\hyph{\text{-}}



%% Our layout configuration.  DO NOT CHANGE.
%% Memoir layout setup

%% NOTE: You are strongly advised not to change any of them unless you
%% know what you are doing.  These settings strongly interact in the
%% final look of the document.

% Dependencies
\usepackage{ETHlogo}

% Turn extra space before chapter headings off.
\setlength{\beforechapskip}{0pt}

\nonzeroparskip
\parindent=0pt
\defaultlists

% Chapter style redefinition
\makeatletter

\if@twoside
  \pagestyle{Ruled}
  \copypagestyle{chapter}{Ruled}
\else
  \pagestyle{ruled}
  \copypagestyle{chapter}{ruled}
\fi
\makeoddhead{chapter}{}{}{}
\makeevenhead{chapter}{}{}{}
\makeheadrule{chapter}{\textwidth}{0pt}
\copypagestyle{abstract}{empty}

\makechapterstyle{bianchimod}{%
  \chapterstyle{default}
  \renewcommand*{\chapnamefont}{\normalfont\Large\sffamily}
  \renewcommand*{\chapnumfont}{\normalfont\Large\sffamily}
  \renewcommand*{\printchaptername}{%
    \chapnamefont\centering\@chapapp}
  \renewcommand*{\printchapternum}{\chapnumfont {\thechapter}}
  \renewcommand*{\chaptitlefont}{\normalfont\huge\sffamily}
  \renewcommand*{\printchaptertitle}[1]{%
    \hrule\vskip\onelineskip \centering \chaptitlefont\textbf{\vphantom{gyM}##1}\par}
  \renewcommand*{\afterchaptertitle}{\vskip\onelineskip \hrule\vskip
    \afterchapskip}
  \renewcommand*{\printchapternonum}{%
    \vphantom{\chapnumfont {9}}\afterchapternum}}

% Use the newly defined style
\chapterstyle{bianchimod}

\setsecheadstyle{\Large\bfseries\sffamily}
\setsubsecheadstyle{\large\bfseries\sffamily}
\setsubsubsecheadstyle{\bfseries\sffamily}
\setparaheadstyle{\normalsize\bfseries\sffamily}
\setsubparaheadstyle{\normalsize\itshape\sffamily}
\setsubparaindent{0pt}

% Set captions to a more separated style for clearness
\captionnamefont{\sffamily\bfseries\footnotesize}
\captiontitlefont{\sffamily\footnotesize}
\setlength{\intextsep}{16pt}
\setlength{\belowcaptionskip}{1pt}

% Set section and TOC numbering depth to subsection
\setsecnumdepth{subsection}
\settocdepth{subsection}

%% Titlepage adjustments
\pretitle{\vspace{0pt plus 0.7fill}\begin{center}\HUGE\sffamily\bfseries}
\posttitle{\end{center}\par}
\preauthor{\par\begin{center}\let\and\\\Large\sffamily}
\postauthor{\end{center}}
\predate{\par\begin{center}\Large\sffamily}
\postdate{\end{center}}

\def\@advisors{}
\newcommand{\advisors}[1]{\def\@advisors{#1}}
\def\@department{}
\newcommand{\department}[1]{\def\@department{#1}}
\def\@thesistype{}
\newcommand{\thesistype}[1]{\def\@thesistype{#1}}

\renewcommand{\maketitlehooka}{\noindent\ETHlogo[2in]}

\renewcommand{\maketitlehookb}{\vspace{1in}%
  \par\begin{center}\Large\sffamily\@thesistype\end{center}}

\renewcommand{\maketitlehookd}{%
  \vfill\par
  \begin{flushright}
    \sffamily
    \@advisors\par
    \@department, ETH Z\"urich
  \end{flushright}
}

\checkandfixthelayout

\setlength{\droptitle}{-48pt}

\makeatother

% This defines how theorems should look. Best leave as is.
\theoremstyle{plain}
\setlength\theorempostskipamount{0pt}

\usepackage[framemethod=tikz]{mdframed}
\newdimen\linenumbersep

\newcommand{\linenumber}[1]{%
  \linenumbersep 4pt%
  \advance\linenumbersep\mdflength{innerleftmargin}%
  \advance\linenumbersep\mdflength{innerlinewidth}%
  \advance\linenumbersep\mdflength{middlelinewidth}%
  \advance\linenumbersep\mdflength{outerlinewidth}%
  \advance\linenumbersep\mdflength{linewidth}%
  \makebox[0pt][r]{{\rmfamily\tiny#1}\hspace*{\linenumbersep}}}

\newenvironment{codeblock}%
   {\medskip\begin{mdframed}\setlength{\parindent}{0cm}}%
   {\end{mdframed}\medskip}
\newcommand{\Ind}{\mbox{}}
\newcommand{\IndI}{\mbox\qquad}
\newcommand{\IndII}{\mbox\qquad\qquad}
\newcommand{\IndIII}{\mbox\qquad\qquad\qquad}
\newcommand{\IndIIII}{\mbox\qquad\qquad\qquad\qquad}

\newenvironment{todo}%
   {\medskip\begin{mdframed}\setlength{\parindent}{0cm}}%
   {\end{mdframed}\medskip}


%%% Local Variables:
%%% mode: latex
%%% TeX-master: "thesis"
%%% End:


%% Theorem environments.  You will have to adapt this for a German
%% thesis.
%% Theorem-like environments

%% This can be changed according to language. You can comment out the ones you
%% don't need.

\numberwithin{equation}{chapter}

% declare theorem style
\declaretheoremstyle[
spaceabove=6pt, spacebelow=6pt,
headfont=\normalfont\bfseries,
notefont=\mdseries, notebraces={(}{)},
bodyfont=\normalfont\itshape,
postheadspace=1em,
%qed=\qedsymbol
]{thm_sty}

% declare definition style
\declaretheoremstyle[
spaceabove=6pt, spacebelow=6pt,
headfont=\normalfont\bfseries,
notefont=\mdseries, notebraces={(}{)},
bodyfont=\normalfont,
postheadspace=1em,
qed=\ensuremath{\lozenge}
]{def_sty}


%% English variants
\declaretheorem[style=thm_sty, name=Theorem, numberwithin=chapter]{theorem}
\declaretheorem[style=thm_sty, name=Observation, sibling=theorem]{observation}
\declaretheorem[style=thm_sty, name=Lemma, sibling=theorem]{lemma}
\declaretheorem[style=def_sty, name=Definition, sibling=theorem]{definition}
%\declaretheorem[style=def_sty, name=Proof]{proof}

%\newtheorem{theorem}{Theorem}[chapter]
% \newtheorem{example}[theorem]{Example}
% \newtheorem{remark}[theorem]{Remark}
% \newtheorem{corollary}[theorem]{Corollary}
% \newtheorem{lemma}[theorem]{Lemma}
% \newtheorem{proposition}[theorem]{Proposition}
% \newtheorem{observation}[theorem]{Observation}

%\theoremstyle{definition}
%\theorembodyfont{\normalfont}
%% end def with blacksquare symbol
%\theoremsymbol{\ensuremath{\lozenge}}
%\newtheorem{definition}[theorem]{Definition}

%%Proof environment with a small square as a "qed" symbol
%\theoremstyle{nonumberplain}
%\theorembodyfont{\normalfont}
%\theoremsymbol{\ensuremath{\square}}
%\theoremseparator{.}
%\newtheorem{proof}{Proof}



%% Helpful macros.
%% Custom commands
%% ===============

%% Special characters for number sets, e.g. real or complex numbers.
\newcommand{\C}{\mathbb{C}}
\newcommand{\D}{\mathbb{D}}
\newcommand{\K}{\mathbb{K}}
\newcommand{\N}{\mathbb{N}}
\newcommand{\M}{\mathbb{M}}
\newcommand{\Q}{\mathbb{Q}}
\newcommand{\R}{\mathbb{R}}
\newcommand{\T}{\mathbb{T}}
\newcommand{\X}{\mathbb{X}}
\newcommand{\Z}{\mathbb{Z}}


\newcommand{\cA}{\mathcal{A}}
\newcommand{\cB}{\mathcal{B}}
\newcommand{\cX}{\mathcal{X}}
\newcommand{\cH}{\mathcal{H}}
\newcommand{\cW}{\mathcal{W}}
\newcommand{\cG}{\mathcal{G}}
\newcommand{\cP}{\mathcal{P}}
\newcommand{\cR}{\mathcal{R}}
\newcommand{\cD}{\mathcal{D}}
\newcommand{\cF}{\mathcal{F}}
\newcommand{\cM}{\mathcal{M}}
\newcommand{\cK}{\mathcal{K}}
\newcommand{\cT}{\mathcal{T}}
\newcommand{\cS}{\mathcal{S}}
\newcommand{\cQ}{\mathcal{Q}}
\newcommand{\cV}{\mathcal{V}}
\newcommand{\cI}{\mathcal{I}}

%define our own code commands
%use capital latters as most of these commands is already defined
\renewcommand{\For}{\textbf{for }}
\renewcommand{\If}{\textbf{if }}
\renewcommand{\Else}{\textbf{else }}
\renewcommand{\Return}{\textbf{return }}
\newcommand{\Then}{\textbf{then }}
\newcommand{\Do}{\textbf{do: }}
\renewcommand{\And}{\textbf{and }}
\newcommand{\Or}{\textbf{or }}
\newcommand{\Run}{\textbf{run }}
\newcommand{\To}{\textbf{to }}
\renewcommand{\Repeat}{\textbf{Repeat }}

%% Fixed/scaling delimiter examples (see mathtools documentation)
\DeclarePairedDelimiter\abs{\lvert}{\rvert}
\DeclarePairedDelimiter\norm{\lVert}{\rVert}

%% Use the alternative epsilon per default and define the old one as \oldepsilon
\let\oldepsilon\epsilon
\renewcommand{\epsilon}{\ensuremath\varepsilon}

%% Also set the alternate phi as default.
\let\oldphi\phi
\renewcommand{\phi}{\ensuremath{\varphi}}

% New command that introduces a tab
\newcommand{\itab}[1]{\hspace{0em}\rlap{#1}}
\newcommand{\tab}[1]{\hspace{.2\textwidth}\rlap{#1}}

\DeclareMathOperator{\la0}{\leftarrow}
\DeclareMathOperator{\ra0}{\rightarrow}

\DeclareMathOperator{\hash}{\mathit{hash}}
\DeclareMathOperator{\CanonicalSuccess}{\mathit{CanonicalSuccess}}
\DeclareMathOperator{\Success}{\mathit{Success}}
\DeclareMathOperator{\poly}{\mathit{poly}}
\DeclareMathOperator{\p}{\mathit{p}}
\DeclareMathOperator{\Time}{\mathit{Time}}
\DeclareMathOperator{\Gen}{\text{Gen}}
\DeclareMathOperator{\FindHash}{\text{FindHash}}
\DeclareMathOperator{\Bad}{\mathit{Bad}}
\DeclareMathOperator{\Good}{\mathit{Good}}
\DeclareMathOperator{\trans}{\mathit{trans}}
\DeclareMathOperator{\Canonical}{\mathit{Canonical}}


%the DWVP for the permutation
\DeclareMathOperator{\PiDWVP}{\Pi_{DWVP}}
%the DWPV for the k-wise product of permutations
\DeclareMathOperator{\kPiDWVP}{\Pi_{DWVP}^{(k)}}

%% Make document internal hyperlinks wherever possible. (TOC, references)
%% This MUST be loaded after varioref, which is loaded in 'extrapackages'
%% above.  We just load it last to be safe.
\usepackage[linkcolor=black,colorlinks=true,citecolor=black,filecolor=black]{hyperref}


%% Document information
%% ====================

\title{On amplification of weakly verifiable dynamic cryptographic primitives}
\author{Grzegorz Makosa}
\thesistype{Master Thesis}
\advisors{Advisors: Prof. Dr. Thomas Holenstein, Dr. Robin Künzler}
\department{Department of Computer Science}
\date{April 8, 2014}

\begin{document}

\frontmatter

% \begin{titlingpage}
%   \calccentering{\unitlength}
%   \begin{adjustwidth*}{\unitlength-24pt}{-\unitlength-24pt}
%     \maketitle
%   \end{adjustwidth*}
% \end{titlingpage}
%\begin{abstract}
% A good abstract explains in one line why the paper is important.
% It then goes on to give a summary of your major results, preferably couched in numbers with error limits.
% The final sentences explain the major implications of your work. A good abstract is concise, readable, and quantitative.
% Length should be ~ 1-2 paragraphs, approx. 400 words.
%
% Absrtracts generally do not have citations.
% Information in title should not be repeated.
% Be explicit.
% Use numbers where appropriate.
% Answers to these questions should be found in the abstract:
% What did you do?
% Why did you do it? What question were you trying to answer?
% How did you do it? State methods.
% What did you learn? State major results.
% Why does it matter? Point out at least one significant implication.

% 1) new hardness proof
% 2) it is possible to start with a weak crypto primitive and obtain strong one.
% 3) a natural puzzle fixing technique similar to the one used in the classical proofs of weak one-way function implying strong ones
% 4) uses an arbitrary monotone function to decide the result
% 5) compare with previous works
% 6)

We give a new proof of hardness amplification for Dynamic Weakly Verifiable Puzzles which is more general than the previous ones.

Our proof applies to the various cryptographic primitives like hash function,
one way functions, signature schemes and bit commitment protocols.

\end{abstract}


\cleartorecto
\tableofcontents
\mainmatter

\chapter{Introduction}
\section{Security Amplification Theorems}
Introduction to security amplification theorems and hardness implication statements.
Example of classical results. Problems captured by weakly verifiable puzzles.
Contribution of this thesis.

\section{Organization of the Thesis}
Overview of the content of the succeeding chapters.

\chapter{Preliminaries}
In this section we set up notation and terminology used in the thesis.
%
\section{Notation and Definitions}
\textbf{(Algorithms, Bitstrings and Circuits)}
We define a Boolean circuit as a directed acyclic graph with input vertices and vertices implementing logical functions \textit{and}, \textit{or}, and \textit{not}.
We denote Boolean circuits using capital letters from the Greek and English alphabet.

We define a \textit{probabilistic circuit} as a Boolean circuit $C_{m,n} : \{0,1\}^{m} \times \{0,1\}^{n} \rightarrow \{0,1\}^{*}$,
For input $x \in \{0,1\}^{m}$ we write to denote $C_{m,n}(x;r)$ where $r \in \{0,1\}^{n}$ is called auxiliary input.
If a probabilistic circuit does not take input $x$, we slightly abuse notation and write $C_{n}(r)$.
Similarly we use $\{C_n\}$ to denote a family of probabilist circuits that takes only auxiliary input.
We make sure that it is clear from the context that probabilist circuits with only auxiliary input
are not confused with circuits that do not take auxiliary input.

For a (probabilist) circuit $C$ we write $\mathit{Size}(C)$ to denote the total number of vertices of $C$.

We define a \textit{two phase circuit} $C := (C_1, C_2)$ as a circuit where in the first phase a circuit $C_1$ is used and in the second phase a circuit $C_2$.
If $C_1$ and $C_2$ are probabilistic circuits we write $C(\delta) := (C_1, C_2)(\delta)$ to denote that in both phases $C_1$ and $C_2$ take
as auxiliary input the same bitstring $\delta$.

It is well known \cite{Arora:2009:CCM:1540612} that a probabilistic polynomial time algorithm can be represented as a circuit of polynomial size.
Moreover, it can be computed in polynomial time and logarithmic space.
Therefore, whenever we state a theorem about circuits we can also apply it to polynomial time algorithms.

We write $\mathit{poly}(\alpha_1, \dots, \alpha_n)$ to denote a polynomial on variables $\alpha_1, \dots, \alpha_n$.
For an algorithm $A$ we write $\mathit{Time}(A)$ to denote the number of steps it takes to execute $A$.
Similarly, as for probabilistic circuits we write the randomness used by a probabilistic algorithm explicitly as a bitstring provided as an auxiliary input.

\textbf{(Probabilities and distributions)}
For a finite set $\cR$ we write $r \xleftarrow{\$} \cR$ to denote that $r$ is chosen from $\cR$ uniformly at random.
For $\delta \in \R : 0 \leq \delta \leq 1$ we write $\mu_{\delta}$ to denote the Bernoulli distribution where outcome $1$ occurs with
probability $\delta$ and $0$ with probability $1-\delta$.
Moreover, we use $\mu_{\delta}^k$ to denote the probability distribution over $k$-tuples
where each element of a $k$-tuple is drawn independently according to $\mu_{\delta}$.
Finally, let $u \leftarrow \mu_{\delta}^k$ denote that a $k$-tuple $u$ is chosen according to $\mu_{\delta}^k$.

Let $(\Omega, \cF, \Pr)$ be a probability space and $n \in \N$. We say that an event $E_n \in \cF$
happens \textit{almost surely} or with \textit{high probability} if $\Pr[E_n] \geq 1 - 2^{-n} \mathit{poly}(n)$.

\begin{todo}
  \textbf{TODO:} Non-negligible function and probability
\end{todo}

\textbf{(Interactive protocols)} We are often interested in situations where two probabilistic algorithms interact with each other according to some protocol.
We limit ourselves to the cases where algorithms interact by means of messages representable by bitstrings.
A protocol execution between two probabilistic algorithms $A$ and $B$ is denoted by $\langle A, B \rangle$.
The output of $A$ in a protocol execution is denoted by $\langle A, B \rangle_A$ and of $B$ by $\langle A, B \rangle_B$.
A sequence of all messages sent by $A$ and $B$ in the protocol execution is called a communication transcript and
is denoted by $\langle A, B \rangle_{\mathit{trans}}$.

\textbf{(Oracle algorithms)}
\begin{todo}
  \textbf{TODO:} set up notation
\end{todo}

\begin{definition}[Polynomial time sampleable distribution]
We say that a distribution is polynomial time sampleable if it can be approximated by polynomial time algorithm
up to exponential factor.
\end{definition}

\begin{definition}[Pairwise independent family of efficient hash functions]
Let $\cD$ and $\cR$ be finite sets and $\cH$ be a family of functions mapping values from $\cD$ to values in $\cR$.
We say that $\cH$ is an \textnormal{efficient family of pairwise independent hash functions}
if $\cH$ has the following properties.

\textbf{(Pairwise independent)} For $\forall x \neq y \in \mathcal{D}$ and $\forall \alpha, \beta \in \cR$, it holds
\begin{displaymath}
\underset{\hash \la0 \cH}{\Pr}[hash(x) = \alpha \mid hash(y) = \beta] = \frac{1}{|\cR|}.
\end{displaymath}

\textbf{(Polynomial time sampleable)} For every $\mathit{hash} \in \cH$ the function $\mathit{hash}$ is sampleable in time $\mathit{poly}(\log|\cD|, \log|\cR|)$.

\textbf{(Efficiently computable)}
For every $hash \in \cH$ there exists an algorithm running in time $\mathit{poly}(\log|\cD|, \log|\cR|)$ which
on input $x \in \cD$ outputs $y \in \cR$ such that $y = hash(x)$.
\end{definition}

We note that the pairwise independence property is equivalent to
\begin{displaymath}
\underset{\hash \la0 \cH}{\Pr}[hash(x) = \alpha \land hash(y) = \beta] = \frac{1}{|\cR|^2}.
\end{displaymath}
It is well know \cite{Carter:1977:UCH:800105.803400} that there exists a family of hash functions meeting the above criteria.

%%% Local Variables:
%%% mode: latex
%%% TeX-master: "master"
%%% End:


\chapter{Weakly Verifiable Cryptographic Primitives}
In this chapter we introduce the notion of weakly verifiable puzzles. In section \ref{def:dwvp} we provide a formal definitions that
is followed by a series of cryptographic primitives that are captured by this notion.
Finally, in Section \ref{st:previous_results} we give an overview of the earlier research in this area
that is primarily covered in \cite{canetti2004hardness}, \cite{Dodis:2009:SAI:1530441.1530450}, and \cite{DBLP:journals/corr/abs-1002-3534}.

\section{Weakly Verifiable Puzzles}
\begin{definition}[Dynamic weakly verifiable puzzle.]
  \label{def:dwvp}
  A dynamic weakly verifiable puzzle (DWVP) is defined by a family of probabilistic circuits $\{P_n\}$.
  A circuit belonging to $\{P_n\}$ is called a problem poser.
  A solver $C := (C_1, C_2)$ for $P_n$ is a probabilistic two phase circuit.
  We write $P_n(\pi)$ to denote the execution of $P_n$ with the randomness fixed to $\pi \in \{0,1\}^n$, and $(C_1,C_2)(\rho)$
  to denote the execution of both $C_1$ and $C_2$ with the randomness fixed to $\rho \in \{0,1\}^{*}$.

  In the first phase, the problem poser $P_n(\pi)$ and the solver $C_1(\rho)$ interact.
  As the result of the interaction $P_n(\pi)$ outputs a verification circuit $\Gamma_{V}$ and a hint circuit $\Gamma_{H}$.
  The circuit $C_1(\rho)$ produces no output.
  The circuit $\Gamma_{V}$ takes as input $q \in Q$, an answer $y \in \{0,1\}^*$,
  and outputs a bit. We say that an answer $(q,y)$ is a correct solution if and only if $\Gamma_V(q,y) = 1$.
  The circuit $\Gamma_H$ on input $q \in Q$ outputs a hint such that $\Gamma_V(q,\Gamma_H(q)) = 1$.

  In the second phase, $C_2$ takes as input $x := \langle P_n(\pi), C_1(\rho) \rangle_{\mathit{trans}}$,
  and has oracle access to $\Gamma_V$ and $\Gamma_H$.
  The execution of $C_2$ with the input $x$ and the randomness fixed to $\rho$
  is denoted by $C_2(x, \rho)$. The queries of $C_2$ to $\Gamma_V$ and $\Gamma_H$ are called verification queries and hint queries respectively.
  The circuit $C_2$ succeeds if and only if it makes a verification query $(q,y)$ such that $\Gamma_V(q,y) = 1$,
  and it has not previously asked for a hint query on $q$.
\end{definition}

\section{Examples}
\subsection{Message Authentication Codes}
\subsection{Public Key Encryption}
\subsection{Bit Commitments}
\subsection{CAPTACHs}

\section{Previous results}
\label{st:previous_results}
\subsection{Results of R.Canetti, S.Halevi, and M.Steiner}
\subsection{Results of Y.Dodis, R.Impagliazzo, R.Jaiswal, V.Kabanets}
\subsection{Results of T.Holenstein and G.Scheonebeck}
\section{Limitations of Security Amplification}

\input{interactive_proof/interactive_proof}

\appendix
\chapter{Appendix}
\section{Concentration Bounds}
\begin{lemma}[Chernoff Bounds]
For independent Bernoulli distributed random variables $X_1, \dotsc, X_n$ with $X := \sum_{i=1}^n X_i$
and $\Pr[X_i = 1] = p_i$ for all $ 1 \leq  i \leq n$ the following inequalities hold
\begin{gather}
\label{ineq:ch0}
\Pr[X \geq (1+\delta) \mathbb{E}[X]] \leq e^{- \mathbb{E}[X] \delta^2/3} \\
\label{ineq:ch1}
\Pr[X \leq (1-\delta) \mathbb{E}[X]] \leq e^{- \mathbb{E}[X] \delta^2/2},
% \label{ineq:ch2}
% \Pr[|X - \mathbb{E}[X]| \geq \delta \mathbb{E}[X]] \leq 2 e^{- \mathbb{E}[X] \delta^2 / 3},
\end{gather}
where $0 \leq \delta \leq 1$.

For independent and identically distributed Bernoulli random variables $X_1, \dotsc, X_n$ with $X := \sum_{i=1}^n X_i$
where $\Pr[X_i = 1] = p$ for some $p \in (0,1)$ and for all $ 1 \leq  i \leq n$ and
for any $\epsilon > 0$ we have
\begin{gather}
\label{ineq:ch3}
\Pr[X \geq (p + \varepsilon)n] < e^{-\frac{\epsilon^2n}{2}} \\
\label{ineq:ch2}
\Pr[| X - \mathbb{E}[X]| \geq \epsilon\mathbb{E}[X]] \leq 2e^{-\frac{\epsilon^2 \mathbb{E}[X]}{2}}.
\end{gather}
\end{lemma}
\vspace*{\fill}
\pagebreak

\section{The Proof of Lemma \ref{lemma:sec_amp_for_p_hash} Under the Simplifying Assumptions}
\label{st:proofSimAssm}
We prove Lemma \ref{lemma:sec_amp_for_p_hash} in the case where $\widetilde{Gen}$ does not find the randomness for which the surplus is large.
For the sake of simplicity we make the following assumptions
\begin{gather}
  \label{pr:always_d}
\underset{
  \mathclap{
  \substack{\pi, \rho \\ x := \langle P^{(1)}(\pi), D_1(\rho) \rangle_{\trans}}}}
{\Pr}[\widetilde{D}_2(x,\rho) \neq \bot] = 1 \\
  \label{pr:low_surpluses}
\forall \pi \in \{0,1\}^{n} : S_{\pi, b} \leq \Big(1 - \frac{1}{k}\Big)\epsilon.
\end{gather}
In \eqref{pr:always_d} we assume that $\widetilde{D}$ always outputs an answer, and in \eqref{pr:low_surpluses} that all surpluses are low.
In the complete proof of Lemma \ref{lemma:sec_amp_for_p_hash} these assumptions fail only slightly such that it is possible
to obtain the desired result. However, the calculations are fairly lengthy.
The following simplified proof is intended to give the intuition behind the full proof.
We have
\begin{align*}
\underset{
  \mathclap{
  \substack{\pi, \rho \\ x := \langle P^{(1)}(\pi), D_1(\rho) \rangle_{\trans}
    \\ (\Gamma_V, \Gamma_H) := \langle P^{(1)}(\pi), D_1 (\rho) \rangle_{P^{(1)}} }}}
{\Pr}[\Gamma_V&(\widetilde{D}_2(x,\rho)) = 1]
  \overset{\eqref{pr:always_d}}{=}
\underset{
  \mathclap{
  \substack{\pi, \rho \\ x := \langle P^{(1)}(\pi), D_1(\rho) \rangle_{\trans}
    \\ (\Gamma_V, \Gamma_H) := \langle P^{(1)}(\pi), D_1 (\rho) \rangle_{P^{(1)}} }}}
{\Pr}[\Gamma_V(\widetilde{D}_2(x,\rho)) = 1 \mid \widetilde{D}_2(x,\rho) \neq \bot] \\
  &\overset{\hphantom{a}(*)\hphantom{a}}{=} \underset{\pi^{(k)}}{\Pr}[c_1 = 1 \mid c \in \cG_1 \setminus \cG_0] \\
  &\overset{\eqref{eq:diff_s01_next}}{=} \underset{\pi^*}{\mathbb{E}}
  \left[\frac{\Pr_{\pi^{(k)}}[c_1 = 1 \mid c \in \cG_1 \setminus \cG_0] \big(\Pr_{\pi^{(k)}}[c \in \cG_1 \setminus \cG_0] - (S_{\pi^*,1} - S_{\pi^*,0})\big)}
  {\underset{u \leftarrow \mu_{\delta}^{k}}{\Pr}[u \in \cG_1 \setminus \cG_0]}\right] \\
  % &\overset{\hphantom{a(*)}\hphantom{a}}{\geq} \underset{\pi^*}{\mathbb{E}}
  % \left[\frac{\Pr[c_1 = 1 \land c \in \cG_1 \setminus \cG_0]  - (S_{\pi^*,1} - S_{\pi^*,0})}
  % {\underset{u \leftarrow \mu_{\delta}^{k}}{\Pr}[u \in \cG_1 \setminus \cG_0]}\right] \\
  &\overset{\eqref{pr:low_surpluses}}{\geq} \underset{\pi^*}{\mathbb{E}}
  \left[\frac{\Pr_{\pi^{(k)}}[g(c)=1] - \Pr_{\pi^{(k)}}[c \in \cG_0]  - (1 - \frac{1}{k})\epsilon + S_{\pi^*,0}}
  {\underset{u \leftarrow \mu_{\delta}^{k}}{\Pr}[u \in \cG_1 \setminus \cG_0]}\right] \\
  &\overset{\eqref{eq:s_pi_b}}{=} \underset{\pi^*}{\mathbb{E}}
  \left[\frac{\Pr_{\pi^{(k)}}[g(c)=1] - \Pr_{u \leftarrow \mu_{\delta}^{k}}[u \in \cG_0] - S_{\pi^*,0}  - (1 - \frac{1}{k})\epsilon + S_{\pi^*,0}}
  {\underset{u \leftarrow \mu_{\delta}^{k}}{\Pr}[u \in \cG_1 \setminus \cG_0]}\right]\\
  &\overset{\substack{\eqref{eq:gu_rel}\\\eqref{eq:lemma_assum}}}{\geq} \underset{\pi^*}{\mathbb{E}}
  \left[\frac{\delta \Pr_{u \leftarrow \mu_{\delta}^{k}}[u \in \cG_1 \setminus \cG_0] + \epsilon  - (1 - \frac{1}{k})\epsilon}
  {\underset{u \leftarrow \mu_{\delta}^{k}}{\Pr}[u \in \cG_1 \setminus \cG_0]}\right] \geq \delta + \frac{\epsilon}{k},
  %
  % &\overset{\hphantom{\eqref{eq:diff_s01_next}}}{=} \frac{\Pr_{\pi^{(k)}}[c_1 = 1 \land c \in \cG_1 \setminus \cG_0]}{\Pr_{\pi^{(k)}}[c \in \cG_1 \setminus \cG_0]} \\
  % &\overset{\eqref{eq:diff_s01_next}}{=} \underset{\pi^*}{\mathbb{E}}\left[\frac{\Pr_{\pi^{(k-1)}}[c_1^* = 1 \land c \in \cG_1 \setminus \cG_0]
  %   \big(\Pr_{\pi^{(k-1)}}[c \in \cG_0 \setminus \cG_1] - (S_{\pi^*,1} - S_{\pi^*,0})\big)}
  % {\underset{\pi^{(k-1)}}{\Pr}[c \in \cG_0 \setminus \cG_1] \underset{u \leftarrow \mu_{\delta}^{k}}{\Pr}[u \in \cG_1 \setminus \cG_0]}\right]  \\
  % &\overset{\hphantom{\eqref{eq:diff_s01_next}}}{\geq} \underset{\pi^*}{\mathbb{E}}\left[\frac{\Pr_{\pi^{(k-1)}}[c_1^* = 1 \land c \in \cG_1 \setminus \cG_0]
  %   \big(\Pr_{\pi^{(k-1)}}[c \in \cG_0 \setminus \cG_1] - (1 - \frac{1}{k})\epsilon\big)}
  % {\underset{\pi^{(k-1)}}{\Pr}[c \in \cG_0 \setminus \cG_1] \underset{u \leftarrow \mu_{\delta}^{k}}{\Pr}[u \in \cG_1 \setminus \cG_0]}\right]  \\
  % &\overset{\hphantom{\eqref{eq:diff_s01_next}}}{\geq} \frac{\Pr_{\pi^{(k)}}[c_1 = 1 \land c \in \cG_1 \setminus \cG_0] -
  %   (1 - \frac{1}{k})\epsilon}{\underset{u \leftarrow \mu_{\delta}^{k}}{\Pr}[u \in \cG_1 \setminus \cG_0]} \\
  % &\overset{\eqref{eq:gu_rel}}{\geq} \frac{\delta \underset{}{\Pr}[u \in \cG_1 \setminus \cG_0] + \epsilon - (1-\frac{1}{k})\epsilon}{\underset{}{\Pr}[u \in \cG_1 \setminus \cG_0]} \\
  % &\overset{\hphantom{\eqref{eq:diff_s01_next}}}{\geq} \delta +  \frac{\epsilon}{k},
\end{align*}
where in $(*)$ we use the facts that $\widetilde{D}_2(x,\rho) \neq \bot$ if and only if $\widetilde{D}_2$ finds $c \in \cG_1 \setminus \cG_0$
and conditioned on $\widetilde{D}_2(x,\rho) \neq \bot$  we have that $\Gamma_V(\widetilde{D}_2(x,r)) = 1$ if and only if $c_1 = 1$.

\backmatter

\bibliographystyle{alpha}
\bibliography{refs}

\end{document}
