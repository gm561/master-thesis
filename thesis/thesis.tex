%% (Master) Thesis template
% Template version used: v1.4
%
% Largely adapted from Adrian Nievergelt's template for the ADPS
% (lecture notes) project.

%% We use the memoir class because it offers a many easy to use features.
\documentclass[11pt,a4paper,titlepage]{memoir}

%% Packages
%% ========

%% LaTeX Font encoding -- DO NOT CHANGE
\usepackage[OT1]{fontenc}

%% Babel provides support for languages.  'english' uses British
%% English hyphenation and text snippets like "Figure" and
%% "Theorem". Use the option 'ngerman' if your document is in German.
%% Use 'american' for American English.  Note that if you change this,
%% the next LaTeX run may show spurious errors.  Simply run it again.
%% If they persist, remove the .aux file and try again.
\usepackage[english]{babel}

%% Input encoding 'utf8'. In some cases you might need 'utf8x' for
%% extra symbols. Not all editors, especially on Windows, are UTF-8
%% capable, so you may want to use 'latin1' instead.
\usepackage[utf8]{inputenc}

%% This changes default fonts for both text and math mode to use Herman Zapfs
%% excellent Palatino font.  Do not change this.
%\usepackage[sc]{mathpazo}

%% The AMS-LaTeX extensions for mathematical typesetting.  Do not
%% remove.
\usepackage{amsmath,amssymb,amsfonts,mathrsfs}

%% NTheorem is a reimplementation of the AMS Theorem package. This
%% will allow us to typeset theorems like examples, proofs and
%% similar.  Do not remove.
%% NOTE: Must be loaded AFTER amsmath, or the \qed placement will
%% break
\usepackage[amsmath,thmmarks]{ntheorem}

%% LaTeX' own graphics handling
\usepackage{graphicx}

%% We unfortunately need this for the Rules chapter.  Remove it
%% afterwards; or at least NEVER use its underlining features.
\usepackage{soul}

%% Some more packages that you may want to use.  Have a look at the
%% file, and consult the package docs for each.
%% See the TeXed file for more explanations

%% [OPT] Multi-rowed cells in tabulars
%\usepackage{multirow}

%% [REC] Intelligent cross reference package. This allows for nice
%% combined references that include the reference and a hint to where
%% to look for it.
\usepackage{varioref}

%% [OPT] Easily changeable quotes with \enquote{Text}
%\usepackage[german=swiss]{csquotes}

%% [REC] Format dates and time depending on locale
\usepackage{datetime}

%% [OPT] Provides a \cancel{} command to stroke through mathematics.
%\usepackage{cancel}

%% [NEED] This allows for additional typesetting tools in mathmode.
%% See its excellent documentation.
\usepackage{mathtools}

%% [ADV] Conditional commands
%\usepackage{ifthen}

%% [OPT] Manual large braces or other delimiters.
%\usepackage{bigdelim, bigstrut}

%% [REC] Alternate vector arrows. Use the command \vv{} to get scaled
%% vector arrows.
%\usepackage[h]{esvect}

%% [NEED] Some extensions to tabulars and array environments.
\usepackage{array}

%% [OPT] Postscript support via pstricks graphics package. Very
%% diverse applications.
%\usepackage{pstricks,pst-all}

%% [?] This seems to allow us to define some additional counters.
%\usepackage{etex}

%% [ADV] XY-Pic to typeset some matrix-style graphics
%\usepackage[all]{xy}

%% [OPT] This is needed to generate an index at the end of the
%% document.
%\usepackage{makeidx}

%% [OPT] Fancy package for source code listings.  The template text
%% needs it for some LaTeX snippets; remove/adapt the \lstset when you
%% remove the template content.
\usepackage{listings}
\lstset{language=TeX,basicstyle={\normalfont\ttfamily}}

%% [REC] Fancy character protrusion.  Must be loaded after all fonts.
\usepackage[activate]{pdfcprot}

%% [REC] Nicer tables.  Read the excellent documentation.
\usepackage{booktabs}

%%pseudocode and algorithms
\usepackage{algpseudocode}
\usepackage{algorithm}

%% define comments in single line
\algnewcommand{\LineComment}[1]{\State \(\triangleright\) #1}

%%placing in the right place
\usepackage{float}

\usepackage{caption}

\DeclareCaptionFormat{algor}{
  \hrulefill\par\offinterlineskip\vskip1pt
    \textbf{#1#2}#3\offinterlineskip\hrulefill}
\DeclareCaptionStyle{algori}{singlelinecheck=off,format=algor,labelsep=space}
\captionsetup[algorithm]{style=algori}

\algnewcommand\algorithmicinput{\textbf{Input:}}
\algnewcommand\Input{\item[\algorithmicinput]}

\algnewcommand\algorithmicauxinput{\textbf{Auxiliary input:}}
\algnewcommand\AuxInput{\item[\algorithmicauxinput]}

%indention in algorithms
\newcommand{\pushcode}[1][1]{\hskip\dimexpr#1\algorithmicindent\relax}

% hyphen in math mode
\def\hyph{\text{-}}



%% Our layout configuration.  DO NOT CHANGE.
%% Memoir layout setup

%% NOTE: You are strongly advised not to change any of them unless you
%% know what you are doing.  These settings strongly interact in the
%% final look of the document.

% Dependencies
\usepackage{ETHlogo}

% Turn extra space before chapter headings off.
\setlength{\beforechapskip}{0pt}

\nonzeroparskip
\parindent=0pt
\defaultlists

% Chapter style redefinition
\makeatletter

\if@twoside
  \pagestyle{Ruled}
  \copypagestyle{chapter}{Ruled}
\else
  \pagestyle{ruled}
  \copypagestyle{chapter}{ruled}
\fi
\makeoddhead{chapter}{}{}{}
\makeevenhead{chapter}{}{}{}
\makeheadrule{chapter}{\textwidth}{0pt}
\copypagestyle{abstract}{empty}

\makechapterstyle{bianchimod}{%
  \chapterstyle{default}
  \renewcommand*{\chapnamefont}{\normalfont\Large\sffamily}
  \renewcommand*{\chapnumfont}{\normalfont\Large\sffamily}
  \renewcommand*{\printchaptername}{%
    \chapnamefont\centering\@chapapp}
  \renewcommand*{\printchapternum}{\chapnumfont {\thechapter}}
  \renewcommand*{\chaptitlefont}{\normalfont\huge\sffamily}
  \renewcommand*{\printchaptertitle}[1]{%
    \hrule\vskip\onelineskip \centering \chaptitlefont\textbf{\vphantom{gyM}##1}\par}
  \renewcommand*{\afterchaptertitle}{\vskip\onelineskip \hrule\vskip
    \afterchapskip}
  \renewcommand*{\printchapternonum}{%
    \vphantom{\chapnumfont {9}}\afterchapternum}}

% Use the newly defined style
\chapterstyle{bianchimod}

\setsecheadstyle{\Large\bfseries\sffamily}
\setsubsecheadstyle{\large\bfseries\sffamily}
\setsubsubsecheadstyle{\bfseries\sffamily}
\setparaheadstyle{\normalsize\bfseries\sffamily}
\setsubparaheadstyle{\normalsize\itshape\sffamily}
\setsubparaindent{0pt}

% Set captions to a more separated style for clearness
\captionnamefont{\sffamily\bfseries\footnotesize}
\captiontitlefont{\sffamily\footnotesize}
\setlength{\intextsep}{16pt}
\setlength{\belowcaptionskip}{1pt}

% Set section and TOC numbering depth to subsection
\setsecnumdepth{subsection}
\settocdepth{subsection}

%% Titlepage adjustments
\pretitle{\vspace{0pt plus 0.7fill}\begin{center}\HUGE\sffamily\bfseries}
\posttitle{\end{center}\par}
\preauthor{\par\begin{center}\let\and\\\Large\sffamily}
\postauthor{\end{center}}
\predate{\par\begin{center}\Large\sffamily}
\postdate{\end{center}}

\def\@advisors{}
\newcommand{\advisors}[1]{\def\@advisors{#1}}
\def\@department{}
\newcommand{\department}[1]{\def\@department{#1}}
\def\@thesistype{}
\newcommand{\thesistype}[1]{\def\@thesistype{#1}}

\renewcommand{\maketitlehooka}{\noindent\ETHlogo[2in]}

\renewcommand{\maketitlehookb}{\vspace{1in}%
  \par\begin{center}\Large\sffamily\@thesistype\end{center}}

\renewcommand{\maketitlehookd}{%
  \vfill\par
  \begin{flushright}
    \sffamily
    \@advisors\par
    \@department, ETH Z\"urich
  \end{flushright}
}

\checkandfixthelayout

\setlength{\droptitle}{-48pt}

\makeatother

% This defines how theorems should look. Best leave as is.
\theoremstyle{plain}
\setlength\theorempostskipamount{0pt}

\usepackage[framemethod=tikz]{mdframed}
\newdimen\linenumbersep

\newcommand{\linenumber}[1]{%
  \linenumbersep 4pt%
  \advance\linenumbersep\mdflength{innerleftmargin}%
  \advance\linenumbersep\mdflength{innerlinewidth}%
  \advance\linenumbersep\mdflength{middlelinewidth}%
  \advance\linenumbersep\mdflength{outerlinewidth}%
  \advance\linenumbersep\mdflength{linewidth}%
  \makebox[0pt][r]{{\rmfamily\tiny#1}\hspace*{\linenumbersep}}}

\newenvironment{codeblock}%
   {\medskip\begin{mdframed}\setlength{\parindent}{0cm}}%
   {\end{mdframed}\medskip}
\newcommand{\Ind}{\mbox{}}
\newcommand{\IndI}{\mbox\qquad}
\newcommand{\IndII}{\mbox\qquad\qquad}
\newcommand{\IndIII}{\mbox\qquad\qquad\qquad}
\newcommand{\IndIIII}{\mbox\qquad\qquad\qquad\qquad}

\newenvironment{todo}%
   {\medskip\begin{mdframed}\setlength{\parindent}{0cm}}%
   {\end{mdframed}\medskip}


%%% Local Variables:
%%% mode: latex
%%% TeX-master: "thesis"
%%% End:


%% Theorem environments.  You will have to adapt this for a German
%% thesis.
%% Theorem-like environments

%% This can be changed according to language. You can comment out the ones you
%% don't need.

\numberwithin{equation}{chapter}

% declare theorem style
\declaretheoremstyle[
spaceabove=6pt, spacebelow=6pt,
headfont=\normalfont\bfseries,
notefont=\mdseries, notebraces={(}{)},
bodyfont=\normalfont\itshape,
postheadspace=1em,
%qed=\qedsymbol
]{thm_sty}

% declare definition style
\declaretheoremstyle[
spaceabove=6pt, spacebelow=6pt,
headfont=\normalfont\bfseries,
notefont=\mdseries, notebraces={(}{)},
bodyfont=\normalfont,
postheadspace=1em,
qed=\ensuremath{\lozenge}
]{def_sty}


%% English variants
\declaretheorem[style=thm_sty, name=Theorem, numberwithin=chapter]{theorem}
\declaretheorem[style=thm_sty, name=Observation, sibling=theorem]{observation}
\declaretheorem[style=thm_sty, name=Lemma, sibling=theorem]{lemma}
\declaretheorem[style=def_sty, name=Definition, sibling=theorem]{definition}
%\declaretheorem[style=def_sty, name=Proof]{proof}

%\newtheorem{theorem}{Theorem}[chapter]
% \newtheorem{example}[theorem]{Example}
% \newtheorem{remark}[theorem]{Remark}
% \newtheorem{corollary}[theorem]{Corollary}
% \newtheorem{lemma}[theorem]{Lemma}
% \newtheorem{proposition}[theorem]{Proposition}
% \newtheorem{observation}[theorem]{Observation}

%\theoremstyle{definition}
%\theorembodyfont{\normalfont}
%% end def with blacksquare symbol
%\theoremsymbol{\ensuremath{\lozenge}}
%\newtheorem{definition}[theorem]{Definition}

%%Proof environment with a small square as a "qed" symbol
%\theoremstyle{nonumberplain}
%\theorembodyfont{\normalfont}
%\theoremsymbol{\ensuremath{\square}}
%\theoremseparator{.}
%\newtheorem{proof}{Proof}



%% Helpful macros.
%% Custom commands
%% ===============

%% Special characters for number sets, e.g. real or complex numbers.
\newcommand{\C}{\mathbb{C}}
\newcommand{\D}{\mathbb{D}}
\newcommand{\K}{\mathbb{K}}
\newcommand{\N}{\mathbb{N}}
\newcommand{\M}{\mathbb{M}}
\newcommand{\Q}{\mathbb{Q}}
\newcommand{\R}{\mathbb{R}}
\newcommand{\T}{\mathbb{T}}
\newcommand{\X}{\mathbb{X}}
\newcommand{\Z}{\mathbb{Z}}


\newcommand{\cA}{\mathcal{A}}
\newcommand{\cB}{\mathcal{B}}
\newcommand{\cX}{\mathcal{X}}
\newcommand{\cH}{\mathcal{H}}
\newcommand{\cW}{\mathcal{W}}
\newcommand{\cG}{\mathcal{G}}
\newcommand{\cP}{\mathcal{P}}
\newcommand{\cR}{\mathcal{R}}
\newcommand{\cD}{\mathcal{D}}
\newcommand{\cF}{\mathcal{F}}
\newcommand{\cM}{\mathcal{M}}
\newcommand{\cK}{\mathcal{K}}
\newcommand{\cT}{\mathcal{T}}
\newcommand{\cS}{\mathcal{S}}
\newcommand{\cQ}{\mathcal{Q}}
\newcommand{\cV}{\mathcal{V}}
\newcommand{\cI}{\mathcal{I}}

%define our own code commands
%use capital latters as most of these commands is already defined
\renewcommand{\For}{\textbf{for }}
\renewcommand{\If}{\textbf{if }}
\renewcommand{\Else}{\textbf{else }}
\renewcommand{\Return}{\textbf{return }}
\newcommand{\Then}{\textbf{then }}
\newcommand{\Do}{\textbf{do: }}
\renewcommand{\And}{\textbf{and }}
\newcommand{\Or}{\textbf{or }}
\newcommand{\Run}{\textbf{run }}
\newcommand{\To}{\textbf{to }}
\renewcommand{\Repeat}{\textbf{Repeat }}

%% Fixed/scaling delimiter examples (see mathtools documentation)
\DeclarePairedDelimiter\abs{\lvert}{\rvert}
\DeclarePairedDelimiter\norm{\lVert}{\rVert}

%% Use the alternative epsilon per default and define the old one as \oldepsilon
\let\oldepsilon\epsilon
\renewcommand{\epsilon}{\ensuremath\varepsilon}

%% Also set the alternate phi as default.
\let\oldphi\phi
\renewcommand{\phi}{\ensuremath{\varphi}}

% New command that introduces a tab
\newcommand{\itab}[1]{\hspace{0em}\rlap{#1}}
\newcommand{\tab}[1]{\hspace{.2\textwidth}\rlap{#1}}

\DeclareMathOperator{\la0}{\leftarrow}
\DeclareMathOperator{\ra0}{\rightarrow}

\DeclareMathOperator{\hash}{\mathit{hash}}
\DeclareMathOperator{\CanonicalSuccess}{\mathit{CanonicalSuccess}}
\DeclareMathOperator{\Success}{\mathit{Success}}
\DeclareMathOperator{\poly}{\mathit{poly}}
\DeclareMathOperator{\p}{\mathit{p}}
\DeclareMathOperator{\Time}{\mathit{Time}}
\DeclareMathOperator{\Gen}{\text{Gen}}
\DeclareMathOperator{\FindHash}{\text{FindHash}}
\DeclareMathOperator{\Bad}{\mathit{Bad}}
\DeclareMathOperator{\Good}{\mathit{Good}}
\DeclareMathOperator{\trans}{\mathit{trans}}
\DeclareMathOperator{\Canonical}{\mathit{Canonical}}


%the DWVP for the permutation
\DeclareMathOperator{\PiDWVP}{\Pi_{DWVP}}
%the DWPV for the k-wise product of permutations
\DeclareMathOperator{\kPiDWVP}{\Pi_{DWVP}^{(k)}}

%% Make document internal hyperlinks wherever possible. (TOC, references)
%% This MUST be loaded after varioref, which is loaded in 'extrapackages'
%% above.  We just load it last to be safe.
\usepackage[linkcolor=black,colorlinks=true,citecolor=black,filecolor=black]{hyperref}


%% Document information
%% ====================

\title{On amplification of weakly verifiable dynamic cryptographic primitives}
\author{Grzegorz Makosa}
\thesistype{Master Thesis}
\advisors{Advisors: Prof. Dr. Thomas Holenstein, Dr. Robin Künzler}
\department{Department of Computer Science}
\date{April 8, 2014}

\begin{document}

\frontmatter

% \begin{titlingpage}
%   \calccentering{\unitlength}
%   \begin{adjustwidth*}{\unitlength-24pt}{-\unitlength-24pt}
%     \maketitle
%   \end{adjustwidth*}
% \end{titlingpage}
%\begin{abstract}
% A good abstract explains in one line why the paper is important.
% It then goes on to give a summary of your major results, preferably couched in numbers with error limits.
% The final sentences explain the major implications of your work. A good abstract is concise, readable, and quantitative.
% Length should be ~ 1-2 paragraphs, approx. 400 words.
%
% Absrtracts generally do not have citations.
% Information in title should not be repeated.
% Be explicit.
% Use numbers where appropriate.
% Answers to these questions should be found in the abstract:
% What did you do?
% Why did you do it? What question were you trying to answer?
% How did you do it? State methods.
% What did you learn? State major results.
% Why does it matter? Point out at least one significant implication.

% 1) new hardness proof
% 2) it is possible to start with a weak crypto primitive and obtain strong one.
% 3) a natural puzzle fixing technique similar to the one used in the classical proofs of weak one-way function implying strong ones
% 4) uses an arbitrary monotone function to decide the result
% 5) compare with previous works
% 6)

We give a new proof of hardness amplification for Dynamic Weakly Verifiable Puzzles which is more general than the previous ones.

Our proof applies to the various cryptographic primitives like hash function,
one way functions, signature schemes and bit commitment protocols.

\end{abstract}


\cleartorecto
\tableofcontents
\mainmatter

\chapter{Introduction}
% use term \textit{puzzle} to denote somewhat-hard automatically-generated computational a problems
\section{Security Amplification Theorems}
% Introduction to security amplification theorems and hardness implication statements.
% Example of classical results. Problems captured by weakly verifiable puzzles.
% Contribution of this thesis.
\section{Weakly verifiable puzzles}
\section{Contribution of the Thesis}
\section{Organization of the Thesis}
% Overview of the content of the succeeding chapters.
%
\chapter{Preliminaries}
In this section we set up notation and terminology used in the thesis.
%
\section{Notation and Definitions}
\textbf{(Algorithms, Bitstrings and Circuits)}
We define a Boolean circuit as a directed acyclic graph with input vertices and vertices implementing logical functions \textit{and}, \textit{or}, and \textit{not}.
We denote Boolean circuits using capital letters from the Greek and English alphabet.

We define a \textit{probabilistic circuit} as a Boolean circuit $C_{m,n} : \{0,1\}^{m} \times \{0,1\}^{n} \rightarrow \{0,1\}^{*}$,
For input $x \in \{0,1\}^{m}$ we write to denote $C_{m,n}(x;r)$ where $r \in \{0,1\}^{n}$ is called auxiliary input.
If a probabilistic circuit does not take input $x$, we slightly abuse notation and write $C_{n}(r)$.
Similarly we use $\{C_n\}$ to denote a family of probabilist circuits that takes only auxiliary input.
We make sure that it is clear from the context that probabilist circuits with only auxiliary input
are not confused with circuits that do not take auxiliary input.

For a (probabilist) circuit $C$ we write $\mathit{Size}(C)$ to denote the total number of vertices of $C$.

We define a \textit{two phase circuit} $C := (C_1, C_2)$ as a circuit where in the first phase a circuit $C_1$ is used and in the second phase a circuit $C_2$.
If $C_1$ and $C_2$ are probabilistic circuits we write $C(\delta) := (C_1, C_2)(\delta)$ to denote that in both phases $C_1$ and $C_2$ take
as auxiliary input the same bitstring $\delta$.

It is well known \cite{Arora:2009:CCM:1540612} that a probabilistic polynomial time algorithm can be represented as a circuit of polynomial size.
Moreover, it can be computed in polynomial time and logarithmic space.
Therefore, whenever we state a theorem about circuits we can also apply it to polynomial time algorithms.

We write $\mathit{poly}(\alpha_1, \dots, \alpha_n)$ to denote a polynomial on variables $\alpha_1, \dots, \alpha_n$.
For an algorithm $A$ we write $\mathit{Time}(A)$ to denote the number of steps it takes to execute $A$.
Similarly, as for probabilistic circuits we write the randomness used by a probabilistic algorithm explicitly as a bitstring provided as an auxiliary input.

\textbf{(Probabilities and distributions)}
For a finite set $\cR$ we write $r \xleftarrow{\$} \cR$ to denote that $r$ is chosen from $\cR$ uniformly at random.
For $\delta \in \R : 0 \leq \delta \leq 1$ we write $\mu_{\delta}$ to denote the Bernoulli distribution where outcome $1$ occurs with
probability $\delta$ and $0$ with probability $1-\delta$.
Moreover, we use $\mu_{\delta}^k$ to denote the probability distribution over $k$-tuples
where each element of a $k$-tuple is drawn independently according to $\mu_{\delta}$.
Finally, let $u \leftarrow \mu_{\delta}^k$ denote that a $k$-tuple $u$ is chosen according to $\mu_{\delta}^k$.

Let $(\Omega, \cF, \Pr)$ be a probability space and $n \in \N$. We say that an event $E_n \in \cF$
happens \textit{almost surely} or with \textit{high probability} if $\Pr[E_n] \geq 1 - 2^{-n} \mathit{poly}(n)$.

\begin{todo}
  \textbf{TODO:} Non-negligible function and probability
\end{todo}

\textbf{(Interactive protocols)} We are often interested in situations where two probabilistic algorithms interact with each other according to some protocol.
We limit ourselves to the cases where algorithms interact by means of messages representable by bitstrings.
A protocol execution between two probabilistic algorithms $A$ and $B$ is denoted by $\langle A, B \rangle$.
The output of $A$ in a protocol execution is denoted by $\langle A, B \rangle_A$ and of $B$ by $\langle A, B \rangle_B$.
A sequence of all messages sent by $A$ and $B$ in the protocol execution is called a communication transcript and
is denoted by $\langle A, B \rangle_{\mathit{trans}}$.

\textbf{(Oracle algorithms)}
\begin{todo}
  \textbf{TODO:} set up notation
\end{todo}

\begin{definition}[Polynomial time sampleable distribution]
We say that a distribution is polynomial time sampleable if it can be approximated by polynomial time algorithm
up to exponential factor.
\end{definition}

\begin{definition}[Pairwise independent family of efficient hash functions]
Let $\cD$ and $\cR$ be finite sets and $\cH$ be a family of functions mapping values from $\cD$ to values in $\cR$.
We say that $\cH$ is an \textnormal{efficient family of pairwise independent hash functions}
if $\cH$ has the following properties.

\textbf{(Pairwise independent)} For $\forall x \neq y \in \mathcal{D}$ and $\forall \alpha, \beta \in \cR$, it holds
\begin{displaymath}
\underset{\hash \la0 \cH}{\Pr}[hash(x) = \alpha \mid hash(y) = \beta] = \frac{1}{|\cR|}.
\end{displaymath}

\textbf{(Polynomial time sampleable)} For every $\mathit{hash} \in \cH$ the function $\mathit{hash}$ is sampleable in time $\mathit{poly}(\log|\cD|, \log|\cR|)$.

\textbf{(Efficiently computable)}
For every $hash \in \cH$ there exists an algorithm running in time $\mathit{poly}(\log|\cD|, \log|\cR|)$ which
on input $x \in \cD$ outputs $y \in \cR$ such that $y = hash(x)$.
\end{definition}

We note that the pairwise independence property is equivalent to
\begin{displaymath}
\underset{\hash \la0 \cH}{\Pr}[hash(x) = \alpha \land hash(y) = \beta] = \frac{1}{|\cR|^2}.
\end{displaymath}
It is well know \cite{Carter:1977:UCH:800105.803400} that there exists a family of hash functions meeting the above criteria.

%%% Local Variables:
%%% mode: latex
%%% TeX-master: "master"
%%% End:


\chapter{Weakly Verifiable Cryptographic Primitives}
\label{ch:intro_weakly}
This chapter gives an overview of weakly verifiable cryptograph primitives.
We start by formulating a definition of a \textit{dynamic interactive weakly verifiable puzzle} in Section \ref{section:wvp}.
To provide the Reader more intuition in Section \ref{section:wvp_examples} we describe a~series of well known cryptographic primitives
that are weakly verifiable. Section \ref{st:previous_results} is devoted to the previous research concerning different types of weakly verifiable puzzles.
%
\section{The Definition}
\label{section:wvp}
Let us combine definitions of puzzles introduced in \cite{dodis2009security} and \cite{holenstein2011general}
and define a \textit{dynamic interactive weakly verifiable puzzle} as follows.
\begin{definition}[Dynamic interactive weakly verifiable puzzle]
  \label{def:dwvp}%
  A~\textit{dynamic interactive weakly verifiable puzzle (DIWVP)} is defined by a family of probabilistic circuits $\{P_n\}_{n \in \N}$.
  A circuit belonging to $\{P_n\}_{n \in \N}$ is called a \textit{problem poser}.
  A \textit{solver} $C := (C_1, C_2)$ is a probabilistic two-phase circuit.
  We write $P_n(\pi)$ to denote the execution of $P_n$ with the randomness fixed to $\pi \in \{0,1\}^n$ and $C(\rho) := (C_1,C_2)(\rho)$
  to denote the execution of both $C_1$ and $C_2$ with the randomness fixed to $\rho \in \{0,1\}^{*}$.

  In the first phase, the problem poser $P_n(\pi)$ and the solver $C_1(\rho)$ interact.
  As a result of the interaction $P_n(\pi)$ outputs a \textit{verification circuit} $\Gamma_{V}$ and a~probabilistic \textit{hint circuit} $\Gamma_{H}$.
  The circuit $C_1(\rho)$ produces no output. The circuit $\Gamma_{V}$ takes as input $q \in \cQ$ (for some set $\cQ$ of indices),
  $y \in \{0,1\}^*$ and outputs a bit. We say that an answer $(q,y)$ is a \textit{correct solution} if and only if $\Gamma_V(q,y) = 1$.
  The circuit $\Gamma_H$ on input $q \in \cQ$ outputs a hint such that $\Gamma_V(q,\Gamma_H(q))~=~1$.

  In the second phase, $C_2$ takes as input $x := \langle P_n(\pi), C_1(\rho) \rangle_{\trans}$
  and has oracle access to $\Gamma_V$ and $\Gamma_H$.
  The execution of $C_2$ with the input $x$ and the randomness fixed to $\rho$
  is denoted by $C_2(x; \rho)$. The queries of $C_2$ to $\Gamma_V$ and $\Gamma_H$ are called \textit{verification queries} and \textit{hint queries}, respectively.
  We say that the circuit $C_2$ \textit{succeeds} if and only if it makes a verification query $(q,y)$ such that $\Gamma_V(q,y)~=~1$,
  and it has not previously asked for a hint query on this $q$.
\end{definition}

If it is clear from the context, we omit the subscript $n$ and write $P$ instead of~$P_n$.

A verification query $(q,y)$ of $C$ for which a hint query on this $q$ has been asked before cannot be a verification query for which $C$ succeeds.
Therefore, without loss of generality throughout this chapter, we make the assumption that $C$ does not ask verification queries on $q$
for which a hint query has been asked before. Moreover, we assume that once $C$ asked a verification query that succeeds,
it does not ask any further hint or verification queries.

There is no loss of generality in assuming that the problem poser and the solver are defined by probabilistic circuits.
Definition \ref{def:dwvp} embraces also the case where the problem poser and the solver are probabilistic polynomial time algorithms.
We use the well known fact \cite{LectureNotesCT} that a probabilistic polynomial time algorithm can be transformed into an equivalent
family of probabilistic Boolean circuits of the polynomial size\footnote{Theorem 6.10 from \cite{LectureNotesCT} is stated for probabilistic, polynomial time,
oracle algorithms with a single bit of output, but it can be adopted to the case where an output is longer than a single bit.}.

We use the term \textit{weakly verifiable} to emphasize that there is no easy way
for the solver to check the correctness of a solution except for asking a verification query.
We call a weakly verifiable puzzle \textit{dynamic} if the number of hint queries is greater than zero.
Furthermore, we say that a weakly verifiable puzzle is \textit{interactive} if in the first
phase the number of messages exchanged between the problem poser and the solver is greater than one.
Finally, we say that a weakly verifiable puzzle is \textit{non-dynamic} if $|\cQ| = 1$
and \textit{non-interactive} if the number of messages sent in the first phase is at most one.

Definition \ref{def:dwvp} generalizes and combines the previous approaches that study
\textit{weakly verifiable puzzles} \cite{canetti2005hardness},
\textit{dynamic weakly verifiable puzzles} \cite{dodis2009security}, and \textit{interactive weakly verifiable puzzles} \cite{holenstein2011general}.

%
\section{The Hardness Amplification Theorem}
\label{section:hardness_amplification_diwvp}
In this section we define the $k$-wise direct product of puzzles and state the hardness amplification theorem for dynamic interactive weakly verifiable puzzles.

\begin{definition}[$k$-wise direct-product of DIWVPs]
  \label{def:k_wise_direct_product}
  Let $g: \{0,1\}^{k}\!\rightarrow\!\{0,1\}$ be a binary monotone function and $P_n^{(1)}$ a problem poser as in Definition~\ref{def:dwvp}.
  The \textit{$k$-wise direct product of $P_n^{(1)}$} is a dynamic interactive weakly verifiable puzzle defined by a circuit $P_{kn}^{(g)}$.
  We write $P_{kn}^{(g)}(\pi^{(k)})$ to denote the execution of $P_{kn}^{(g)}$ with the randomness fixed to $\pi^{(k)} := (\pi_1, \dots, \pi_k)$
  where $\pi_i \in \{0,1\}^n$ for all $1 \leq i \leq n$. Let $(C_1, C_2)(\rho)$ be a probabilistic two-phase circuit called a \textit{solver}.
  In the first phase, the problem poser $P_{kn}^{(g)}(\pi^{(k)})$ sequentially interacts in $k$ rounds with $C_1(\rho)$.
  In the $i$-th round $C_1(\rho)$ interacts with $P_n^{(1)}(\pi_i)$, and as a result $P_{n}^{(1)}(\pi_i)$ generates circuits $\Gamma_V^i, \Gamma_H^i$.
  Finally, after $k$ rounds $P_{kn}^{(g)}(\pi^{(k)})$ outputs a verification circuit
\begin{align*}
  \Gamma_V^{(g)} (q, y_1, \dots, y_k) := g(\Gamma_V^{1}(q, y_1), \dotsc, \Gamma_V^{k}(q, y_k))
\end{align*}
and a hint circuit
\begin{align*}
  \Gamma_H^{(k)} (q) := (\Gamma_H^{1}(q), \dotsc, \Gamma_H^{k}(q)).
\end{align*}
\end{definition}

For the $k$-wise direct product of puzzles we require the solver to make a verification query on a single index $q \in \cQ$.
Otherwise, if verification queries of the form $((q_1,y_1), \dotsc, (q_k, y_k))$ were allowed,
solving the $k$-wise direct product of puzzles would be trivial.
One could fix some $\bar{q} \in \cQ$ and ask
$k$ hint queries such that the $i$-th hint query is of the form $(q_1, \dotsc,q_{i-1}, \bar{q}, q_{i+1}, \dotsc, q_k)$ where
for all $1 \leq j \leq k$ such that $i \neq j$ it holds $q_j \in \cQ \setminus \{\bar{q}\}$.
The solver obtains correct solutions for $\bar{q}$ to the puzzles on all $k$ positions and
can trivially make a successful verification query.

In the following code listing we define an experiment $\Success$ such that
it outputs $1$ if and only if $C$ asks a successful verification query.
%
%\begin{codeblock}
\begin{restatable}{codeblock}{success}
  \textbf{Experiment} $\Success^{P, C}(\pi, \rho)$
  \medskip \hrule
  \textbf{Oracles:} A problem poser $P$, a solver $C := (C_1, C_2)$ for $P$.\\
  \textbf{Input:}  Bitstrings $\pi \in \{0,1\}^n$, $\rho \in \{0,1\}^*$.\\
  \textbf{Output:} A bit $b \in \{0,1\}$.
  \medskip\hrule
  \Run $\langle P(\pi), C_1(\rho) \rangle$ \\
  \IndI $(\Gamma_V, \Gamma_H) := \langle P(\pi), C_1(\rho) \rangle_{P}$ \\
  \IndI $x := \langle P(\pi), C_1(\rho) \rangle_{\mathit{trans}}$ \\ \\
  \Run $C_2^{\Gamma_V,\Gamma_H}(x; \rho)$ \\
  \IndI \If $C_2^{\Gamma_V, \Gamma_H}(x; \rho)$ asks a verification query $(q, y)$ s.t. $\Gamma_V(q, y) = 1$ \Then \\
  \IndII \Return $1$ \\
  \Return $0$
\end{restatable}
%\end{codeblock}
%
We define the \textit{success probability} of $C$ in solving a puzzle defined by $P$ as
\begin{align}
 \underset{\pi, \rho}{\Pr}[\Success^{P,C}(\pi, \rho) = 1].
\end{align}
Furthermore, for fixed $P$ we say that $C$ \textit{succeeds} for $\pi$, $\rho$ if $\Success^{P,C}(\pi, \rho) = 1$.

We now state our main theorem. Loosely speaking, we claim that it is possible to reduce a solver for the $k$-wise direct product of $P$
to a solver for a single puzzle defined by $P$. This implies that if there is no good solver for $P$, then also a good solver for
the $k$-wise direct product of $P$ does not exist.
%
\begin{restatable}[Hardness amplification for dynamic interactive weakly verifiable puzzles]{theorem}{hardnessAmpfDiwvp}
\label{th:sec_amp_for_dwvp}
Let $P_{n}^{(1)}$ be a fixed problem poser as in Definition \ref{def:dwvp}
and $P_{kn}^{(g)}$ a problem poser for the $k$-wise direct product of $P_{n}^{(1)}$ as in Definition~\ref{def:k_wise_direct_product}.
Additionally, let $C$ be a solver for $P_{kn}^{(g)}$ asking at most $h$ hint queries and $v$ verification queries.
There exists a probabilistic algorithm $\Gen$ with oracle access to $C$,
a binary monotone function $g:\{0,1\}^k \rightarrow \{0,1\}$, and a problem poser $P_{n}^{(1)}$.
The algorithm $\Gen$ takes as input parameters $n$, $\varepsilon$, $\delta$, $k$, $h$, $v$,
and outputs a two-phase probabilistic solver circuit $D$ for $P_{n}^{(1)}$ such that: \\
If $C$ satisfies
  \begin{align}
    \label{th_sec_amp_dwvp_assum}
    \underset{\substack{\pi^{(k)} \in \{0,1\}^{kn} \\ \rho \in \{0,1\}^{*}}}{\Pr}\left[\mathit{Success}^{P_{kn}^{(g)}, C}(\pi^{(k)}, \rho) = 1\right]
    \geq 16(h+v)\Bigl(\underset{u \leftarrow \mu_\delta^k}{\Pr}[g(u) = 1] + \varepsilon \Bigr),
  \end{align}
then $D$ almost surely over the randomness of $\Gen$ satisfies
  \begin{align}
    \underset{\substack{\pi \in \{0,1\}^{n} \\ \rho \in \{0,1\}^{*}}}
    {\Pr}\left[\Success^{P_{n}^{(1)},D}(\pi, \rho) = 1\right] \geq \delta + \frac{\epsilon}{6k}.
  \end{align}
Additionally, $D$ requires oracle access to $g$, $P_{n}^{(1)}$, $C$, hint and verification circuits generated by
the problem poser after the first phase and asks at most $\frac{6k}{\epsilon}\log\left(\frac{6k}{\epsilon}\right) h$
hint queries and one verification query. Finally, $\Time(\Gen) = \poly(k, \frac{1}{\varepsilon}, n, v, h)$ with oracle calls.
\end{restatable}

The above theorem is very general in the sense that it does not pose any constrains
on the size of the circuits or the time complexity of the interactive protocol.
Additionally, we count each oracle call as a single step.

Let us consider, as an example, the case where $C$ and $P_{kn}^{(g)}$ are polynomial time probabilistic algorithms.
Furthermore, we assume that $C$ is such that it satisfies \eqref{th_sec_amp_dwvp_assum}
and $\frac{1}{\epsilon}$ and $k$ are bounded by some polynomial $p(n)$.
Clearly, there are polynomial size families of probabilistic circuits that correspond to $P_{kn}^{(g)}$ and $C$.
%Clearly the solver and the problem poser can be represented as polynomial size families of probabilistic circuits.
The running time of $Gen$ is polynomial, therefore the size of $D$ must also be polynomial.
Finally, the circuit $D$ can be executed by a polynomial time algorithm.
Thereby, we obtain a polynomial time reduction similar as described in the literature \cite{Arora:2009:CCM:1540612, LectureNotesCrypo}.

Theorem \ref{th:sec_amp_for_dwvp} holds with high probability over the randomness of $\Gen$.
More precisely, the circuit output by $\Gen$ satisfies the condition of the theorem with probability
at least $1 - p(k, n, \frac{1}{\epsilon}) \cdot 2^{-n}$. Therefore, Theorem~\ref{th:sec_amp_for_dwvp} is meaningful if there exists $p(n)$
that bounds~$k$~and~$\frac{1}{\epsilon}$.

We emphasize that the number of hint queries asked by $D$ is greater than the number of hint queries asked by $C$ whereas the number of verification queries
is limited to at most one. For many cryptographic constructions making such an assumption about the number of hint and verification queries is reasonable.
In particular, we cannot assume that a solver for a single puzzle may ask more verification queries than a solver for a $k$-wise direct product of puzzles.

In Chapter \ref{ch:preliminaries} we defined a binary monotone function.
The monotone restriction on $g$ in Theorem \ref{th:sec_amp_for_dwvp} is essential. For $g(b) := 1 - b$ a solver circuit that deliberately gives incorrect
answers satisfies $g$ with probability $1$ whereas a circuit that solves a puzzle successfully with probability
$\gamma > 0$ succeeds only with probability $1 - \gamma$.

We prove Theorem \ref{th:sec_amp_for_dwvp} in Chapter~\ref{ch:main_result}.

%%% Local Variables:
%%% mode: latex
%%% TeX-master: "thesis"
%%% End:

\section{Examples}
\label{section:wvp_examples}
In this section we give examples of cryptographic constructions that are different types of weakly verifiable puzzles.

\subsection{Message Authentication Codes}
We consider the setting in which two parties a \textit{sender} and a \textit{receiver} communicate over an insecure channel.
The messages of the sender may be intercepted, modified, and replaced by a third party called an \textit{adversary}.
The receiver needs a way to ensure that received messages have been indeed sent by the sender and have not been modified by the adversary.
The solution is to use \textit{message authentication codes}.

Loosely speaking, the message authentication codes may be explained as follows.
Let sender, receiver, and adversary be polynomial time algorithms, and messages be represented as bitstrings.
Furthermore, we assume that the sender and the receive share a secrete key to which an adversary has no access.
The sender appends to every message a tag which is computed as a function of the key and the message.
The receiver, using the key, has a way to check whether an appended tag is valid for a received message.
The receiver accepts a message if the tag is valid, otherwise it rejects.
We require that it is hard for the adversary to find a tag and a message that is accepted by the receiver with non-negligible probability.
We give the following formal definition of \textit{Message Authentication Code} based on \cite{LectureNotesCrypo} and \cite{Goldreich:2004:FCV:975541}.
\begin{definition}[Message Authentication Codes]
  Let $\cM$ be a set of messages, $\cK$ a set of keys and $\cT$ a set of tags where $n \in \N$.
  We define the \textnormal{message authentication code (MAC)} as an efficiently computable function $\cM \times \cK \rightarrow \cT$.
  Furthermore, we say that MAC is \textit{secure} if it satisfies the following condition:

  Let $k \xleftarrow{\$} \cK$ be fixed and $\Gamma_H$ be a polynomial size circuit that takes as input a message $m \in \cM$ and outputs a tag $t \in \cT$
  such that $f(m,k) = t$ where a key $k$ is hard-coded in $\Gamma_H$. We say that MAC is secure if there is no probabilistic polynomial time algorithm
  with oracle access to $\Gamma_H$ that with non-negligible probability outputs a message $m \in \cM$
  as well as a corresponding tag $t \in \cT$ such that $f(m, k) = t$, and $\Gamma_H$ has not been queried for a tag of message $m$.
\end{definition}
%
We show how MAC is captured by notion of dynamic interactive weakly verifiable puzzles.
For fixed $n$ the sender corresponds to a problem poser, the adversary to a problem solver, and
the key is a bitstring $\pi \in \{0,1\}^{n}$ taken as auxiliary input by a problem poser.
In the first phase, which is non interactive, the problem poser outputs a hint circuit
$\Gamma_H$  and a verification circuit $\Gamma_V$ where both circuits have hard-coded $\pi$.
The circuit $\Gamma_H$ takes as input a message and outputs a tag for this message.
The circuit $\Gamma_V$ that as input $m \in \cM$ and $t \in \cT$ and outputs a bit $1$ if and only if $f(m, \pi) = t$.

In the second phase an adversary takes no input ($x^*$ is empty string) and is given oracle access to $\Gamma_H$ and $\Gamma_V$.
A task of finding by an adversary a valid tag $t \in \cT$ for a message $m \in \cM$ such that a hint for $m$ has no been asked before
corresponds to asking a successful verification query by a problem poser to $\Gamma_V$.
%
\subsection{Public Key Signature Scheme}
% \begin{todo}
%   \textbf{TODO:} Add introduction that gives intuition about the Public Key Signature Schemes
% \end{todo}
First we give a definition of public key encryption scheme, and what it means for such a scheme to be secure.
These definitions are based on \cite{Goldreich:2004:FCV:975541}.

\begin{definition}[Public key signature scheme]
Let $\cQ$ be the set of messages. A \textnormal{public key signature scheme} is defined by a triple of probabilistic polynomial time algorithms:
$G$ -- the key generation algorithm,
$V$ -- the verification algorithm,
$S$ -- the signing algorithm,
such that the following conditions are satisfied:
\begin{enumerate}[-]
  \item $G(1^n)$ outputs a pair of bitstrings $k_{priv} \in \{0,1\}^{n}$ and $k_{pub} \in \{0,1\}^{n}$ where $n$ is a security parameter.
    We call $k_{priv}$ a private key and $k_{pub}$ a public key.
  \item The signing algorithm $S$ takes as input $k_{priv} \in \{0,1\}^{n}$, $q \in \cQ$ and outputs a signature $s \in S$.
  \item The verification algorithm $V$ takes as input $k_{pub} \in \{0,1\}^{n}$, $q \in \cQ$, and $s \in S$ and outputs a bit $b \in \{0,1\}$.
  \item For every $k_{priv}$, $k_{pub}$ output by $G$ and every $q \in \cQ$ it holds
    \begin{align*}
      \Pr[V(k_{pub}, q, S(k_{priv}, q))] = 1,
    \end{align*}
    where the probability is over the random coins of $V$ and $S$.
\end{enumerate}
\end{definition}
We say that $s \in S$ is a \textit{valid} signature for $q \in \cQ$ if and only if $V(k_{pub}, q, s) = 1$.
%
%TODO efficiency of the algorithms
%
\begin{definition}\textbf{(Security of public key signature scheme with respect to a chosen message attack)}
Let an \textnormal{adversary} $A$ be a probabilistic polynomial time algorithm that takes as input $k_{pub}$ and has oracle access to $S$.
We say that $A$ \textnormal{succeeds} if it finds a signature $s \in S$ for a message $q \in \cQ$ such that $V(k_{pub}, q, s) = 1$
and the oracle $S$ has not been queried for a signature of $q$.
The public key encryption scheme is \textnormal{secure} if there is no polynomial time adversary that succeeds with non-negligible probability.
\end{definition}
%
We will show now that a public key signature scheme defined as above can be represented as a dynamic weakly verifiable puzzle.
Let a problem poser correspond to entity that generates $k_{pub}$ and $k_{priv}$ and a problem solver to an adversary.
In the first phase, the problem poser uses algorithm $G(1^n)$ to obtain $k_{pub}$, $k_{priv}$ and sends to the adversary the public key $k_{pub}$.
Then the problem poser generates a hint circuit $\Gamma_H$ and a verification circuit $\Gamma_V$.
The hint circuit $\Gamma_H$ takes as input $q \in \cQ$ and outputs a signature for $q$. The verification circuit
$\Gamma_V$ takes as input $s \in S$ and $q \in \cQ$ and checks whether $s \in S$ is a valid signature for $q \in \cQ$.
In the second phase, the problem solver takes as input a transcript of message from the first round which consists solely of a single message $k_{pub}$.
Additionally, it is given oracle access to $\Gamma_V$ and $\Gamma_H$.
It is clear that if the adversary asks a successful verification query $(q,s)$,
then it also forges a signature.

Thus, a game of forging a signature of a public key signature schemes is a weakly verifiable puzzle that
is dynamic but not interactive as in the first phase only a single message is sent.
%
\subsection{Bit Commitments}
Let us consider the following \textit{bit commitment protocol} that involves two parties a \textit{sender} and a \textit{receiver}.
We suppose that the sender and the receiver are polynomial time probabilistic algorithms.
The protocol consists of a \textit{commit phase} and a \textit{reveal phase}.
In the commit phase the sender and the receiver interact, as the result the sender commits to a value $b \in \{0,1\}$.
In the reveal phase the sender opens the commitment by sending to the receiver a pair $(y,b')$ where $y \in \{0,1\}^{*}$ and $b' \in \{0,1\}$.
We require that after the commit phase it is hard for the receiver to correctly guess $b$. Additionally, in the \textit{reveal phase}
it should be hard for the sender to persuade the receiver that it was committed to the value $\lnot b$.

We base the following definition of \textit{bit commitment protocol} on \cite{LectureNotesComThCrypto}.
\begin{definition}[Bit Commitment Protocol]
  \label{def:bit_commitment}
For a security parameter $n~\in~\N$ a \textnormal{bit commitment protocol} is defined by a pair $(S_n, R_n)$
where $S_n = (S_1, S_2)$ is a two phase probabilistic circuit, and $R_n$ is a probabilistic circuit.
We call $S_n$ a sender and $R_n$ a receiver. The circuit $S_1$, used in the commit phase,
takes as input a pair $(b, \rho_S)$ where $b \in \{0,1\}$ is interpreted as a bit to which $S_n$
commits, and $\rho_S \in \{0,1\}^{n}$ is the randomness used by the algorithm $S_n$.
The receiver $R_n$ takes only auxiliary input $\rho_R \in \{0,1\}^{*}$ that is the randomness used by $R_n$.
The protocol consists of two phases. In the commit phase, circuits $S_1$ and $R_n$ engage in the protocol execution.
As the result $S_1$ commits to $b$ and $R_n$ generates a verification circuit $\Gamma_V$.
The circuit $\Gamma_V$ takes as input a bit $b' \in \{0,1\}$ and a bitstring $y \in \{0,1\}^{*}$ and outputs a bit.
In the reveal phase the circuit $S_2$ takes as input a transcript communication transcript from the first phase
$\lange S_1, R_n \rangle_{\mathit{trans}}$, the bitstring $\rho_s$ and returns $(b', y)$.
We require a bit commitment protocol to have the following properties:
\begin{enumerate}[]
\item{\textnormal{\textbf{(Correctness)}}} For a fixed $b \in \{0,1\}$ we have
  \begin{align*}
    \underset{\substack{\rho_S \in \{0,1\}^{*}, \rho_R \in \{0,1\}^{n} \\
        \Gamma_V := \langle S_1(b,\rho_S), R_n(\rho_r) \rangle_{R_n} \\
        (b',y) := S_2(\langle S_1(b,\rho_s), R_n(\rho_R) \rangle,\rho_S)}}{\Pr}\Big[\Gamma_V(b',y) = 1 \Big] \geq 1 - \epsilon(n),
  \end{align*}
where $\epsilon(n)$ is a negligible function of $n$.
\item{\textnormal{\textbf{(Hiding)}}}
  \begin{todo}
    \textbf{TODO:} Describe it using equations, define somehow the guess of R? Maybe as a last message in the first phase of communication
  \end{todo}
  Probability over random coins of $S_n$ and $R_n$ that any polynomial size circuit
  can guess bit $b$ correctly after the commit phase is at most $\frac{1}{2} + \epsilon(n)$ where $\epsilon(n)$ is a negligible function of $n$.
\item{\textnormal{\textbf{(Binding)}}}
  For every polynomial size circuit $S_n$ we have
  \begin{align*}
    \underset{\substack{\Gamma_V := \langle S_1, R \rangle_{R} \\ (b',y) := S_2(\rho_S)}}{\Pr}[\Gamma_V(0,y_0) = 1 \land \Gamma_V(1,y_1) = 1] \leq \epsilon(k),
  \end{align*}
  where $\epsilon(k)$ is a negligible function in $k$.
\end{enumerate}
\end{definition}

\begin{todo}
  \textbf{TODO:} This is not clear ... access to the oracle etc.\\
  \textbf{TODO:} Why it is not possible to verify -- i.e. the sender does not even
  know the function that is used by the receiver to validate decommitment.
\end{todo}

The bit commitment protocols can be generalized as an interactive weakly verifiable puzzle as follows.
The number of hint queries amounts to zero and the number of the verification queries is at most one.
The sender corresponds to a problem solver, and the receiver is a problem poser.
Additionally, we require the problem solver to ask a verification query $(b',y)$ only on $b' := \lnot b$ where $b$
is a bit to which the problem solver is committed after the first phase.
The first phase corresponds to the commit phase.
The second phase is the reveal phase where the problem poser tries to find a bitstring $y$ such that $\Gamma_V(\lnot b, y) = 1$.

\subsection{Automated Turing Tests}
The goal of \textit{Automated Turing Tests} is to distinguish humans from computers which
is frequently used to prevent computer programs from accessing resources for humans.
An example is \textit{CAPTCHA} defined first in \cite{von2003captcha}.
Loossly speaking, CAPTCHA is a test that human can solve with probability close to 1, but it is hard to write a computer program
that has a success probability comparable to the one achieved by humans.
An example of CAPTACHA is an image depicting a distorted text. Most humans guess the text which is displayed on the image correctly, but it might be hard to write
a program for which it would also be easy. We note that the definition of hardness has not been particular well defined ,
and bases on opinions AI community opinions that distinguish between hard and easy AI problems \cite{von2003captcha}.

CAPTCHAs based on guessing the distorted text are weakly verifiable puzzles.
In the first round the problem poser and problem solver engage in interactive protocol, such that
after the execution of the protocol the problem poser has a way to verify the solution.
The problem poser in the second round takes as input a distorted image, and try to guess the text that was used to generated it.
The standard CAPTCHAs are non-dynamic, as the problem poser does not gain access to the hint oracle and
asks only a single verification query.

Our definition captures also the above type of problems, additionally it is also applicable in the broader context for a different
AI problems.

As it is not know how good the possible algorithm can be to recognize CAPTCHA it is likely that the gap between human
performance and a performance of computer programs may be small. Therefore, it is of interest to find a way to amplify this gap.
It turns out that it is indeed possiThe first ble which for not dynamic puzzles was proved in \cite{DBLP:journals/corr/abs-1002-3534}.
The proof presented in Chapter \ref{ch:main_result} applies also to the dynamic context.

\begin{todo}
  \textbf{TODO:} Give an optimization problem for gap amplification
\end{todo}

% why is it hard to automatically check the solution for CAPTCHAs
%todo define information theoretic security
% \subsection{Information theoretically secure constructions}
% The definition presented in \ref{def:dwvp} applies also to information theoretic secure constructions.

%%% Local Variables:
%%% mode: latex
%%% TeX-master: "thesis"
%%% End:

\label{st:previous_results}
In the last chapter we gave an overview of different types of cryptographic primitives that motivated studies of weakly verifiable puzzles.
The focus of this chapter is on providing an outline of previous research.
We give a short overview of techniques used in the series of papers \cite{canetti2005hardness, dodis2009security, holenstein2011general}
and provide some intuition and insight into the hardness amplification results for weakly verifiable puzzles.
First, we describe weakly verifiable puzzles studied in \cite{canetti2005hardness} that are neither interactive nor dynamic.
Then, in Section~\ref{section:dijk} we bring our focus on dynamic non-interactive puzzles studied in \cite{dodis2009security}.
Finally, in Section~\ref{section:iwvp} we give an overview of the results of Holenstein and Schoenebeck \cite{holenstein2011general}
where non-dynamic but interactive weakly verifiable puzzles are studied.
%
% dodis2009security holenstein2011general
\section{Weakly Verifiable Puzzles}
\label{subsec:chs}
The notion of \textit{weakly verifiable puzzles} was coined by Canetti et al. in the paper
\textit{Hardness amplification of weakly verifiable puzzles} \cite{canetti2005hardness}.
The puzzles considered there are non-dynamic and non-interactive.
Moreover, the number of verification queries is limited to one. This constitutes a special case of Definition \ref{def:dwvp}.
In this section we define weakly verifiable puzzles (WVPs) and state the hardness amplification theorem for WVPs
in a similar vein as in \cite{canetti2005hardness}.
Finally, we give the intuition behind the proof of this theorem. It is noteworthy that the main proof of this thesis,
presented in Chapter~\ref{ch:main_result}, uses many ideas of the work of Canetti et al. \cite{canetti2005hardness}.
%
\subsection{The definition}
We give the definition of WVP from \cite{canetti2005hardness}. However, we use the notation and terminology defined in this thesis.

\begin{definition}[Weakly Verifiable Puzzle, \cite{canetti2005hardness}]
  \label{def:wvp}
A \textit{weakly verifiable puzzle} is defined by a pair of polynomial time algorithms:
a probabilistic puzzle-generation algorithm $G$ and a deterministic verification algorithm $V$.
For a security parameter $k \in \N$ we write $G(1^k; \rho)$ to denote that $G$ takes as input $1^k$
and uses the randomness $\rho \in \{0,1\}^{*}$.
The algorithm $G$ outputs $p \in \{0,1\}^{*}$ and a check information $c \in \{0,1\}^{*}$.
The \textit{verifier} $V$ takes as input $p$, $c$, an answer $a \in \{0,1\}^{*}$ and outputs $b \in \{0,1\}$.

A \textit{solver} $S$ for $G$ is a probabilistic polynomial time algorithm that
takes as input $p$ and outputs $a$. We denote the randomness used by $S$ as $\pi \in \{0,1\}^{*}$
and define the \textit{success probability} of $S$ in solving a puzzle defined by $(V,G)$ as
\begin{align*}
  \underset{\substack{\rho \in \{0,1\}^{*}, \pi \in \{0,1\}^{*} \\ (p,c):=G(1^k; \rho) \\ a := S(p, \pi)}}{\Pr}\Big[ V(p,c,a) = 1\Big].
\end{align*}
\end{definition}

We write $P := (G,V)$ to denote a weakly verifiable puzzle $P$ defined by algorithms $G$ and $V$.

Let us argue that Definition \ref{def:wvp} is a special case of Definition \ref{def:dwvp}.
First, we note that if $G$ takes as input $1^k$, then the length of the randomness used by $G$ is bounded by $\poly(k)$.
For a fixed $k$, without loss of generality, we can represent $G(1^k; \rho)$ as a probabilistic circuit of polynomial size
that takes as input only the randomness $\rho$.
In Definition \ref{def:wvp} a verification algorithm $V$ takes as input $p$, $c$, $a$.
Again, without loss of generality, we can assume that bitstrings $p$ and $c$ are hard-coded
in the circuit $\Gamma_V$ from Definition \ref{def:dwvp}.
Furthermore, for weakly verifiable puzzles we have that $|\cQ| = 1$ and thus,
it is not necessary to pass $q$ as the parameter to $\Gamma_V$.
Hence, for fixed $p$ and $c$ the algorithm $V$ corresponds to $\Gamma_V$.
The puzzles considered in Definition \ref{def:wvp} are non--dynamic.
Thus, there is no element corresponding to the hint circuit $\Gamma_H$, and without loss of generality, we can assume that
$\Gamma_H$ outputs $\bot$ for every query.
Finally, we note that the puzzles described in Definition~\ref{def:wvp} are non-interactive.

\subsection{The hardness amplification theorem}
We now give the definition of the $n$-wise direct product of weakly verifiable puzzles.
%
\begin{definition}[$n$-wise direct product of weakly verifiable puzzles, \cite{canetti2005hardness}]
  \label{def:n-fold-rep}
  Let $n~\in~\N$ and a weakly verifiable puzzle $P := (G,V)$ be fixed.
  We define the $n$-wise direct product of $P$ as a weakly verifiable puzzle where the puzzle-generation algorithm
  $G^{(n)}$ takes as input $1^{k \cdot n}$, uses the randomness $\rho \in \{0,1\}^{*}$,
  and outputs tuples $p^{(n)} := (p_1, \dotsc, p_n) \in \{0,1\}^{*}$ and $c^{(n)} := (c_1, \dotsc, c_k) \in \{0,1\}^{*}$
  where for each $1 \leq i \leq n$ a pair $(p_i, c_i)$ is an independent instance of a weakly verifiable puzzle defined by $G$ and $V$ with the security parameter $k$.
  Finally, the verification algorithm $V^{(n)}$ takes as input $p^{(n)}$, $c^{(n)}$, an answer $a^{(n)}$, and outputs $b \in \{0,1\}$
  such that $b = 1$ if and only if for all $1 \leq i \leq n$ it holds $V(p_i, c_i, a_i) = 1$.
 \end{definition}
%
Let us set up the additional notation and terminology. We write $P^{(n)} := (G^{(n)}, V^{(n)})$ to denote the $n$-wise direct product of $P$.
For $P^{(n) } := (G^{(n)},V^{(n)})$ we say \textit{puzzle on the $i$-th coordinate} to refer to the $i$-th puzzle of the $n$-wise direct product of $P$
(this puzzle corresponds to the one generated by $G^{(n)}_i$,  $V^{(n)}_i$).

The $n$-wise direct product of WVP is solved successfully if and only if all $n$ puzzles are solved successfully.
In contrast, in Definition~\ref{def:k_wise_direct_product} we are interested in a more general situation where a monotone function $g: \{0,1\}^{n} \rightarrow \{0,1\}$
is used to decide on which coordinates the puzzles of the $n$-wise direct product have to be solved successfully.
Clearly, we can assume that $g$ is such that the puzzles on all coordinates have to be solved successfully
which matches the case considered in Definition \ref{def:n-fold-rep}.

The main theorem proved in \cite{canetti2005hardness} states that it is possible to turn a good solver for $P^{(n)}$
into a good solver for $P$.
%
\begin{theorem}[Hardness amplification for weakly verifiable puzzles, \cite{canetti2005hardness}]
  \label{thm:wvp_chs}
Let $n,q \in \N$ and $\delta: \N \rightarrow (0,1)$ be an efficiently computable function.
Moreover, let $P := (G,V)$ be a weakly verifiable puzzle. We denote the running time of the
puzzle-generation algorithm $G$ by $T_G$ and of the verification algorithm $V$ by $T_V$.
If $S^{(n)}$ is a solver for the $n$-wise direct product of $P$ that success probability is at least $\delta^{n}$
and the running time is $T$, then there exists a solver $S$ for $P$ with oracle access to $S^{(n)}$ that success
probability is at least $\delta(1-\frac{1}{q})$ and the running time is $O\Big(\frac{nq^3}{\delta^{2n-1}}(T + nT_G + nT_V)\Big)$.
\end{theorem}
%
The parameter $q$ is introduced as it is not possible to achieve perfect hardness amplification.
%Let $S^{(n)}$ be a solver for $P^{(n)}$ that success probability is at least $\delta^{n}$.
The following algorithm CHS-solver from \cite{canetti2005hardness} has oracle access to $S^{(n)}$ and
solves a puzzle defined by $P$ with probability at least $\delta(1  - \frac{1}{q})$.
We denote by $p \in \{0,1\}^{*}$ the output of $G$, which is also the input taken by the CHS-solver.
To make the notation shorter in the following code excerpts we do not write the randomness used by $G$ explicitly.
In the analysis of the running time of the CHS-solver we explicitly take into account the time needed for the oracle calls to $S^{(n)}$, $V$, $G$.

\vspace*{\fill}
\pagebreak
\begin{codeblock}
  \textbf{Algorithm:} $\text{CHS-solver}^{S^{(n)},V,G}(p, n, k, q, \delta)$
  \medskip\hrule
  \textbf{Oracle:} A solver $S^{(n)}$ for $P^{(n)}$, a verification algorithm $V$ for $P$, a puzzle-generation algorithm $G$ for $P$.\\
  \textbf{Input:}  A bistring $p \in \{0,1\}^{*}$, parameters $n, k, q, \delta$.
  \medskip\hrule
  $\mathit{prefix} := \emptyset$\\
  \For $i := 1$ \To $n\!-\!1$ \Do \\
  \IndI $p^* := \mathit{ExtendPrefix}^{S^{(n)}, V, G}(\mathit{prefix}, i, n, k, q, \delta)$\\
  \IndI \If $p^* = \bot$ \Then \Return $\mathit{OnlinePhase}^{S^{(n)}, V, G}(\mathit{prefix}, p, i, n, k, q, \delta)$ \\
  \IndI \Else $\mathit{prefix} := \mathit{prefix} \circ p^*$\\
  $ a^{(n)} := S^{(n)}(\mathit{prefix} \circ p)$ \\
  \Return $a_n$
\end{codeblock}
%
\begin{codeblock}
  \textbf{Algorithm:} $\mathit{OnlinePhase^{S^{(n)}, V, G}(\mathit{prefix}, p, i, n, k, q, \delta)}$
  \medskip \hrule
  \textbf{Oracle:} A solver algorithm $S^{(n)}$ for $P^{(n)}$, a puzzle-generation algorithm $G$ for $P$, a~verification algorithm $V$ for~$P$.\\
  \textbf{Input:} A $(i-1)$--tuple of bitstrings $\mathit{prefix}$, a bitstring $p \in \{0,1\}^{*}$, \\ parameters $i, n, k, q, \delta$.
  \medskip\hrule
  \Repeat $\Big\lceil\frac{6q \ln (6q)}{\delta^{n-i+1}}\Big\rceil$ times \\
  \IndI $((p_{i+1}, \dotsc, p_{n}),(c_{i+1}, \dots, c_n)) := G^{(n-i)}(1^{k\cdot(n-i)})$\\
  \IndI $a^{(n)} := S^{(n)}(\mathit{prefix}, p, p_{i+1}, \dotsc, p_n)$\\
  \IndI \If $\forall_{i+1 \leq j \leq n} V(p_j, c_j, a_j) = 1$ \Then \Return $a_i$\\
  \Return $\bot$
\end{codeblock}
%
\begin{codeblock}
  \textbf{Algorithm:} $\mathit{ExtendPrefix^{S^{(n)}, V, G}(prefix, i, n, k, q, \delta)}$
  \medskip \hrule
  \textbf{Oracle:} A solver algorithm $S^{(n)}$ for $P^{(n)}$, a puzzle-generation algorithm $G$ for $P$, a~verification algorithm $V$ for~$P$.\\
  \textbf{Input:} A $(i-1)$--tuple of puzzles $\mathit{prefix}$, parameters $i, n, k, q, \delta$.
  \medskip\hrule
  \Repeat $\Big\lceil \frac{6q}{\delta^{n-v+1}} \ln (\frac{18qn}{\delta}) \Big\rceil$ times \\
  \IndI $(p^*, c^*) := G(1^k) $\\
  \IndI $\bar{\nu}_i := \mathit{EstimateResSuccProb}^{S^{(n)},G,V}(\mathit{prefix} \circ p^*, i, n, k, q, \delta)$\\
  \IndI \If $\bar{\nu}_i \geq \delta^{n-i}$ \Then \Return $p^*$ \\
  \Return $\bot$
\end{codeblock}
%
\begin{codeblock}
  \textbf{Algorithm:} $\mathit{EstimateResSuccProb}^{S^{(n)},V, G}(\mathit{prefix}, i, n, k, q, \delta)$
  \medskip \hrule
  \textbf{Oracle:} A solver $S^{(n)}$ for $P^{(n)}$, a verification algorithm $V$ for $P$, a~puzzle-generation algorithm $G$ for~$P$\\
  \textbf{Input:} A $i$--tuple of puzzles $\mathit{prefix}$, parameters $i, n, k, q, \delta$.
  \medskip\hrule
  $successes := 0$ \\
  $N_i := \big\lceil \frac{6q}{\delta^{n-i+1}} \ln(\frac{18qn}{\delta}) \big\rceil $\\
  \Repeat $K := \Big\lceil \frac{84q^2}{\delta^{n-i}} \ln \Big(\frac{18qn \cdot N_i}{\delta} \Big) \Big\rceil$ times \\
  \IndI $((p_{i+1}, \dotsc, p_n), (c_{i+1}, \dotsc, c_n)) := G^{(n-i)}(1^{k\cdot(n-i)})$\\
  \IndI $a^{(n)} := A(\mathit{prefix}, p_{i+1}, \dotsc, p_{n})$\\
  \IndI \If $\forall_{i + 1\leq j \leq n} : V(p_j, c_j, a_j) = 1$ \Then $\mathit{successes := successes + 1}$ \\
  \Return $successes / K$
\end{codeblock}
%
A full proof of Theorem \ref{thm:wvp_chs} is presented in \cite{canetti2005hardness}.
We limit ourselves to providing the intuition why the CHS-solver transforms a good solver
for the $n$--wise direct product of $P$ into a good solver for a single puzzle defined by $P$.

Let us consider the $n$-wise direct product of $P$ and, for simplicity, a deterministic solver $S^{(n)}$ for $P^{(n)} := (G^{(n)}, V^{(n)})$.
We write $p^{(n)}, c^{(n)}$ to denote the output of $G^{(n)}$.
We define a matrix $M$ as follows. The columns of $M$ are labeled with all possible bitstrings $p_1$
whereas the rows are labeled with all possible tuples $(p_2, \dotsc, p_n)$ output by $G^{(n)}$ when executed with the different randomness.
A cell of $M$ contains a binary $n$-tuple such that the $i$-th bit equals $1$ if and only if $V_i(p_i, c_i, a_i) = 1$ where
 $a^{(n)} := S^{(n)}(p^{(n)})$ and $p^{(n)}$ is a tuple of bitstrings inferred by a column and a row of the cell.
We make the following observation.
%
\begin{observation}[\cite{canetti2005hardness}]
\label{obs:wvp_matrix}
Let $S^{(n)}$ be a deterministic polynomial time solver for $P^{(n)}$ that success probability is at least $\delta^{n}$.
Then, the matrix $M$ defined as above has either a column with a $\delta^{(n-1)}$ fraction of cells that are $1^n$ tuples, or
a conditional probability that a cell is of the form $1^n$ given that the last $(n-1)$ bits of this cell are equal $1$ is at least $\delta$.
\end{observation}
%
Let us explain, at least intuitively using Observation \ref{obs:wvp_matrix}, how the CHS-solver
that has oracle access to $S^{(n)}$ for $P^{(n)}$ can be used to solve a puzzle defined by $P$
with substantial probability.
We refer to a puzzle solved by the CHS-solver as an \textit{input puzzle}.
The CHS-solver starts with the first position and tries to fix a bitstring $p^*$ used to generate the puzzle on this position such
that the success probability of $S^{(n)}$ on the remaining $(n-1)$ position is at least $\delta^{(n-1)}$.
If it is possible to find $p^*$ such that this condition is satisfied, then we use $p^*$ to generate the puzzle on this position
and repeat the whole process again in the consecutive iteration for the next position.
If the CHS-solver fails to find a bitstring $p^*$, then we assume that there is no column of $M$ that contains a $\delta^{(n-1)}$ fraction
of cells that are of the form $1^n$. We use Observation~\ref{obs:wvp_matrix} to conclude that the conditional probability of
solving the first puzzle given that all puzzles on the remaining position are solved successfully is at least~$\delta$.
We place the input puzzle $p$ on this position and note that the remaining puzzles are generated by the CHS-solver.
Thus, it is possible to efficiently verify whether these puzzles are successfully solved by $S^{(n)}$.

Obviously, the CHS-solver can still fail. First, it may happen that it does not find a column
with a high fraction of puzzles that are solved successfully, although such a column exists.
Secondly, we cannot exclude a situation where no such column exists, but the algorithm fails to find a cell such that last $(n\!-\!1)$ bits are 1.
Finally, it is also possible that the estimate returned by $\mathit{EstimateResSuccProb}$ is incorrect.

It is possible to show that all these events happen with small probability such that
the success probability of the CHS-solver is at least $\delta(1\!-\!\frac{1}{q})$ almost surely.

In Chapter \ref{ch:main_result} we study a more general class of puzzles that are not only weakly verifiable, but also dynamic and interactive.
Furthermore, we allow a situation where the solver successfully solves the $n$-wise direct product of $P$,
although it succeeds only on a subset of coordinates of the $n$-wise direct product of $P$.
It turns out that it is possible to use a similar technique of fixing puzzles on the consecutive positions of the $n$-wise direct product of
puzzles to prove hardness amplification in this more general setting.
%
\section{Dynamic Weakly Verifiable Puzzles}
\label{section:dijk}
Some of the cryptographic primitives presented in Chapter~\ref{ch:examples_wvcp}
are not only weakly verifiable but also dynamic (MAC and SIG). This type of puzzles is defined and studied in \cite{dodis2009security}.
We give a short overview of this work and define a \textit{dynamic weakly verifiable puzzle} (DWVP) as in \cite{dodis2009security}.
Finally, we describe important parts of the proof of hardness amplification for DWVPs.

\subsection{The definition}
We now define \textit{dynamic weakly verifiable puzzles} as in \cite{dodis2009security}.
\begin{definition}[Dynamic Weakly Verifiable Puzzle, \cite{dodis2009security}]
  \label{def:dwvp_dodis}
  A \textit{dynamic weakly verifiable puzzle} is defined by a distribution $\cD$ on pairs $(x, \alpha)$ where
  $\alpha$ is secret information and $x$ is a bitstring.
  Furthermore, we consider a set $\cQ$ (for some set of indices $\cQ$) and a probabilistic polynomial time computable verification relation $V$ such that
  $V(\alpha, q, r) = 1$ if and only if $r \in \{0,1\}^{*}$ is a correct answer to $q \in \cQ$
  on the set determined by $\alpha$. Finally, let $H$ be a probabilistic polynomial time computable \textit{hint} function
  that on input $\alpha$, $q$ outputs a bitstring $\{0,1\}^{*}$.

  A solver $S$ takes as input $x$ and can make hint queries on $q \in \cQ$ which are answered using $H(\alpha, q)$ and verification
  queries $(q,r)$ answered by means of $V(\alpha, q, r)$.
  We say that $S$ succeeds if and only if it makes a verification query on $(q,r)$ such that
  $V(\alpha,q,r) = 1$, and it has not previously asked for a hint query on this $q$.
  We write $P := (\cD, V, H)$ to denote a DWVP with the distribution $\cD$ and $V$, $H$ being a verification and hint relations, respectively.
\end{definition}
%
Dynamic weakly verifiable puzzles generalize games of breaking the security of message authentication codes and public key signature schemes.
In the case of MAC $x$ is an empty bitstring. For the public key signature schemes $x$ is a public key.

The above definition is a special case of Definition \ref{def:dwvp}.
To see this we observe that instead of considering a distribution $\cD$ on pairs $(x,\alpha)$
we can use a probabilistic problem poser that outputs circuits $\Gamma_H$ and $\Gamma_V$ with hard-coded $\alpha$ that correspond to $H$ and $V$, respectively.
It is clear that the solver $S$ from Definition \ref{def:dwvp_dodis} can be turned into a family of probabilistic polynomial size circuits
with oracle access to $\Gamma_H$ and $\Gamma_V$. Furthermore, in the first phase, which is non-interactive,
a bitstring $x$ is sent by the problem poser to the solver.

\subsection{The hardness amplification theorem}
We now give the definition of the $n$-wise direct product of DWVPs.

\begin{definition}[$n$-wise direct product of DWVPs, \cite{dodis2009security}]
  \label{def:n-wdp-dwvp}
For a dynamic weakly verifiable puzzle $P~:=~(\cD, V, H)$ we define the $n$-wise direct product of $P$
as a DWVP with a distribution $\cD^{(n)}$ on tuples $(x_1, \alpha_1), \dotsc, (x_n, \alpha_n)$ where
each $(x_i, \alpha_i)$ is drawn from the distribution $\cD$.
Furthermore, a hint relation is defined by $H^{(n)}(q, \alpha_1, \dotsc, \alpha_n) := (H(\alpha_1, q), \dotsc, H(\alpha_n, q))$ and
a verification relation $V^{(n)}(\alpha_1, \dotsc, \alpha_n, r_1, \dotsc, r_n, q)$ evaluates to $1$ if and only if
for at least $n - (1 - \gamma)\delta n$ of bitstrings $r_1, \dotsc, r_n $ it holds $V(\alpha_i, q, r_i)~=~1$ where $0~\leq~\gamma,\delta~\leq~1$.
%
\end{definition}
We write $P^{(n)} := (D^{(n)}, H^{(n)}, V^{(n)})$ to denote the $n$-wise direct product of $P$.

The above definition is more general than the one given in the previous section for the $n$-wise direct product of WVP, as
in Definition \ref{def:n-fold-rep} it is enough if the solver succeeds only on a fraction of coordinates.

We write $(\cH_{\mathit{hint}}, \cV_{\mathit{verif}}) \leftarrow S(x; \delta)$ to denote
the execution of $S$ with the input $x \in \{0,1\}^{*}$ and using the randomness $\delta \in \{0,1\}^{*}$
where $\cH_{hint}$ is the set of all hint queries asked by $S$, and $\cV_{verif}$ is
the set of all verification queries asked in the execution of $S$.

With no loss of generality, we make the assumption that once $S$
made a successful verification query it does not ask any further hint or verification queries.

We define the \textit{success probability} of $S$ as
\begin{align*}
  \underset{\substack{\delta \in \{0,1\}^{*} \\(x,\alpha) \leftarrow \cD \\ (\cH_{hint}, \cV_{verif}) \leftarrow S(x;\delta))}}
  {\Pr}\big[\exists (q,r) \in \cV_{verif} : q \notin \cH_{hint} \land V(\alpha, q, r) = 1 \big].
\end{align*}

\begin{theorem}[Hardness amplification for dynamic weakly verifiable puzzles \cite{dodis2009security}]
\label{lemma:dwvp}
Let $S^{(n)}$ be a probabilistic algorithm for $P^{(n)}$ that succeeds with
probability at least $\epsilon$, where $\epsilon \geq (800/\gamma\delta) \cdot (h+v) \cdot e^{-\gamma^2\delta n/40}$ and $h$ and $v$
denote the number of hint and verification queries asked by $S^{(n)}$, respectively.
There exists a probabilistic algorithm $S$ that solves a puzzle defined by $P$ with probability at least $1-\delta$.
Furthermore, $S$ makes $O(h(h+v)/\epsilon) \cdot \log(1/\gamma\delta)$ hint queries and at most one verification query.
The running time of $S$ is $\poly(h,v,\frac{1}{\epsilon}, t, \omega, \log(1/\gamma\delta))$ where
$\omega$ is the time needed to make a single hint or verification query.
\end{theorem}

It is worth seeing why the approach presented in the previous section that works well for the direct product of WVPs
cannot be applied for the direct product of DWVPs (moving aside for a moment the issue of solving only a fraction of puzzles successfully).
Loosely speaking, for DWVP the algorithm CHS-solver breaks in the $\mathit{OnlinePhase}$ where $S^{(n)}$ can be called multiple times.
It is possible that in one of these calls $S^{(n)}$ asks a hint query on $q$
which prevents in one of the later runs a verification query $(q,r)$ from succeeding.
The fact that a hint query on $q$ has been asked before makes it impossible to make a successful verification query on this $q$.
Thus, we cannot dismiss a situation where the success probability of $S^{(n)}$ decreases with the number of iterations.

The solution proposed in \cite{dodis2009security} is to partition $\cQ$ into a set of \textit{attacking queries} $\cQ_{\mathit{attack}}$
and a set of \textit{advice queries} $\cQ_{\mathit{adv}}$. The idea is to allow a solver to ask hint
queries only on $q \in \cQ_{\mathit{adv}}$, and to halt the execution whenever a hint query is asked on $q \in \cQ_{\mathit{attack}}$.

More formally, for a function $\hash:\cQ \rightarrow \{0,1,\dotsc, 2(h+v)\!-\!1 \}$
we define $\cQ_{\mathit{attack}} := \{q \in \cQ : \hash(q) = 0 \}$ and $\cQ_{adv} := \{q \in \cQ: \hash(q) \neq 0\}$.
It is possible, for a fixed solver $S$ that asks at most $h$ hint queries and $v$ verification queries,
to find a function $hash: \cQ \rightarrow \{0,1,\dotsc 2(h+v)-1\}$ such that the success probability of $S$ with respect to
$\cQ_{\mathit{attack}}$ and $\cQ_{\mathit{adv}}$ is multiplied by $\frac{1}{8(h+v)}$.
If $h$ and $v$ are not too big, then the success probability of $S$ can be still substantial.
More formally, the following lemma is proved in \cite{dodis2009security}.
\begin{lemma}[Success probability in solving a dynamic weakly verifiable puzzle with respect to a function hash, \cite{dodis2009security}]
  \label{lemma:hash_function_previous}
Let $S$ be a solver for DWVP which success probability is at least $\delta$, the running time is at most $t$,
and the number of hint and verification queries is at most $h$ and $v$, respectively.
There exists a probabilistic algorithm that runs in time $poly(h,v,\frac{1}{\delta},t)$
that outputs a function $\hash : \cQ \rightarrow \{0,1, \dotsc, 2(h+v)-1\}$
that partitions $\cQ$ into $\cQ_{attack}$ and $\cQ_{adv}$ such that
with probability at least $\frac{\delta}{8(h+v)}$ the first successful verification query $(q',a)$ asked by $S$ is such that $q' \in \cQ_{attack}$
and all previous hint and verification queries have been asked on $q \in \cQ_{adv}$.
\end{lemma}
A function $\hash$ can be found by means of a natural sampling technique.
We follow exactly the same approach of partitioning $\cQ$ in Section \ref{st:domain_partition}.

Let $H_{\alpha}(q)$ be a polynomial time probabilistic algorithm that takes as input $q$, has hard-coded $\alpha$ and computes $H(\alpha, q)$.
Similarly, we use $V_{\alpha}(q,r)$ to denote a polynomial time probabilistic algorithm that computes $V(\alpha, q, r)$ and has hard-coded $\alpha$.
The following algorithm DWVP-solver from \cite{dodis2009security} has oracle access to $S^{(n)}$, $V_{\alpha}$ and $H_{\alpha}$ as well as a function $\hash$
as in Lemma \ref{lemma:hash_function_previous}.
%
\begin{codeblock}
  \textbf{Algorithm:} $\text{DWVP-solver}^{S^{(n)}, \hash, H_{\alpha}^{(n)}, V_{\alpha}^{(n)}}(x)$
  \medskip
  \hrule
  \textbf{Oracle:}  A solver $S^{(n)}$ for $P^{(n)}$, algorithms $V_{\alpha}$ and $H_{\alpha}$, a function $hash : \cQ \rightarrow \{0,1, \dotsc, 2(h+v)-1\}$.\\
  \textbf{Input:} A bistring $x \in \{0,1\}^{*}$.
  \medskip\hrule
  \Repeat at most $O(\frac{h+v}{\epsilon} \cdot \log(\frac{1}{\gamma\delta}))$ times \\
  \IndI Let $i \xleftarrow{\$} \{1, \dotsc, n\}$ be a position for $x$.\\
  \IndI Generate $(x_1, \alpha_1), \dotsc, (x_{i-1}, \alpha_{i-1}), (x_{i+1}, \alpha_{i+1}), \dotsc, (x_n, \alpha_n)$ \\
  \IndI using $(n-1)$ calls to $P$ each time with the fresh randomness.\\
  \IndI \Run $S^{(n)}(x_1, \dotsc, x_{i-1}, x, x_{i+1}, \dotsc, x_n)$\\
  \IndII \If $S^{(n)}$ asks a hint query on $q$ \Then \\
  \IndIII \If $hash(q) \neq 0$ \Then abort the current run of $S^{(n)}$\\
  \IndIII Ask a verification query $r := H(q)$.\\
  \IndIII Let $(r_1, \dotsc, r_{i-1}, r_{i+1}, \dotsc, r_{n})$ be hints for the query $q$ for \\
  \IndIII puzzles $(x_1, \dotsc, x_{i-1}, x_{i+1}, x_n)$.\\
  \IndIII Answer the hint query of $S^{(n)}$ using $(r_1, \dots, r_{i-1}, r, r_{i+1}, r_n)$.\\
  \IndII \If $S^{(n)}$ asks a verification query $(q, r_1, \dots, r_n)$ \Then \\
  \IndIII \If $hash(q) = 0$ \Then answer the query with $0$\\
  \IndIII Let $m := |j: V(q,r_j) = 1, j \neq i|$\\
  \IndIII \If $m \geq n - n(1-\gamma)\delta$ \Then \\
  \IndIIII make a verification query $(q, r_i)$ and halt.\\
  \IndIII \Else with probability $\rho^{m - n(1-\gamma)\delta}$ ask a verification query \\
  \IndIIII $(q, r_i)$ and halt. \\
  \IndIII Halt the current run of $S^{(n)}$ and go to the next iteration.\\
  \Return $\bot$
\end{codeblock}

In each iteration of the DWVP-solver a position for the input puzzle is chosen uniformly at random.
The remaining $(n-1)$ puzzles are generated by the DWVP-solver, thus it is possible to answer
all hint and verification queries for these puzzles.
%We assume that $\hash$ is such that the success probability of $S^{(n)}$ with respect to $\hash$ is at least $\frac{\delta}{8(h+v)}$.
The DWVP-solver calls $S^{(n)}$ multiple times, but the function $\hash$ is used
to partition the query domain such that
if a hint query is asked on $q$ for which $hash(q) = 0$ then the current execution of $S^{(n)}$
is aborted, and the DWVP-solver goes to the next iteration.
Thus, we ensure that the DWVP-solver never asks a hint query that could prevent a verification query from succeeding.
If a verification query is asked on $q$ such that $hash(q) \neq 0$ we answer this query with $0$.

Finally, in the case where $S^{(n)}$ asks a verification query on $q$ such that $hash(q) = 0$,
a \textit{soft decision system} is used to decide whether to ask a verification query.
The idea is that if there are many puzzles among the ones generated by the algorithm that are solved successfully,
then it is likely that also the input puzzle is solved successfully.
We discount $\gamma\delta n$ to take into account that not all puzzles have to be solved successfully.
Detail calculations provided in \cite{dodis2009security} show that this approach
yields a demanded result.

In Chapter \ref{ch:main_result} we consider weakly verifiable puzzles that are not only dynamic but also interactive.
We use a very similar technique to partition $\cQ$ into advice and attacking queries.
Instead of the requirement to succeed on at least a fraction of puzzles we consider a more general approach where
an arbitrary binary monotone function $g : \{0,1\}^{n} \rightarrow \{0,1\}$ is used to
determine on which coordinates the solver has to succeed in order to successfully solve the $n$-wise direct product of puzzles.

\section{Interactive Weakly Verifiable Puzzles}
\label{section:iwvp}
Hardness amplification for interactive but non-dynamic weakly verifiable puzzles
has been studied by Holenstein and Schoenebeck in \cite{holenstein2011general}.
We will now give an overview of this work and compare it with our approach.

\subsection{The definition}
In \cite{holenstein2011general} interactive weakly verifiable puzzles are introduced.
\begin{definition}[Interactive Weakly Verifiable Puzzle, \cite{holenstein2011general}]
  \label{def:iwvp}
An \textit{interactive weakly verifiable puzzle} is defined by a protocol given by two probabilistic algorithms $P$ and $S$.
The algorithm $P$ is called a \textit{problem poser} and produces as output a verification circuit $\Gamma$.
The algorithm $S$ called \textit{a solver} produces no output.
Furthermore, the \textit{success probability} of the algorithm $S^*$ in solving an interactive weakly verifiable puzzle defined by $(P,S)$ amounts
\begin{align*}
  \underset{\substack{\rho \in \{0,1\}^{*}, \pi \in \{0,1\}^{n} \\ \Gamma := \langle P(\pi), S^*(\rho) \rangle_{P}}}{\Pr}\Big[\Gamma(\langle P(\pi),S^*(\rho) \rangle_{S^*}) = 1 \Big].
\end{align*}
\end{definition}
It is not hard to see that Definition \ref{def:iwvp} is a special case of Definition \ref{def:dwvp}.
The puzzles in Definition \ref{def:iwvp} are non-dynamic, thus $|\cQ| = 1$. Furthermore, we can assume that $\Gamma$
corresponds to the verification circuit from Definition \ref{def:dwvp} and the circuit $\Gamma_H$ always outputs $\bot$.
Thus, with no loss of generality, we can assume that the solver makes no hint queries.
Finally, at most one verification query is allowed.

\subsection{The hardness amplification theorem}
In this section we briefly describe approach of Holenstein and Schoenebeck used to amplify hardness for interactive weakly verifiable puzzles.

Similarly as in the previous sections we define the $k$-wise direct product of puzzles.
\begin{definition}[$k$-wise direct product of interactive weakly verifiable puzzles, \cite{holenstein2011general}]
Let $g: \{0,1\}^{k} \rightarrow \{0,1\}$ be a monotone function and $(P,S)$ be a fixed interactive weakly verifiable puzzle.
The $k$-wise direct product of $(P,S)$ denoted by $(P^{(g)}, S^{(g)})$ is an interactive weakly verifiable puzzle where the problem poser and the solver
sequentially interact in $k$ rounds. In each round $(P,S)$ is used to produce an instance of the puzzle.
As a result circuits $\Gamma^{1}, \dotsc, \Gamma^{k}$ for $P$ are generated.
Finally, $P^{(g)}$ outputs the circuit $\Gamma^{(g)}(y_1, \dotsc, y_k) := g(\Gamma^{1}(y_1), \dotsc, \Gamma^{k}(y_k))$.
\end{definition}

The following hardness amplification theorem is proved in \cite{holenstein2011general}.
\begin{theorem}[Hardness amplification for interactive weakly verifiable puzzles, \cite{holenstein2011general}]
There exists an algorithm $\mathit{Gen}(C,g,\epsilon, \delta, n)$ which takes as input a solver circuit $C$ for the $k$-wise
direct product of $(P,S)$, a monotone function $g: \{0,1\}^{*} \rightarrow \{0,1\}$, and parameters $\epsilon,\delta,n$.
The algorithm $\mathit{Gen}$ outputs a solver circuit $D$ for $P$ such that the following holds. \\
If $C$ is such that
\begin{align*}
 \underset{\substack{\pi \in \{0,1\}^{kn}, \rho \in \{0,1\}^{*} \\ \Gamma^{(g)} := \langle P^{(g)}(\pi),C(\rho) \rangle_{P^{(g)}}}}{\Pr}
 \Big[\Gamma^{(g)}(\langle P^{(g)}(\pi), C(\rho) \rangle_C) = 1\Big] \geq \Pr_{u \leftarrow \mu_{\delta}^{(k)}} \Big[ g(u) = 1 \Big] + \epsilon,
\end{align*}
then $D$ satisfies almost surely over the randomness of $\Gen$
\begin{align*}
 \underset{\substack{\pi \in \{0,1\}^{n}, \rho \in \{0,1\}^{*} \\ \Gamma := \langle P(\pi),C(\rho) \rangle_{P}}}
 {\Pr}\Big[ \Gamma(\langle P(\pi), D(\rho)\rangle_{D}) = 1\Big] \geq \delta + \frac{\epsilon}{6k},
\end{align*}
Additionally, $\mathit{Gen}$ and $D$ only require oracle access to $g$ and $C$.
Furthermore, the running time of Gen is $poly(k, \frac{1}{\epsilon}, n)$ with oracle calls and $\mathit{Size}(D) \leq \mathit{Size}(C)\cdot\frac{6k}{\epsilon}\log(\frac{6k}{\epsilon})$.
\end{theorem}

The above definition does not impose any restrictions on the time complexity of the poser and the solver.
The algorithm $\mathit{Gen}$ is used to define a polynomial time reduction between a solver for the $k$-wise direct product of puzzles to a solver for a single puzzle.
In the previous sections we considered solvers for the $k$-wise direct product that
must solve all puzzles \cite{canetti2005hardness} or allow a fraction of puzzles to be solved incorrectly \cite{dodis2009security}.
In the above definition a more general approach is considered where a binary monotone function $g$ is used.

In Chapter \ref{ch:main_result} we use a very similar approach as Holenstein and Schoenebeck.
The difference is that the hardness amplification theorem proven in this thesis captures dynamic and interactive puzzles.

% in Definition \ref{def:dwvp}, the puzzles studied by Holenstein and Schoenebeck
% are non-dynamic. Thus, only a verification circuit $\Gamma$ is generated and no hint circuit is ever used.
% \begin{todo}
%   \textbf{TODO:} Compare to the work of CHS i.e. what we use there to compare the puzzles
% \end{todo}

% In order to estimate how much better a solver circuit $C$ for the $n$-wise direct product performs when
% a puzzle on the first position is fixed a notion of a surplus $S_{\pi^*, b}$ is introduced:
% \begin{align*}
% S_{\pi^*, b} := \Pr_{\pi^{(k)}} [ c \in \cG_b | \pi_1 = \pi^* ] - \Pr_{u \leftarrow \mu_{\delta}^{k}} [u \in \cG_b],
% \end{align*}
% which intuitively tells us how much better a solver $C$ performs when a first puzzle is always solved correctly (case when $b = 1$)
% or is always solved incorrectly (when $b = 0$).
% Now we observe the following fact. If there exists a puzzle which is fixed on the first position for which the surplus is bigger than
% $(1 - \frac{1}{k})\epsilon$ then we can fixed a this first puzzle and inductively solve the problem for the $(k-1)$-direct product of puzzles.
% \begin{todo}
%   \textbf{TODO:} what do it mean
% \end{todo}
% On the other hand, if we there is no such puzzle to fix on the first position it means that when we fix the first bit of $g$ then
% the performance between the solver $C$ and an algorithm that solves puzzle on each position independently with probability $\delta$
% is similar. However, we know that when the first bit of a function $g$ is not fixed then the solver $C$ is better.
% Thus, we draw a conclusion that the puzzle on the first position has to be solved unusually often.

% In a case it would be possible to fix all $(k-1)$ puzzles except the last one, the proof become trivial as we know that a function
% $g$ with the first $k-1$ bits fixed is either the identity or a constant function.
% %TODO write why it is true in this case

% Thus, it is enough to show that if it is not possible to find an estimate that is low then
% if we place an input puzzle on this position and we can find remaining $k-1$ puzzle such that
% $c \in \cG_1 \setminus \cG_0$ then this puzzle is solved with substantial probability.
% The whole proof is given in \cite{holenstein2011general}, and requires some probability manipulations.

% %cite tell more what is used in our proof which technique do we use.
% Our proof of the hardness amplification for dynamic interactive weakly verifiable puzzles closely follows the one given in \cite{holenstein2011general}.
% \begin{todo}
%   \textbf{TODO:} Why do we consider fixing 0/1 on the first position
%   \textbf{TODO:} How the technique is generalized to approach of CHS \\
%   \textbf{TODO:} Explain why we compare to such a probability i.e. why we consider with $\mu$ \\
%   \textbf{TODO:} Why the requirement for $g$ being a monotone function is interesting \\
%   \textbf{TODO:} Give the intuition behind the proof. \\
%   \textbf{TODO:} Give a proof under the simplified assumptions? \\
%   1) The algorithm always output an answer \\
%   2) For every pi the surpluses $S_{\pi^*, 0}$ and $S_{\pi^*, 1}$ re less than $(1-\frac{1}{k})\epsilon$.\\
% \end{todo}

% Local Variables:
% mode: latex
% TeX-master: "thesis"
% End:

%
\section{Limitations of Security Amplification}
%
% the main theorem
\chapter{Hardness amplification for weakly verifiable puzzles}
\label{ch:main_result}
In the previous chapter we gave an overview of the former studies of different types of weakly verifiable puzzles.
We also defined a~more general notion of a~dynamic interactive weakly verifiable puzzle.
The focus of this chapter is on a constructive proof of hardness amplification for dynamic interactive weakly verifiable puzzles.
In Section \ref{st:main_theorem} we formulate the theorem which is then proved in the succeeding sections.
We begin with constructing an algorithm that finds an efficiently computable function that is used
to partition the domain of hint and verification queries. Next, we give a proof of hardness amplification
under the assumption that the domain is well partitioned. Finally, in Section \ref{st:put_together}
we finish the proof by combining the previous steps.
%
% section{Main theorem}
\input{interactive_proof/interactive_proof}
%
% section{Domain partitioning}

% Let $hash:Q\rightarrow\{0,1,\dots, 2(h+v)-1\}$, then a set $P_{hash} \subseteq Q$,
% defined with respect to $hash$, is the set of preimages of $0$ for $hash$.
Let $hash:Q\rightarrow\{0,1,\dots, 2(h+v)-1\}$,
the idea is to partition $Q$ such that the set of preimages of $0$ for $hash$ contains $q \in Q$ on which $C$ is not allowed to ask hint queries,
and the first successful verification query $(q,y)$ of $C$ is such that $hash(q) = 0$.
Therefore, if $C$ makes a verification query $(q,y)$ such that $hash(q) = 0$, then we know that no hint query is ever asked on this $q$.

In the experiment $CanonicalSuccess$ we denote the $i$-th query of $C$ by $q_i$ if it is a hint query, and by $(q_i, y_i)$ if it is a verification query.
A solver circuit $C$ succeeds in the experiment $CanonicalSuccess$ if it asks a successful verification query $(q_j, y_j)$ such that $hash(q_j) = 0$,
and no hint query $q_i$ is asked before $(q_j, y_j)$ such that $hash(q_i) = 0$.
%
\begin{codeblock}
  \textbf{Experiment $CanonicalSuccess^{P, C, hash}(\pi, \rho)$}
  \medskip

  \hrule

  \medskip
  \textbf{Oracle:} A problem poser $P$, a solver circuit $C = (C_1, C_2)$.\\
  \IndII A function $hash: Q \rightarrow \{0, \dots, 2(h+v) - 1\}$.\\
  \textbf{Input:}  Bitstrings $\pi \in \{0,1\}^n$ and $\rho \in \{0,1\}^*$. \\
  \textbf{Output:} A bit $b \in \{0,1\}$.

  \medskip\hrule\medskip
  Run $\langle P(\pi), C_1(\rho) \rangle$ \\
  \IndI $(\Gamma_V, \Gamma_H) := \langle P(\pi), C_1(\rho) \rangle_{P}$ \\
  \IndI $x := \langle P(\pi), C_1(\rho) \rangle_{\text{trans}}$ \\ \\
  Run $C_2^{\Gamma_V, \Gamma_H} (x, \rho)$ \\
  \IndI Let $(q_j,y_j)$ be the first verification query of $C_2$ such that $\Gamma_v(q_j, y_j) = 1$.\\
  \IndI If $C_2$ does not succeed let $(q_j, y_j)$ be an arbitrary verification query.\\
  \\
  \textbf{If} $(\forall i < j :  hash(q_i) = 0)$ \And $(hash(q_j) = 1)$ \And $(\Gamma_V(q_j, y_j) = 1)$ \then \\
  \IndI \textbf{return} 1\\
  \textbf{else}\\
  \IndI \textbf{return} 0
\end{codeblock}
%
We define the \textit{canonical success probability} of a solver $C$ for $P$ with respect to a function $hash$ as
\begin{align}
 \underset{\pi, \rho}{\Pr}[CanonicalSuccess^{P, C, hash}(\pi, \rho) = 1].
\end{align}
%
For fixed $hash$ and a problem poser $P$ a \textit{canonical success} of $C$ for $\pi, \rho$ is a situation where $CanonicalSuccess^{P, C, hash}(\pi, \rho) = 1$.

We show that if a solver circuit $C$ for $P^{(g)}$ often succeeds in the experiment $Success$, then it is
also often successful in the experiment $CanonicalSuccess$.

\begin{lemma}\textbf{(\boldmath{Success probability in solving a $k$-wise direct product of $P^{(1)}$ with respect to a function $hash$.)}}
\label{lemma:hash_function_probability}
For fixed $P^{(g)}$ let $C$ be a solver for $P^{(g)}$ with the success probability at least $\gamma$,
asking at most $h$ hint queries and $v$ verification queries.
There exists a probabilistic algorithm \textbf{FindHash} that takes as input:
parameters $\gamma$, $n$, $k$, the number of verification queries $v$ and hint queries $h$, and has
oracle access to $C$ and $P^{(g)}$. Furthermore, \textbf{FindHash} runs in time $O((h+v)^4/\gamma^4)$,
and with high probability outputs a function $hash \in \cH$
such that the canonical success probability of $C$ with respect to $hash$ is at least $\frac{\gamma}{8(h+v)}$.
\end{lemma}
%
\begin{proof}
We fix $P^{(g)}$ and a solver $C$ for $P^{(g)}$ in the whole proof of Lemma \ref{lemma:hash_function_probability}.
Let $\cH$ be a family of pairwise independent hash functions $Q \rightarrow \{0,1, \dots,2(h+v)-1\}$.
For all $m,n \in \{1, \dots, (h+v)\}$ and $k,l \in \{0,1,\dots,2(h+v)-1\}$ by the pairwise independence property of $\cH$, we have
\begin{align}
  \label{eq:hash_pr}
 \forall q_m,q_n \in Q, q_m \neq q_n : \underset{\textit{hash} \leftarrow \cH}{\Pr}[hash(q_m) = k \mid hash(q_n) = l] = \underset{\textit{hash} \leftarrow \cH}{\Pr}[hash(q_m) = k] = \frac{1}{2(h+v)}.
\end{align}
%
Let $\cP_{Success}$ be a set containing all $(\pi^{(k)},\rho)$ for which $Success^{P^{(g)}, C}(\pi^{(k)}, \rho) = 1$.
We choose uniformly at random $hash \leftarrow \cH$, and consider the experiment $CanonicalSuccess^{P^{(g)}, C, hash}(\pi^{(k)}, \rho)$.
We are interested in the probability of the event that for a fixed $(\pi, rho) \in \cP_{Success}$ the solver $C$ succeeds canonically.
Let $(q_j, y_j)$ denote the first query such that $\Gamma_V(q_j, y_j) = 1$.
We have
\begin{align*}
  &\underset{\textit{hash} \leftarrow \cH}{\Pr}[hash(q_j) = 0 \land (\forall i < j : hash(q_i) \neq 0)]\\
  &\IndII = \underset{\textit{hash} \leftarrow \cH}{\Pr}[\forall i < j : hash(q_i) \neq 0 \mid hash(q_j) = 0] \underset{\textit{hash} \leftarrow \cH}{\Pr}[hash(q_j) = 0] \\
  &\IndII \stackrel{(\ref{eq:hash_pr})}{=} \frac{1}{2(h+v)}\left(1 -\underset{\textit{hash} \leftarrow \cH}{\Pr}[\exists i < j : hash(q_i) = 0 \mid hash(q_j) = 0] \right) \\
  &\IndII \stackrel{(\ref{eq:hash_pr})}{=} \frac{1}{2(h+v)} \left( 1 -\underset{\textit{hash} \leftarrow \cH}{\Pr}[\exists i < j : hash(q_i) = 0] \right) \\
  &\IndII \stackrel{(\text{u.b})}{\geq} \frac{1}{2(h+v)} \left( 1 - \sum_{i < j} \underset{\textit{hash} \leftarrow \cH}{\Pr}[hash(q_i) = 0] \right) \\
  &\IndII \stackrel{(\ref{eq:hash_pr})}{\geq} \frac{1}{4(h+v)}.
\end{align*}

We denote the set of those $(\pi^{(k)},\rho)$ for which $CanonicalSuccess^{P^{(g)}, C, hash}(\pi^{(k)}, \rho) = 1$ by $\cP_{Canonical}$.
For $(\pi^{(k)}, \rho)$ for which $C$ succeeds canonically, we have $Success^{P^{(g)}, C}(\pi^{(k)}, \rho) = 1$.
Hence, $\cP_{Canonical} \subseteq \cP_{Success}$, and we conclude
\begin{align}
  \label{ineq:hash_high_prob}
\underset{\substack{\textit{hash} \leftarrow \cH \\ \pi^{(k)}, \rho}}{\Pr}\left[CanonicalSuccess^{P^{(g)}, C, hash}(\pi^{(k)}, \rho) = 1\right] &=
\underset{{(\pi^{(k)},\rho) \in \cP_{Success}}}{\mathbb{E}}\left[\underset{\substack{\textit{hash} \leftarrow \cH}}{\Pr}[X = 1]\right] \notag\\
&\geq \frac{\gamma}{4(h+v)}.
\end{align}
%
\begin{codeblock}
  \textbf{Algorithm: FindHash}$(\gamma, n, k, h, v)$
  \medskip
  \hrule
  \medskip
  \textbf{Oracle:} A problem poser $P^{(g)}$, a solver circuit $C$ for $P^{(g)}$.\\
  \textbf{Input:} Parameters $\gamma, n, k, h,v $\\
  \textbf{Output:} A function $hash:Q \rightarrow \{0,1, \dots, 2(h+v)-1 \}$.
  \medskip\hrule\medskip
  Let $\cH$ be a family of pairwise independent hash functions $Q \rightarrow \{0,1,\dots, 2(h+v)-1\}$\\
  \For $i = 1$ \To $32(h+v)^2/\gamma^2$ \Do \\
  \IndI $hash \xleftarrow{\$} \cH$ \\
  \IndI $count := 0$ \\
  \IndI \For $j := 1$ to $32(h+v)^2/\gamma^2$ \Do \\
  \IndII $\pi^{(k)} \xleftarrow{\$} \{0,1\}^{kn} $\\
  \IndII $\rho \xleftarrow{\$} \{0,1\}^*$ \\
  \IndII \If $CanonicalSuccess^{P^{(g)}, C, hash}(\pi^{(k)}, \rho) = 1$ \then \\
  \IndIII $count := count + 1$\\
  \IndI \If $\frac{\gamma^2}{32(h+v)^2} count \geq \frac{\gamma}{6(h+v)}$ \then \\
  \IndII \return $hash$\\
  \return $\bot$
\end{codeblock}
We show that \textbf{FindHash} chooses $hash$ such that the canonical success probability of $C$
with respect to $hash$ is at least $\frac{\gamma}{4(h+v)}$ almost surely.
Let $\cH_{Good}$ denote a family of functions $hash \in \cH$ for which
\begin{align}
  \label{eq:hash_good}
\underset{\pi^{(k)}, \rho}{\Pr}\left[CanonicalSuccess^{P^{(g)}, C, hash}(\pi^{(k)}, \rho) = 1\right] \geq \frac{\gamma}{8(h+v)},
\end{align}
and $\cH_{Bad}$ be the family of functions $hash \in \cH$ such that
\begin{align}
  \label{eq:hash_bad}
\underset{\pi^{(k)}, \rho}{\Pr}\left[CanonicalSuccess^{P^{(g)}, C^{(\cdot, \cdot)}, hash}(\pi^{(k)}, \rho) = 1\right] \leq \frac{\gamma}{16(h+v)}.
\end{align}
%
Let $N$ denote the number of iterations of the inner loop of \textbf{FindHash}.
For a fixed $hash$, we define binary random variables $X_1, \dots, X_{N}$ such that
\begin{align*}
  X_i =
  \begin{cases}
    1 & \text{if in the $i$-th iteration of the inner loop $count$ is increased}\\
    0 & \text{otherwise.}
  \end{cases}
\end{align*}
We show now that \textbf{FindHash} is unlikely to return $hash \in \cH_{Bad}$.
For $hash \in \cH_{Bad}$ by (\ref{eq:hash_bad}) we have $\mathbb{E}_{\pi^{(k)},\rho}[X_i] \leq \frac{\gamma}{16(h+v)}$.
Therefore, for any fixed $hash \in \cH_{Bad}$ using the Chernoff bound we get
\footnote{For $X = \sum_{i=1}^N X_i$ and $0 < \delta \leq 1$ we use the Chernoff bounds in the form
$\Pr[X \geq (1+\delta) \mathbb{E}[X]] \leq e^{- \mathbb{E}[X] \delta^2/3}$ and
$\Pr[X \leq (1-\delta) \mathbb{E}[X]] \leq e^{- \mathbb{E}[X] \delta^2/2}$.}
\begin{align*}
  \underset{\pi^{(k)},\rho}{\Pr} \left[\frac{1}{N} \sum_{i=1}^{N} X_i \geq \frac{\gamma}{12(h+v)} \right] \leq
  \underset{\pi^{(k)}, \rho}{\Pr}\left[\frac{1}{N} \sum_{i=1}^{N} X_i \geq (1 + \frac{1}{4}) \mathbb{E}[X_i]\right] \leq
  e^{-{\frac{\gamma}{16(h+v)}} N /48} \leq e^{-\frac{1}{24}\frac{(h+v)}{\gamma}}.
\end{align*}
%
The probability that $hash \in \cH_{Good}$, when picked, is not returned amounts
\begin{align*}
  \underset{\pi^{(k)}, \rho}{\Pr}\left[\frac{1}{N} \sum_{i=1}^{N} X_i \leq \frac{\gamma}{12(h+v)}\right] \leq
  \underset{\pi^{(k)}, \rho}{\Pr}\left[\frac{1}{N} \sum_{i=1}^{N} X_i \leq (1 - \frac{1}{3})\mathbb{E}[X_i]\right]
  \leq e^{-{\frac{\gamma}{8(h+v)}} N / 18} \leq e^{-\frac{2}{9} \frac{(h+v)}{\gamma}},
\end{align*}
where we once more used the Chernoff bound.
Now we show that the probability of picking a $hash \in \cH_{Good}$ is at least $\frac{\gamma}{8(h+v)}$.
We proof this statement by contradiction. We assume otherwise, namely that
$\underset{hash \leftarrow \cH}{\Pr}[hash \in \cH_{Good}] < \frac{\gamma}{8(g+v)}$.
We have
\begin{align*}
  &\underset{\substack{hash \leftarrow \cH \\ \pi, \rho}}{\Pr}[CanonicalSuccess^{P,C,hash}(\pi, \rho) = 1] \\
  &\IndI = \underset{\substack{hash \leftarrow \cH \\ \pi, \rho}}{\Pr}[CanonicalSuccess^{P,C,hash}(\pi, \rho) = 1 \mid hash \in \cH_{Good}]
  \underset{hash \leftarrow \cH}{\Pr}[hash \in \cH_{Good}] \\
  & \IndII + \underset{\substack{hash \leftarrow \cH \\ \pi, \rho}}{\Pr}[CanonicalSuccess^{P,C,hash}(\pi, \rho) = 1 \mid hash \notin \cH_{Good}]
  \underset{hash \leftarrow \cH}{\Pr}[hash \notin \cH_{Good}] \\
  & \IndI \leq \underset{hash \leftarrow \cH}{\Pr}[hash \in \cH_{Good}] +
  \underset{\substack{hash \leftarrow \cH \\ \pi, \rho}}{\Pr}[CanonicalSuccess^{P,C,hash}(\pi, \rho) = 1 \mid hash \notin \cH_{Good}] \\
  & \IndI < \frac{\gamma}{8(h+v)} + \frac{\gamma}{8(h+v)} = \frac{\gamma}{4(h+v)}.
\end{align*}
But this contradicts (\ref{ineq:hash_high_prob}).
Finally, we show that \textbf{FindHash} picks in one of its iteration $hash \in \cH_{Good}$ almost surely.
Let $K$ be the number of iterations of the outer loop of \textbf{FindHash}.
Let $Y_i$ be a random variable for the event
that in the $i$-th iteration of the outer loop $hash \in \cH_{Good}$ is picked.
Using $\underset{hash \leftarrow \cH}{\Pr}[hash \in \cH_{Good}] < \frac{\gamma}{8(g+v)}$ and  $K \leq \frac{32(h+v)^2}{\gamma^2}$ we conclude
\begin{align*}
  \underset{hash \leftarrow \cH}{\Pr}[ \bigcap_{1 \leq i \leq K} Y_i ] \leq \left(1 - \frac{\gamma}{8(h+v)}\right)^{K}
    \leq e^{-\frac{\gamma}{8(h+v)} K}
    \leq e^{-\frac{4(h+v)}{\gamma}}.
\end{align*}
\end{proof}
%%% Local Variables:
%%% mode: latex
%%% TeX-master: "../master"
%%% End:

%
% section{Amplification proof for partitioned domain}
%
\subsection{The hardness amplification proof for partitioned domains}
\label{st:amplification_proof}
\begin{todo}
  \textbf{TODO:} Add short introduction
\end{todo}

Let $C := (C_1, C_2)$ be a two-phase solver circuit as in Definition \ref{def:dwvp}.
We write $C_2^{(\cdot, \cdot)}$ to emphasize that $C_2$ does not obtain direct access to the hint and verification oracles.
Instead, whenever $C_2$ asks a hint or verification query, it is answered explicitly
as in the following code excerpt of the circuit $\widetilde{C}_2$.

\begin{codeblock}
  \textbf{Circuit} $\widetilde{C}_2^{\Gamma_H, C_2, \hash} (x, \rho)$
  \medskip \hrule
  \textbf{Oracle:} A hint circuit $\Gamma_H$, a circuit $C_2$, \\ \IndII a function $\hash : \cQ \rightarrow \{0,1,\dots, 2(h+v)-1\}$. \\
  \textbf{Input:} Bitstrings $x \in \{0,1\}^{*}$, $\rho \in \{0,1\}^{*}$. \\
  \textbf{Output:} A pair $(q, y)$ where $q \in \cQ$ and $y \in \{0,1\}^{*}$.
  \medskip\hrule
  \Run $C_2^{(\cdot, \cdot)}(x, \rho)$ \\
  \IndI \If $C_2^{(\cdot, \cdot)}(x, \rho)$ asks a hint query on $q$ \Then\\
  \IndII \If $\hash(q) = 0$ \Then\\
  \IndIII \Return $\bot$\\
  \IndII \textbf{else}\\
  \IndIII answer the query of $C_2^{(\cdot, \cdot)}(x, \rho)$ using $\Gamma_H(q)$\\
  \\
  \IndI \If $C_2^{(\cdot, \cdot)}(x, \rho)$ asks a verification query $(q, y)$ \Then \\
  \IndII \If $\hash(q) = 0 $ \textbf{then} \\
  \IndIII \Return $(q, y)$ \\
  \IndII \textbf{else} \\
  \IndIII answer the verification query of $C_2^{(\cdot, \cdot)}(x, \rho)$ with 0 \\
  \Return $\bot$
\end{codeblock}
%
Given $C := (C_1, C_2)$ we define a circuit $\widetilde{C} := (C_1, \widetilde{C}_2)$.
Every hint query $q$ asked by $\widetilde{C}$ is such that $hash(q) \neq 0$.
Furthermore, $\widetilde{C}$ asks no verification queries, instead it returns $(q,y)$ such that $hash(q) = 0$ or $\bot$.

For fixed $\pi$, $\rho$, and $hash$ we say that the circuit $\widetilde{C}$ \textit{succeeds} if
for $x := \langle P(\pi), C_1(\rho) \rangle_{\mathit{trans}}$,
$(\Gamma_V, \Gamma_H) := \langle P(\pi), C_1(\rho) \rangle_{P}$, we have
\begin{align*}
\Gamma_V(\widetilde{C}_2^{\Gamma_H, C_2, \hash}(x, \rho)) = 1.
\end{align*}
%
\begin{todo}
  \textbf{TODO:} Show the intuitive meaning of this lemma
\end{todo}

\begin{lemma}
  \label{lemma:ctilda_c}
  For fixed $P$, $C := (C_1, C_2)$, and $hash$ it holds
  \begin{align*}
    \underset{\pi, \rho}{\Pr}[\CanonicalSuccess^{P, C, \hash}(\pi, \rho) = 1]
    \leq
    \mkern13mu
    \underset{
      \mathclap{
      \substack{
        \pi, \rho \\
        x := \langle P(\pi), C_1(\rho) \rangle_{\mathit{trans}} \\
        (\Gamma_V, \Gamma_H) := \langle P(\pi), C_1(\rho) \rangle_{P}}}}
  {\Pr}[\Gamma_V(\widetilde{C}_2^{\Gamma_H, C_2, hash}(x, \rho)) = 1].
  \end{align*}
\end{lemma}
%
\begin{todo}
  \textbf{TODO:} Give an overview of this Lemma
\end{todo}
\begin{proof}[Proof of Lemma \ref{lemma:ctilda_c}]
If for some $\pi$, $\rho$, and $\hash$ the circuit $C := (C_1, C_2)$ succeeds canonically,
then for the same $\pi$, $\rho$, and $\hash$ the circuit $\widetilde{C} := (C_1, \widetilde{C}_2)$ also succeeds.
Using this observation, we conclude that
\begin{align*}
  &\underset{\pi, \rho}{\Pr}\left[\CanonicalSuccess^{P, C, \hash}(\pi, \rho) = 1\right] \\
  &\IndII \leq
  \mkern33mu
    \underset{
      \mathclap{
        \substack{\pi, \rho \\
        x := \langle P(\pi), C_1(\rho) \rangle_{\mathit{trans}} \\
        (\Gamma_V, \Gamma_H) := \langle P(\pi), C_1(\rho) \rangle_{P}
      }}}
    {\mathbb{E}}\mkern13mu\big[\Gamma_V(\widetilde{C}_2^{\Gamma_H, C_2, \hash}(x, \rho)) = 1\big] \\
  &\IndII =
  \mkern33mu
    \underset{
      \mathclap{
        \substack{\pi, \rho \\
        x := \langle P(\pi), C_1(\rho) \rangle_{\mathit{trans}} \\
        (\Gamma_V, \Gamma_H) := \langle P(\pi), C_1(\rho) \rangle_{P}
      }}}
    {\Pr}\mkern13mu\big[\Gamma_V(\widetilde{C}_2^{\Gamma_H, C_2, \hash}(x, \rho)) = 1\big]
\end{align*}
\end{proof}
%
%
\begin{todo}
  \textbf{TODO:} intuition behind the lemma \\
  \textbf{TODO:} bases on \cite{holenstein2011general}
\end{todo}
The following Lemma is analogous to Theorem 10 from \cite{holenstein2011general}.
\begin{lemma}[Hardness amplification for a dynamic interactive weakly verifiable puzzle with respect to $\hash$]
  \label{lemma:sec_amp_for_p_hash}
  Let $g: \{0,1\}^{k} \rightarrow \{0,1\}$ be a monotone function, $P_n^{(1)}$ a fixed
  problem poser and $\widetilde{C} := (C_1, \widetilde{C}_2)$ a probabilistic two-phase circuit
  with oracle access to a function $\hash: \cQ \rightarrow \{0,1,\dots, 2(h+v)-1\}$
  and a solver $C := (C_1, C_2)$ for $P_{kn}^{(g)}$ that asks at most $h$ hint queries and $v$ verification queries.
  There exists an algorithm Gen that takes as input parameters $\varepsilon$, $\delta$, $n$, $k$,
  has oracle access to $P_n^{(1)}$,  $\widetilde{C}$, $\hash$, $g$,
  and outputs a probabilistic two-phase circuit $D := (D_1, D_2)$ such that the following holds: \\
  If $\widetilde{C}$ is such that
  \begin{align*}
    \underset{\mathclap{\substack{
          \pi^{(k)} \in \{0,1\}^{kn}, \rho \in \{0,1\}^{*} \\
          x:= \langle P^{(g)}(\pi^{(k)}), {C}_1(\rho) \rangle_{\mathit{trans}} \\
          (\Gamma_H^{(k)}, \Gamma_V^{(g)}) := \langle P^{(g)}(\pi^{(k)}), C_1(\rho) \rangle_{P^{(g)}}}}}
    {\Pr}[\Gamma_V^{(g)}(\widetilde{C}_2^{\Gamma_H^{(k)}, C_2, \hash}(x,\rho)) = 1]
    \geq \underset{u \leftarrow \mu_\delta^k}{\Pr}[g(u) = 1] + \varepsilon,
  \end{align*}
  then $D$ satisfies almost surely over the randomness of Gen
  \begin{align*}
    \underset{
      \mathclap{
      \substack{
        \pi \in \{0,1\}^{n}, \rho \in \{0,1\}^{*} \\
        x := \langle P^{(1)}(\pi), D_1^{\widetilde{C}}(\rho) \rangle_{\mathit{trans}} \\
        (\Gamma_H, \Gamma_V) := \langle P^{(1)}(\pi), D_1^{\widetilde{C}}(\rho) \rangle_{P^{(1)}}}}}
    {\Pr}[\Gamma_V(D_2^{P^{(1)}, \widetilde{C}, \hash, g, \Gamma_H}(x, \rho)) = 1] \geq \delta + \frac{\varepsilon}{6k}.
  \end{align*}
  Furthermore, $D$
  asks at most $\frac{6k}{\epsilon}\log\left(\frac{6k}{\epsilon}\right) h$ hint queries and no verification queries.
  Finally, the running time of $\mathit{Gen}$ is polynomial in $k, \frac{1}{\varepsilon}, n$ with oracle calls.
\end{lemma}
We note that the circuit $D$ from Lemma \ref{lemma:sec_amp_for_p_hash} does not ask any verification queries,
instead it outputs a pair $(q, y)$ such that $\hash(q) = 0$ or $\bot$.

Before we give the proof of Lemma \ref{lemma:sec_amp_for_p_hash} we define additional algorithms.
First, in the following code listing the algorithm $\Gen$ from Lemma \ref{lemma:sec_amp_for_p_hash} is defined.
The procedures and circuits used by $\Gen$ are presented on the succeeding code listings.
\begin{codeblock}
  \textbf{Algorithm} $\Gen^{P^{(1)}, \widetilde{C}, g, \mathit{hash}}(\epsilon, \delta, n, k)$
  \medskip \hrule
  \textbf{Oracle:} A poser $P^{(1)}$, a solver $\widetilde{C}$ for $P^{(g)}$, functions $g: \{0,1\}^{k} \rightarrow \{0,1\}$, $hash:\cQ \rightarrow \{0,1, \dots, 2(h + v) - 1\}$. \\
  \textbf{Input:}  Parameters $\epsilon$, $\delta$, $n$, $k$.\\
  \textbf{Output:} A circuit $D$.
  \medskip\hrule
  \For $i:=1$ \To $\frac{6k}{\epsilon}n$ \Do \\
  \IndI $\pi^* \xleftarrow{\$} \{0,1\}^{n}$\\
  \IndI $\widetilde{S}_{\pi^*,0} := \text{EstimateSurplus}^{P^{(1)},  \widetilde{C}, g, hash}(\pi^*, 0, k, \epsilon, \delta,n)$\\
  \IndI $\widetilde{S}_{\pi^*,1} := \text{EstimateSurplus}^{P^{(1)},  \widetilde{C}, g, hash}(\pi^*, 1, k, \epsilon, \delta,n)$\\
  \IndI \If $ \exists b \in \{0,1\}: \widetilde{S}_{\pi^*,b} \geq (1 - \frac{3}{4k}) \epsilon$ \Then \\
  \IndII Let $C_1'$ have oracle access to $\widetilde{C}$, and have hard-coded $\pi^*$. \\
  \IndII Let $\widetilde{C}_2'$ have oracle access to $\widetilde{C}$, and have hard-coded $\pi^*$. \\
  \IndII $\widetilde{C}' := (C_1', \widetilde{C}_2')$ \\
  \IndII $g'(b_2, \dots, b_k) := g(b, b_2, \dots, b_k)$\\
  \IndII\Return $Gen^{P^{(1)}, \widetilde{C}', g', hash}(\epsilon, \delta, n, k-1)$ \\
  \textit{// all estimates are lower than $(1-\frac{3}{4k})\varepsilon$}\\
  \Return $D^{P^{(1)}, \widetilde{C}, hash, g}$
\end{codeblock}
We are interested in the probability that for $u \leftarrow \mu_{\delta}^k$ and a bit $b$ we have $g(b,u_2, \dotsc, u_k) = 1$.
The estimate of this probability is calculated by the algorithm EstimateFunctionProbability.
%
\begin{codeblock}
  \textbf{Algorithm} $\text{EstimateFunctionProbability}^{g}(b, k, \epsilon, \delta, n)$
  \medskip\hrule
  \textbf{Oracle:} A function $g : \{0,1\}^{k} \rightarrow \{0,1\}$.\\
  \textbf{Input:} A bit $b \in \{0,1\}$, parameters $k$, $\epsilon$, $\delta$, $n$. \\
  \textbf{Output:} An estimate $\widetilde{g}_b$ of $\Pr_{u \leftarrow \mu_{\delta}^{k}}[g(b,u_2, \dotsc, u_k) = 1]$.
  \medskip\hrule
  \For $i:=1$ \To $N := \frac{64k^2}{\epsilon^2} n$ \Do \\
  \IndI $u \leftarrow \mu_{\delta}^{k}$ \\
  \IndI $g_i := g(b, u_2, \dotsc, u_k)$ \\
  \Return $\frac{1}{N} \sum_{i=1}^{N} g_i$
\end{codeblock}
%
For fixed $\pi^{(k)}$, $\rho$, and $hash$ we say that the circuit $\widetilde{C} := (C_1, \widetilde{C}_2)$ \textit{succeeds on the $i$-th coordinate}
if for $x := \langle P^{(g)}(\pi^{(k)}), C_1(\rho) \rangle_{\mathit{trans}}$, $(\Gamma_V^{(g)}, \Gamma_H^{(k)}) := \langle P^{(g)}(\pi), C_1(\rho) \rangle_{P^{(g)}}$ and
$(q, y^{(k)}) := \widetilde{C}_2(x, \rho)$ we have
\begin{align*}
  \Gamma_V^i(q, y_i) = 1.
\end{align*}
%
\begin{lemma}
  \label{lemma:estimate_of_g}
  The algorithm $\text{EstimateFunctionProbability}^{g}(b, k, \epsilon, \delta, n)$ outputs an estimate $\widetilde{g}_b$
  such that $| \widetilde{g}_b - \Pr_{u \leftarrow \mu_{\delta}^{k}}\left[g(b,u_2, \dots, u_k) = 1\right] | \leq \frac{\epsilon}{8k}$ almost surely.
\end{lemma}
%
\begin{proof}
We fix notation as in the code excerpt of the algorithm EstimateFunctionProbability.
Let us define independent and identically distributed binary random variables $K_1, K_2, \dots, K_N$
such that for each $1 \leq i \leq N$ the random variable $K_i$ takes value $g_i$. We use the Chernoff bound to obtain
\begin{align*}
  &\Pr \Bigl[ \Bigl| \widetilde{g}_b - \Pr_{u \leftarrow \mu_{\delta}^{k}}\left[g(b,u_2, \dots, u_k) = 1\right] \Bigr| \geq \frac{\epsilon}{8k} \Bigr]\\
  &\IndII = \Pr \Bigl[\Bigl|\Bigl(\frac{1}{N} \sum_{i=1}^N K_i \Bigr) - \mathbb{E}_{u \leftarrow \mu_{\delta}^k}[g(b,u_2, \dots, u_k)]\Bigr|
    \geq \frac{\epsilon}{8k} \Bigr] \leq 2 \cdot e^{-n/3}.
\end{align*}
\end{proof}
%
The algorithm $\text{EvalutePuzzles}^{P^{(1)}, \widetilde{C}, \hash}(\pi^{(k)}, \rho, n, k)$
evaluates which of the $k$ puzzles of the $k$-wise direct product of $P^{(1)}$ are solved successfully by $\widetilde{C}(\rho) := (C_1,\widetilde{C}_2)(\rho)$.
To decide whether the $i$-th puzzle of the $k$-wise direct product is solved successfully we need to gain access to the verification circuit
for the puzzle generated in the $i$-th round of the interaction between $P^{(g)}$ and $\widetilde{C}$.
Therefore, the algorithm EvalutePuzzles runs $k$ times $P^{(1)}(\pi_i)$ to simulate the interaction with
$C_1(\rho)$ where in each round of interaction a fresh random bitstring $\pi_i \in \{0,1\}^{n}$ is used.

Let us introduce the additional notation.
We write $\langle P^{(1)}(\pi_i), C_1(\rho)\rangle^i$ to denote the $i$-th round of the sequential interaction.
Let $\langle P^{(1)}(\pi_i), C_1(\rho)\rangle^i_{P^{(1)}}$ be the output of $P^{(1)}(\pi_i)$ in the $i$-th round.
Finally, we write $\langle P^{(1)}(\pi_i), C_1(\rho)\rangle^i_{\mathit{trans}}$ to denote the transcript of communication in the $i$-th round.
We note that the $i$-th round of the interaction between $P^{(1)}$ and $C_1$ is well defined only if all previous rounds have been executed before.

For simplicity of the notation in the code excerpts of circuits $C_2$, $D_2$, and EvalutePuzzles we omit superscripts of some oracles.

Exemplary, for $\widetilde{C}_2^{\Gamma_H^{(k)}, C, \hash}$ we omit the superscript $C$ and instead write $\widetilde{C}_2^{\Gamma_H^{(k)}, \hash}$.
We make sure that it is clear from the context which oracles are used.

\begin{todo}
  \textbf{TODO:} Introduce this algorithm
\end{todo}

\begin{codeblock}
  \textbf{Algorithm} $\text{EvaluatePuzzles}^{P^{(1)}, \widetilde{C}, \hash}(\pi^{(k)}, \rho, n, k)$
  \medskip \hrule
  \textbf{Oracle:}  A problem poser $P^{(1)}$, a solver circuit $\widetilde{C} = (C_1, \widetilde{C}_2)$ for $P^{(g)}$,\\
  \IndII a function $hash : \cQ \rightarrow \{0,1,\dots, 2(h+v)-1\}$.\\
  \textbf{Input:} Bitstrings $\pi^{(k)} \in \{0,1\}^{kn}$, $\rho \in \{0,1\}^{*}$, parameters $n$, $k$.\\
  \textbf{Output}: A tuple $(c_1, \dots, c_k) \in \{0,1\}^{k}$.
  \medskip\hrule
  %
  \For $i:=1$ \To $k$ \Do \IndII \textit{//simulate $k$ rounds of interaction} \\
  \IndI $(\Gamma_V^{i}, \Gamma_H^{i}) := \langle P^{(1)}(\pi_i), C_1(\rho) \rangle_{P^{(1)}}^i$\\
  \IndI $x_i := \langle P^{(1)}(\pi_i), C_1(\rho) \rangle^i_{\mathit{trans}}$ \\
  $x := (x_1, \dots, x_k)$ \\
  $\Gamma_H^{(k)} := (\Gamma_H^1, \dotsc, \Gamma_H^k)$ \\
  $(q, y_1, \dots, y_k) := \widetilde{C}_2^{\Gamma_H^{(k)}, hash} (x, \rho)$ \\
  \If $(q, y_1, \dots, y_k) = \bot$ \Then \\
  \IndI \Return $(0, \dotsc, 0)$ \\
  $(c_1, \dotsc, c_k) := (\Gamma_V^{1}(q, y_1), \dotsc, \Gamma_V^{k}(q, y_k))$\\
  \Return $(c_1, \dotsc, c_k)$
\end{codeblock}
%
All puzzles used by EvalutePuzzles are generated internally. Thus, the algorithm can answer all queries of $\widetilde{C}_2$ itself.

We are interested in the success probability of $\widetilde{C}$ with the bitstring $\pi_1$ fixed to $\pi^*$ where
the fact whether $\widetilde{C}$ succeeds in solving the input puzzle defined by $P^{(1)}(\pi_1)$ placed on the first position is neglected,
and instead a bit $b$ is used. More formally, we define the surplus $S_{\pi^*, b}$ as
\begin{align}
  \label{eq:s_pi_b}
S_{\pi^*, b} = \underset{\pi^{(k)}, \rho}{\Pr}\left[g(b, c_2, \dots, c_k) = 1 \mid \pi_1 = \pi^*\right] - \underset{u \leftarrow \mu^{k}_{\delta}}{\Pr}\left[g(b, u_2, \dots, u_k) = 1\right],
\end{align}
where $(c_2, c_3, \dotsc, c_k)$ is obtained as in EvalutePuzzles.

The algorithm EstimateSurplus returns an estimate $\widetilde{S}_{\pi^*, b}$ for $S_{\pi^*, b}$.
%
\begin{codeblock}
  \textbf{Algorithm} $\text{EstimateSurplus}^{P^{(1)}, \widetilde{C}, g, \hash}(\pi^*, b, k, \epsilon, \delta, n)$
  \medskip\hrule
  \textbf{Oracle:} A problem poser $P^{(1)}$, a circuit $\widetilde{C}$ for $P^{(g)}$, functions \\
  \IndII $g: \{0,1\}^{k} \rightarrow \{0,1\}$ and  $\hash : \cQ \rightarrow \{0,1,\dots, 2(h+v)-1\}$.\\
  \textbf{Input:} A bistring $\pi^* \in \{0,1\}^{n}$, a bit $b \in \{0,1\}$, parameters $k$, $\epsilon$, $\delta$, $n$.\\
  \textbf{Output:} An estimate $\widetilde{S}_{\pi^*, b}$ for $S_{\pi^*, b}$.
  \medskip\hrule
  \For $i:=1$ \To $N := \frac{64k^2}{\epsilon^2}n$ \Do \\
  \IndI $(\pi_{2}, \dots, \pi_k) \xleftarrow{\$} \{0,1\}^{(k-1)n}$\\
  \IndI $\rho \xleftarrow{\$} \{0,1\}^{*}$\\
  \IndI $(c_1, \dots, c_k) := \text{EvalutePuzzles}^{P^{(1)}, \widetilde{C}, \hash}((\pi^*, \pi_2, \dots, \pi_k), \rho, n, k)$\\
  \IndI $\widetilde{s}_{\pi^*,b}^i := g(b, c_{2}, \dots, c_k)$\\
  $\widetilde{g}_b := \text{EstimateFunctionProbability}^{g}(b, k, \epsilon, \delta, n)$ \\
  \textbf{return} $\Bigl(\frac{1}{N} \sum_{i=1}^N \widetilde{s}_{\pi^*,b}^i \Bigr) - \widetilde{g}_b$
\end{codeblock}
%
\begin{lemma}
  \label{lemma:surplus_estimate}
The estimate $\widetilde{S}_{\pi^*,b}$ returned by EstimateSurplus differs from $S_{\pi^*, b}$ by at most $\frac{\epsilon}{4k}$ almost surely.
\end{lemma}

\begin{proof}
We use the union bound and similar argument as in Lemma \ref{lemma:estimate_of_g}
which yields that $\frac{1}{N} \sum_{i=1}^{N} \widetilde{s}_{\pi^*,b}^i$ differs from
$\mathbb{E}[g(b, c_2, \dots, c_k)]$ by at most $\frac{\epsilon}{8k}$ almost surely. Together, with Lemma $\ref{lemma:estimate_of_g}$ we conclude that the surplus estimate
returned by EstimateSurplus differs from $S_{\pi^*,b}$ by at most $\frac{\epsilon}{4k}$ with probability at least $1 - 2e^{-n}$.
\end{proof}
%
We define the following solver circuit $C' = (C_1', C_2')$ for the $(k-1)$--wise direct product of $P^{(1)}$.
\begin{todo}
  \textbf{TODO:} Give more intuition why we need this circuit and where it is used
\end{todo}
\begin{codeblock}
  \textbf{Circuit} $C_1'^{\widetilde{C}, P^{(1)}}(\rho)$
  \medskip \hrule
  \textbf{Oracle:} A solver circuit $\widetilde{C} = (C_1, \widetilde{C}_2)$ for $P^{(g)}$, a poser $P^{(1)}$. \\
  \textbf{Input:}  A bitstring $\rho \in \{0,1\}^{*}$. \\
  \textbf{Hard-coded:} A bitstring $\pi^* \in \{0,1\}^{n}$.
  \medskip\hrule
  Simulate $\langle P^{(1)}(\pi^*), C_1(\rho)\rangle^1$ \\
  Use $C_1(\rho)$ for the remaining $k-1$ rounds of interaction.
\end{codeblock}
%
\begin{codeblock}
  \textbf{Circuit} $\widetilde{C}_2'^{\Gamma_H^{(k-1)}, \widetilde{C}, \hash}(x^{(k-1)}, \rho)$
  \medskip \hrule
  \textbf{Oracle:} A hint oracle $\Gamma_H^{(k-1)} := (\Gamma_H^{2}, \dots, \Gamma_H^{k})$,\\
  \IndII a solver circuit $\widetilde{C} = (C_1, \widetilde{C}_2)$ for $P^{(g)}$, \\
  \IndII a function $\hash: \cQ \rightarrow \{0,1,\dots, 2(h+v)-1\}$. \\
  \textbf{Input:}  A transcript of $k-1$ rounds of interaction \\
  \IndII $x^{(k-1)} := (x_2, \dotsc, x_{k}) \in \{0,1\}^{*}$, a bitstring $\rho \in \{0,1\}^{*}$.\\
  \textbf{Hard-coded:} A bitstring $\pi^* \in \{0,1\}^{n}$. \\
  \textbf{Output:} A tuple $(q, y_2, \dots, y_k)$.
  \medskip\hrule
  Simulate $\langle P^{(1)}(\pi^*), C_1(\rho) \rangle^{1}$ \\
  \IndI $(\Gamma_H^*, \Gamma_V^*) := \langle P^{(1)}(\pi^*), C_1(\rho) \rangle^{1}_{P^{(1)}}$ \\
  \IndI $x^* := \langle P^{(1)}(\pi^*), C_1(\rho) \rangle^{1}_{\mathit{trans}}$ \\
  $\Gamma_H^{(k)} := (\Gamma_H^*, \Gamma_H^{2}, \dots, \Gamma_H^{k})$ \\
  $x^{(k)} := (x^*, x_2, \dots, x_{k})$ \\
  $(q, y_1, \dots, y_k) := \widetilde{C}_2^{\Gamma_H^{(k)}, \mathit{hash}}(x^{(k)}, \rho)$ \\
  \Return $(q, y_2, \dots, y_k)$
\end{codeblock}
%
We are ready to define the solver circuit $D = (D_1, D_2)$ for $P^{(1)}$ output by $\Gen$.
%
\begin{codeblock}
  \textbf{Circuit} $D_1^{\widetilde{C}}(r)$
  \medskip \hrule
  \textbf{Oracle:} A solver circuit $\widetilde{C} = (C_1, \widetilde{C}_2)$ for $P^{(g)}$.\\
  \textbf{Input:} A pair $r := (\rho, \sigma)$ where $ \rho \in \{0,1\}^{*}$ and $\sigma \in \{0,1\}^{*}$.
  \medskip\hrule
  Interact with the problem poser $\langle P^{(1)}, C_1(\rho) \rangle^1$. \\
  Let $x^* := \langle P^{(1)}, C_1(\rho) \rangle^1_{\mathit{trans}}$.
\end{codeblock}
%
\begin{codeblock}
  \textbf{Circuit} $D_2^{P^{(1)}, \widetilde{C}, \mathit{hash}, g,  \Gamma_H}(x^*, r)$
  \medskip \hrule
  \textbf{Oracle:} A poser $P^{(1)}$, a solver circuit $\widetilde{C} = (C_1, \widetilde{C}_2)$ for $P^{(g)}$, \\
  \IndII functions $hash : \cQ \rightarrow \{0,1, \dots, 2(h+v)-1\}$, $g:\{0,1\}^k \rightarrow \{0,1\}$, \\
  \IndII a hint circuit $\Gamma_H$ for $P^{(1)}$. \\
  \textbf{Input:} A communiation transcript $x^* \in \{0,1\}^{*}$, a bitstring $r := (\rho, \sigma)$ \\
  \IndII where $\rho \in \{0,1\}^{*}$ and $\sigma \in \{0,1\}^{*}$\\
  \textbf{Output}: A pair $(q, y^*)$.
  \medskip \hrule
  \For at most $\frac{6k}{\epsilon} \log(\frac{6k}{\epsilon})$ iterations \Do \\
  \IndI $(\pi_2, \dots, \pi_k) \leftarrow$ read next $(k-1)\cdot n$ bits from $\sigma$ \\
  \IndI Use $x^*$ to simulate the first round of interaction of $C_1(\rho)$ \\
  \IndI with the problem poser $P^{(1)}$.\\
  \IndI \For $i:=2$ \To $k$ \Do \\
  \IndII \Run $\langle P^{(1)}(\pi_i), C_1(\rho)\rangle^i$ \\
  \IndIII $(\Gamma_V^{i}, \Gamma_H^{i}) := \langle P^{(1)}(\pi_i), C_1(\rho) \rangle^i_{P^{(1)}}$ \\
  \IndIII $x_i := \langle P^{(1)}(\pi_i), C_1(\rho) \rangle^i_{\mathit{trans}}$ \\
  \IndI $\Gamma_H^{(k)}(q) := (\Gamma_H(q), \Gamma_H^{2}(q), \dots, \Gamma_H^{k}(q))$ \\
  \IndI $(q, y^*, y_2, \dots, y_k) := \widetilde{C}_2^{\Gamma_H^{(k)}, \hash}((x^*, x_2, \dotsc, x_k), \rho)$\\
  \IndI $(c_2, \dots, c_k) := (\Gamma_V^2(q, y_2), \dotsc, \Gamma_V^{k}(q, y_k))$ \\
  \IndI \If $g(1, c_{2}, \dots, c_k) = 1$ \And $g(0,c_{2}, \dots, c_k) = 0$ \Then \\
  \IndII \Return $(q, y^*)$ \\
  \Return $\bot$
%
\end{codeblock}
%
%
\begin{proof}[of Lemma \ref{lemma:sec_amp_for_p_hash}]
First, let us consider the case where $k=1$. The function $g: \{0,1\} \rightarrow \{0,1\}$ is either the identity or a constant function.
In the latter case, when $g$ is a constant function, Lemma \ref{lemma:sec_amp_for_p_hash} is vacuously true.
If $g$ is the identity function, then the circuit $D$ returned by Gen directly uses $\widetilde{C}$ to find a solution.
From the assumptions of Lemma \ref{lemma:sec_amp_for_p_hash} it follows that $\widetilde{C}$ succeeds with probability at least
$\delta + \epsilon$. Hence, $D$ trivially satisfies Lemma~\ref{lemma:sec_amp_for_p_hash}.

For the general case, we consider two possibilities.
Namely, either Gen in one of the iterations finds an estimate with high surplus such that $\widetilde{S}_{\pi, b} \geq (1-\frac{3}{4k})\epsilon$ and recurses,
or in all iterations it fails and outputs the circuit~$D$.

If it is possible to find an estimate with high surplus, then we introduce a new monotone function $g': \{0,1\}^{k-1} \rightarrow \{0,1\}$
such that $g'(b_2, \dots, b_k) := g(b, b_2, \dots, b_k)$ and a new circuit $\widetilde{C}' = (C_1', \widetilde{C}_2')$
with oracle access to $\widetilde{C} := (C_1, \widetilde{C}_2)$.
W apply Lemma \ref{lemma:surplus_estimate} and conclude that almost surely it holds
\begin{align*}
S_{\pi^*,b} \geq \widetilde{S}_{\pi^*, b} - \frac{\epsilon}{4k} \geq \Bigl(1 - \frac{1}{k}\Bigr)\epsilon.
\end{align*}
It follows that $\widetilde{C}'$ succeeds in solving the $(k\!-\!1)$--wise direct product of puzzles with probability at least
\begin{align*}
\Pr_{u \leftarrow \mu^{(k-1)}_{\delta}}[g'(u_1,\dots, u_{k-1} )] + \Bigl(1 - \frac{1}{k}\Bigr)\epsilon.
\end{align*}
We see that in this case $\widetilde{C}'$ satisfies the conditions of Lemma \ref{lemma:sec_amp_for_p_hash} for the $(k\!-\!1)$--wise direct product of puzzles.
Therefore, the recursive call to Gen with access to $g'$ and $\widetilde{C}$ returns $D = (D_1, D_2)$ that with high probability satisfies
\begin{align}
  \underset{
    \mathclap{
      \substack{
        \pi, \rho \\
        x := \langle P^{(1)}(\pi), D_1^{\widetilde{C}}(\rho) \rangle_{\mathit{trans}} \\
        (\Gamma_H, \Gamma_V) := \langle P^{(1)}(\pi), D_1^{\widetilde{C}}(\rho) \rangle_{P^{(1)}}}}}
  {\Pr}\Big[\Gamma_V\big(D_2^{P^{(1)}, \widetilde{C}, \hash, g, \Gamma_H}(x, \rho)\big) = 1\Big]
  &\geq \delta + \Bigl(1 - \frac{1}{k}\Bigr)\frac{\epsilon}{6(k-1)} \notag\\
  &= \delta + \frac{\epsilon}{6k}.
\end{align}
%
Let us bring our attention to the case where none of the estimates is greater than $(1-\frac{3}{4k})\epsilon$.
If all surpluses $S_{\pi,0}$ and $S_{\pi,1}$ were lower than $(1-\frac{1}{k})\epsilon$, then it would mean that $\widetilde{C}$
does not succeed on the remaining $k-1$ puzzles with much higher probability than an algorithm that solves each puzzle
independently with success probability $\delta$. However, from the assumptions of Lemma~\ref{lemma:sec_amp_for_p_hash}
we know that on all $k$ puzzles the success probability of $\widetilde{C}$ is higher.
Hence, we suspect that the first puzzle is correctly solved unusually often.
It remains to show that the fact that $\Gen$ fails to find a surplus estimate that is large implies that
with high probability there are only few surpluses greater than $(1-\frac{1}{k})\epsilon$ and their influence
is can be neglected. Additionally, we have to show that the circuit $D$ finds outputs an answer almost surely.

We fix notation as in the code listing of the circuit $D_2$.
Let us consider a single iteration of the outer loop of $D_2$ where values $\pi_2, \dotsc, \pi_k$ are fixed.
We write $\pi_1$ to denote randomness used by the problem poser to generate the input puzzle.
Furthermore, we define $c_1 := \Gamma_V(q,y_1)$ where $\Gamma_V$ is the verification circuit generated
by $P^{(1)}(\pi_1)$ in the first phase when interacting with $D_1(r)$.
We write $c := (c_1, c_2, \dotsc, c_k)$, and for $b \in \{0,1\}$ we define a set
\begin{align*}
\cG_{b}~:=~\big\{(b_1, b_2, \dots, b_k) : g(b, b_2, \dots, b_k) = 1 \big\}.
\end{align*}
Using this notation we express
\begin{align}
  \label{eqs:set_g}
  \underset{u \leftarrow \mu_{\delta}^k}{\Pr}[u \in \cG_b] = \underset{u \leftarrow \mu_{\delta}^k}{\Pr}[g(b, u_2, \dots, u_k) = 1]\notag\\
 \underset{\pi^{(k)}, \rho}{\Pr}[c \in \cG_b] = \underset{\pi^{(k)}, \rho}{\Pr}[g(b, c_2, \dots, c_k) = 1].
\end{align}
Let us fix randomness $\pi_1$ used by the problem poser to generate the input puzzle to $\pi^*$.
We use \eqref{eq:s_pi_b} and \eqref{eqs:set_g} to obtain
\begin{multline}
\label{eq:diff_s01}
\underset{u \leftarrow \mu_{\delta}^k}{\Pr}[u \in \cG_1] - \underset{u \leftarrow \mu_{\delta}^k}{\Pr}[u \in \cG_0] \\
 = \underset{\pi^{(k)}, \rho}{\Pr}[c \in \cG_1 \mid \pi_1 = \pi^*] - \underset{\pi^{(k)}, \rho}{\Pr}[c \in \cG_0 \mid \pi_1 = \pi^*] - (S_{\pi^*, 1} - S_{\pi^*,0})
\end{multline}
By monotonicity of $g$ it holds $\cG_0 \subseteq \cG_1$, and we write \eqref{eq:diff_s01} as
\begin{align}
  \label{eq:diff_s01_next}
  \underset{u \leftarrow \mu_{\delta}^k}{\Pr}[u \in \cG_1 \setminus \cG_0] = \underset{\pi^{(k)}, \rho}{\Pr}[c \in \cG_1 \setminus \cG_0 \mid \pi_1 = \pi^*] - (S_{\pi^*,1} - S_{\pi^*,0}).
\end{align}
Let us multiply both sides of \eqref{eq:diff_s01_next} by
\begin{align*}
\underset{
  \mathclap{
    \substack{r \\ x^* := \langle P^{(1)}(\pi^*), D_1(r) \rangle_{\mathit{trans}}
    \\ (\Gamma_V, \Gamma_H) := \langle P^{(1)}(\pi^*), D_1(r) \rangle_{P^{(1)}} }}}
{\Pr}\mkern13mu [\Gamma_V(D_2(x^*, r)) = 1]
 \mkern11mu / \underset{u \leftarrow \mu_{\delta}^k}{\Pr}[ u \in \cG_1 \setminus\cG_0],
\end{align*}
%
which yields
\begin{align}
\label{eq:pr_d_succ_0}
&\IndII\underset{
  \mathclap{
    \substack{r \\ x^* := \langle P^{(1)}(\pi^*), D_1(r) \rangle_{\mathit{trans}} \\ (\Gamma_V, \Gamma_H) := \langle P^{(1)}(\pi^*), D_1(r) \rangle_{P^{(1)}} }}}
{\Pr}\mkern13mu[\Gamma_V(D_2(x^*, r)) = 1] \notag\\
%
&\IndIII = \mkern13mu
  \underset{
    \mathclap{
      \substack{r \\ x^* := \langle P^{(1)}(\pi^*), D_1 (r) \rangle_{\mathit{trans}} \\ (\Gamma_V, \Gamma_H) := \langle P^{(1)}(\pi^*), D_1 (r) \rangle_{P^{(1)}} }}}
  {\Pr}\mkern13mu[\Gamma_V(D_2(x^*, r)) = 1]
  \underset{\pi^{(k)},\rho}{\Pr}[c \in \cG_1 \setminus \cG_0 \mid \pi_1 = \pi^*]
\frac{1}{\underset{u \leftarrow \mu_{\delta}^k}{\Pr}[ u \in \cG_1 \setminus \cG_0]}\notag\\
%
&\IndIIII - \mkern13mu
\underset{
  \mathclap{
  \substack{r \\ x^* := \langle P^{(1)}(\pi^*), D_1(r) \rangle_{\mathit{trans}} \\ (\Gamma_V, \Gamma_H) := \langle P^{(1)}(\pi^*), D_1(r) \rangle_{P^{(1)}} }}}
{\Pr}\mkern13mu[\Gamma_V(D_2(x^*, r)) = 1](S_{\pi^*,1} - S_{\pi^*,0})
\frac{1}{\underset{u \leftarrow \mu_{\delta}^k}{\Pr}[ u \in \cG_1 \setminus\cG_0]}.
\end{align}
Let us study the first summand of \eqref{eq:pr_d_succ_0}. First, we have
\begin{align}
  \label{eq:pr_gamma_v_0}
  \IndII &\underset{
    \mathclap{
      \substack{r \\
        x^* := \langle P^{(1)}(\pi^*), D_1 (r) \rangle_{\mathit{trans}} \\
        (\Gamma_V, \Gamma_H) := \langle P^{(1)}(\pi^*), D_1(r) \rangle_{P^{(1)}} }}}
  {\Pr}\mkern13mu[\Gamma_V(D_2(x^*, r)) = 1] \notag\\
  &\IndI = \underset{
    \mathclap{
      \substack{r \\
        x^* := \langle P^{(1)}(\pi^*), D_1 (r) \rangle_{\mathit{trans}} \\
        (\Gamma_V, \Gamma_H) := \langle P^{(1)}(\pi^*), D_1(r) \rangle_{P^{(1)}} }}}
  {\Pr}[\Gamma_V(D_2(x^*, r)) = 1 | D_2(x^*,r) \neq \bot]
  \underset{\mathclap{\substack{r \\ x^* = \langle P^{(1)}(\pi^*), D_1(r) \rangle_{\mathit{trans}}}}}{\Pr}[D_2(x^*,r) \neq \bot] \notag\\
  &\IndI \stackrel{(*)}{=}
  \underset{\pi^{(k)}, \rho}{\Pr}[c_1 = 1 \mid c \in \cG_1 \setminus \cG_0, \pi_1 = \pi^*]
  \underset{\mathclap{\substack{r \\ x^* = \langle P^{(1)}(\pi^*), D_1(r) \rangle_{\mathit{trans}}}}} {\Pr}[D_2(x^*,r) \neq \bot]
\end{align}
where in $(*)$ we use the observation that $D_2(x^*, r) \neq \bot$ occurs if and only if $D_2(x^*, r)$ finds $\pi^{(k)}$ such that $c \in \cG_1 \setminus \cG_0$.
Furthermore, conditioned on $c \in \cG_1 \setminus \cG_0$ we have that $\Gamma_V(D_2(x^*,r)) = 1$ happens if and only if $c_1 = 1$.
Inserting \eqref{eq:pr_gamma_v_0} to the numerator of the first summand of (\ref{eq:pr_d_succ_0}) yields
\begin{align}
  \label{ineq:start_for_case}
\IndI &\underset{
  \mathclap{
  \substack{r \\
    x^* := \langle P^{(1)}(\pi^*), D_1 (r) \rangle_{\mathit{trans}} \\
    (\Gamma_V, \Gamma_H) := \langle P^{(1)}(\pi^*), D_1(r) \rangle_{P^{(1)}} }}}
{\Pr}\mkern13mu[\Gamma_V(D_2(x^*, r)) = 1]
\underset{\pi^{(k)},\rho}{\Pr}[c \in \cG_1 \setminus \cG_0 \mid \pi_1 = \pi^*] \notag\\
  &\IndI = \underset{\mathclap{\substack{r
      \\ x^* = \langle P^{(1)}(\pi^*), D_1(r) \rangle_{\mathit{trans}}}}}
  {\Pr}\mkern13mu[D_2(x^*,r) \neq \bot]
  \underset{\pi^{(k)}, \rho}{\Pr}[c_1 = 1 \mid c \in \cG_1 \setminus \cG_0, \pi_1 = \pi^*]
  \underset{\pi^{(k)}, \rho}{\Pr}[c \in \cG_1 \setminus \cG_0 \mid \pi_1 = \pi^*].
\end{align}
We consider the following two cases. First, if $\Pr_{\pi^{(k)}, \rho}[ c \in \cG_1 \setminus \cG_0 \mid \pi_1 = \pi^*] \leq \frac{\epsilon}{6k}$ then
\begin{align}
  \label{ineq:case_0}
  \underset{\pi^{(k)}, \rho}{\Pr}[c_1 = 1 \mid c \in \cG_1 \setminus \cG_0, \pi_1 = \pi^*] \underset{\pi^{(k)}, \rho}{\Pr}[c \in \cG_1 \setminus \cG_0 \mid \pi_1 = \pi^*] \leq \frac{\epsilon}{6k}.
\end{align}
In case when $\Pr_{\pi^{(k)}, \rho}[c \in \cG_1 \setminus \cG_0 \mid \pi_1 = \pi^*] > \frac{\epsilon}{6k}$ the circuit $D_2$ outputs $\bot$
if and only if it fails in all $\frac{6k}{\epsilon} \log(\frac{6k}{\epsilon})$ iterations to find $\pi^{(k)}$ such that $c \in \cG_1 \setminus \cG_0$
which happens with probability
\begin{align}
  \label{ineq:case_1}
\underset{
  \mathclap{
    \substack{
      r \\
      x^* := \langle P^{(1)}(\pi^*), D_1(r) \rangle_{\mathit{trans}}}}}
{\Pr}[D_2(x^*,r) = \bot]
\leq \Big(1 - \frac{\epsilon}{6k}\Big)^{\frac{6k}{\epsilon}\log(\frac{6k}{\epsilon})} \leq \frac{\epsilon}{6k}.
\end{align}
We conclude that in both aforementioned cases using \eqref{ineq:start_for_case}, \eqref{ineq:case_0} and \eqref{ineq:case_1} the following holds
\begin{align}
  \label{ineq:first_part}
  &\underset{
    \mathclap{
    \substack{r \\
      x^* := \langle P^{(1)}(\pi^*), D_1(r) \rangle_{\mathit{trans}}}}}
  {\Pr}\mkern13mu[D_2(x^*,r) \neq \bot]
  \underset{\pi^{(k)}, \rho}{\Pr}[c_1 = 1 \mid c \in \cG_1 \setminus \cG_0, \pi_1 = \pi^*]
  \underset{\pi^{(k)}, \rho}{\Pr}[c \in \cG_1 \setminus \cG_0 \mid \pi_1 = \pi^*] \notag\\
  &\IndII \stackrel{\hphantom{(\ref{eq:s_pi_b})}}{\geq}
  \underset{\pi^{(k)}, \rho}{\Pr}[c_1 = 1 \mid c \in \cG_1 \setminus \cG_0, \pi_1 = \pi^*]\underset{\pi^{(k)}, \rho}
  {\Pr}[c \in \cG_1 \setminus \cG_0 \mid \pi_1 = \pi^*] - \frac{\epsilon}{6k} \notag\\
  &\IndII \stackrel{\hphantom{(\ref{eq:s_pi_b})}}{=}
  \underset{\pi^{(k)}, \rho}{\Pr}[c_1 = 1 \land c \in \cG_1 \setminus \cG_0 \mid \pi_1 = \pi^*] - \frac{\epsilon}{6k} \notag\\
  &\IndII \stackrel{\hphantom{(\ref{eq:s_pi_b})}}{=}
  \underset{\pi^{(k)}, \rho}{\Pr}[g(c) = 1 \mid \pi_1 = \pi^*] -  \underset{\pi^{(k)}, \rho}{\Pr}[c \in \cG_0 \mid \pi_1 = \pi^*] - \frac{\epsilon}{6k} \notag\\
  &\IndII \stackrel{(\ref{eq:s_pi_b})}{=}
   \underset{\pi^{(k)}, \rho}{\Pr}[g(c) = 1 \mid \pi_1 = \pi^*] -  \underset{u \leftarrow \mu_{\delta}^{(k)}}{\Pr}[u \in \cG_0]  - S_{\pi^*,0} - \frac{\epsilon}{6k}.
\end{align}
We insert \eqref{ineq:first_part} into \eqref{eq:pr_d_succ_0} and calculate the expected value over $\pi^*$ which yields
\begin{align}
  \label{ineq:ex_pr_gamma}
\underset{
  \mathclap{
    \substack{\pi, r \\ x := \langle P^{(1)}(\pi), D_1(r) \rangle_{\mathit{trans}} \\ (\Gamma_V, \Gamma_H) := \langle P^{(1)}(\pi), D_1(r) \rangle_{P^{(1)}} }}}
{\Pr}[\Gamma_V(D_2(x, r)) = 1]
&\geq \mathbb{E}_{\pi^*}\left[\frac{\Pr_{\pi^{(k)}, \rho}[g(c) = 1 | \pi_1 = \pi^*]
  - \Pr_{u \leftarrow \mu_{\delta}^{(k)}}[u \in \cG_0] - \frac{\epsilon}{6k}}{\Pr_{u \leftarrow \mu_{\delta}^{(k)}}[u \in \cG_1 \setminus \cG_0]}\right] \notag\\
&- \mathbb{E}_{\pi^*}\Bigl[\Bigl(
\underset{\mathclap{
  \substack{r \\ x^* := \langle P^{(1)}(\pi^*), D_1(r) \rangle_{\mathit{trans}} \\ (\Gamma_V, \Gamma_H) := \langle P^{(1)}(\pi^*), D_1(r) \rangle_{P^{(1)}} }}}
{\Pr}[\Gamma_V(D_2(x^*, r)) = 1](S_{\pi^*,1} - S_{\pi^*,0})
 + S_{\pi^*,0}\Bigr)
\frac{1}{\underset{u \leftarrow \mu_{\delta}^k}{\Pr}[ u \in \cG_1 \setminus\cG_0]}\Bigr].
\end{align}
We now show that if Gen does not recurse, then the majority of estimates is low almost surely.
Let us assume that
\begin{align}
\underset{\pi}{\Pr}\left[\left(S_{\pi,0} \leq (1 - \frac{1}{2k})\epsilon\right) \land \left( S_{\pi,1} \leq (1-\frac{1}{2k})\epsilon\right)\right] < 1 - \frac{\epsilon}{6k},
\end{align}
then Gen recurses almost surely, because the probability that
Gen does not find $\widetilde{S}_{\pi, b} \geq (1-\frac{3}{4k})\epsilon$ in all of the $\frac{6k}{\epsilon}n$ iterations is at most
\begin{align*}
  \Bigl(1 - \frac{\epsilon}{6k}\Bigr)^{\frac{6k}{\epsilon}n} \leq e^{-n}
\end{align*}
almost surely, where we used Lemma \ref{lemma:surplus_estimate}.
Therefore, under the assumption that Gen does not recurse with high probability it holds
\begin{align}
\underset{\pi, \rho}{\Pr}\left[\left(S_{\pi,0} \leq (1 - \frac{1}{2k})\epsilon\right) \land \left( S_{\pi,1} \leq (1-\frac{1}{2k})\epsilon\right)\right] \geq 1 - \frac{\epsilon}{6k}.
\end{align}
Let us define a set
\begin{align}
  \cW = \left\{ \pi :  \left(S_{\pi,0} \leq (1 - \frac{1}{2k})\epsilon\right) \land \left( S_{\pi,1} \leq (1-\frac{1}{2k})\epsilon \right) \right\}.
\end{align}
Additionally, let $\overline{\cW}$ denote the complement of $\cW$.
We bound the numerator of the second summand in (\ref{ineq:ex_pr_gamma})
\begin{align}
  \label{ineq:second_eq}
&\mathbb{E}_{\pi^*}\Big[ S_{\pi^*,0}
\mkern23mu
+
\mkern23mu
\underset{
  \mathclap{
  \substack{r \\ x^* := \langle P^{(1)}(\pi^*), D_1(r) \rangle_{\mathit{trans}}
    \\ (\Gamma_V, \Gamma_H) := \langle P^{(1)}(\pi^*), D_1 (r) \rangle_{P^{(1)}} }}}
{\Pr}\mkern13mu[\Gamma_V(D_2(x^*, r)) = 1]
(S_{\pi^*,1} - S_{\pi^*,0})\Big] \notag\\
%
&\IndII = \mathbb{E}_{\pi^*}\Bigl[ S_{\pi^*,0}
\mkern23mu + \mkern23mu
\underset{
  \mathclap{
  \substack{r \\ x^* := \langle P^{(1)}(\pi^*), D_1(r) \rangle_{\mathit{trans}}
    \\ (\Gamma_V, \Gamma_H) := \langle P^{(1)}(\pi^*), D_1 (r) \rangle_{P^{(1)}} }}}
{\Pr}\mkern13mu[\Gamma_V(D_2(x^*, r) = 1]
  (S_{\pi^*,1} - S_{\pi^*,0}) \bigm| \pi^* \in \overline{\cW}\Bigr] \Pr_{\pi^*}[\pi^* \in \overline{\cW}]\notag\\
&\IndIII +  \mathbb{E}_{\pi^*}\Bigl[ S_{\pi^*,0} \mkern13mu + \mkern13mu
\underset{
  \mathclap{
  \substack{r \\ x^* := \langle P^{(1)}(\pi^*), D_1(r) \rangle_{\mathit{trans}}
    \\ (\Gamma_V, \Gamma_H) := \langle P^{(1)}(\pi^*), D_1 (r) \rangle_{P^{(1)}} }}}
{\Pr}\mkern13mu[\Gamma_V(D_2(x^*, r)) = 1]
(S_{\pi^*,1} - S_{\pi^*,0})  \bigm| \pi^* \in \cW\Bigr] \Pr_{\pi^*}[\pi^* \in \cW] \notag\\
&\IndII \leq \frac{\epsilon}{6k} + \mathbb{E}_{\pi^*}\Bigl[ S_{\pi^*,0} \mkern23mu + \mkern23mu
\underset{
  \mathclap{
  \substack{r \\ x := \langle P^{(1)}(\pi^*), D_1(r) \rangle_{\mathit{trans}}
    \\ (\Gamma_V, \Gamma_H) := \langle P^{(1)}(\pi^*), D_1 (r) \rangle_{P^{(1)}} }}}
{\Pr}\mkern13mu\big[\Gamma_V(D_2^{\widetilde{C}}(x^*, r)) = 1\big]
\big(\bigl(1 - \frac{1}{2k}\bigr)\epsilon - S_{\pi^*,0}\big)  \bigm| \pi^* \in \cW \Bigr] \notag\\
% \Pr_{\pi^*}[\pi^* \in \cW] \notag\\
&\IndII \leq \frac{\epsilon}{6k} + (1 - \frac{1}{2k})\epsilon = (1 - \frac{1}{3k}) \epsilon.
\end{align}
Finally, we insert \eqref{ineq:second_eq} into \eqref{ineq:ex_pr_gamma} which yields
\begin{align*}
  \IndI
\underset{
  \mathclap{
  \substack{\pi, \rho \\ x := \langle P^{(1)}(\pi), D_1(\rho) \rangle_{\text{trans}}
    \\ (\Gamma_V, \Gamma_H) := \langle P^{(1)}(\pi), D_1 (\rho) \rangle_{P^{(1)}} }}}
{\Pr}\big[\Gamma_V(D_2(x, \rho)) = 1\big]
&\geq \underset{\pi^*}{\mathbb{E}}\left[\frac{{\Pr}_{\pi^{(k)}, \rho}[g(c) = 1 \mid \pi_1 = \pi^*] -
{\Pr}_{u \leftarrow \mu_{\delta}^{k}}[u \in G_0] - (1 - \frac{1}{6k})\epsilon} {\Pr_{u \leftarrow \mu_{\delta}^{k}}[u \in \cG_1 \setminus \cG_0]}\right] \notag.
 \end{align*}
 From the assumptions of Lemma \ref{lemma:sec_amp_for_p_hash} it follows that
 \begin{align}
   \label{eq:lemma_assum}
   \Pr_{\pi^{(k)}, \rho} [g(c) = 1] \geq \Pr_{u \leftarrow \mu_{\delta}^{(k)}}[g(u) = 1] + \epsilon.
 \end{align}
We observe that
\begin{align}
  \label{eq:gu_rel}
\underset{u \leftarrow \mu_{\delta}^k}{\Pr}[g(u) = 1]
&= \Pr[u \in \cG_0 \lor ( u \in \cG_1 \setminus \cG_0 \land u_1 = 1)] \notag\\
&= \Pr[u \in \cG_0] + \delta \Pr[u \in \cG_1 \setminus \cG_0].
\end{align}
 Using \eqref{eq:gu_rel} and \eqref{eq:lemma_assum} we obtain
 \begin{align}
   \label{eq:proof_final}
   \IndI
\underset{
  \mathclap{
  \substack{\pi, \rho \\ x := \langle P^{(1)}(\pi), D_1(\rho) \rangle_{\text{trans}}
    \\ (\Gamma_V, \Gamma_H) := \langle P^{(1)}(\pi), D_1 (\rho) \rangle_{P^{(1)}} }}}
{\Pr}\mkern13mu[\Gamma_V(D_2(x, \rho)) = 1]
 &\stackrel{\hphantom{\eqref{eq:gu_rel}}}{\geq} \frac{ {\Pr}_{u \leftarrow \mu_{\delta}^{k}}[g(u) = 1] + \epsilon -
 \Pr_{u \leftarrow \mu_{\delta}^{k}}[u \in \cG_0] - (1 - \frac{1}{6k})\epsilon} {\Pr_{u \leftarrow \mu_{\delta}^{k}}[u \in \cG_1 \setminus \cG_0]} \notag\\
 &\stackrel{\eqref{eq:gu_rel}}{\geq} \frac{\epsilon + \delta\Pr_{u \leftarrow \mu_{\delta}^{k}}[u \in \cG_1 \setminus \cG_0] - (1 - \frac{1}{6k})\epsilon}
{\Pr_{u \leftarrow \mu_{\delta}^{k}}[u \in \cG_1 \setminus \cG_0]} \geq \delta + \frac{\epsilon}{6k}.
\end{align}
Clearly, the running time of $\Gen$ is bounded by some polynomial $p(k, \frac{1}{\epsilon}, n)$ with oracle calls.
Furthermore, the algorithm $\Gen$ outputs a circuit $D$ such that it satisfies with probability at least $1 - \p(k, \frac{1}{\epsilon}, n) \cdot 2^n$
the statement of Lemma \ref{lemma:sec_amp_for_p_hash}. This concludes the proof of Lemma~\ref{lemma:sec_amp_for_p_hash}.
\end{proof}
%
%%% Local Variables:
%%% mode: latex
%%% TeX-master: "../master"
%%% End:

%
% subsection{Putting it together}
%
\begin{proof}[Theorem \ref{th:sec_amp_for_dwvp}]
We define the following circuits.
%
\begin{codeblock}
  \textbf{Circuit} $\widetilde{D}_2^{D, P^{(1)}, \hash, g, \Gamma_V, \Gamma_H}(x, \rho)$
  \medskip
  \hrule
  \medskip
  \textbf{Oracle:} A circuit $D :=(D_1, D_2)$ from Lemma \ref{lemma:sec_amp_for_p_hash}, a problem poser $P^{(1)}$, \\
  \IndII functions $\hash: Q \rightarrow \{0,1, \dots, 2(h+v) - 1\}$, $g: \{0,1\}^{k} \rightarrow \{0,1\}$ \\
  \IndII a verification oracle $\Gamma_V$, a hint oracle $\Gamma_H$.\\
  \textbf{Input:}  Bitstrings $x \in \{0,1\}^{*}$, $\rho \in \{0,1\}^{*}$.
  \medskip\hrule\medskip
  %
  $(q, y) := D_2^{P^{(1)}, \widetilde{C}, \hash, g, \Gamma_H}(x, \rho)$ \\
  Make a verification query to $\Gamma_V$ using $(q,y)$
\end{codeblock}
%
\begin{codeblock}
  \textbf{Algorithm} $\widetilde{\text{Gen}}^{P^{(1)}, g, C}(n, \epsilon, \delta, k, h, v)$
  \medskip \hrule \medskip
  \textbf{Oracle:} A problem poser $P^{(1)}$, a function $g: \{0,1\}^{k} \rightarrow \{0,1\}$, \\
  \IndII a solver circuit $C$ for $P^{(g)}$.  \\
  \textbf{Input:} Parameters $n$, $\epsilon$, $\delta$, $k$, $h$, $v$.
  \medskip\hrule\medskip
  %
  $hash := \text{FindHash}((h+v)\epsilon, n, h, v)$ \\
  Let $\widetilde{C} := (C_1, \widetilde{C}_2)$ be as in Lemma \ref{lemma:ctilda_c} with oracle access to $C$, $\hash$. \\
  $D := Gen^{P^{(1)},  \widetilde{C},  g, \hash}(\epsilon, \delta, n, k)$ \\
  \Return $\widetilde{D} := (D_1, \widetilde{D}_2)$
\end{codeblock}
%
We show that Theorem \ref{th:sec_amp_for_dwvp} follows from Lemma \ref{lemma:hash_function_probability} and Lemma \ref{lemma:sec_amp_for_p_hash}.
We fix $P^{(1)}$, $g$, $P^{(g)}$. Given a solver circuit $C = (C_1, C_2)$, asking $h$ hint queries and $v$ verification queries, such that
\begin{align*}
    \underset{\pi^{(k)}, \rho}{\Pr}\left[\Success^{P^{(g)}, C}(\pi^{(k)}, \rho) = 1\right] \geq 16(h+v)\left(\underset{u \leftarrow \mu_\delta^k}{\Pr}\left[g(u) = 1\right] + \varepsilon\right)
\end{align*}
we satisfy conditions of Lemma \ref{lemma:hash_function_probability}. Therefore, $\widetilde{\text{Gen}}$ can use the algorithm FindHash to find $\hash$ such that
\begin{align*}
    \underset{\pi^{(k)}, \rho}{\Pr}\left[\CanonicalSuccess^{P^{(g)}, C, \hash}(\pi^{(k)}, \rho) = 1\right] \geq \underset{u \leftarrow \mu_\delta^k}{\Pr}\left[g(u) = 1\right] + \varepsilon
\end{align*}
almost surely.
By Lemma \ref{lemma:ctilda_c} we know that it is possible to build $\widetilde{C} = (C_1, \widetilde{C}_2)$ such that
\begin{align*}
    \underset{
      \mathclap {
      \substack{\pi^{(k)}, \rho \\
        x := \langle P^{(g)}(\pi^{(k)}), C_1(\rho) \rangle_{\mathit{trans}} \\
        (\Gamma_V^{(g)}, \Gamma_H^{(k)}) := \langle P^{(g)}(\pi^{(k)}), C_1(\rho) \rangle_{P^{(g)}}
      }}}
    {\Pr}\mkern13mu[\Gamma_V^{(g)}(\widetilde{C}_2^{\Gamma_H^{(k)}, C_2, \hash}(x, \rho)) = 1]
    \geq
\underset{u \leftarrow \mu_\delta^k}{\Pr}\left[g(u) = 1\right] + \varepsilon.
\end{align*}
Now, we use Gen to obtain a circuit $D = (D_1, D_2)$, which by Lemma \ref{lemma:sec_amp_for_p_hash} satisfies
\begin{align}
  \label{eq:succ_prob_d}
    \underset{
      \mathclap{
      \substack{\pi, \rho \\ x := \langle P^{(1)}(\pi), D_1^{\widetilde{C}}(\rho) \rangle_{\mathit{trans}} \\
        (\Gamma_H, \Gamma_V) := \langle P^{(1)}(\pi), D_1^{\widetilde{C}}(\rho) \rangle_{P^{(1)}}}}}
    {\Pr}\mkern13mu[\Gamma_V(D_2^{P^{(1)}, \widetilde{C}, hash, g, \Gamma_H}(x, \rho)) = 1] \geq (\delta + \frac{\varepsilon}{6k})
\end{align}
almost surely.
Finally, $\widetilde{\text{Gen}}$ outputs $\widetilde{D} = (D_1, \widetilde{D}_2)$ with oracle access to $D$, $P^{(1)}$, $hash$, $g$ such that with high probability it holds
\begin{align*}
    \underset{\pi, \rho}{\Pr}\left[\Success^{P^{(1)},\widetilde{D}}(\pi, \rho) = 1\right] \geq (\delta + \frac{\varepsilon}{6k}).
\end{align*}
The running time of FindHash is $\mathit{poly}(h,v,\frac{1}{\epsilon},n)$ with oracle calls and of Gen $\mathit{poly(k, \frac{1}{\epsilon}, n)}$ with oracle calls.
Thus, the overall running time of $\widetilde{\mathit{Gen}}$ is  $\mathit{poly(k,\frac{1}{\epsilon},h,v,n,t)}$ with oracle calls.
Furthermore, the circuit $\widetilde{D}$ asks at most $\frac{6k}{\epsilon} \log(\frac{6k}{\epsilon})h$ hint queries and one verification query.
Finally, we have $\mathit{Size}(\widetilde{D}) \leq \mathit{Size}(C) \cdot \frac{6k}{\epsilon}$.
This finishes the proof of Theorem \ref{th:sec_amp_for_dwvp}.
\end{proof}

%%% Local Variables:
%%% mode: latex
%%% TeX-master: "../master"
%%% End:

%
\section{Discussion}
%
%
\appendix
\chapter{Appendix}
\section{Concentration Bounds}
\begin{lemma}[Chernoff Bounds]
For independent Bernoulli distributed random variables $X_1, \dotsc, X_n$ with $X := \sum_{i=1}^n X_i$
and $\Pr[X_i = 1] = p_i$ for all $ 1 \leq  i \leq n$ the following inequalities hold
\begin{gather}
\label{ineq:ch0}
\Pr[X \geq (1+\delta) \mathbb{E}[X]] \leq e^{- \mathbb{E}[X] \delta^2/3} \\
\label{ineq:ch1}
\Pr[X \leq (1-\delta) \mathbb{E}[X]] \leq e^{- \mathbb{E}[X] \delta^2/2},
% \label{ineq:ch2}
% \Pr[|X - \mathbb{E}[X]| \geq \delta \mathbb{E}[X]] \leq 2 e^{- \mathbb{E}[X] \delta^2 / 3},
\end{gather}
where $0 \leq \delta \leq 1$.

For independent and identically distributed Bernoulli random variables $X_1, \dotsc, X_n$ with $X := \sum_{i=1}^n X_i$
where $\Pr[X_i = 1] = p$ for some $p \in (0,1)$ and for all $ 1 \leq  i \leq n$ and
for any $\epsilon > 0$ we have
\begin{gather}
\label{ineq:ch3}
\Pr[X \geq (p + \varepsilon)n] < e^{-\frac{\epsilon^2n}{2}} \\
\label{ineq:ch2}
\Pr[| X - \mathbb{E}[X]| \geq \epsilon\mathbb{E}[X]] \leq 2e^{-\frac{\epsilon^2 \mathbb{E}[X]}{2}}.
\end{gather}
\end{lemma}
\vspace*{\fill}
\pagebreak

\section{The Proof of Lemma \ref{lemma:sec_amp_for_p_hash} Under the Simplifying Assumptions}
\label{st:proofSimAssm}
We prove Lemma \ref{lemma:sec_amp_for_p_hash} in the case where $\widetilde{Gen}$ does not find the randomness for which the surplus is large.
For the sake of simplicity we make the following assumptions
\begin{gather}
  \label{pr:always_d}
\underset{
  \mathclap{
  \substack{\pi, \rho \\ x := \langle P^{(1)}(\pi), D_1(\rho) \rangle_{\trans}}}}
{\Pr}[\widetilde{D}_2(x,\rho) \neq \bot] = 1 \\
  \label{pr:low_surpluses}
\forall \pi \in \{0,1\}^{n} : S_{\pi, b} \leq \Big(1 - \frac{1}{k}\Big)\epsilon.
\end{gather}
In \eqref{pr:always_d} we assume that $\widetilde{D}$ always outputs an answer, and in \eqref{pr:low_surpluses} that all surpluses are low.
In the complete proof of Lemma \ref{lemma:sec_amp_for_p_hash} these assumptions fail only slightly such that it is possible
to obtain the desired result. However, the calculations are fairly lengthy.
The following simplified proof is intended to give the intuition behind the full proof.
We have
\begin{align*}
\underset{
  \mathclap{
  \substack{\pi, \rho \\ x := \langle P^{(1)}(\pi), D_1(\rho) \rangle_{\trans}
    \\ (\Gamma_V, \Gamma_H) := \langle P^{(1)}(\pi), D_1 (\rho) \rangle_{P^{(1)}} }}}
{\Pr}[\Gamma_V&(\widetilde{D}_2(x,\rho)) = 1]
  \overset{\eqref{pr:always_d}}{=}
\underset{
  \mathclap{
  \substack{\pi, \rho \\ x := \langle P^{(1)}(\pi), D_1(\rho) \rangle_{\trans}
    \\ (\Gamma_V, \Gamma_H) := \langle P^{(1)}(\pi), D_1 (\rho) \rangle_{P^{(1)}} }}}
{\Pr}[\Gamma_V(\widetilde{D}_2(x,\rho)) = 1 \mid \widetilde{D}_2(x,\rho) \neq \bot] \\
  &\overset{\hphantom{a}(*)\hphantom{a}}{=} \underset{\pi^{(k)}}{\Pr}[c_1 = 1 \mid c \in \cG_1 \setminus \cG_0] \\
  &\overset{\eqref{eq:diff_s01_next}}{=} \underset{\pi^*}{\mathbb{E}}
  \left[\frac{\Pr_{\pi^{(k)}}[c_1 = 1 \mid c \in \cG_1 \setminus \cG_0] \big(\Pr_{\pi^{(k)}}[c \in \cG_1 \setminus \cG_0] - (S_{\pi^*,1} - S_{\pi^*,0})\big)}
  {\underset{u \leftarrow \mu_{\delta}^{k}}{\Pr}[u \in \cG_1 \setminus \cG_0]}\right] \\
  % &\overset{\hphantom{a(*)}\hphantom{a}}{\geq} \underset{\pi^*}{\mathbb{E}}
  % \left[\frac{\Pr[c_1 = 1 \land c \in \cG_1 \setminus \cG_0]  - (S_{\pi^*,1} - S_{\pi^*,0})}
  % {\underset{u \leftarrow \mu_{\delta}^{k}}{\Pr}[u \in \cG_1 \setminus \cG_0]}\right] \\
  &\overset{\eqref{pr:low_surpluses}}{\geq} \underset{\pi^*}{\mathbb{E}}
  \left[\frac{\Pr_{\pi^{(k)}}[g(c)=1] - \Pr_{\pi^{(k)}}[c \in \cG_0]  - (1 - \frac{1}{k})\epsilon + S_{\pi^*,0}}
  {\underset{u \leftarrow \mu_{\delta}^{k}}{\Pr}[u \in \cG_1 \setminus \cG_0]}\right] \\
  &\overset{\eqref{eq:s_pi_b}}{=} \underset{\pi^*}{\mathbb{E}}
  \left[\frac{\Pr_{\pi^{(k)}}[g(c)=1] - \Pr_{u \leftarrow \mu_{\delta}^{k}}[u \in \cG_0] - S_{\pi^*,0}  - (1 - \frac{1}{k})\epsilon + S_{\pi^*,0}}
  {\underset{u \leftarrow \mu_{\delta}^{k}}{\Pr}[u \in \cG_1 \setminus \cG_0]}\right]\\
  &\overset{\substack{\eqref{eq:gu_rel}\\\eqref{eq:lemma_assum}}}{\geq} \underset{\pi^*}{\mathbb{E}}
  \left[\frac{\delta \Pr_{u \leftarrow \mu_{\delta}^{k}}[u \in \cG_1 \setminus \cG_0] + \epsilon  - (1 - \frac{1}{k})\epsilon}
  {\underset{u \leftarrow \mu_{\delta}^{k}}{\Pr}[u \in \cG_1 \setminus \cG_0]}\right] \geq \delta + \frac{\epsilon}{k},
  %
  % &\overset{\hphantom{\eqref{eq:diff_s01_next}}}{=} \frac{\Pr_{\pi^{(k)}}[c_1 = 1 \land c \in \cG_1 \setminus \cG_0]}{\Pr_{\pi^{(k)}}[c \in \cG_1 \setminus \cG_0]} \\
  % &\overset{\eqref{eq:diff_s01_next}}{=} \underset{\pi^*}{\mathbb{E}}\left[\frac{\Pr_{\pi^{(k-1)}}[c_1^* = 1 \land c \in \cG_1 \setminus \cG_0]
  %   \big(\Pr_{\pi^{(k-1)}}[c \in \cG_0 \setminus \cG_1] - (S_{\pi^*,1} - S_{\pi^*,0})\big)}
  % {\underset{\pi^{(k-1)}}{\Pr}[c \in \cG_0 \setminus \cG_1] \underset{u \leftarrow \mu_{\delta}^{k}}{\Pr}[u \in \cG_1 \setminus \cG_0]}\right]  \\
  % &\overset{\hphantom{\eqref{eq:diff_s01_next}}}{\geq} \underset{\pi^*}{\mathbb{E}}\left[\frac{\Pr_{\pi^{(k-1)}}[c_1^* = 1 \land c \in \cG_1 \setminus \cG_0]
  %   \big(\Pr_{\pi^{(k-1)}}[c \in \cG_0 \setminus \cG_1] - (1 - \frac{1}{k})\epsilon\big)}
  % {\underset{\pi^{(k-1)}}{\Pr}[c \in \cG_0 \setminus \cG_1] \underset{u \leftarrow \mu_{\delta}^{k}}{\Pr}[u \in \cG_1 \setminus \cG_0]}\right]  \\
  % &\overset{\hphantom{\eqref{eq:diff_s01_next}}}{\geq} \frac{\Pr_{\pi^{(k)}}[c_1 = 1 \land c \in \cG_1 \setminus \cG_0] -
  %   (1 - \frac{1}{k})\epsilon}{\underset{u \leftarrow \mu_{\delta}^{k}}{\Pr}[u \in \cG_1 \setminus \cG_0]} \\
  % &\overset{\eqref{eq:gu_rel}}{\geq} \frac{\delta \underset{}{\Pr}[u \in \cG_1 \setminus \cG_0] + \epsilon - (1-\frac{1}{k})\epsilon}{\underset{}{\Pr}[u \in \cG_1 \setminus \cG_0]} \\
  % &\overset{\hphantom{\eqref{eq:diff_s01_next}}}{\geq} \delta +  \frac{\epsilon}{k},
\end{align*}
where in $(*)$ we use the facts that $\widetilde{D}_2(x,\rho) \neq \bot$ if and only if $\widetilde{D}_2$ finds $c \in \cG_1 \setminus \cG_0$
and conditioned on $\widetilde{D}_2(x,\rho) \neq \bot$  we have that $\Gamma_V(\widetilde{D}_2(x,r)) = 1$ if and only if $c_1 = 1$.

\backmatter

\bibliographystyle{alpha}
\bibliography{refs}

\end{document}
