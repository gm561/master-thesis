In this section we set up notation and terminology used in the thesis.
%
\section{Notation}
\textbf{(Algorithms, Bitstrings and Circuits)}
We define a circuit as a directed acyclic graph with input vertices and vertices implementing logical functions \textit{and}, \textit{or}, and \textit{not}.
We denote circuits using capital letters from the Greek and English alphabet.
For a circuit $C$ we write $\mathit{Size}(C)$ to denote total number of vertices of $C$.
We define a \textit{family of probabilistic circuits} $\{C_n\}$ as a family of circuits taking as part of the input a random bitstring.
A circuit $C_n \in \{C_n\}$ is called a probabilistic circuit.
We define a \textit{two phase circuit} $C := (C_1, C_2)$ as a circuit where in the first phase a circuit $C_1$
is used and in the second phase a circuit $C_2$.
It is well known \cite{Arora:2009:CCM:1540612} that a probabilistic polynomial time algorithm can be represented as a circuit of polynomial size.
Additionally it can be computed in polynomial time and logarithmic space. Therefore, whenever we state a theorem about
circuits we can also apply it to polynomial time algorithms.

We write $\mathit{poly}(\alpha_1, \dots, \alpha_n)$ to denote a polynomial on variables $\alpha_1, \dots, \alpha_n$.
For an algorithm $A$ we write $\mathit{Time}(A)$ to denote the number of steps it takes to execute $A$.
We often write the randomness used by a probabilistic algorithm explicitly as a bitstring provided as an auxiliary input.

We write $\{0,1\}^{n}$ to denote a set of all bitstring of length $n$, and $\{0,1\}^{*}$ for a set of bitstrings that length is arbitrary.
We note that for probabilistic algorithms and circuits the length of an arbitrary bitstring provided as an input is naturally bounded by
running time of an algorithm or number of input vertices of a circuit.

\textbf{(Probabilities and distributions)}
For a finite set $\cR$ we write $r \xleftarrow{\$} \cR$ to denote that $r \in \cR$ is chosen from $\cR$ uniformly at random.
For $\delta \in \R : 0 \leq \delta \leq 1$ we write $\mu_{\delta}$ to denote the Bernoulli distribution where outcome $1$ occurs with
probability $\delta$ and $0$ with probability $1-\delta$.
Moreover, we use $\mu_{\delta}^k$ to denote the probability distribution over $k$-tuples
where each element of a $k$-tuple is drawn independently according to $\mu_{\delta}$.
Finally, let $u \leftarrow \mu_{\delta}^k$ denote that a $k$-tuple $u$ is chosen according to $\mu_{\delta}^k$.

Let $(\Omega, \cF, \Pr)$ be a probability space and $n \in \N$. We say that an event $E_n \in \cF$
happens \textit{almost surely} or with \textit{high probability} if $\Pr[E_n] \geq 1 - 2^{-n} \mathit{poly}(n)$.

\textbf{(Interactive protocols)} We are often interested in situations where two probabilistic algorithms interact with each other according to some protocol.
A protocol execution between two probabilistic algorithms $A$ and $B$ is denoted by $\langle A, B \rangle$.
We limit ourselves to the cases where $A$ and $B$ interact by means of messages that can be represented as bitstrings.
The output of $A$ in a protocol execution is denoted by $\langle A, B \rangle_A$ and of $B$ by $\langle A, B \rangle_B$.
A sequence of all messages sent by $A$ and $B$ in the protocol execution is called a communication transcript and
is denoted by $\langle A, B \rangle_{\mathit{trans}}$.
%
\section{Pairwise independent family of hash functions}
\begin{definition}[Polynomial time sampleable function]
  \begin{todo}
    \textbf{TODO:}
  \end{todo}
\end{definition}

\begin{definition}[Pairwise independent family of efficient hash functions]
Let $\cD$ and $\cR$ be finite sets and $\cH$ be a family of functions mapping values from $\cD$ to values in $\cR$.
We say that $\cH$ is an \textnormal{efficient family of pairwise independent hash functions}
if $\cH$ has the following properties.

\textbf{(Pairwise independent)} For $\forall x \neq y \in \mathcal{D}$ and $\forall \alpha, \beta \in \cR$, it holds
\begin{displaymath}
\underset{\hash \la0 \cH}{\Pr}[hash(x) = \alpha \mid hash(y) = \beta] = \frac{1}{|\cR|}.
\end{displaymath}

\textbf{(Polynomial time sampleable)} For every $\mathit{hash} \in \cH$ the function $\mathit{hash}$ is sampleable in time $\mathit{poly}(\log|\cD|, \log|\cR|)$.

\textbf{(Efficiently computable)}
For every $hash \in \cH$ there exists an algorithm running in time $\mathit{poly}(\log|\cD|, \log|\cR|)$ which
on input $x \in \cD$ outputs $y \in \cR$ such that $y = hash(x)$.
\end{definition}

We note that the pairwise independence property is equivalent to
\begin{displaymath}
\underset{\hash \la0 \cH}{\Pr}[hash(x) = \alpha \land hash(y) = \beta] = \frac{1}{|\cR|^2}.
\end{displaymath}
It is well know \cite{Carter:1977:UCH:800105.803400} that there exists a family of hash functions meeting the above criteria.

%TODO: How to efficiently implement the random function
% \section{Efficient implementation of random functions}
% It is possible to implement a random function $\hash: Q \rightarrow \{0,1,\dotsc, 2(h+v)-1\}$ efficiently by for example building its function table on the fly.

\section{Oracle Algorithms}
Explain the motivation to model our problem using algorithms with oracle access.
Clarify the case when an algorithm is run with different oracles, or the oracle calls are simulated and answered by the algorithm.
Description how the oracle calls are counted.

%the difference between traditional approach and our apporach
%algorithms circuit equivalence

%%% Local Variables:
%%% mode: latex
%%% TeX-master: "master"
%%% End:
