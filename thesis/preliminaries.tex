In this section we define the notation used in this thesis. Furthermore, we give definitions of some notations
used in further chapters.
\section{Notation}
\textbf{(Probabilities and distributions)}
Let $\cR$ be a finite set, then we write $r \leftarrow \cR$ to denote that $r \in \cR$ is chosen from $\cR$ uniformly at random.
For $\delta \in \R : 0 \leq \delta \leq 1$ we write $\mu_{\delta}$ to denote the Bernoulli distribution where outcome $1$ occurs with
probability $\delta$ and $0$ with probability $1-\delta$.
Moreover, we use $\mu_{\delta}^k$ to denote a probability distribution over $k$-tuples
where each element of a $k$-tuple is drawn independently according to $\mu_{\delta}$.
Finally, let $u \leftarrow \mu_{\delta}^k$ denote that a $k$-tuple $u$ is chosen according to $\mu_{\delta}^k$.

Let $(\Omega, \cF, \Pr)$ be a probability space. We say that an event $E_n \in \cF$
happens \textit{almost surely} or with \textit{high probability} if $\Pr[E_n] \geq 1 - 2^{-n} \mathit{poly}(n)$.

\textbf{(Interactive protocols)} We are often interested in situations where two probabilistic circuits interact with each other according to some protocol.
A protocol execution between two probabilistic circuits $A$ and $B$ is denoted by $\langle A, B \rangle$.
The output of $A$ in such a protocol execution is denoted by $\langle A, B \rangle_A$ and of $B$ by $\langle A, B \rangle_B$.
We consider a case when $A$ and $B$ interact by means of messages that can be represented as bitstrings.
A sequence of all messages sent by $A$ and $B$ in the protocol execution is called a communication transcript and
is denoted by $\langle A, B \rangle_{\mathit{trans}}$.

\textbf{(Algorithms, Circuits and Bitstrings)}
We write $\mathit{poly}(\alpha_1, \dots, \alpha_n)$ to denote a polynomial on variables $\alpha_1, \dots, \alpha_n$.
%TODO more precise?
We define a circuit $C$ as a directed acyclic graph with vertices that degree equals 0 - input vertices,
1 - vertices implementing logical $\mathit{Negation}$, or 2 - vertices implementing logical functions $\mathit{And}$, $\mathit{Or}$.
We denote circuits using capital letters from Greek and English alphabet.
For a circuit $C$ we write $\mathit{Size}(C)$ to denote the total number of vertices.

For an algorithm $A$ we write $\mathit{Time}(A)$ to denote the number of steps it takes to execute $A$.
We often write the randomness used by a probabilistic algorithm explicitly as a bitstring taken as an input.

We write $\{0,1\}^{n}$ to denote a bitstring of length $n$.
We also use $\{0,1\}^{*}$ which should be understood that the length of the bitstring is arbitrary.
Exemplary, for a probabilistic algorithm $A$ that uses bitstring $\delta = \{0,1\}^{*}$ as a source of randomness
the length of $\delta$ is naturally bounded by $\mathit{Time}(A)$.

\subsection{Pairwise independent hash functions}
\begin{definition}[Efficient pairwise independent family of hash functions]
Let $\cD$ and $\cR$ be finite sets and $\cH$ be a family of functions mapping values from $\cD$ to values in $\cR$.
We say that $\cH$ is \textnormal{the efficient family of pairwise independent hash functions}
if $\cH$ has the following properties.

\textbf{(Pairwise independent)} For $\forall x \neq y \in \mathcal{D}$ and $\forall \alpha, \beta \in \cR$, we have
\begin{displaymath}
\underset{\hash \la0 \cH}{\Pr}[hash(x) = \alpha \mid hash(y) = \beta] = \frac{1}{|\cR|}.
\end{displaymath}

\textbf{(Polynomial time sampleable)} For every $\mathit{hash} \in \cH$ the function $\mathit{hash}$ is sampleable in time $\mathit{poly}(\log|\cD|, \log|\cR|)$.

\textbf{(Efficiently computable)}
For every $hash \in \cH$ there exists an algorithm running in time $\mathit{poly}(\log|\cD|, \log|\cR|)$ which
on input $x \in \cD$ outputs $y \in \cR$ such that $y = hash(x)$.
\end{definition}

We note that the pairwise independence property is equivalent to
\begin{displaymath}
\underset{\hash \la0 \cH}{\Pr}[hash(x) = \alpha \land hash(y) = \beta] = \frac{1}{|\cR|^2}.
\end{displaymath}
%TODO: How to efficiently implement the random function

\section{Oracel Machines and Circuits}
% algorithms = circuits equivalence
% how to write

We define a family of probabilistic circuit $\{C_n\}$ as a family of circuits taking as part of the input a random bitstring.
A circuit $C_n \in \{C_n\}$ is called a probabilistic circuit.

We define a \textit{two phase circuit} $C := (C_1, C_2)$ as a circuit where in the first phase the circuit $C_1$
is used and in the second phase the circuit $C_2$.

\section{Algorithm simulation with access to the oracle}

\section{Basic inequalities}
Chernoff ...
%%% Local Variables:
%%% mode: latex
%%% TeX-master: "master"
%%% End:
