In this section we set up notation and terminology used in the thesis.
%
\section{Notation and Definitions}
\textbf{(Algorithms, Bitstrings and Circuits)}
We define a Boolean circuit as a directed acyclic graph with input vertices and vertices implementing logical functions \textit{and}, \textit{or}, and \textit{not}.
We denote Boolean circuits using capital letters from the Greek and English alphabet.

We define a \textit{probabilistic circuit} as a Boolean circuit $C_{m,n} : \{0,1\}^{m} \times \{0,1\}^{n} \rightarrow \{0,1\}^{*}$,
For input $x \in \{0,1\}^{m}$ we write to denote $C_{m,n}(x;r)$ where $r \in \{0,1\}^{n}$ is called auxiliary input.
If a probabilistic circuit does not take input $x$, we slightly abuse notation and write $C_{n}(r)$.
Similarly we use $\{C_n\}$ to denote a family of probabilist circuits that takes only auxiliary input.
\begin{todo}
  \textbf{TODO:} What $n$ denotes?
\end{todo}

We make sure that it is clear from the context that probabilist circuits with only auxiliary input
are not confused with circuits that do not take auxiliary input.

For a (probabilist) circuit $C$ we write $\mathit{Size}(C)$ to denote the total number of vertices of $C$.

We define a \textit{two phase circuit} $C := (C_1, C_2)$ as a circuit where in the first phase a circuit $C_1$ is used and in the second phase a circuit $C_2$.
If $C_1$ and $C_2$ are probabilistic circuits we write $C(\delta) := (C_1, C_2)(\delta)$ to denote that in both phases $C_1$ and $C_2$ take
as auxiliary input the same bitstring $\delta$.

It is well known \cite{Arora:2009:CCM:1540612} that a probabilistic polynomial time algorithm can be represented as a circuit of polynomial size.
Moreover, it can be computed in polynomial time and logarithmic space.
Therefore, whenever we state a theorem about circuits we can also apply it to polynomial time algorithms.

We write $\mathit{poly}(\alpha_1, \dots, \alpha_n)$ to denote a polynomial on variables $\alpha_1, \dots, \alpha_n$.
For an algorithm $A$ we write $\mathit{Time}(A)$ to denote the number of steps it takes to execute $A$.
We say that the algorithm $A$ runs in polynomial time if number of steps required to evaluate $A$ is bounded by $poly(|x|)$, where $|x|$ denotes
the length of the input that $A$ takes.
Similarly, as for probabilistic circuits we write the randomness used by a probabilistic algorithm explicitly as a bitstring provided as an auxiliary input.

\textbf{(Probabilities and distributions)}
For a finite set $\cR$ we write $r \xleftarrow{\$} \cR$ to denote that $r$ is chosen from $\cR$ uniformly at random.
For $\delta \in \R : 0 \leq \delta \leq 1$ we write $\mu_{\delta}$ to denote the Bernoulli distribution where outcome $1$ occurs with
probability $\delta$ and $0$ with probability $1-\delta$.
Moreover, we use $\mu_{\delta}^k$ to denote the probability distribution over $k$-tuples
where each element of a $k$-tuple is drawn independently according to $\mu_{\delta}$.
Finally, let $u \leftarrow \mu_{\delta}^k$ denote that a $k$-tuple $u$ is chosen according to $\mu_{\delta}^k$.

Let $(\Omega, \cF, \Pr)$ be a probability space and $n \in \N$. We say that an event $E_n \in \cF$
happens \textit{almost surely} or with \textit{high probability} if $\Pr[E_n] \geq 1 - 2^{-n} \mathit{poly}(n)$.

\begin{todo}
  \textbf{TODO:} Non-negligible function and probability
define computational security
\end{todo}
\begin{todo}
  \textbf{TODO:} Define a protocol?
\end{todo}
\begin{todo}
  \textbf{TODO:} Define a protocol running time of the protocol
\end{todo}
\begin{todo}
 \textbf{TODO}: efficiently computable function
\end{todo}

\textbf{(Interactive protocols)} We are often interested in situations where two probabilistic algorithms interact with each other according to some protocol.
We limit ourselves to the cases where algorithms interact by means of messages representable by bitstrings.
A protocol execution between two probabilistic algorithms $A$ and $B$ is denoted by $\langle A, B \rangle$.
The output of $A$ in a protocol execution is denoted by $\langle A, B \rangle_A$ and of $B$ by $\langle A, B \rangle_B$.
A sequence of all messages sent by $A$ and $B$ in the protocol execution is called a communication transcript and
is denoted by $\langle A, B \rangle_{\mathit{trans}}$.

\textbf{(Oracle algorithms)}
\begin{todo}
  \textbf{TODO:} set up notation
\end{todo}

\begin{definition}[Polynomial time sampleable distribution]
We say that a distribution is polynomial time sampleable if it can be approximated by polynomial time algorithm
up to exponential factor.
\end{definition}

\begin{definition}[Pairwise independent family of efficient hash functions]
Let $\cD$ and $\cR$ be finite sets and $\cH$ be a family of functions mapping values from $\cD$ to values in $\cR$.
We say that $\cH$ is an \textnormal{efficient family of pairwise independent hash functions}
if $\cH$ has the following properties.

\textbf{(Pairwise independent)} For $\forall x \neq y \in \mathcal{D}$ and $\forall \alpha, \beta \in \cR$, it holds
\begin{displaymath}
\underset{\hash \la0 \cH}{\Pr}[hash(x) = \alpha \mid hash(y) = \beta] = \frac{1}{|\cR|}.
\end{displaymath}

\textbf{(Polynomial time sampleable)} For every $\mathit{hash} \in \cH$ the function $\mathit{hash}$ is sampleable in time $\mathit{poly}(\log|\cD|, \log|\cR|)$.

\textbf{(Efficiently computable)}
For every $hash \in \cH$ there exists an algorithm running in time $\mathit{poly}(\log|\cD|, \log|\cR|)$ which
on input $x \in \cD$ outputs $y \in \cR$ such that $y = hash(x)$.
\end{definition}

We note that the pairwise independence property is equivalent to
\begin{displaymath}
\underset{\hash \la0 \cH}{\Pr}[hash(x) = \alpha \land hash(y) = \beta] = \frac{1}{|\cR|^2}.
\end{displaymath}
It is well know \cite{Carter:1977:UCH:800105.803400} that there exists a family of hash functions meeting the above criteria.

%%% Local Variables:
%%% mode: latex
%%% TeX-master: "master"
%%% End:
