\chapter{Preliminaries}
\section{Notation}
For $\delta \in \R : 0 \leq \delta \leq 1$ we write $\mu_{\delta}$ to denote the Bernoulli distribution where outcome $1$ occurs with
probability $\delta$ and $0$ with probability $1-\delta$.
Moreover, we use $\mu_{\delta}^k$ to denote a probability distribution over $k$-tuples
where each element of a $k$-tuple is drawn independently according to $\mu_{\delta}$.
Finally, let $u \leftarrow \mu_{\delta}^k$ denote that a $k$-tuple $u$ is chosen according to $\mu_{\delta}^k$.

We use capital letters to denote boolean circuits. We define a family of probabilistic circuit $\{C_n\}$
as a family of circuits taking as part of the input a random bitstring.
A circuit $C_n \in \{C_n\}$ is called a probabilistic circuit.

We are often interested in situations where two probabilistic circuits interact with each other according to some protocol.
A protocol execution between two such circuits $A$ and $B$ is denoted by $\langle A, B \rangle$.
The output of $A$ in such a protocol execution is denoted by $\langle A, B \rangle_A$ and of $B$ by $\langle A, B \rangle_B$.
We define a transcript of the protocol execution between $A$ and $B$ as all messages sent by both $A$ and $B$.
We assume that messages of both $A$ and $B$ can be represented as bitstrings, and transcript of communication between $A$ and $B$
can be repressed as a tuple consisting from messages sent by $A$ and $B$.
We denote the transcript of communication between $A$ and $B$ by $\langle A, B \rangle_{\mathit{trans}}$.

We define a \textit{two phase circuit} $C := (C_1, C_2)$ as a circuit where in the first phase the circuit $C_1$
is used and in the second phase the circuit $C_2$.

We say that an event happens \textit{almost surely} or with \textit{high probability} if
it occurs with probability at least $1 - 2^{-n} \mathit{poly}(n)$.

For a circuit $C$ let $\mathit{Size}(C)$ denote the number of gates of $C$. For an algorithm $A$ we write $\mathit{Time}(A)$ to denote the running time of $A$.
Finally, we use $\mathit{poly}(\alpha_1, \dots, \alpha_n)$ to denote a polynomial on variables $\alpha_1, \dots, \alpha_n$.

\section{Hardness amplification results}

\section{Weakly verifiable puzzles}

\section{Examples}

\section{Algorithm simulation with access to the oracle}


%%% Local Variables:
%%% mode: latex
%%% TeX-master: "master"
%%% End:
