\begin{abstract}
% A good abstract explains in one line why the paper is important.
% It then goes on to give a summary of your major results, preferably couched in numbers with error limits.
% The final sentences explain the major implications of your work. A good abstract is concise, readable, and quantitative.
% Length should be ~ 1-2 paragraphs, approx. 400 words.
%
% Abstracts generally do not have citations.
% Information in title should not be repeated.
% Be explicit.
% Use numbers where appropriate.
% Answers to these questions should be found in the abstract:
% What did you do?
% Why did you do it? What question were you trying to answer?
% How did you do it? State methods.
% What did you learn? State major results.
% Why does it matter? Point out at least one significant implication.
%
%
% 1) new hardness proof
% 2) it is possible to start with a weak crypto primitive and obtain strong one.
% 3) a natural puzzle fixing technique similar to the one used in the classical proofs of weak one-way function implying strong ones
% 4) uses an arbitrary monotone function to decide the result
% 5) compare with previous works
%
We give a proof of hardness amplification for a new type of weakly verifiable puzzles that generalize the previous approaches.

An important question in cryptography is whether it is possible to build cryptographic construction that is strongly hard
from a one that is only weakly hard. An example is the well known result of Yao which shows that it is possible to build a strong one-way function
from functions that are only weakly one-way hard.

Weakly verifiable puzzles can be seen as problems where a solution cannot be efficiently verified by a party that solves a puzzle.
The problem of hardness amplification for various variants of weakly verifiable puzzles has been
studied in a series of papers \cite{canetti2004hardness, Dodis:2009:SAI:1530441.1530450, DBLP:journals/corr/abs-1002-3534}.
Dodis, Impagliazzo, Jaiswal, and Kabanets \cite{Dodis:2009:SAI:1530441.1530450} define dynamic weakly verifiable puzzles
that generalize cryptographic primitives like message authentication codes and signature schemes.
They prove that it is possible to amplify hardness for this type of puzzles.
Holenstein and Schoenebeck \cite{DBLP:journals/corr/abs-1002-3534} introduce interactive weakly verifiable puzzles
that generalize constructions like bit commitment protocols and also give the proof of hardness amplification for this case.

In this thesis we introduce the notion of \textit{dynamic interactive weakly verifiable puzzles} that combines the previous definitions of dynamic and interactive puzzles.
Similar as in the previous works we state the hardness amplification theorem for these puzzles.
More precisely, we show that if there is no efficient probabilistic algorithm that solves a single puzzle with the substantial probability,
then it is not possible to find an efficient probabilistic algorithm that is a good solver for several independent instances of puzzles.
\end{abstract}

