\begin{abstract}
Weakly verifiable puzzles can be seen as problems where a solution
cannot be efficiently verified by the solver. They are introduced by
Canetti et al. (Theory of Cryptography, 2005), who give a first hardness amplification
result. Dodis et al. (Theory of cryptography, 2009) define dynamic weakly verifiable
puzzles that generalize, for example, MACs and signature schemes. They
prove a hardness amplification result for the case where the solver
must solve some fraction of puzzles. Holenstein and Schönebeck (Theory of Cryptography,
2011) introduce interactive weakly verifiable puzzles that generalize,
for example, commitment schemes. They study hardness amplification for
these puzzles and give the right bound even for the setting where a
monotone function applied to the output of the checking circuit of the
puzzle has to evaluate to one.

In this thesis we introduce dynamic interactive weakly verifiable
puzzles by combining the definitions of Dodis et al. and Holenstein et
al. We combine their proofs to obtain a hardness amplification result
for monotone functions in this case.
\end{abstract}

% Local Variables:
% mode: latex
% TeX-master: "thesis"
% End:
