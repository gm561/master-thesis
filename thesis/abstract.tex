\begin{abstract}
% A good abstract explains in one line why the paper is important.
% It then goes on to give a summary of your major results, preferably couched in numbers with error limits.
% The final sentences explain the major implications of your work. A good abstract is concise, readable, and quantitative.
% Length should be ~ 1-2 paragraphs, approx. 400 words.
%
% Absrtracts generally do not have citations.
% Information in title should not be repeated.
% Be explicit.
% Use numbers where appropriate.
% Answers to these questions should be found in the abstract:
% What did you do?
% Why did you do it? What question were you trying to answer?
% How did you do it? State methods.
% What did you learn? State major results.
% Why does it matter? Point out at least one significant implication.

% 1) new hardness proof
% 2) it is possible to start with a weak crypto primitive and obtain strong one.
% 3) a natural puzzle fixing technique similar to the one used in the classical proofs of weak one-way function implying strong ones
% 4) uses an arbitrary monotone function to decide the result
% 5) compare with previous works
% 6)

We give a new proof of hardness amplification for Dynamic Weakly Verifiable Puzzles which is more general than the previous ones.

Our proof applies to the various cryptographic primitives like hash function,
one way functions, signature schemes and bit commitment protocols.

\end{abstract}
